%!TeX root = main.tex
\chapter{Mathematische Grundlagen}

\begin{comment}
Hier werde ich mich auf \cite{sabbah_cimpa90} und \cite{coutinho1995primer}
beziehen.
\end{comment}


\begin{defn}[Direkte Summe] \cite[4(Categories).5.1]{stacks-project}
Seien $x,y\in \Ob(\cC)$, eine \emph{direkte Summe}, oder das \emph{Coprodukt}
von $x$ und $y$ ist ein Objekt $x\oplus y\in \Ob(\cC)$ zusammen mit
Morphismen $i\in\Mor_\cC(x,x\oplus y)$ und $j\in\Mor_\cC(y,x\oplus y)$, so
dass die folgende universelle Eigenschaft gilt: für jedes $w\in Ob(\cC)$ mit
Morphismen $\alpha\in\Mor_\cC(x,w)$ und $\beta\in\Mor_\cC(y,w)$ existiert ein
eindeutiges $\gamma\in\Mor_\cC(x\oplus y,w)$, so dass das Diagramm
\begin{center}
\begin{tikzpicture} [scale=3.3, descr/.style={fill=white,inner sep=2.5pt} ]
\matrix (m) [
  matrix of math nodes
  , row sep=2em
  , column sep=3em
  %, text height=3em
  %, text depth=0.25em
]{
    & y         &  &   \\
  x & x\oplus y &  &   \\
    &           &  & w \\
};
%TODO: Pfeile
\path[->,font=\scriptsize,>=angle 90]
(m-1-2) edge node[left]{$j$} (m-2-2)
(m-2-1) edge node[above]{$i$} (m-2-2)
(m-1-2) edge node[right]{$\beta$} (m-3-4)
(m-2-1) edge node[below]{$\alpha$} (m-3-4)
;
\path[->,font=\scriptsize,>=angle 90,dashed]
(m-2-2) edge node[above]{$\exists!\gamma$} (m-3-4)
;
\end{tikzpicture}
\end{center}
kommutiert.
\end{defn}

\begin{defn}[Exakte Sequenz]
Eine Sequenz
\begin{center}
\begin{tikzpicture} [scale=3.3, descr/.style={fill=white,inner sep=2.5pt} ]
  \matrix (m) [
    matrix of math nodes
    , row sep=2em
    , column sep=3em
    %, text height=3em
    %, text depth=0.25em
  ]{
    %M_0 
    & \cdots & M_{i-1} & M_i & M_{i+1} & \cdots & 
    %M_n 
  \\
  };
  %TODO: Pfeile
  \path[->,font=\scriptsize,>=angle 90]
  %(m-1-1) edge node[above]{$f_0$} (m-1-2)
  (m-1-2) edge 
    %node[above]{$f_{i-2}$} 
  (m-1-3)
  (m-1-3) edge node[above]{$f_{i-1}$} (m-1-4)
  (m-1-4) edge node[above]{$f_i$} (m-1-5)
  (m-1-5) edge 
    %node[above]{$f_{i+1}$} 
  (m-1-6)
  %(m-1-6) edge node[above]{$f_{n-1}$} (m-1-7)
  ;
\end{tikzpicture}
\end{center}
heißt exakt, wenn für alle $i$ gilt, dass
$\im(f_{i-1})=\ker{f_i}$.
\end{defn}
\begin{defn}[Kurze exakte Sequenz]
Eine kurze exakte Sequenz ist eine Sequenz
\begin{center}
\begin{tikzpicture} [scale=3.3, descr/.style={fill=white,inner sep=2.5pt} ]
  \matrix (m) [
    matrix of math nodes
    , row sep=2em
    , column sep=3em
    %, text height=3em
    %, text depth=0.25em
  ]{
    0 & M' & M & M'' & 0 \,,\\
  };
  %TODO: Pfeile
  \path[->,font=\scriptsize,>=angle 90]
  (m-1-1) edge node[above]{$  $} (m-1-2)
  (m-1-2) edge node[above]{$f$} (m-1-3)
  (m-1-3) edge node[above]{$g$} (m-1-4)
  (m-1-4) edge node[above]{$  $} (m-1-5)
  ;
\end{tikzpicture}
\end{center}
welche exakt ist.
\end{defn}

\begin{defn}[Kokern]
Ist $f:M'\rightarrow M$ eine Abbildung, so ist der \emph{Kokern} von $f$
definiert als $\coker(f)=M/\im(f)$.
\end{defn}

\begin{prop}
Ist $f:M'\rightarrow M$ eine injektive Abbildung, so ist
\begin{center}
\begin{tikzpicture} [scale=3.3, descr/.style={fill=white,inner sep=2.5pt} ]
  \matrix (m) [
    matrix of math nodes
    %, row sep=2em
    , column sep=3em
    %, text height=3em
    %, text depth=0.25em
  ]{
    0 & M' & M & M/f(M')      & 0 \\
      &    & m & m \mod f(M') &   \\
  };
  %TODO: Pfeile
  \path[->,font=\scriptsize,>=angle 90]
  (m-1-1) edge (m-1-2)
  (m-1-2) edge node[above]{$f$} (m-1-3)
  (m-1-3) edge node[above]{$\pi$} (m-1-4)
  (m-1-4) edge (m-1-5)
  ;
  \path[|->,font=\scriptsize,>=angle 90]
  (m-2-3) edge (m-2-4)
  ;
\end{tikzpicture}
\end{center}
eine kurze exakte Sequenz und $M/f(M')=\coker(f)$ ist der \emph{Kokern} von
$f$.
\end{prop}
\begin{proof}
    
\end{proof}

\begin{comment}
\begin{defn}[Filtrierung] \label{defn:Filtrierung}
\cite[Def 10.13.1.]{stacks-project}
\cite[Rem 2.5.]{elliottDmod}
Eine \emph{aufsteigende Filtrierung $F$} von einem Objekt (Ring) $A$ ist eine
Familie von $(F_iA)_{i\in\Z}$ von Unterobjekten (Unterring), so dass 
\[ 0\subset\cdots\subset F_i\subset F_{i+1} \subset \cdots \subset A \]
und definiere weiter $gr_i^FA:=F_iA\slash F_{k-1}A$ und damit das zu $A$ mit
Filtrierung $F$ \emph{assoziierte graduierte Modul} 
\[gr^FA:=\bigoplus_{k\in\Z}gr_i^FA \,. \]
\end{defn}

\begin{defn}
\cite{ayoubIntro}
\cite[Def 3.2.1]{sabbah_cimpa90}
Eine Filtrierung heißt \emph{gut}, falls ...
\end{defn}
\end{comment}

% vim: set ft=tex :
