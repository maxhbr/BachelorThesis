%!TeX root = main.tex
\chapter{Mathematische Grundlagen}

\begin{comment}
Hier werde ich mich auf \cite{sabbah_cimpa90} und \cite{coutinho1995primer}
beziehen.
\end{comment}

In dieser Arbeit spielen die folgenden Ringe eine große Rolle:
\begin{itemize}
\item $\C[x]:=\{ \sum^{N}_{i=1}a_ix^i | N\in \N \}$
\item $\C\{x\}:=\{ \sum^{\infty}_{i=1}a_ix^i | \mbox{pos.
  Konvergenzradius} \}$
\item $\C\llbracket x\rrbracket:=\{ \sum^{\infty}_{i=1}a_ix^i \}$
\item $K:=\Ckxl:=\C\{x\}[x^{-1}]$ der Ring der Laurent Reihen.
\item $\hat{K}:=\Cfxl:=\C\llbracket x\rrbracket[x^{-1}]$ der Ring der
formalen Laurent Reihen.
\end{itemize}
Wobei offensichtlich die Inclulsionen $\C[x]\subsetneq\C\{x\}\subsetneq\Cfx$ und
$K\subsetneq\hat K$ gelten.

\begin{comment}
Es bezeichnet der Hut ($ \, \hat \,\, $) das jeweils formale äquivalent zu
einem konvergentem Objekt.
\end{comment}

\begin{comment}
\begin{lem}[Seite 2]
ein paar eigenschaften
\begin{enumerate}
\item $\C[x]$ ist ein graduierter Ring, durch die Grad der
Polynome. Diese graduierung induziert eine aufsteigende Filtrierung.

alle Ideale haben die form $(x-a)$ mit $a\in \C$
\item wenn $\mathfrak{m}$ das maximale Ideal von $\C[x]$ (erzeugt von
$x$ ist), so ist
\[
  \C[[x]]=
  \underset{k}{\underleftarrow{\lim}} \C[X]\backslash\mathfrak{m}^k
\]
The ring $\C[[x]]$ ist ein nöterscher lokaler Ring:
jede Potenzreihe mit konstantem term $\neq 0$ ist invertierbar.

Der ring ist ebenfalls ein diskreter ??? Ring (discrete valuation
ring)

Die Filtrierung nach grad des Maximalen Ideals, genannt
$\mathfrak{m}$-adische Fitration, ist die Filtrierung
$\mathfrak{m}^k=\{f\in \C[[x]]|v(f)\geq k\}$

und es gilt $gr_\mathfrak{m}(\C[[x]])=\C[x]$
\end{enumerate}
\end{lem}
\end{comment}

\begin{defn}[Direkte Summe] \cite[4(Categories).5.1]{stacks-project}
Seien $x,y\in \Ob(\cC)$, eine \emph{Direkte Summe} oder das \emph{coprodukt}
von $x$ und $y$ ist ein Objekt $x\oplus y\in \Ob(\cC)$ zusammen mit
Morphismen $i\in\Mor_\cC(x,x\oplus y)$ und $j\in\Mor_\cC(y,x\oplus y)$ so
dass die folgende universelle Eigenschaft gilt: für jedes $w\in Ob(\cC)$ mit
Morphismen $\alpha\in\Mor_\cC(x,w)$ und $\beta\in\Mor_\cC(y,w)$ existiert ein
eindeutiges $\gamma\in\Mor_\cC(x\oplus y,w)$ so dass das Diagram
\begin{center}
\begin{tikzpicture} [scale=3.3, descr/.style={fill=white,inner sep=2.5pt} ]
\matrix (m) [
  matrix of math nodes
  , row sep=2em
  , column sep=3em
  %, text height=3em
  %, text depth=0.25em
]{
    & y         &  &   \\
  x & x\oplus y &  &   \\
    &           &  & w \\
};
%TODO: Pfeile
\path[->,font=\scriptsize,>=angle 90]
(m-1-2) edge node[left]{$j$} (m-2-2)
(m-2-1) edge node[above]{$i$} (m-2-2)
(m-1-2) edge node[right]{$\beta$} (m-3-4)
(m-2-1) edge node[below]{$\alpha$} (m-3-4)
;
\path[->,font=\scriptsize,>=angle 90,dashed]
(m-2-2) edge node[above]{$\exists!\gamma$} (m-3-4)
;
\end{tikzpicture}
\end{center}
kommutiert.
\end{defn}

\begin{defn}[Tensorprodukt]
\cite[3(Algebra).11.21]{stacks-project}
\begin{comment}
Faserprodukt: \cite[4(Categories).6.1]{stacks-project}
\end{comment}
% von Vorlesung Algebra 2

\begin{center}
\begin{tikzpicture} [scale=3.3, descr/.style={fill=white,inner sep=2.5pt} ]
\matrix (m) [
  matrix of math nodes
  , row sep=2em
  , column sep=3em
  %, text height=3em
  %, text depth=0.25em
]{
  M\times N & M\otimes_RN \\
            & T \\
};
%TODO: Pfeile
\path[->,font=\scriptsize,>=angle 90]
(m-1-1) edge node[above]{$  $} (m-1-2)
(m-1-1) edge node[below]{$f$} (m-2-2)
;
\path[->,font=\scriptsize,>=angle 90,dashed]
(m-1-2) edge node[right]{$\exists!\gamma$} (m-2-2)
;
\end{tikzpicture}
\end{center}
\end{defn}
%TODO: tensorprodukt / faserprodukt zum basiswechsel

\begin{defn}[Exacte Sequenz]
Eine Sequenz
\begin{center}
\begin{tikzpicture} [scale=3.3, descr/.style={fill=white,inner sep=2.5pt} ]
\matrix (m) [
  matrix of math nodes
  , row sep=2em
  , column sep=3em
  %, text height=3em
  %, text depth=0.25em
]{
  %M_0 
  & \cdots & M_{i-1} & M_i & M_{i+1} & \cdots & 
  %M_n 
\\
};
%TODO: Pfeile
\path[->,font=\scriptsize,>=angle 90]
%(m-1-1) edge node[above]{$f_0$} (m-1-2)
(m-1-2) edge 
  %node[above]{$f_{i-2}$} 
(m-1-3)
(m-1-3) edge node[above]{$f_{i-1}$} (m-1-4)
(m-1-4) edge node[above]{$f_i$} (m-1-5)
(m-1-5) edge 
  %node[above]{$f_{i+1}$} 
(m-1-6)
%(m-1-6) edge node[above]{$f_{n-1}$} (m-1-7)
;
\end{tikzpicture}
\end{center}
heißt exact, wenn für alle $i$ gilt, dass
$\im(f_{i-1})=\ker{f_i}$.
\end{defn}
\begin{defn}[Kurze exacte Sequenz]
Eine kurze exacte Sequenz ist eine Sequenz
\begin{center}
\begin{tikzpicture} [scale=3.3, descr/.style={fill=white,inner sep=2.5pt} ]
\matrix (m) [
  matrix of math nodes
  , row sep=2em
  , column sep=3em
  %, text height=3em
  %, text depth=0.25em
]{
  0 & M' & M & M'' & 0 \\
};
%TODO: Pfeile
\path[->,font=\scriptsize,>=angle 90]
(m-1-1) edge node[above]{$  $} (m-1-2)
(m-1-2) edge node[above]{$f$} (m-1-3)
(m-1-3) edge node[above]{$g$} (m-1-4)
(m-1-4) edge node[above]{$  $} (m-1-5)
;
\end{tikzpicture}
\end{center}
welche exact ist.
\end{defn}

\begin{defn}[Filtrierung] \label{defn:Filtrierung}
\cite[Def 10.13.1.]{stacks-project}
Eine \emph{aufsteigende Filtrierung $F$} von einem Objekt (Ring) $A$ ist eine
Familie von $(F_iA)_{i\in\Z}$ von Unterobjekten (Unterring), so dass 
\[ 0\subset\cdots\subset F_i\subset F_{i+1} \subset \cdots \subset A \]
und definiere weiter $gr_i^FA:=F_iA\slash F_{k-1}A$ und damit
$gr^FA:=\bigoplus_{k\in\Z}gr_i^FA$.
\begin{comment}
$gr_i^F$ als was??
\end{comment}
\end{defn}

\begin{defn}
\cite{ayoubIntro}
\cite[Def 3.2.1]{sabbah_cimpa90}
Eine Filtrierung heißt \emph{gut}, falls ...
\end{defn}

\begin{defn}[Kommutator]%zula seite 15
Sei $R$ ein Ring. Für $a,b\in R$ wird
\[[a,b]=a\cdot b-b\cdot a\]
als der \emph{Kommutator von a und b} definiert.
\end{defn}
% komutativ, dann immer kommutator gleich 0

\begin{prop} \label{prop:d-modul-komutator-regeln}
Sei $k\in \{ \C[x],\Ckx,\Cfx,K,\hat K \}$.
Sei $\partial_x:k\rightarrow k$ der gewohnte Ableitungsoperator
nach $x$, so gilt 
% geklaut aus Zula Barbara
\begin{enumerate}
\item $[ \partial_x,x] = \partial_xx-x\partial_x=1 $
%und damit ist $\cD_k$ insbesondere nicht kommutativ.
\item für $f\in k$ ist
\[ [\partial_x,f] = \frac{\partial f}{\partial x} \,. \]
\item Es gelten die Formeln\\
\begin{align*}
[\partial_x,x^k]   &= kx^{k-1}\\
[\partial_x^j,x]   &= j\partial_x^{j-1}\\
[\partial_x^j,x^k] &= \sum_{i\geq1}\frac{k(k-1)\cdots(k-i+1)
  \cdot j(j-1)\cdots(j-i+1)}{i!}x^{k-i}\partial_x^{j-i} \\
\end{align*}
\end{enumerate}
\end{prop}
\begin{proof}
\begin{enumerate}
\item Klar.
\item Für ein Testobjekt $g\in k$ ist
\[
[\partial_x,f]\cdot g=\partial_x(fg)-f\partial_xg=
  (\partial_xf)g+\underset{=0}{\underbrace{ 
  f(\partial_xg)-f(\partial_xg)}}=
  (\partial_xf)g
\]
\item Siehe \cite[???]{ZulaBarbara}
\end{enumerate}
\end{proof}

% vim: set ft=tex :
