%!TeX root = main.tex
% das gleiche wie Meromorphe zusammenhänge?

\chapter{Moduln über $\cD_k$}
Ich werde hier die Weyl Algebra, wie in \cite[Chapter~1]{sabbah_cimpa90}, in
einer Veränderlichen einführen. 
Ab hier sei $k\in \{ \C[x],\Ckx,\Cfx,K,\hat K \}$.

\section{Weyl-Algebra und der Ring $\cD_k$} 
%TODO: sabah-cimpa90.pdf  seite 3
%script ginzburg.pdf seite 34 als garbe definiert
Sei dazu $\frac{\partial}{\partial x}=\partial_x$ der Ableitungsoperator nach
$x$ und sei $f \in k$.
Man hat die folgende
Kommutations-Relation zwischen dem \emph{Ableitungsoperator} und dem
\emph{Multiplikations Operator} $f$:
\begin{equation}\label{eq:weyl_relation}
[\frac{\partial}{\partial x},f]=\frac{\partial f}{\partial x}
\end{equation}
wobei die Rechte Seite die Multiplikation mit $\frac{\partial f}{\partial x}$
darstellt. Dies bedeutet, für alle $g\in\C[x]$ hat man
\[
[\frac{\partial}{\partial x},f]\cdot g
=\frac{\partial fg}{\partial x} - f\frac{\partial g}{\partial x}
=\frac{\partial f}{\partial x} \cdot g \,.
\]
%\begin{defn}[Weyl Algebra, $\cD$]
%%TODO: eigentlich ist \cD ja das wichtigste!
%Definiere nun die Weyl Algebra $A_1(\C)$ (bzw. die Algebra $\cD$ von linearen
%Operatoren mit Koeffizienten in $\C\{x\}$ bzw. die Algebra $\hat{\cD}$
%(Koeffizienten in $\C\llbracket x\rrbracket$)) als die Quotientenalgebra der
%freien Algebra, welche von dem Koeffizientenring zusammen mit dem Element
%$\partial_x$, erzeugt wird, Modulo der Relation \eqref{eq:weyl_relation}.
%\end{defn}
\begin{defn}
Definiere nun den Ring $\cD_k$ als die Quotientenalgebra der freien Algebra,
welche von dem Koeffizientenring in $k$ zusammen mit dem Element $\partial_x$,
erzeugt wird, Modulo der Relation \eqref{eq:weyl_relation}.  Wir schreiben
diesen Ring auch als
\begin{itemize}
\item $A_1(\C):=\C[x]<\partial_x>$ falls $k=\C[x]$, und nennen ihn die
\emph{Weyl Algebra}
\item $\cD:=\C\{x\}<\partial_x>$ falls $k=\C\{x\}$
\item $\hat\cD:=\Cfx<\partial_x>$ falls $k=\Cfx$
\item $\cD_K:=\Ckxl<\partial_x>$ falls
$k=K\overset{\mbox{def}}{=}\C\{x\}[x^{-1}]$
\item $\cD_{\hat K}:=\Cfxl<\partial_x>$ falls $k=\hat
K\overset{\mbox{def}}{=}\Cfx[x^{-1}]$
\end{itemize}
\end{defn}
\begin{bem}
Es gilt $\cD[x^{-1}]=\cD_K$ und $\hat \cD[x^{-1}]=\cD_{\hat K}$.
\end{bem}
%Wir werden die Notation $A_1(\C):=\C[x]<\partial_x>$ (bzw.
%$\cD:=\C\{x\}<\partial_x>$ bzw. 
%$\hat{\cD}:=\C\llbracket x\rrbracket<\partial_x>$) verwenden.

%TODO: vlt umsortieren
\begin{comment}
\begin{lem} %TODO: vervollständigen
Sei $A$ einer der 3 soeben eingeführten Objekten, so definieren die Addition 
\[ +:A\times A \rightarrow A \]
und die Multiplikation
\[ \cdot:A\times A \rightarrow A \]
eine Ringstruktur auf $A$.
\end{lem}
\begin{proof}
\cite[Kapittel 2 Section 1]{ZulaBarbara}
\end{proof}
\end{comment}

%\begin{rem}
%%TODO: alle sind
%$A_1(\C),~\cD$ und $\hat\cD$ sind nicht kommutative Algebren.
%\end{rem}

%\begin{prop} \label{prop:weyl_eindeutige_schreibung}
%\cite[Proposition 1.2.3]{sabbah_cimpa90}
%Jedes Element in $A_1(\C)$ (bzw. $\cD$ oder $\hat{\cD}$) kann auf eindeutige
%weiße als $P=\sum_{i=0}^na_i(x)\partial_x^i$, mit $a_i(x)\in A_1(\C)$ (bzw.
%$\cD$ oder $\hat{\cD}$), geschrieben werden. 
%\end{prop}
\begin{prop} \label{prop:weyl_eindeutige_schreibung}
\cite[Proposition 1.2.3]{sabbah_cimpa90}
Jedes Element in $\cD_k$ kann auf eindeutige
weiße als $P=\sum_{i=0}^na_i(x)\partial_x^i$, mit $a_i(x)\in k$, geschrieben
werden.
\end{prop}
\begin{proof}
Siehe \cite[Proposition 1.2.3]{sabbah_cimpa90}
\begin{comment}
ein teil des Beweises ist "left as an exersice"
\end{comment}
\end{proof}

\begin{comment}
Besser?:\\
erst Filtrierung definieren und dadurch dann den Grad?
\end{comment}
\begin{defn}
Sei $P=\sum_{i=0}^na_i(x)\partial_x^i$, wie in Proposition
\ref{prop:weyl_eindeutige_schreibung}, gegeben, so definiere 
\[
\deg P:=\max\{i|a_i\neq 0\}
\]
als den \emph{Grad} von $P$.
\end{defn}
In natürlicher Weise erhält man die aufsteigende Filtrierung
$F_N\cD:=\{P\in\cD|\deg P\leq N\}$ mit
\[
\cdots\subset F_{-1}\cD\subset F_{0}\cD\subset
F_{1}\cD\subset\cdots\subset\cD
\]
und erhalte $gr_k^F\cD\underset{\mbox{def}}{=}F_N\cD\slash F_{N-1}\cD
=\{P\in\cD|\deg P=N\}\cong\C\{x\}$.

\begin{proof}
Sei $P\in F_N\cD$ so betrachte den Isomorphismus:
\[
F_N\cD\slash F_{N-1}\cD\rightarrow \C\{x\}; [P]=P+F_{N-1}\cD\mapsto a_n(x)
\]
\end{proof}

\begin{prop}
Es gilt:
\begin{center}
\begin{tikzpicture} [descr/.style={fill=white,inner sep=2.5pt}]
\matrix (m) [
  matrix of math nodes,
  row sep=1em,
  %column sep=-0.7em,
  text height=1.5ex,
  text depth=0.25ex]
{
  gr^F\cD &
  := \bigoplus_{N\in\Z}gr_N^F\cD = \bigoplus_{N\in\N_0}gr_N^F\cD \cong
  \bigoplus_{N\in\N_0}\C\{x\} \cong \C\{x\}[\xi] = &
  \bigoplus_{N\in\N_0}\C\{x\}\cdot \xi^N \\
};
\path[solid]
(m-1-1) edge [bend right=15] node[descr]{$\cong$}
  node[below]{$\mbox{isomorph als grad. Ringe}$} (m-1-3);
\end{tikzpicture}
\end{center}
also $gr^F\cD \cong \bigoplus_{N\in\N_0}\C\{x\}\cdot \xi^N$ als graduierte
Ringe.
\end{prop}
\begin{proof} TODO
\begin{comment}
Treffen?
\end{comment}
\end{proof}

%\begin{comment}
%\subsection{Weyl Algebra als Graduierter Ring}
%% Treffen 2
%Sei $A$ nun einer der drei Koeffizienten Ringe, welche zuvor behandelt
%wurden.  Der Ring $A<\partial_x>$ kommt zusammen mit einer aufsteigenden
%Filtrierung, welche wir mit $F(A<\partial_x)$ bezeichen werden.  Sei $P$ ein
%bzgl. \ref{prop:weyl_eindeutige_schreibung} minimal geschriebener Operator,
%so ist $P$ in $F_k$ falls der maximale Grad von $\partial_x$ in $P$ kleiner
%oder gleich $k$. So definiere den Grad $deg P$ von $P$ als die Eindeutige
%ganze Zahl $k$ mit $P\in F_kA<\partial_x>\slash F_{k-1}<\partial_x>$

%Unabhängigkeit von Schreibung wird in Sabbah Script behauptet
%\end{comment}

\subsection{Alternative Definition / Sichtweise}
%Nur abgeschrieben
\cite[Chap 1.1.]{kashiwara2003d}
Sei $X$ eine $1$-Dimensionale Complexe Mannigfaltigkeit und $\cO_X$ die Garbe
der holomorphen Funktionen auf $X$. Ein (holomorpher) \emph{differential
Operator} auf $X$ ist ein Garben-Morphismus $P:\cO_X\rightarrow \cO_X$, lokal
in der Koordinate $x$ und mit holomorphen Funktionen $a_n(x)$ als
\[
(Pu)(x)=\sum_{n\geq0}a_n(x)\partial_x^nu(x)
\]
geschrieben (für $u\in\cO_X$). Zusätzlich nehmen wir an, dass $a_n(x)\equiv 0$
für fast alle $n\in \N$ gilt. Wir setzten
$\partial_x^nu(x)=\frac{\partial^nu}{\partial x^n}(x)$. Wir sagen ein Operator
hat Ordnung $m$, falls $\forall n\geq m: \alpha_n(x)\equiv0$.  Mit $\cD_X$
bezeichnen wir die Garbe von Differentialoperatoren auf $X$. Die Garbe $\cD_X$
hat eine Ring Struktur mittels der Komposition als Multiplikation und $\cO_X$
ist ein Unterring von $\cD_X$. Sei $\Theta_X$ die Garbe der Vektorfelder über
über $X$. Es gilt, dass $\Theta_X$ in $\cD_X$ enthalten ist. Bemerke auch, dass
$\Theta_X$ ein links $\cO_X$-Untermodul, aber kein rechts $\cO_X$-Untermodul ist.

\begin{prop}
\cite[Exmp 1.1]{ArkhipovDmod}
Sei $X=\A^1=\C$, $\cO_X=\C[t]$ und $\Theta_X=\C[t]\partial$. Wobei $\partial$
als $\partial(t^n)=nt^{n-1}$ wirkt. Dann sind die Differentialoperatoren 
\begin{align*}
\cD_X &= \C[t,\partial], & \mbox{mit} & & \partial t-t\partial & =1.
\end{align*}
Somit stimmt die Alternative Definition schon mal mit der Einfachen überein.
\end{prop}

\begin{comment}
\begin{defn} \cite[Defn 2.1]{ArkhipovDmod}
Sei $X=\A^1$, $\cO_X=\C[t]$ und $\cD_X=[t,\partial]$ mit der Relation
$[\partial,t]=1$. Dann definieren wir die links $\cD$-Moduln über $\A^1$ als
die $\C[t,\partial]$-Moduln. Sie werden geschrieben als $\cD-mod(\A^1)$
\end{defn}
\end{comment}

\section{(Links) $\cD$-Moduln}
Da $\cD$ ein nichtkommutativer Ring ist, muss man vorsichtig sein und zwischen
links unr rechts $\cD$-Moduln unterschiden. Wenn ich im folgendem von
$\cD$-Moduln rede, werde ich mich immer, wie auch \cite[Chapter
1.6.]{arapuraNotes}, auf links $\cD$-Moduln beziehen.

\begin{exmp}[Einfachste links $\cD$-Moduln]
% Sergey-Arkhipov-MAT1191_Lecture_Notes.pdf Chapter 2.1
\cite[Exmp 2.2]{ArkhipovDmod}
%Sei $X=\A^1$ und $\sO_X=\C[t]$.
\begin{enumerate}
\item $\cD$ ist ein links und rechts $\cD$-Modul
\item $\cM=\C[t]$ durch
\begin{itemize}
\item $\partial(f(t))=\frac{\partial f}{\partial t}$ und
$t\cdot f(t)=tf$
\item oder \cite[Exmp 3.1.2]{ginzburg} $\C[t]=\cD\cdot1=\cD/\cD\cdot\partial$.
\end{itemize}
\item $\cM=\C[t,t^{-1}]$ mit $t\cdot t^{m}=t^{m+1}$ und
$\partial(t^m)=mt^{m-1}$
\end{enumerate}
\end{exmp}
\begin{exmp}[Weiter $\cD$-Moduln] 
\begin{enumerate}
\item \cite[Exmp 2.2]{ArkhipovDmod}
Führe formal, also ohne jeglichen analytischen Hintergurnd, ein Symbol
$\exp(\lambda t)$ ein, mit $\partial(f(t)\exp(\lambda t))=\frac{\partial
f}{\partial t}\exp(\lambda t)+f\lambda\exp(\lambda t)$.  So ist
$\cM=\sO_X\exp(\lambda t)$ ein $\cD$-Modul.
\item \cite[Exmp 3.1.4]{ginzburg}
Führe formal ein Symbol $\log(x)$ mit den Eigenschaften
$\partial\cdot\log(x)=\frac{1}{x}$ ein. Erhalte nun das $\cD$-Modul
$\C[x]\log(x)+\C[x,x^{-1}]$. Dieses Modul ist über $\cD$ erzeugt durch
$\log(x)$ und man hat
\[
\C[x]\log(x)+\C[x,x^{-1}]=\cD\cdot\log(x)=\cD\slash\cD(\partial x\partial) \,.
\]
\end{enumerate}
%TODO: Variable zu t machen?
\end{exmp}

\begin{lem}\cite[Lem 2.3.3.]{sabbah_cimpa90}
Sei $\cM$ ein links $\cD$-Modul von endlichem Typ, welches auch von endlichem
Typ über $\Ckx$ ist. Dann ist $\cM$ bereits ein freies $\C\{x\}$-Modul.
\end{lem}
\begin{proof}
Siehe \cite[Lem 2.3.3.]{sabbah_cimpa90}.
\end{proof}
\begin{cor} \cite[Cor 2.3.4.]{sabbah_cimpa90}
Falls $\cM$ ein links $\cD$-Modul von endlichem typ, welches außerdem ein
endich dimensionaler Vektorraum ist, so ist schon $\cM=\{0\}$.
\end{cor}

\subsection{Holonome $\cD$-Moduln}
\begin{comment}
TODO: defn of Car als Charakteristische Varietät
\end{comment}
\begin{defn} \cite[Def 3.3.1.]{sabbah_cimpa90}
Sei $\cM$ lineares Differentialsystem 
(linear differential system) %TODO: defn
.  Man sagt, $\cM$ ist holonom, falls $\cM=0$ oder falls $\Car\cM\subset
\{x=0\}\cup{\xi=0}$.
\end{defn}
\begin{lem} \cite[Lem 3.3.8.]{sabbah_cimpa90}
Ein $\cD$-Modul ist holonom genau dann, wenn $\dim_{gr^F\cD,0}gr^F\cM=1$.
\end{lem}
\begin{proof}
Siehe \cite[Lem 3.3.8.]{sabbah_cimpa90}
\end{proof}

\begin{comment}
\section{Lokalisierung von $\Ckx$-Moduln}
\cite[Chap 4.1.]{sabbah_cimpa90}
Sei $M$ ein $\Ckx$-Modul. Wir schreiben $M[x^{-1}]$ für den $K$-Vektor Raum
$M\otimes_{\Ckx}K$. Im allgemeinen gilt, falls $M$ von andlichen Typ über
$\Ckx$ ist, so ist $C[x^{-1}]$ von endlichem Typ über $K$. Bemerke aber, dass
$M[x^{-1}]$ generell nicht von endlichem Typ über $\Ckx$ ist.
\end{comment}

\section{Lokalisierung eines (holonomen) $\cD$-Moduls}
\cite[Chap 4.2.]{sabbah_cimpa90}
%TODO: was heißt holonom (holonomic)
Sei $\cM$ ein links $\cD$-Modul. Betrachte $\cM$ als $\C\{x\}$-Modul und
definiere darauf
\[ \cM[x^{-1}]:=\cM\otimes_{\C\{x\}}K \]
als die Lokalisierung von $\cM$.
\begin{prop} \cite[Prop 4.2.1.]{sabbah_cimpa90}
$\cM[x^{-1}]$ bekommt in natürlicher weiße eine $\cD$-Modul Struktur.
\end{prop}
\begin{proof} \cite[Prop 4.2.1.]{sabbah_cimpa90}
mit:
\[
\partial_x(m\otimes x^{-k})=((\partial_xm)\otimes x^{-k})-km\otimes x^{-k-1}
\]
\begin{comment}
beweis der $\cD$-linearität ist als übung gelassen
\end{comment}
\end{proof}


%vim: set ft=tex :
