% das gleiche wie Meromorphe zusammenhänge?

\chapter{Der Ring $\cD$}

\section{Weyl-Algebra und der Ring $\cD$} 
%TODO: sabah-cimpa90.pdf  seite 3
%script ginzburg.pdf seite 34 als garbe definiert
Ich werde hier die Weyl Algebra, wie in
\cite[Chapter~1]{sabbah_cimpa90}, in einer Veränderlichen einführen.
Sei $\frac{\partial}{\partial x}=\partial_x$ der Ableitungsoperator nach $x$
und sei $f \in\C[x]$ (bzw. $\C\{x\}$ bzw. $\C\llbracket x\rrbracket$).
Man hat die folgende Kommutations-Relation zwischen dem
\emph{Ableitungsoperator}
und dem \emph{Multiplikations Operator} $f$:
% TODO; ist das der Kommutator??
\begin{equation}\label{eq:weyl_relation}
[\frac{\partial}{\partial x},f]=\frac{\partial f}{\partial x}
\end{equation}
wobei die Rechte Seite die Multiplikation mit $\frac{\partial f}{\partial x}$
darstellt. Dies bedeutet, für alle $g\in\C[x]$ hat man:
\[
[\frac{\partial}{\partial x},f]\cdot g
=\frac{\partial fg}{\partial x} - f\frac{\partial g}{\partial x}
=\frac{\partial f}{\partial x} \cdot g
\]
\begin{defn}[Weyl Algebra, $\cD$]
%TODO: eigentlich ist \cD ja das wichtigste!
Definiere nun die Weyl Algebra $A_1(\C)$ (bzw. die Algebra $\cD$ von
linearen Operatoren mit Koeffizienten in $\C\{x\}$ bzw. die Algebra
$\hat{\cD}$ (Koeffizienten in $\C\llbracket x\rrbracket$)) als die
Quotientenalgebra der freien Algebra, welche von dem Koeffizientenring
zusammen mit dem Element $\partial_x$, erzeugt wird, Modulo der Relation
\eqref{eq:weyl_relation}.
\end{defn}
\begin{comment}
\begin{defn}
Definiere nun den Ring $\cD_k$ als die
Quotientenalgebra der freien Algebra, welche von dem Koeffizientenring
zusammen mit dem Element $\partial_x$, erzeugt wird, Modulo der Relation
\eqref{eq:weyl_relation}.
Wir schreiben diesen Ring als
\begin{itemize}
\item $A_1(\C)=\C[x]<\partial_x>$ falls $k=\C[x]$, und nennen ihn die
\emph{Weyl Algebra}
\item $\cD=\C\{x\}<\partial_x>$ falls $k=\C\{x\}$
\item $\hat\cD=\Cfx<\partial_x>$ falls $k=\Cfx$
\item $\cD_K=\Ckxl<\partial_x>$ falls
$k=K\overset{\mbox{def}}{=}\C\{x\}[x^{-1}]$
\item $\cD_{\hat K}=\Cfxl<\partial_x>$ falls $k=\hat
K\overset{\mbox{def}}{=}\Cfx[x^{-1}]$
\end{itemize}
\end{defn}
\end{comment}
Wir werden die Notation $A_1(\C):=\C[x]<\partial_x>$ (bzw.
$\cD:=\C\{x\}<\partial_x>$ bzw. 
$\hat{\cD}:=\C\llbracket x\rrbracket<\partial_x>$) verwenden.

\begin{comment}
Beispiele und Alternative Definition:\\
Sergey-Arkhipov-MAT1191\_Lecture\_Notes.pdf Chapter 2.1
\end{comment}

%TODO: vlt umsortieren
\begin{lem} %TODO: vervollständigen
Sei $A$ einer der 3 soeben eingeführten Objekten, so definieren die Addition 
\[ +:A\times A \rightarrow A \]
und die Multiplikation
\[ \cdot:A\times A \rightarrow A \]
eine Ringstruktur auf $A$.
\end{lem}
\begin{proof}
\cite[Kapittel 2 Section 1]{ZulaBarbara}
\end{proof}

\begin{rem}
%TODO: alle sind
$A_1(\C),~\cD$ und $\hat\cD$ sind nicht kommutative Algebren.
\end{rem}

% ist das nichtkommutativ??
\begin{defn}[Kommutator]%zula seite 15
Sei $R$ ein Ring. Für $a,b\in R$ wird
\[[a,b]=a\cdot b-b\cdot a\]
der \emph{Kommutator von a und b} genannt.
\end{defn}
% komutativ, dann immer kommutator gleich 0

\begin{prop} % geklaut aus Zula Barbara
\begin{enumerate}
\item Es gilt
\[[ \partial_x,x] = \partial_xx-x\partial_x=1 \]
\item Sei $f\in \C[x]$, so gilt:
\[ [\partial_x,f] = \frac{\partial f}{\partial x} \,. \]
Denn für $g\in \C[x]$ ist
\[
[\partial_x,f]\cdot g=\partial_x(fg)-f\partial_xg=
  (\partial_xf)g+\underset{=0}{\underbrace{ 
  f(\partial_xg)-f(\partial_xg)}}=
  (\partial_xf)g
\]
\item Es gelten die Formeln\\
\begin{align*}
[\partial_x,x^k]   &= kx^{k-1}\\
[\partial_x^j,x]   &= j\partial_x^{j-1}\\
[\partial_x^j,x^k] &= \sum_{i\geq1}\frac{k(k-1)\cdots(k-i+1)
  \cdot j(j-1)\cdots(j-i+1)}{i!}x^{k-i}\partial_x^{j-i} \\
\end{align*}
\end{enumerate}
\end{prop}
\begin{proof}
\cite{ZulaBarbara}
\end{proof}

\begin{prop} \label{prop:weyl_eindeutige_schreibung}
Jedes Element in $A_1(\C)$ (bzw. $\cD$ oder $\hat{\cD}$) kann auf eindeutige
weiße als $P=\sum_{i=0}^na_i(x)\partial_x^i$, mit $a_i(x)\in A_1(\C)$ (bzw.
$\cD$ oder $\hat{\cD}$), geschrieben werden. 
\end{prop}
\begin{proof}
\cite[Proposition 1.2.3]{sabbah_cimpa90}
\begin{comment}
ein teil des Beweises ist "left as an exersice"
\end{comment}
\end{proof}

%TODO: Beispiele??

%TODO: definition Filtrierung

\begin{defn}
Sei $P=\sum_{i=0}^na_i(x)\partial_x^i$ gegeben, so definiere 
\[
\deg P:=\max\{i|a_i\neq 0\}
\]
In natürlicher Weise erhält man $F_N\cD:=\{P\in\cD|\deg P\leq N\}$ sowie die
entsprechende aufsteigende Filtrierung
\[
\cdots\subset F_{-1}\cD\subset F_{0}\cD\subset
F_{1}\cD\subset\cdots\subset\cD
\]
und erhalte $gr_k^F\cD\underset{\mbox{def}}{=}F_N\cD\slash F_{N-1}\cD
=\{P\in\cD|\deg P=N\}\cong\C\{x\}$.
\end{defn}

\begin{proof}
Sei $P\in F_N\cD$ so betrachte den Isomorphismus:
\[
F_N\cD\slash F_{N-1}\cD\rightarrow \C\{x\}; [P]=P+F_{N-1}\cD\mapsto a_n(x)
\]
\end{proof}

\begin{prop}
Es gilt:
\begin{center}
\begin{tikzpicture} [descr/.style={fill=white,inner sep=2.5pt}]
\matrix (m) [
  matrix of math nodes,
  row sep=1em,
  %column sep=-0.7em,
  text height=1.5ex,
  text depth=0.25ex]
{
  gr^F\cD &
  := \bigoplus_{N\in\Z}gr_N^F\cD = \bigoplus_{N\in\N_0}gr_N^F\cD \cong
  \bigoplus_{N\in\N_0}\C\{x\} \cong \C\{x\}[\xi] = &
  \bigoplus_{N\in\N_0}\C\{x\}\cdot \xi^N \\
};
\path[solid]
(m-1-1) edge [bend right=15] node[descr]{$\cong$}
  node[above]{$\mbox{isomorph als grad. Ringe}$} (m-1-3);
\end{tikzpicture}
\end{center}
also
\[ gr^F\cD \cong \bigoplus_{N\in\N_0}\C\{x\}\cdot \xi^N \,. \]
\end{prop}
\begin{proof} TODO
\begin{comment}
Treffen?
\end{comment}
\end{proof}

\begin{comment}
\subsection{Weyl Algebra als Graduierter Ring}
% Treffen 2
Sei $A$ nun einer der drei Koeffizienten Ringe, welche zuvor behandelt
wurden.  Der Ring $A<\partial_x>$ kommt zusammen mit einer aufsteigenden
Filtrierung, welche wir mit $F(A<\partial_x)$ bezeichen werden.  Sei $P$ ein
bzgl. \ref{prop:weyl_eindeutige_schreibung} minimal geschriebener Operator,
so ist $P$ in $F_k$ falls der maximale Grad von $\partial_x$ in $P$ kleiner
oder gleich $k$. So definiere den Grad $deg P$ von $P$ als die Eindeutige
ganze Zahl $k$ mit $P\in F_kA<\partial_x>\slash F_{k-1}<\partial_x>$

Unabhängigkeit von Schreibung wird in Sabbah Script behauptet
\end{comment}

\section{(Links) $\cD$-Moduln}

\begin{exmp}[Einfachste links $\cD$-Moduln]
% Sergey-Arkhipov-MAT1191_Lecture_Notes.pdf Chapter 2.1
Sei $X=\A^1$ und $\sO_X=\C[t]$.
\begin{enumerate}
\item $\cD$ ist ein $\cD$-Modul
\item $\cM=\sO_X$ durch
\begin{itemize}
\item $\partial(f(t))=\frac{\partial f}{\partial t}$ und
$t\cdot f(t)=tf$
\item oder \cite[Exmp 3.1.2]{ginzburg} $\sO_X=\cD\cdot1=\cD/\cD\cdot\partial$.
\end{itemize}
\item $\cM=\sO_X\exp(\lambda t)$ mit $\partial(f(t)\exp(\lambda
t))=\frac{\partial f}{\partial t}\exp(\lambda t)+f\lambda\exp(\lambda t)$
\item $\cM=\C[t,t^{-1}]$ mit $t\cdot t^{m}=t^{m+1}$ und
$\partial(t^m)=mt^{m-1}$
\end{enumerate}
\end{exmp}

%vim: set ft=tex :
