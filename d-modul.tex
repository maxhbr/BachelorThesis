%!TeX root = main.tex
% das gleiche wie Meromorphe zusammenhänge?

\chapter{Moduln über $\cD_k$}
In diesem Kapittel wird die Weyl Algebra, wie in
\cite[Chapter~1]{sabbah_cimpa90}, in einer Veränderlichen einführen. Ähnlich
wie auch in \cite[Kapittel~2]{ZulaBarbara}. Allgemeiner und in
mehreren Veränderlichen wird die Weyl-Algebra beispielsweise in
\cite[Chapter~1]{coutinho1995primer} definiert.

\begin{defn}[Kommutator]%zula seite 15
Sei $R$ ein Ring. Für $a,b\in R$ wird
\[[a,b]=a\cdot b-b\cdot a\]
als der \emph{Kommutator von a und b} definiert.
\end{defn}
% komutativ, dann immer kommutator gleich 0

\begin{prop} \label{prop:d-modul-komutator-regeln}
Sei $k= \C[x]$ (bzw. $\Ckx$ bzw. $\Cfx$) ein Ring der Potenzreihen in $x$ über
$\C$. Sei $\partial_x:k\rightarrow k$ der gewohnte Ableitungsoperator nach $x$,
so gilt 
% geklaut aus Zula Barbara
\begin{enumerate}
\item $[ \partial_x,x] = \partial_xx-x\partial_x=1 $
%und damit ist $\cD_k$ insbesondere nicht kommutativ.
\item für $f\in k$ ist
\begin{equation} \label{eq:kommutator1}
[\partial_x,f] = \frac{\partial f}{\partial x} \,. 
\end{equation}
\item Es gelten die Formeln
\begin{align}
[\partial_x,x^k] &\!\!\overset{(\ref{eq:kommutator1})}{=}
  \frac{\partial x^k}{\partial x} = kx^{k-1}
  \label{eq:kommutator1b}\\
[\partial_x^j,x]   &= j\partial_x^{j-1}
  \label{eq:kommutator2}\\
[\partial_x^j,x^k] &= \sum_{i\geq1}\frac{k(k-1)\cdots(k-i+1)
  \cdot j(j-1)\cdots(j-i+1)}{i!}x^{k-i}\partial_x^{j-i}
  \label{eq:kommutator3}
\end{align}
\end{enumerate}
\end{prop}
\begin{proof}
\begin{enumerate}
\item Klar.
\item Für ein Testobjekt $g\in k$ ist
\[
[\partial_x,f]\cdot g=\partial_x(fg)-f\partial_xg=
  (\partial_xf)g+\underset{=0}{\underbrace{ 
  f(\partial_xg)-f(\partial_xg)}}=
  (\partial_xf)g \,.
\]
\item Siehe \cite[1.2.4.]{sabbah_cimpa90} oder \cite[Kor 2.8]{ZulaBarbara}.
\end{enumerate}
\end{proof}

\section{Weyl-Algebra und der Ring $\cD_k$}
%TODO: sabah-cimpa90.pdf  seite 3
%script ginzburg.pdf seite 34 als garbe definiert
Sei dazu $\frac{\partial}{\partial x}=\partial_x$ der Ableitungsoperator nach
$x$ und sei $f \in \C[x]$ (bzw. $\Ckx$ bzw. $\Cfx$).
Man hat die folgende
Kommutations-Relation zwischen dem \emph{Ableitungsoperator} und dem
\emph{Multiplikations Operator} $f$:
\begin{equation}\label{eq:weyl_relation}
[\frac{\partial}{\partial x},f]=\frac{\partial f}{\partial x}
\end{equation}
wobei die rechte Seite die Multiplikation mit $\frac{\partial f}{\partial x}$,
also dem bereits abgeleiteten $f$, darstellt. Dies bedeutet, für alle
$g\in\C[x]$ hat man
\begin{align*}
[\frac{\partial}{\partial x},f]\cdot g
  &=\myubracket{\frac{\partial (fg)}{\partial x}}
    - f\cdot \frac{\partial g}{\partial x}
\\&= \myobracket{\frac{\partial f}{\partial x}\cdot g
  + \rlap{$\underbrace{\phantom{
  f\cdot \frac{\partial g}{\partial x}
  - f\cdot \frac{\partial g}{\partial x}
  }}_{=0}$}
  f\cdot \frac{\partial g}{\partial x}}
  - f\cdot \frac{\partial g}{\partial x}
\\&=\frac{\partial f}{\partial x} \cdot g \,.
\end{align*}
\begin{defn}
Definiere nun den Ring $\cD_k$ als die Quotientenalgebra der freien Algebra,
welche von dem Koeffizientenring in $k$ zusammen mit dem Element $\partial_x$,
erzeugt wird, Modulo der Relation \eqref{eq:weyl_relation}.  Wir schreiben
diesen Ring auch als:
\begin{itemize}
\item $A_1(\C)$ falls $k=\C[x]:=\{ \sum^{N}_{i=1}a_ix^i \mid N\in \N \}$, und
nennen ihn die \emph{Weyl Algebra}.
\item $\cD$ falls $k=\C\{x\}:=\{ \sum^{\infty}_{i=1}a_ix^i \mid \mbox{pos.
Konvergenzradius}\}$ die konvergenten Potenzreihen.
\item $\hat\cD$ falls $k=\Cfx:=\{\sum^{\infty}_{i=1}a_ix^i\}$ die formalen
Potenzreihen.
\item $\cD_K$ falls $k=K:=\Ckxl:=\C\{x\}[x^{-1}]$ der Ring der Laurent Reihen.
\item $\cD_{\hat K}$ falls $k=\hat K:=\Cfxl:=\Cfx[x^{-1}]$ der Ring der
formalen Laurent Reihen
\footnote{Wird in \cite{ZulaBarbara} mit $\hat\cD_{\hat K}$ bezeichnet.}.
\end{itemize}
\end{defn}
\begin{bem}
\begin{enumerate}
\item Es bezeichnet der Hut ($ \, \hat \,\, $) das jeweils formale Pendant
zu einem konvergentem Objekt. Dementsprechend könnte man auch $\Cfx$ für
$\hat{\Ckx}$ schreiben.
\item Es gilt $\cD[x^{-1}]=\cD_K$ und $\hat \cD[x^{-1}]=\cD_{\hat K}$.
\item Offensichtlich erhält $\cD_k$ in kanonischer Weise eine nichtkommutative
Ringstruktur, dies ist in \cite[Kapittel 2 Section 1]{ZulaBarbara} genauer
ausgeführt.
\end{enumerate}
\end{bem}

\begin{prop} \label{prop:weyl_eindeutige_schreibung}
Jedes Element in $\cD_k$ kann auf eindeutige
Weise als $P=\sum_{i=0}^na_i(x)\partial_x^i$, mit $a_i(x)\in k$, geschrieben
werden.
\end{prop}
\begin{proof}
Siehe \cite[Proposition 1.2.3]{sabbah_cimpa90}.
%TODO: ein teil des Beweises ist "left as an exersice"
\end{proof}
\begin{comment}
Gilt das folgende??
\[
\alpha_i(x)\partial_x^i \equiv \frac{\alpha_i}{x^i}(x\partial_x)^i \mod
F_{i-1}\cD
\]
\end{comment}

\begin{comment}
Besser?:\\
erst Filtrierung definieren und dadurch dann den Grad?
\end{comment}
\begin{defn}
Sei $P=\sum_{i=0}^na_i(x)\partial_x^i$, wie in Proposition
\ref{prop:weyl_eindeutige_schreibung}, gegeben, so definiere
\[
\deg P:=\max\Big\{\{i\mid a_i\neq 0\}\cup\{-\infty\}\Big\}
\]
als den \emph{Grad (oder den $\partial_x$-Grad)}
von $P$.
\end{defn}
\begin{comment}
In natürlicher Weise erhält man die aufsteigende Filtrierung
$F_N\cD:=\{P\in\cD|\deg P\leq N\}$ mit
\[
\cdots\subset F_{-1}\cD\subset F_{0}\cD\subset
F_{1}\cD\subset\cdots\subset\cD
\]
und erhalte $gr_k^F\cD\bydef F_N\cD\slash F_{N-1}\cD
=\{P\in\cD|\deg P=N\}\cong\C\{x\}$.

\begin{proof}[Beweisidee]
Sei $P\in F_N\cD$, so betrachte den Isomorphismus $F_N\cD\slash
F_{N-1}\cD\rightarrow \C\{x\}$ definiert durch $[P]=P+F_{N-1}\cD\mapsto
a_n(x)$.
\end{proof}

\begin{prop}
Es gilt:
\begin{center}
\begin{tikzpicture} [descr/.style={fill=white,inner sep=2.5pt}]
\matrix (m) [
  matrix of math nodes,
  row sep=1em,
  %column sep=-0.7em,
  text height=1.5ex,
  text depth=0.25ex]
{
  gr^F\cD &
  := \bigoplus_{N\in\Z}gr_N^F\cD = \bigoplus_{N\in\N_0}gr_N^F\cD \cong
  \bigoplus_{N\in\N_0}\C\{x\} \cong \C\{x\}[\xi] = &
  \bigoplus_{N\in\N_0}\C\{x\}\cdot \xi^N \\
};
\path[solid]
(m-1-1) edge [bend right=15] node[descr]{$\cong$}
  node[below]{$\mbox{isomorph als grad. Ringe}$} (m-1-3);
\end{tikzpicture}
\end{center}
also $gr^F\cD \cong \bigoplus_{N\in\N_0}\C\{x\}\cdot \xi^N$ als gradierte
Ringe.
\end{prop}
\begin{proof}
TODO: Treffen?
\end{proof}
\end{comment}

\begin{comment}
\subsection{Alternative Definition / Sichtweise}
\cite[Chap 1.1.]{kashiwara2003d}
Sei $X$ eine $1$-dimensionale komplexe Mannigfaltigkeit und $\cO_X$ die Garbe
der holomorphen Funktionen auf $X$. Ein \emph{(holomorpher)
Differenzialoperator} auf $X$ ist ein Garben-Morphismus $P:\cO_X\rightarrow
\cO_X$, lokal in der Koordinate $x$ und mit holomorphen Funktionen $a_n(x)$ als
\[
(Pu)(x)=\sum_{n\geq0}a_n(x)\partial_x^nu(x)
\]
geschrieben (für $u\in\cO_X$). Zusätzlich nehmen wir an, dass $a_n(x)\equiv 0$
für fast alle $n\in \N$ gilt. Wir setzten
$\partial_x^nu(x)=\frac{\partial^nu}{\partial x^n}(x)$. Wir sagen, ein Operator
hat höchstens Ordnung $m$, falls $\forall n\geq m: \alpha_n(x)\equiv0$.
\begin{defn}
Mit $\cD_X$ bezeichnen wir die \emph{Garbe von Differentialoperatoren} auf $X$.
\end{defn}
Die Garbe $\cD_X$ hat eine Ringstruktur mittels der Komposition als
Multiplikation und $\cO_X$ ist ein Unterring von $\cD_X$. Sei $\Theta_X$ die
Garbe der Vektorfelder über $X$. Es gilt, dass $\Theta_X$ in $\cD_X$
enthalten ist.  Bemerke auch, dass $\Theta_X$ ein Links-$\cO_X$-Untermodul,
aber kein rechts $\cO_X$-Untermodul ist.

\begin{prop}
\cite[Exmp 1.1]{ArkhipovDmod}
Sei $X=\A^1=\C$, $\cO_X=\C[t]$ und $\Theta_X=\C[x]\partial_x$, wobei
$\partial_x$ als $\partial_x(x^n)=nx^{n-1}$ wirkt, dann sind die
Differentialoperatoren
\begin{align*}
\cD_X &= \C[x,\partial_x], & \mbox{mit} & & \partial_x x-x\partial_x & =1.
\end{align*}
Somit stimmt die alternative Definition bereits mit der einfachen überein.
\end{prop}
\end{comment}

\begin{comment}
\begin{defn} \cite[Defn 2.1]{ArkhipovDmod}
Sei $X=\A^1$, $\cO_X=\C[x]$ und $\cD_X=[x,\partial_x]$ mit der Relation
$[\partial_x,x]=1$. Dann definieren wir die Links-$\cD$-Moduln über $\A^1$ als
die $\C[x,\partial_x]$-Moduln. Sie werden geschrieben als $\cD-mod(\A^1)$
\end{defn}
\end{comment}

\section{(Links-) $\cD$-Moduln}
Da $\cD$ ein nichtkommutativer Ring ist, muss man vorsichtig sein und zwischen
Links- und Rechts-$\cD$-Moduln unterscheiden.
Wenn im folgendem von $\cD$-Moduln gesprochen wird, werden immer
%, wie auch \cite[Chapter 1.6.]{arapuraNotes},
Links-$\cD$-Moduln gemeint.

\begin{exmp}
Hier einige Beispiele für (Links-) $\cD$-Moduln
\begin{enumerate}
%
\item $\cD$ ist ein Links- und Rechts-$\cD$-Modul
%
\item $\cM=\C[x]$ oder $\cM=\C[x,x^{-1}]$ jeweils durch $x\cdot x^{m}=x^{m+1}$
und $\partial(x^m)=mx^{m-1}$
%
\item Führe formal, also ohne analytischen Hintergrund, ein Objekt
$\exp(\lambda x)$ ein, mit $\partial(f(x)\exp(\lambda x))=\frac{\partial
f}{\partial x}\exp(\lambda x)+f\lambda\exp(\lambda x)$.  So ist
$\cM=\C[x]\exp(\lambda x)$ ein $\cD$-Modul.
%
\begin{comment}
\cite[Exmp 2.2]{ArkhipovDmod}
\end{comment}
\item Führe analog ein Symbol $\log(x)$ mit den Eigenschaften
$\partial_x\log(x)=\frac{1}{x}$ ein. Erhalte nun das $\cD$-Modul
$\C[x]\log(x)+\C[x,x^{-1}]$. Dieses Modul ist über $\cD$ durch $\log(x)$
erzeugt und man hat
\[
\C[x]\log(x)+\C[x,x^{-1}]=\cD\cdot\log(x)=\cD\slash\cD(\partial_x x\partial_x) \,.
\]
%
\begin{comment}
\cite[Exmp 3.1.4]{ginzburg}
\end{comment}
\end{enumerate}
%TODO: Variable zu t machen?
\end{exmp}

\begin{comment}
\begin{lem}\cite[Lem 2.3.3.]{sabbah_cimpa90}
Sei $\cM$ ein Links-$\cD$-Modul von endlichem Typ, welches auch von endlichem
Typ über $\Ckx$ ist. Dann ist $\cM$ bereits ein freies $\C\{x\}$-Modul.
\end{lem}
\begin{proof}
Siehe \cite[Lem 2.3.3.]{sabbah_cimpa90}.
\end{proof}
\begin{cor} \cite[Cor 2.3.4.]{sabbah_cimpa90}
Falls $\cM$ ein Links-$\cD$-Modul von endlichem Typ, welches außerdem ein
endlich dimensionaler Vektorraum ist, so ist schon $\cM=\{0\}$.
\end{cor}
\end{comment}

\section{Holonome $\cD_k$-Moduln}
\begin{defn}
Sei $\cM_k$ ein Links-$\cD_k$-Modul ungleich $\cD_k$. $\cM_k$ heißt
\emph{holonom}, falls es ein Element $m\in\cM_k$ gibt, das $\cM_k$ als
$\cD_k$-Modul erzeugt. Im speziellen folgt damit, dass $\cM_k\cong
\cD_k/\mathfrak{a}$ für ein $0\neq\mathfrak{a}\vartriangleleft\cD_k$.
\end{defn}
\begin{bem}
\begin{comment}
Dies hier ist eine sehr vereinfachte, aber für unsere Zwecke völlig
ausreichende, Definition von holonom.
\end{comment}
In \cite{coutinho1995primer} wird der Begriff holonom über die Dimension
definiert und bei \cite{sabbah_cimpa90} über die charakteristische Varietät.
\end{bem}

\begin{bem} 
Nach \cite[Prop 10.1.1]{coutinho1995primer} gilt
\begin{itemize}
\item Submoduln und Quotienten von holonomen $\cD_k$-Moduln sind holonom
\item sowie endliche Summen von holonomen $\cD_k$-Moduln sind holonom
\end{itemize}
und laut \cite[Thm. 4.2.3]{sabbah_cimpa90} gilt, dass
\begin{itemize}
\item für ein holonomes $\cD_{\C\{x\}}$-Modul $\cM_{\C\{x\}}$
(bzw. ein $\cD_{\Cfx}$-Modul $\cM_{\Cfx}$)
ist die Lokalisierung
\begin{align*}
\cM_{\C\{x\}}[x^{-1}]&:=\cM_{\C\{x\}}\otimes_{\C\{x\}}K
&&\text{(bzw. }\cM_{\Cfx}[x^{-1}]:=\cM_{\Cfx}\otimes_{\Cfx}\hat K\text{ ),}
\end{align*}
mit der $\cD_{\C\{x\}}$ (bzw. $\cD_{\Cfx}$) Modul Struktur durch
\[
\partial_x(m\otimes x^{-k})=((\partial_xm)\otimes x^{-k})-km\otimes x^{-k-1}
\]
wieder holonom.
\end{itemize}
\end{bem}

\begin{thm}
Sei $\cM_k$ ein holonomes $\cD_k$-Modul, dann gilt, dass seine Lokalisierung
isomorph zu $\cD_k/\cD_k\cdot P$, mit einem $P\in \cD_k$ ungleich Null, ist.
\end{thm}
\begin{proof}
Siehe \cite[Cor 4.2.8]{sabbah_cimpa90}.
\end{proof}

\begin{comment}
\subsubsection{Alternative Definition B} %nach Sabbah
\begin{defn} \cite[Def 3.3.1.]{sabbah_cimpa90}
Sei $\cM$ lineares Differentialsystem
(linear differential system) %TODO: defn
.  Man sagt, $\cM$ ist holonom, falls $\cM=0$ oder falls $\Car\cM\subset
\{x=0\}\cup{\xi=0}$.
\end{defn}
\begin{lem} \cite[Lem 3.3.8.]{sabbah_cimpa90}
Ein $\cD$-Modul ist holonom genau dann, wenn $\dim_{gr^F\cD,0}gr^F\cM=1$.
\end{lem}
\begin{proof}
Siehe \cite[Lem 3.3.8.]{sabbah_cimpa90}
\end{proof}
\end{comment}

\begin{comment}
\subsubsection{Alternative Definition A} %nach Countinho
\begin{defn}[Holonome $\cD$-Moduln]
\cite[Chap 10 §1]{coutinho1995primer}
Ein endlich generierter $\cD$-Modul $\cM$ ist \emph{holonom}, falls $\cM=0$
gilt, oder falls es die Dimension $1$ hat.
\end{defn}
\begin{bem}
\cite[Chap 10 §1]{coutinho1995primer}
Sei $\mathfrak{a}\neq 0$ ein Links-Ideal von $\cD$. Es gilt nach
\cite[Corollary 9.3.5]{coutinho1995primer}, dass $d(\cD/\mathfrak{a})\leq 1$.
Falls $\mathfrak{a}\neq\cD$, dann gilt nach der \emph{Bernstein's inequality}
\cite[Chap 9 §4]{coutinho1995primer}, dass $d(\cD/\mathfrak{a})=1$. Somit ist
$\cD/\mathfrak{a}$ ein holonomes $\cD$-Modul.
\end{bem}
\end{comment}

%vim: set ft=tex :
