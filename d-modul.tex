% das gleiche wie Meromorphe zusammenhänge?

\chapter{Links $\cD$-Moduln}

\begin{exmp}[Einfachste links $\cD$-Moduln]
% Sergey-Arkhipov-MAT1191_Lecture_Notes.pdf Chapter 2.1
Sei $X=\A^1$ und $\sO_X=\C[t]$.
\begin{enumerate}
\item $\cD$ ist ein $\cD$-Modul
\item $\cM=\sO_X$ mit $\partial(f(t))=\frac{\partial f}{\partial t}$ und
$t\cdot f(t)=tf$.
\item $\cM=\sO_X\exp(\lambda t)$ mit $\partial(f(t)\exp(\lambda
t))=\frac{\partial f}{\partial t}\exp(\lambda t)+f\lambda\exp(\lambda t)$
\item $\cM=\C[t,t^{-1}]$ mit $t\cdot t^{m}=t^{m+1}$ und
$\partial(t^m)=mt^{m-1}$
\end{enumerate}
\end{exmp}

\begin{comment}
  %TODO: unsere sind bereits Lokalisiert?

  \section{Lokalisierung eines $\C\{x\}$-Moduls}

  \begin{defn}
    Sei $M$ ein $\C\{x\}$-Modul und $K=\C\{x\}[x^{-1}]$, dann ist die
    Lokalisierung
    \[ M[x^{-1}]:=M\otimes_{\C\{x\}}K \,. \]
  \end{defn}

  \section{Lokalisierung eines holonomen $\cD$-Moduls}
\end{comment}

%vim: set ft=tex :
