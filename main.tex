\documentclass[ngerman
  ,numbers=noenddot % obsolete
  ,headsepline
  ,parskip=half*
  ,openany
  ,DIV=12
  ,fleqn % mv align to left
]{scrbook}
  %,12pt
  %,titelpage % unused
  %,biblography=totoc  % unused
  %,landscape,twocolumn
\usepackage[ngerman]{babel}
\usepackage[T1]{fontenc}
\usepackage[utf8]{inputenc}
%\usepackage[ansinew]{inputenc}
\usepackage{lmodern} %Type1-Schriftart für nicht-englische Texte

\setlength{\unitlength}{1mm}
\emergencystretch 2em

\usepackage{lscape} % alt: pdflscape
\usepackage{amsmath,amsthm,amssymb}
\usepackage{latexsym}
\usepackage{enumerate}
\usepackage{verbatim}
\usepackage{graphicx}

\usepackage{listings}

\allowdisplaybreaks

%\usepackage{thmbox}

%\usepackage{natbib}

\usepackage[automark]{scrpage2} % Headline styles
\usepackage[square,numbers]{natbib}


\pagestyle{scrheadings}
%A better solution would be to use typearea package.


% Section Numbers moved left out
%\usepackage{sectsty}
%\makeatletter\def\@seccntformat#1{%
  %\protect\makebox[0pt][r]{\csname the#1\endcsname\hspace{12pt}}}\makeatother

% semibold instead of bold
%\renewcommand{\bfdefault}{sb}

\renewcommand*{\thefootnote}{[\arabic{footnote}]}

% Zeichen
\usepackage{nicefrac}
\usepackage{mathrsfs}
\usepackage{stmaryrd}
%simplewick

\usepackage{tikz}
\usetikzlibrary{matrix,arrows,decorations.pathmorphing}
\usepackage{subfig}

\newcommand{\myNewtonPlot}[6]{
  \draw[color=black,thick] #2;
  \foreach \pos in #1 { \fill[blue,opacity=.2] (-.5,#5) rectangle \pos; }
  %\clip (-.5,#4) rectangle (#3,#5);
  \draw[->] (-.5,0) -- (#3+.7,0);
  \draw[->] (0,#4-.2) -- (0,#5+.2);
  \draw (1,0) -- (1,-.1);
  \draw (0,1) -- (-.1,1);
  \foreach \pos in #1 { \node[draw,circle,inner sep=1.5pt,fill=white] at \pos {}; }
  \node [below right] at (#3,#5/2) {#6};
}

\numberwithin{equation}{chapter}
\numberwithin{figure}{chapter}

%%%%  Abkürzungsverzeichnis  %%%%%%%%%%%%%%%%%%%%%%%%%%%%%%%%%%%%%%%%
\usepackage{nomencl}
% Befehl umbenennen in abk
\let\abk\nomenclature
% Deutsche Überschrift
\renewcommand{\nomname}{Abkürzungsverzeichnis}
% Punkte zw. Abkürzung und Erklärung
\setlength{\nomlabelwidth}{.20\hsize}
\renewcommand{\nomlabel}[1]{#1 \dotfill}
% Zeilenabstände verkleinern
\setlength{\nomitemsep}{-\parsep}
\makenomenclature

%% BASH: makeindex main.nlo -s nomencl.ist -o main.nls

%%%%  Theoreme Styles  %%%%%%%%%%%%%%%%%%%%%%%%%%%%%%%%%%%%%%%%%%%%%%
\makeatletter

\theoremstyle{plain}% default
\newtheorem{thm}{Satz}[chapter]
%\newtheorem{thm}[satz]{Satz}
\newtheorem{lemma}[thm]{Lemma}
\newtheorem{lem}[thm]{Lemma}
\newtheorem{kor}[thm]{Korollar}
\newtheorem{prop}[thm]{Proposition}
\newtheorem{cor}[thm]{Korollar}

\theoremstyle{definition}
\newtheorem{defn}[thm]{Definition}
\newtheorem{conj}[thm]{Conjecture}
\newtheorem{exmp}[thm]{Beispiel}
\newtheorem{bsp}[thm]{Beispiel}

\theoremstyle{remark}
\newtheorem{rem}[thm]{Bemerkung}
\newtheorem{bem}[thm]{Bemerkung}
\newtheorem{note}[thm]{Notiz}
\newtheorem{case}{Fall}

%custom pdf-metadata
%\pdfinfo{
%/Title (Title)
%/Subject (Subject)
%/Author (Maximilian Huber)
%/Keywords (keywords) }

\usepackage[german]{varioref}
%\usepackage[nameinlink,german]{cleveref}
\usepackage[colorlinks=true,linkcolor=black]{hyperref}

%% Autoref Names %%%%%%%%%%%%%%%%%
%\crefname{lemma}{Lemma}{Lemmas}
%\crefname{equation}{Gleichung}{Gleichungen}
%\crefname{definition}{Definition}{Definitionen}
%\crefname{algorithmus}{Algorithmus}{Algorithmen}
%\crefname{kor}{Korollar}{Korollare}

%%%%  new Commands  %%%%%%%%%%%%%%%%%%%%%%%%%%%%%%%%%%%%%%%%%%%%%%%%%
\def\bme{\boldsymbol{e}}

\def\A{\ensuremath \mathbb{A}}
%\def\B{\ensuremath \mathbb{B}}
\def\C{\ensuremath \mathbb{C}}
%\def\D{\ensuremath \mathbb{D}}
%\def\E{\ensuremath \mathbb{E}}
%\def\F{\ensuremath \mathbb{F}}
%\def\G{\ensuremath \mathbb{G}}
%\def\H{\ensuremath \mathbb{H}}
\def\I{\ensuremath \mathbb{I}}
%\def\J{\ensuremath \mathbb{J}}
%\def\K{\ensuremath \mathbb{K}}
%\def\L{\ensuremath \mathbb{L}}
%\def\M{\ensuremath \mathbb{M}}
\def\N{\ensuremath \mathbb{N}}
%\def\O{\ensuremath \mathbb{O}}
%\def\P{\ensuremath \mathbb{P}}
%\def\Q{\ensuremath \mathbb{Q}}
\def\R{\ensuremath \mathbb{R}}
%\def\S{\ensuremath \mathbb{S}}
%\def\T{\ensuremath \mathbb{T}}
%\def\U{\ensuremath \mathbb{U}}
%\def\V{\ensuremath \mathbb{V}}
%\def\W{\ensuremath \mathbb{W}}
%\def\X{\ensuremath \mathbb{X}}
%\def\Y{\ensuremath \mathbb{Y}}
\def\Z{\ensuremath \mathbb{Z}}

%\def\cA{\ensuremath \mathcal{A}}
%\def\cB{\ensuremath \mathcal{B}}
\def\cC{\ensuremath \mathcal{C}}
\def\cD{\ensuremath \mathcal{D}}
%\def\cE{\ensuremath \mathcal{E}}
%\def\cF{\ensuremath \mathcal{F}}
%\def\cG{\ensuremath \mathcal{G}}
%\def\cH{\ensuremath \mathcal{H}}
%\def\cI{\ensuremath \mathcal{I}}
%\def\cJ{\ensuremath \mathcal{J}}
%\def\cK{\ensuremath \mathcal{K}}
%\def\cL{\ensuremath \mathcal{L}}
\def\cM{\ensuremath \mathcal{M}}
\def\cN{\ensuremath \mathcal{N}}
%\def\cO{\ensuremath \mathcal{O}}
\def\cP{\ensuremath \mathcal{P}}
%\def\cQ{\ensuremath \mathcal{Q}}
%\def\cR{\ensuremath \mathcal{R}}
%\def\cS{\ensuremath \mathcal{S}}
%\def\cT{\ensuremath \mathcal{T}}
%\def\cU{\ensuremath \mathcal{U}}
%\def\cV{\ensuremath \mathcal{V}}
%\def\cW{\ensuremath \mathcal{W}}
%\def\cX{\ensuremath \mathcal{X}}
%\def\cY{\ensuremath \mathcal{Y}}
%\def\cZ{\ensuremath \mathcal{Z}}

%\def\sA{\ensuremath \mathscr{A}}
%\def\sB{\ensuremath \mathscr{B}}
%\def\sC{\ensuremath \mathscr{C}}
%\def\sD{\ensuremath \mathscr{D}}
\def\sE{\ensuremath \mathscr{E}}
%\def\sF{\ensuremath \mathscr{F}}
%\def\sG{\ensuremath \mathscr{G}}
%\def\sH{\ensuremath \mathscr{H}}
%\def\sI{\ensuremath \mathscr{I}}
%\def\sJ{\ensuremath \mathscr{J}}
%\def\sK{\ensuremath \mathscr{K}}
%\def\sL{\ensuremath \mathscr{L}}
%\def\sM{\ensuremath \mathscr{M}}
%\def\sN{\ensuremath \mathscr{N}}
\def\sO{\ensuremath \mathscr{O}}
%\def\sP{\ensuremath \mathscr{P}}
%\def\sQ{\ensuremath \mathscr{Q}}
%\def\sR{\ensuremath \mathscr{R}}
%\def\sS{\ensuremath \mathscr{S}}
%\def\sT{\ensuremath \mathscr{T}}
%\def\sU{\ensuremath \mathscr{U}}
%\def\sV{\ensuremath \mathscr{V}}
%\def\sW{\ensuremath \mathscr{W}}
%\def\sX{\ensuremath \mathscr{X}}
%\def\sY{\ensuremath \mathscr{Y}}
%\def\sZ{\ensuremath \mathscr{Z}}

%\renewcommand{\headrulewidth}{0.2pt}

\newcommand{\myhr}{\rule{0.3\textwidth}{1pt}}

\let\epsilon\varepsilon
\let\phi\varphi

\DeclareMathOperator{\modulo}{mod}
\DeclareMathOperator{\conv}{conv}
\DeclareMathOperator{\Cof}{Cof}
\DeclareMathOperator{\grad}{grad}
\DeclareMathOperator{\Id}{Id}
\DeclareMathOperator{\dist}{dist}
\DeclareMathOperator{\diam}{diam}
\DeclareMathOperator{\co}{co}
\DeclareMathOperator{\supp}{supp}
\DeclareMathOperator{\graph}{graph}
\DeclareMathOperator{\slopes}{slopes}
\DeclareMathOperator{\Ob}{Ob}
\DeclareMathOperator{\Mor}{Mor}

% specific:
\newcommand{\Dm}{{\cD\mbox{-Modul}}}

\newcommand{\Cfx}{{\mathbb C\llbracket x\rrbracket}}
\newcommand{\Cft}{{\mathbb C\llbracket t\rrbracket}}
\newcommand{\Cfu}{{\mathbb C\llbracket u\rrbracket}}
\newcommand{\Cfxl}{{\mathbb C (\!(x)\!)}}
\newcommand{\Cftl}{{\mathbb C (\!(t)\!)}}
\newcommand{\Cful}{{\mathbb C (\!(u)\!)}}

\makeatother

\newcommand{\overbox}[2]{\ensuremath\begin{array}[b]{c}%
\makebox[0pt]{\fbox{\scriptsize#2}}\\[-2pt]\text{\small$\downarrow$}\\[-3pt]%
{\displaystyle#1}\end{array}}%

%%%%%%%%%%%%%%%%%%%%%%%%%%%%%%%%%%%%%%%%%%%%%%%%%%%%%%%%%%%%%%%%%%%%%
%\usepackage{xcolor}
\usepackage{color}
\usepackage{framed}

\newenvironment{fshaded}{%
\def\FrameCommand{\fcolorbox{framecolor}{shadecolor}}%
\MakeFramed {\FrameRestore}}%
{\endMakeFramed}

%\newenvironment{comment}{\definecolor{shadecolor}{rgb}{1,.6,.6}%
%\definecolor{framecolor}{rgb}{0,0,0}%
%\begin{fshaded}}{\end{fshaded}}

\renewcommand{\comment}{\definecolor{shadecolor}{rgb}{1,.6,.6}%
\definecolor{framecolor}{rgb}{0,0,0}%
\fshaded}
\renewcommand{\endcomment}{\endfshaded}

%Zeilenhöhe, für bessere lesbarkeit
\linespread{1.3}

%\usepackage{thmbox}

\cfoot{\today}
%%%%%%%%%%%%%%%%%%%%%%%%%%%%%%%%%%%%%%%%%%%%%%%%%%%%%%%%%%%%%%%%%%%%%

\begin{document}

%TODO: set titel variable??

%%%%%%%%%%%%%%%%%%%%%%%%%%%%%%%%%%%%%%%%%%%%%%%%%%%%%%%%%%%%%%%%%%%%%
\frontmatter

%TODO; titelpage bitte nicht seitlich versetzt
\begin{titlepage}
  \thispagestyle{empty}
  \newcommand{\Rule}{%
    \textcolor{black}{\rule{\textwidth}{0.5mm}}%
  }
  \begin{center}\sffamily
    \normalfont\sffamily\large
    Bachelorarbeit
    %\vfill
    \Rule
    \vspace{5mm}
    \Huge{Explizite Berechnung der Levelt-Turrittin-Zerlegung für spezielle
      $\cD$-Moduln}
    %{Explizite Berechnung der Levelt-Turrittin-Zerlegung einer Klasse
      %von Fourier-Transformationen}
    \vspace{1mm}
    \Rule
  \end{center}
    %\vfill
    \normalfont\sffamily\large vorgelegt von \Large Maximilian Huber\\
    \normalfont\sffamily\large am            \Large Institut für Mathematik\\
    \normalfont\sffamily\large der           \Large Universität Augsburg\\
    \normalfont\sffamily\large betreut durch \Large Prof. Dr. Marco Hien\\
    \normalfont\sffamily\large abgegeben am  \Large 04.07.2013\\
    \vfill
    \vfill
\end{titlepage}
% vim: set ft=tex :


%\begin{center}
  %\today
%\end{center}
\tableofcontents{}
%\printnomenclature

%\newpage

\chapter{Einleitung}

Die Theorie der $\cD$-Moduln ist aus einem Versuch, Systeme von partiellen
Differentialgleichungen algebraisch zu betrachten, entstanden. Seit den 1960er
Jahren wurde diese Theorie von beispielsweise B. Malgrange, J. Bernstein, P.
Deligne, M.Sato, und M. Kashiwara entwickelt und vorangetrieben.
Auch aktuell wird viel auf diesem Thema geforscht, so ist hier vor allem
C. Sabbah zu erwähnen, der auch zwei der Hauptquellen dieser Arbeit, namentlich
\cite{sabbah_cimpa90} und \cite{sabbah_Fourier-local}, verfasst hat.

Eines der wichtigsten Resultate für $\cD$-Moduln ist die
Riemann-Hilbert-Korrespondenz. Sie beschreibt mittels des de-Rahm-Funktors
\textbf{DR} die Kategorienäquivalenz
\begin{center}
\begin{tikzcd}[column sep=large]
  & \left\{ \overset{\text{\normalsize Regulär singuläre}}{
   \underset{\text{\normalsize holonome $\cD$-Moduln}}{\phantom{0}}}
   \right\}\arrow[yshift=0.7ex]{r}{\textbf{DR}}&
  \left\{ \text{Perverse Garben} \right\}
   \arrow[yshift=-0.7ex]{l}\\
  \left\{ \overset{\text{\normalsize Regulär singuläre}}{
   \underset{\text{\normalsize Zusammenhänge}}{\text{meromorphe}}}
   \right\} \arrow[leftrightarrow]{r}[description]{$1:1$}&
  \left\{ \overset{\text{\normalsize Regulär singuläre}}{
   \underset{\text{\normalsize holonome $\cD$-Moduln}}{\text{lokalisierte}}}
   \right\}\arrow[yshift=0.7ex]{r} \arrow[hook]{u}&
  \left\{ \text{Lokale Systeme} \right\} \,.
   \arrow[yshift=-0.7ex]{l} \arrow[hook]{u}
\end{tikzcd}
\end{center}
\iffalse
  \begin{center}
  \begin{tikzcd}[column sep=huge]
    \left\{ \overset{\text{\normalsize Regulär singuläre}}{
     \underset{\text{\normalsize holonome $\cD$-Moduln}}{\phantom{0}}}
     \right\}\arrow[yshift=0.7ex]{r}{\textbf{DR}}&
    \left\{ \text{Perverse Garben} \right\} \,,
     \right\}\arrow[yshift=-0.7ex]{l}
  \end{tikzcd}
  \end{center}
  wobei \textbf{DR} der de-Rahm-Funktor ist.
  Für unseren Fall, da wir uns auf meromorphe Zusammenhänge bzw. lokalisierte
  holonome $\cD$-Moduln beschränken wollen, ergibt sich
  \begin{center}
  \begin{tikzcd}[column sep=large]
    \left\{ \overset{\text{\normalsize Regulär singuläre}}{
     \underset{\text{\normalsize holonome $\cD$-Moduln}}{\text{lokalisierte}}}
     \right\} \arrow[leftrightarrow]{r}[description]{$1:1$}&
    \left\{ \overset{\text{\normalsize Regulär singuläre}}{
     \underset{\text{\normalsize Zusammenhänge}}{\text{meromorphe}}}
     \right\}\arrow[yshift=0.7ex]{r}&
    \left\{ \text{Lokale Systeme} \right\} \,.
     \right\}\arrow[yshift=-0.7ex]{l}
  \end{tikzcd}
  \end{center}
\fi
Vergleiche hierzu \cite[Sec 6]{REFKashiwara1984} oder \cite[Thm
7.2.1]{hotta2007d}.

Allerdings lässt sich dieses Resultat über regulär singuläre holonome
$\cD$-Moduln leider nicht kanonisch auf irregulär singuläre fortsetzen.
Jedoch gibt es zumindest für formale irregulär singuläre $\cD$-Moduln bzw. 
formale irregulär singuläre meromorphe Zusammenhänge eine allgemeine
Strukturaussage: das Levelt-Turrittin-Theorem, welches im Rahmen dieser Arbeit
genauer betrachtet werden soll.

Das Levelt-Turrittin-Theorem beschreibt, wie sich ein formaler meromorpher
Zusammenhang $\cM$ nach einem möglicherweise notwendigem Pullback in die
direkte Summe
\begin{center}
$ \cM \cong \bigoplus_{i=0}^n \sE^{\psi_i}\otimes R_i $
\end{center}
zerlegen lässt.
Die $\sE^{\psi_i}\otimes R_i$ stellen dabei elementare meromorphe Zusammenhänge
dar und $R_i$ ist jeweils einer der bereits erwähnten regulär singulären
meromorphen Zusammenhänge.

Das Ziel dieser Arbeit ist es, in die Theorie der $\cD$-Moduln (Kapitel
\ref{chap:dModuln}) bzw. der meromorphen Zusammenhänge (Kapitel
\ref{chap:meromZsh} und \ref{chap:operationen}) kurz einzuführen und danach
eine Levelt-Turrittin-Zerlegung explizit an einem Beispiel auszuführen und zu
berechnen.
\begin{comment}
Die Riemann-Hilbert-Korrespondenz und die Theorie der Stokes-Strukturen werden
in dieser Arbeit nicht weiter betrachtet.
\end{comment}

\begin{comment}
Es wird in dieser Arbeit kein Vorwissen über $\cD$-Moduln bzw. meromorphe
Zusammenhänge vorausgesetzt, diese beiden Begriffe werden in den ersten Zwei
Kapiteln eingeführt.
\end{comment}

Im ersten Kapitel erfolgt ein kurzer Überblick über die Theorie der
$\cD$-Moduln, dabei wird der Schwerpunkt auf diejenigen Themen gelegt, die für
das weitere Verständnis der Arbeit nötig sind. Eine ausführliche Einführung,
z.B. zur genauen Definition von Holonomie, kann in \cite{sabbah_cimpa90} oder
in \cite{hotta2007d} nachgeschlagen werden.\\
\iffalse
  Im ersten Kapitel wird sehr knapp die Theorie der $\cD$-Moduln eingeführt, da
  uns diese im später sehr nützlich sein wird. Hier wurde, vor allem bei der
  Definition von holonomen $\cD$-Moduln, viel Hintergrund weggelassen, da
  dieser zum Verständnis dieser Arbeit nicht nötig ist.\\
\fi
\begin{comment} \end{comment}
Im zweiten Kapitel werden die meromorphen Zusammenhänge definiert.
Diese sind das wichtigste mathematische Objekt in dieser Arbeit.
Ein wichtiges Resultat in diesem Kapitel ist die Eins-zu-Eins Korrespondenz
von meromorphen Zusammenhängen zu holonomen lokalisierten $\cD$-Moduln.\\
\begin{comment} \end{comment}
Das dritte Kapitel beschäftigt sich ausschließlich mit Operationen auf
meromorphen Zusammenhängen.
Vorgestellt werden Tensorprodukt, Pullback und Pushforward,
Fouriertransformation, die sogenannte \glqq{}Betrachtung bei Unendlich\grqq{} und das Twisten
eines meromorphen Zusammenhangs.
Diese Operationen werden in den folgenden zwei Kapiteln Verwendung finden.\\
\begin{comment} \end{comment}
Im vierten Kapitel soll das Levelt-Turrittin-Theorem vorgestellt werden.
Es erlaubt, einen meromorphen Zusammenhang in elementare meromorphe
Zusammenhänge zu zerlegen.\\
\begin{comment} \end{comment}
Das letzte Kapitel hat das Ziel, die Levelt-Turrittin-Zerlegung auf eine Klasse
von meromorphen Zusammenhängen anzuwenden.
Dazu wird zu Beginn ein Rezept gegeben, welches die gewünschten Zusammenhänge
beschreibt.
Zu diesen Zusammenhängen werden zunächst allgemeine Aussagen getroffen, bevor
in Abschnitt \ref{sec:LT-speziell} ganz konkret eine Levelt-Turrittin-Zerlegung
zu einem meromorphen Zusammenhang berechnet wird.
Zuletzt wird in Abschnitt \ref{sec:konvergenzDerPotReihen} das
Konvergenzverhalten der berechneten Zusammenhänge analysiert.

\begin{comment}
Ich möchte diese Stelle nutzen, um Herrn Prof. Dr. Hien dafür zu danken, dass
er mir ermöglicht hat, mich mit diesem Thema zu beschäftigen.
Auch bedanke ich mich für die hervorragende Betreuung, welche diese Arbeit erst
ermöglicht hat.
\end{comment}

% vim:set ft=tex foldmethod=marker foldmarker={{{,}}}:


\mainmatter
%%%%%%%%%%%%%%%%%%%%%%%%%%%%%%%%%%%%%%%%%%%%%%%%%%%%%%%%%%%%%%%%%%%%%
%\part{Theorie}

%!TeX root = main.tex
\chapter{Mathematische Grundlagen}

\begin{comment}
Hier werde ich mich auf \cite{sabbah_cimpa90} und \cite{coutinho1995primer}
beziehen.
\end{comment}

Wir betrachten $\C$ hier als Komplexe Mannigfaltigkeit mit der klassischen
Topologie.
In dieser Arbeit spielen die folgenden Funktionenräume eine große Rolle:
\begin{itemize}
\item $\C[x]:=\{ \sum^{N}_{i=1}a_ix^i | N\in \N \}$ die einfachen Potenzreihen
\item $\C\{x\}:=\{ \sum^{\infty}_{i=1}a_ix^i | \mbox{pos.
Konvergenzradius}\}$ (\cite[Chap 5.1.1]{hotta2007d})
\item $\C\llbracket x\rrbracket:=\{ \sum^{\infty}_{i=1}a_ix^i \}$ die formalen
Potenzreihen
\item $K:=\Ckxl:=\C\{x\}[x^{-1}]$ der Ring der Laurent Reihen.
\item $\hat{K}:=\Cfxl:=\C\llbracket x\rrbracket[x^{-1}]$ der Ring der
formalen Laurent Reihen.
\item $\tilde\cO$ als der Raum der Keime aller (möglicherweise mehrdeutigen)
Funktionen. (bei \cite{hotta2007d} mit $\tilde K$ bezeichnet)
\end{itemize}
Wobei offensichtlich die Inclulsionen $\C[x]\subsetneq\C\{x\}\subsetneq\Cfx$
und $K\subsetneq\hat K$ gelten.

Es bezeichnet der Hut ($ \, \hat \,\, $) das jeweils formale Äquivalent zu
einem konvergentem Objekt.

\iffalse
  \begin{comment}
  \begin{lem}[Seite 2]
  ein paar eigenschaften
  \begin{enumerate}
  \item $\C[x]$ ist ein graduierter Ring, durch die Grad der
  Polynome. Diese graduierung induziert eine aufsteigende Filtrierung.

  alle Ideale haben die form $(x-a)$ mit $a\in \C$
  \item wenn $\mathfrak{m}$ das maximale Ideal von $\C[x]$ (erzeugt von
  $x$ ist), so ist
  \[
    \C[[x]]=
    \underset{k}{\underleftarrow{\lim}} \C[X]\backslash\mathfrak{m}^k
  \]
  The ring $\C[[x]]$ ist ein nöterscher lokaler Ring:
  jede Potenzreihe mit konstantem term $\neq 0$ ist invertierbar.

  Der ring ist ebenfalls ein diskreter ??? Ring (discrete valuation
  ring)

  Die Filtrierung nach grad des Maximalen Ideals, genannt
  $\mathfrak{m}$-adische Fitration, ist die Filtrierung
  $\mathfrak{m}^k=\{f\in \C[[x]]|v(f)\geq k\}$

  und es gilt $gr_\mathfrak{m}(\C[[x]])=\C[x]$
  \end{enumerate}
  \end{lem}
  \end{comment}
\fi

Für $v=(v_1,\dots,v_n)$ ein Vektor, bezeichnet 
\[
\,^tv:= \begin{pmatrix}
  v_{1}\\
  \vdots\\
  v_{n}
\end{pmatrix}
\]
den transponierten Vektor. Es bezeichnet $M(n\times m,k)$ die Menge der $n$ mal
$m$ dimensionalen Matritzen mit Einträgen in $k$.

Sei $R$ ein Ring, dann bezeichnet $R^\times$ die Einheitengruppe von $R$.

\begin{defn}[Direkte Summe] \cite[4(Categories).5.1]{stacks-project}
Seien $x,y\in \Ob(\cC)$, eine \emph{direkte Summe}, oder das \emph{Coprodukt}
von $x$ und $y$ ist ein Objekt $x\oplus y\in \Ob(\cC)$ zusammen mit
Morphismen $i\in\Mor_\cC(x,x\oplus y)$ und $j\in\Mor_\cC(y,x\oplus y)$, so
dass die folgende universelle Eigenschaft gilt: für jedes $w\in Ob(\cC)$ mit
Morphismen $\alpha\in\Mor_\cC(x,w)$ und $\beta\in\Mor_\cC(y,w)$ existiert ein
eindeutiges $\gamma\in\Mor_\cC(x\oplus y,w)$, so dass das Diagramm
\begin{center}
\begin{tikzpicture} [scale=3.3, descr/.style={fill=white,inner sep=2.5pt} ]
\matrix (m) [
  matrix of math nodes
  , row sep=2em
  , column sep=3em
  %, text height=3em
  %, text depth=0.25em
]{
    & y         &  &   \\
  x & x\oplus y &  &   \\
    &           &  & w \\
};
%TODO: Pfeile
\path[->,font=\scriptsize,>=angle 90]
(m-1-2) edge node[left]{$j$} (m-2-2)
(m-2-1) edge node[above]{$i$} (m-2-2)
(m-1-2) edge node[right]{$\beta$} (m-3-4)
(m-2-1) edge node[below]{$\alpha$} (m-3-4)
;
\path[->,font=\scriptsize,>=angle 90,dashed]
(m-2-2) edge node[above]{$\exists!\gamma$} (m-3-4)
;
\end{tikzpicture}
\end{center}
kommutiert.
\end{defn}

\begin{defn}[Tensorprodukt]
\cite[3(Algebra).11.21]{stacks-project}
% von Vorlesung Algebra 2
\begin{center}
\begin{tikzpicture} [scale=3.3, descr/.style={fill=white,inner sep=2.5pt} ]
  \matrix (m) [
    matrix of math nodes
    , row sep=2em
    , column sep=3em
    %, text height=3em
    %, text depth=0.25em
  ]{
    M\times N & M\otimes_RN \\
              & T \\
  };
  %TODO: Pfeile
  \path[->,font=\scriptsize,>=angle 90]
  (m-1-1) edge node[above]{$  $} (m-1-2)
  (m-1-1) edge node[below]{$f$} (m-2-2)
  ;
  \path[->,font=\scriptsize,>=angle 90,dashed]
  (m-1-2) edge node[right]{$\exists!\gamma$} (m-2-2)
  ;
\end{tikzpicture}
\end{center}
Für eine Abbildung $f:M\rightarrow M'$ definiere das Tensorprodukt davon über
$R$ mit $N$ als
\[
\id_N \otimes f:
\begin{array}[t]{ccc}
N\otimes_{R}M & \rightarrow & N\otimes_{R}M'\\
n\otimes m & \mapsto & n\otimes f(m)
\end{array}
\]
\end{defn}
\begin{bem} \label{bem:Rechenregeln-Tensorprodukt}
Hier einige Rechenregeln für das Tensorprodukt,
\begin{align}
(M\otimes_R N)\otimes_S L &\cong M\otimes_R (N \otimes_S L)
  \label{bem:Rechenregeln-Tensorprodukt1}\\
M\otimes_R R &\cong M \label{bem:Rechenregeln-Tensorprodukt2}
\end{align}
Sei $f:M'\rightarrow M$ eine Abbildung, so gilt
\begin{align}
N\otimes_R(M/\im(f)) &\cong N\otimes_R M / \im(\id_{R}\otimes f)
\label{bem:Rechenregeln-Tensorprodukt3}
\end{align}
\end{bem}
%TODO: tensorprodukt / faserprodukt zum basiswechsel

\begin{defn}[Exakte Sequenz]
Eine Sequenz
\begin{center}
\begin{tikzpicture} [scale=3.3, descr/.style={fill=white,inner sep=2.5pt} ]
  \matrix (m) [
    matrix of math nodes
    , row sep=2em
    , column sep=3em
    %, text height=3em
    %, text depth=0.25em
  ]{
    %M_0 
    & \cdots & M_{i-1} & M_i & M_{i+1} & \cdots & 
    %M_n 
  \\
  };
  %TODO: Pfeile
  \path[->,font=\scriptsize,>=angle 90]
  %(m-1-1) edge node[above]{$f_0$} (m-1-2)
  (m-1-2) edge 
    %node[above]{$f_{i-2}$} 
  (m-1-3)
  (m-1-3) edge node[above]{$f_{i-1}$} (m-1-4)
  (m-1-4) edge node[above]{$f_i$} (m-1-5)
  (m-1-5) edge 
    %node[above]{$f_{i+1}$} 
  (m-1-6)
  %(m-1-6) edge node[above]{$f_{n-1}$} (m-1-7)
  ;
\end{tikzpicture}
\end{center}
heißt exakt, wenn für alle $i$ gilt, dass
$\im(f_{i-1})=\ker{f_i}$.
\end{defn}
\begin{defn}[Kurze exakte Sequenz]
Eine kurze exakte Sequenz ist eine Sequenz
\begin{center}
\begin{tikzpicture} [scale=3.3, descr/.style={fill=white,inner sep=2.5pt} ]
  \matrix (m) [
    matrix of math nodes
    , row sep=2em
    , column sep=3em
    %, text height=3em
    %, text depth=0.25em
  ]{
    0 & M' & M & M'' & 0 \,,\\
  };
  %TODO: Pfeile
  \path[->,font=\scriptsize,>=angle 90]
  (m-1-1) edge node[above]{$  $} (m-1-2)
  (m-1-2) edge node[above]{$f$} (m-1-3)
  (m-1-3) edge node[above]{$g$} (m-1-4)
  (m-1-4) edge node[above]{$  $} (m-1-5)
  ;
\end{tikzpicture}
\end{center}
welche exakt ist.
\end{defn}

\begin{defn}[Kokern]
Ist $f:M'\rightarrow M$ eine Abbildung, so ist der \emph{Kokern} von $f$
definiert als $\coker(f)=M/\im(f)$.
\end{defn}

\begin{prop}
Ist $f:M'\rightarrow M$ eine injektive Abbildung, so ist
\begin{center}
\begin{tikzpicture} [scale=3.3, descr/.style={fill=white,inner sep=2.5pt} ]
  \matrix (m) [
    matrix of math nodes
    %, row sep=2em
    , column sep=3em
    %, text height=3em
    %, text depth=0.25em
  ]{
    0 & M' & M & M/f(M')      & 0 \\
      &    & m & m \mod f(M') &   \\
  };
  %TODO: Pfeile
  \path[->,font=\scriptsize,>=angle 90]
  (m-1-1) edge (m-1-2)
  (m-1-2) edge node[above]{$f$} (m-1-3)
  (m-1-3) edge node[above]{$\pi$} (m-1-4)
  (m-1-4) edge (m-1-5)
  ;
  \path[|->,font=\scriptsize,>=angle 90]
  (m-2-3) edge (m-2-4)
  ;
\end{tikzpicture}
\end{center}
eine kurze exakte Sequenz und $M/f(M')=\coker(f)$ ist der \emph{Kokern} von
$f$.
\end{prop}
\begin{proof}
    
\end{proof}

\begin{comment}
\begin{defn}[Filtrierung] \label{defn:Filtrierung}
\cite[Def 10.13.1.]{stacks-project}
\cite[Rem 2.5.]{elliottDmod}
Eine \emph{aufsteigende Filtrierung $F$} von einem Objekt (Ring) $A$ ist eine
Familie von $(F_iA)_{i\in\Z}$ von Unterobjekten (Unterring), so dass 
\[ 0\subset\cdots\subset F_i\subset F_{i+1} \subset \cdots \subset A \]
und definiere weiter $gr_i^FA:=F_iA\slash F_{k-1}A$ und damit das zu $A$ mit
Filtrierung $F$ \emph{assoziierte graduierte Modul} 
\[gr^FA:=\bigoplus_{k\in\Z}gr_i^FA \,. \]
\end{defn}

\begin{defn}
\cite{ayoubIntro}
\cite[Def 3.2.1]{sabbah_cimpa90}
Eine Filtrierung heißt \emph{gut}, falls ...
\end{defn}
\end{comment}

% vim: set ft=tex :


%!TeX root = main.tex
% das gleiche wie Meromorphe zusammenhänge?
\chapter{Moduln über $\cD_k$} \label{chap:dModuln}
In diesem Kapitel wird die Weyl Algebra, wie in
\cite[Chapter~1]{sabbah_cimpa90} oder in \cite[Kapittel~2]{ZulaBarbara} , in
einer Veränderlichen einführen.  Allgemeiner und in mehreren Veränderlichen
wird die Weyl-Algebra beispielsweise in \cite[Chapter~1]{coutinho1995primer}
definiert.

\begin{defn}[Kommutator]%zula seite 15
Sei $R$ ein Ring. Für $a,b\in R$ wird
\[[a,b]:=a\cdot b-b\cdot a\]
als der \emph{Kommutator von a und b} definiert.
\end{defn}
% komutativ, dann immer kommutator gleich 0

\begin{prop} \label{prop:d-modul-komutator-regeln}
Sei $k= \C[x]$ (bzw. $\Ckx$ bzw. $\Cfx$) ein Ring der Polynome (bzw. der
konvergenten Potenzreihnen bzw. der formalen Potenzreihen) in $x$ über $\C$.
Sei $\partial_x:k\rightarrow k$ der gewohnte Ableitungsoperator nach $x$, so
gilt
% geklaut aus Zula Barbara
\begin{enumerate}
\item $[ \partial_x,x] = \partial_xx-x\partial_x=1 $
%und damit ist $\cD_k$ insbesondere nicht kommutativ.
\item für $f\in k$ ist
\begin{equation} \label{eq:kommutator1}
[\partial_x,f] = \frac{\partial f}{\partial x} \,. 
\end{equation}
\item Es gelten die Formeln
\begin{align}
[\partial_x,x^k] &\!\!\overset{(\ref{eq:kommutator1})}{=}
  \frac{\partial x^k}{\partial x} = kx^{k-1}
  \label{eq:kommutator1b}\\
[\partial_x^j,x]   &= j\partial_x^{j-1}
  \label{eq:kommutator2}\\
[\partial_x^j,x^k] &= \sum_{i\geq1}\frac{k(k-1)\cdots(k-i+1)
  \cdot j(j-1)\cdots(j-i+1)}{i!}x^{k-i}\partial_x^{j-i}
  \label{eq:kommutator3}
\end{align}
\end{enumerate}
\end{prop}
\begin{proof}
Die erste Aussage ist klar. Für die zweite Aussage gilt für ein Testobjekt
$g\in k$:
\begin{align*}
[\partial_x,f]\cdot g&=\myubracket{\partial_x(fg)}-f\partial_xg
\\&= \myobracket{(\partial_xf)g
  +\rlap{$\underset{=0}{\underbrace{ \phantom{f(\partial_xg)-f(\partial_xg)}}}$}
  f(\partial_xg)}-f(\partial_xg)
\\&= (\partial_xf)g \,.
\end{align*}
Der Rest der Proposition wird beispielsweise in \cite[1.2.4.]{sabbah_cimpa90}
oder \cite[Kor 2.8]{ZulaBarbara} bewiesen.
\end{proof}

\section{Weyl-Algebra und der Ring $\cD_k$}
%TODO: sabah-cimpa90.pdf  seite 3
%script ginzburg.pdf seite 34 als garbe definiert
Sei dazu $\frac{\partial}{\partial x}=\partial_x$ der Ableitungsoperator nach
$x$ und sei $f \in \C[x]$ (bzw. $\Ckx$ bzw. $\Cfx$).
Man hat die folgende
Kommutationsrelation zwischen dem \emph{Ableitungsoperator} und dem
\emph{Multiplikationsoperator} $f$:
\begin{equation}\label{eq:weyl_relation}
\left[\frac{\partial}{\partial x},f\right]=\frac{\partial f}{\partial x} \,,
\end{equation}
wobei die rechte Seite die Multiplikation mit $\frac{\partial f}{\partial x}$,
also dem bereits abgeleiteten $f$, darstellt. Dies bedeutet, für alle
$g\in\C[x]$ hat man $[\frac{\partial}{\partial x},f]\cdot g
=\frac{\partial f}{\partial x} \cdot g$, denn
\begin{align*}
[\frac{\partial}{\partial x},f]\cdot g
  &=\myubracket{\frac{\partial (fg)}{\partial x}}
    - f\cdot \frac{\partial g}{\partial x}
\\&= \myobracket{\frac{\partial f}{\partial x}\cdot g
  + \rlap{$\underbrace{\phantom{
  f\cdot \frac{\partial g}{\partial x}
  - f\cdot \frac{\partial g}{\partial x}
  }}_{=0}$}
  f\cdot \frac{\partial g}{\partial x}}
  - f\cdot \frac{\partial g}{\partial x}
=\frac{\partial f}{\partial x} \cdot g \,.
\end{align*}
\begin{defn}
Definiere nun den Ring $\cD_k$ als die Quotientenalgebra der freien Algebra,
welche von dem Koeffizientenring in $k$ zusammen mit dem Element $\partial_x$
erzeugt wird, modulo der Relation \eqref{eq:weyl_relation}.  Wir schreiben
diesen Ring auch als:
\begin{itemize}
\item $A_1(\C)$, falls $k=\C[x]:=\{ \sum^{N}_{i=0}a_ix^i \mid N\in \N \}$, und
nennen ihn die \emph{Weyl Algebra}.
\item $\cD$, falls $k=\C\{x\}:=\{ \sum^{\infty}_{i=0}a_ix^i \mid \mbox{pos.
Konvergenzradius}\}$ die konvergenten Potenzreihen.
\item $\hat\cD$, falls $k=\Cfx:=\{\sum^{\infty}_{i=0}a_ix^i\}$ die formalen
Potenzreihen.
\item $\cD_K$, falls $k=K:=\Ckxl:=\C\{x\}[x^{-1}]$ der Ring der Laurent Reihen.
\item $\cD_{\hat K}$, falls $k=\hat K:=\Cfxl:=\Cfx[x^{-1}]$ der Ring der
formalen Laurent Reihen\footnote{Wird in \cite{ZulaBarbara} mit $\hat\cD_{\hat
K}$ bezeichnet.}.
\end{itemize}
\end{defn}
\begin{bem}
\begin{enumerate}
\item Es bezeichnet der Hut ($ \, \hat \,\, $) das jeweils formale Pendant
zu einem konvergenten Objekt. Dementsprechend könnte man auch $\hat{\Ckx}$ für
$\Cfx$ schreiben.
\item Es gilt $\cD[x^{-1}]=\cD_K$ und $\hat \cD[x^{-1}]=\cD_{\hat K}$.
\item Offensichtlich erhält $\cD_k$ in kanonischer Weise eine nichtkommutative
Ringstruktur, dies ist in \cite[Kapittel 2 Section 1]{ZulaBarbara} genauer
ausgeführt.
\end{enumerate}
\end{bem}
\begin{bem} \label{bem:konvToFormal}
Jede Aussage bzw. Definition, die in dieser Arbeit über $K$ bzw. $\cD_K$
getroffen wird, gilt auch über $\hat K$ bzw. $\cD_{\hat K}$. Die andere
Richtung gilt im Allgemeinen nicht.
\end{bem}

\begin{prop} \label{prop:weyl_eindeutige_schreibung}
Jedes Element in $\cD_K$ kann auf eindeutige
Weise als $P=\sum_{i=0}^na_i(x)\partial_x^i$, mit $a_i(x)\in k$, geschrieben
werden.
\end{prop}
\begin{proof}
Siehe \cite[Proposition 1.2.3]{sabbah_cimpa90}.
\end{proof}
\begin{comment}
Gilt das folgende??
\[
\alpha_i(x)\partial_x^i \equiv \frac{\alpha_i}{x^i}(x\partial_x)^i \mod
F_{i-1}\cD
\]
\end{comment}

\begin{comment}
Besser?:\\
erst Filtrierung definieren und dadurch dann den Grad?
\end{comment}
\begin{defn}
Sei $P=\sum_{i=0}^na_i(x)\partial_x^i$, wie in Proposition
\ref{prop:weyl_eindeutige_schreibung} gegeben, so definiere
\[
\deg P:=\max\Big\{\{i\mid a_i\neq 0\}\cup\{-\infty\}\Big\}
\]
als den \emph{Grad (oder den $\partial_x$-Grad)}
von $P$.
\end{defn}
\begin{comment}
In natürlicher Weise erhält man die aufsteigende Filtrierung
$F_N\cD:=\{P\in\cD|\deg P\leq N\}$ mit
\[
\cdots\subset F_{-1}\cD\subset F_{0}\cD\subset
F_{1}\cD\subset\cdots\subset\cD
\]
und erhalte $gr_k^F\cD\bydef F_N\cD\slash F_{N-1}\cD
=\{P\in\cD|\deg P=N\}\cong\C\{x\}$.

\begin{proof}[Beweisidee]
Sei $P\in F_N\cD$, so betrachte den Isomorphismus $F_N\cD\slash
F_{N-1}\cD\rightarrow \C\{x\}$ definiert durch $[P]=P+F_{N-1}\cD\mapsto
a_n(x)$.
\end{proof}

\begin{prop}
Es gilt:
\begin{center}
\begin{tikzpicture} [descr/.style={fill=white,inner sep=2.5pt}]
\matrix (m) [
  matrix of math nodes,
  row sep=1em,
  %column sep=-0.7em,
  text height=1.5ex,
  text depth=0.25ex]
{
  gr^F\cD &
  := \bigoplus_{N\in\Z}gr_N^F\cD = \bigoplus_{N\in\N_0}gr_N^F\cD \cong
  \bigoplus_{N\in\N_0}\C\{x\} \cong \C\{x\}[\xi] = &
  \bigoplus_{N\in\N_0}\C\{x\}\cdot \xi^N \\
};
\path[solid]
(m-1-1) edge [bend right=15] node[descr]{$\cong$}
  node[below]{$\mbox{isomorph als grad. Ringe}$} (m-1-3);
\end{tikzpicture}
\end{center}
also $gr^F\cD \cong \bigoplus_{N\in\N_0}\C\{x\}\cdot \xi^N$ als gradierte
Ringe.
\end{prop}
\end{comment}

\begin{comment}
\subsection{Alternative Definition / Sichtweise}
\cite[Chap 1.1.]{kashiwara2003d}
Sei $X$ eine $1$-dimensionale komplexe Mannigfaltigkeit und $\cO_X$ die Garbe
der holomorphen Funktionen auf $X$. Ein \emph{(holomorpher)
Differenzialoperator} auf $X$ ist ein Garben-Morphismus $P:\cO_X\rightarrow
\cO_X$, lokal in der Koordinate $x$ und mit holomorphen Funktionen $a_n(x)$ als
\[
(Pu)(x)=\sum_{n\geq0}a_n(x)\partial_x^nu(x)
\]
geschrieben (für $u\in\cO_X$). Zusätzlich nehmen wir an, dass $a_n(x)\equiv 0$
für fast alle $n\in \N$ gilt. Wir setzten
$\partial_x^nu(x)=\frac{\partial^nu}{\partial x^n}(x)$. Wir sagen, ein Operator
hat höchstens Ordnung $m$, falls $\forall n\geq m: \alpha_n(x)\equiv0$.
\begin{defn}
Mit $\cD_X$ bezeichnen wir die \emph{Garbe von Differentialoperatoren} auf $X$.
\end{defn}
Die Garbe $\cD_X$ hat eine Ringstruktur mittels der Komposition als
Multiplikation und $\cO_X$ ist ein Unterring von $\cD_X$. Sei $\Theta_X$ die
Garbe der Vektorfelder über $X$. Es gilt, dass $\Theta_X$ in $\cD_X$
enthalten ist.  Bemerke auch, dass $\Theta_X$ ein Links-$\cO_X$-Untermodul,
aber kein rechts $\cO_X$-Untermodul ist.

\begin{prop}
\cite[Exmp 1.1]{ArkhipovDmod}
Sei $X=\A^1=\C$, $\cO_X=\C[t]$ und $\Theta_X=\C[x]\partial_x$, wobei
$\partial_x$ als $\partial_x(x^n)=nx^{n-1}$ wirkt, dann sind die
Differentialoperatoren
\begin{align*}
\cD_X &= \C[x,\partial_x], & \mbox{mit} & & \partial_x x-x\partial_x & =1.
\end{align*}
Somit stimmt die alternative Definition bereits mit der einfachen überein.
\end{prop}
\end{comment}

\begin{comment}
\begin{defn} \cite[Defn 2.1]{ArkhipovDmod}
Sei $X=\A^1$, $\cO_X=\C[x]$ und $\cD_X=[x,\partial_x]$ mit der Relation
$[\partial_x,x]=1$. Dann definieren wir die Links-$\cD$-Moduln über $\A^1$ als
die $\C[x,\partial_x]$-Moduln. Sie werden geschrieben als $\cD-mod(\A^1)$
\end{defn}
\end{comment}

%\section{(Links-) $\cD$-Moduln}
Da $\cD$ ein nichtkommutativer Ring ist, muss man vorsichtig sein und zwischen
Links- und Rechts-$\cD$-Moduln unterscheiden.
Wenn im Folgendem von $\cD$-Moduln gesprochen wird, sind damit immer
Links-$\cD$-Moduln gemeint.

\begin{exmp}
Hier einige Beispiele für (Links-) $\cD$-Moduln
\begin{enumerate}
%
\item $\cD$ ist ein Links- und Rechts-$\cD$-Modul.
%
\item $\cM=\C[x]$ oder $\cM=\C[x,x^{-1}]$ mit jeweils $x\cdot x^{m}=x^{m+1}$
und $\partial_x(x^m)=mx^{m-1}$ ist ein Links-$\cD$-Modul.
%
\item Führt man formal, also ohne analytischen Hintergrund, ein Objekt
$\exp(\lambda x)$ ein, mit $\partial(f(x)\exp(\lambda x))=\frac{\partial
f}{\partial x}\exp(\lambda x)+f\lambda\exp(\lambda x)$, so ist
$\cM=\C[x]\exp(\lambda x)$ ein $\cD$-Modul.
%
\begin{comment}
\cite[Exmp 2.2]{ArkhipovDmod}
\end{comment}
\item Führt man analog ein Symbol $\log(x)$ mit den Eigenschaften
$\partial_x\log(x)=\frac{1}{x}$ ein, so erhält man nun den $\cD$-Moduln
$\C[x]\log(x)+\C[x,x^{-1}]$. Dieses Modul ist über $\cD$ durch $\log(x)$
erzeugt und es gilt
\[
\C[x]\log(x)+\C[x,x^{-1}]=\cD\cdot\log(x)=\cD\slash\cD(\partial_x x\partial_x) \,.
\]
\end{enumerate}
\end{exmp}

\begin{comment}
\begin{lem}\cite[Lem 2.3.3.]{sabbah_cimpa90}
Sei $\cM$ ein Links-$\cD$-Modul von endlichem Typ, welches auch von endlichem
Typ über $\Ckx$ ist. Dann ist $\cM$ bereits ein freies $\C\{x\}$-Modul.
\end{lem}
\begin{proof}
Siehe \cite[Lem 2.3.3.]{sabbah_cimpa90}.
\end{proof}
\begin{cor} \cite[Cor 2.3.4.]{sabbah_cimpa90}
Falls $\cM$ ein Links-$\cD$-Modul von endlichem Typ, welches außerdem ein
endlich dimensionaler Vektorraum ist, so ist schon $\cM=\{0\}$.
\end{cor}
\end{comment}

\section{Holonome $\cD_K$-Moduln}
In diesem Abschitt werden die Holonomen $\cD_K$-Moduln nur sehr einfach und
ohne großen Hintergrund eingeführt, da wir diese Theorie nicht benötigen. Eine
detailreichere Definition ist bei \cite[Def. 3.3.1]{sabbah_cimpa90} zu finden.
\begin{defn}
Sei $\cM_K$ ein Links-$\cD_K$-Modul ungleich $\cD_K$. $\cM_K$ heißt
\emph{holonom}, falls es ein Torsionselement $m\in\cM_K$ gibt, das $\cM_K$ als
$\cD_K$-Moduln erzeugt. Im Speziellen folgt damit, dass $\cM_K\cong
\cD_K/\mathfrak{a}$ für ein $0\neq\mathfrak{a}\vartriangleleft\cD_K$.
\end{defn}
\begin{bem}
\begin{comment}
Dies hier ist eine sehr vereinfachte, aber für unsere Zwecke völlig
ausreichende, Definition von holonom.
\end{comment}
In \cite{coutinho1995primer} wird der Begriff holonom über die Dimension
definiert und bei \cite{sabbah_cimpa90} über die charakteristische Varietät.
Letzteres ist die übliche Definition, da sich diese gut verallgemeinern lässt.
\end{bem}

\begin{bem} 
Nach \cite[Prop 10.1.1]{coutinho1995primer} gilt
\begin{itemize}
\item Submoduln und Quotienten von holonomen $\cD_K$-Moduln sind holonom
\item sowie endliche Summen von holonomen $\cD_K$-Moduln sind holonom
\end{itemize}
und laut \cite[Thm. 4.2.3]{sabbah_cimpa90} gilt, dass
\begin{itemize}
\item für einen holonomen $\cD_{\C\{x\}}$-Modul $\cM_{\C\{x\}}$
(bzw. einen $\cD_{\Cfx}$-Moduln $\cM_{\Cfx}$)
ist die Lokalisierung
\begin{align*}
\cM_{\C\{x\}}[x^{-1}]&:=\cM_{\C\{x\}}\otimes_{\C\{x\}}K
&&\text{(bzw. }\cM_{\Cfx}[x^{-1}]:=\cM_{\Cfx}\otimes_{\Cfx}\hat K\text{ ),}
\end{align*}
mit der $\cD_{\C\{x\}}-$ (bzw. $\cD_{\Cfx}$-) Modul-Struktur durch
\[
\partial_x(m\otimes x^{-k})=((\partial_xm)\otimes x^{-k})-km\otimes x^{-k-1}
\]
wieder holonom.
\end{itemize}
\end{bem}

Nach \cite[Cor 4.2.8]{sabbah_cimpa90} gilt der folgende wichtige Satz.
\begin{thm} \label{thm:lokalHoloZuQuot}
Sei $\cM_K$ ein holonomes $\cD_K$-Modul, dann gilt, dass seine Lokalisierung
isomorph zu $\cD_K/\cD_K\cdot P$, mit einem $P\in \cD_K$ ungleich Null, ist.
\end{thm}

\begin{comment}
\subsubsection{Alternative Definition B} %nach Sabbah
\begin{defn} \cite[Def 3.3.1.]{sabbah_cimpa90}
Sei $\cM$ lineares Differentialsystem
(linear differential system) %TODO: defn
.  Man sagt, $\cM$ ist holonom, falls $\cM=0$ oder falls $\Car\cM\subset
\{x=0\}\cup{\xi=0}$.
\end{defn}
\begin{lem} \cite[Lem 3.3.8.]{sabbah_cimpa90}
Ein $\cD$-Modul ist holonom genau dann, wenn $\dim_{gr^F\cD,0}gr^F\cM=1$.
\end{lem}
\begin{proof}
Siehe \cite[Lem 3.3.8.]{sabbah_cimpa90}
\end{proof}
\end{comment}

\begin{comment}
\subsubsection{Alternative Definition A} %nach Countinho
\begin{defn}[Holonome $\cD$-Moduln]
\cite[Chap 10 §1]{coutinho1995primer}
Ein endlich generierter $\cD$-Modul $\cM$ ist \emph{holonom}, falls $\cM=0$
gilt, oder falls es die Dimension $1$ hat.
\end{defn}
\begin{bem}
\cite[Chap 10 §1]{coutinho1995primer}
Sei $\mathfrak{a}\neq 0$ ein Links-Ideal von $\cD$. Es gilt nach
\cite[Corollary 9.3.5]{coutinho1995primer}, dass $d(\cD/\mathfrak{a})\leq 1$.
Falls $\mathfrak{a}\neq\cD$, dann gilt nach der \emph{Bernstein's inequality}
\cite[Chap 9 §4]{coutinho1995primer}, dass $d(\cD/\mathfrak{a})=1$. Somit ist
$\cD/\mathfrak{a}$ ein holonomes $\cD$-Modul.
\end{bem}
\end{comment}

%vim: set ft=tex :


%!TeX root = main.tex
% TODO; wie darf ich einen meromorphen verändern, (das P verändern) ohne das
% sich effektiv was ändert?
% TODO: \cM = \cM_K ... replace
% TODO: Dimension eines meromorphen Zusammenhang
\chapter{Meromorphe Zusammenhänge}
\begin{comment}
Sei $\cM$ ein $\cD$-Modul ungleich Null von endlichem Typ. Falls die
links-Multiplikation mit $x$ bijektiv ist, so nennen wir $\cM$ einen
meromorphen Zusammenhang. \cite[Chap 4]{sabbah_cimpa90}
%TODO: erklären, erwähnen oder entfernen!
\end{comment}

\begin{comment}
\cite[Chap 5.1.1]{hotta2007d} %abgeschrieben
\end{comment}
Wir beginnen mit der klassischen Theorie von Gewöhnlichen
Differentialgleichungen.
Sei $V$ ein geeigneter Funktionenraum, beispielsweise der Raum der holomorphen
Funktionen.
%TODO: posibly multivalued
\begin{defn}[Systeme von ODEs]
Für eine Matrix $A=(a_{ij}(x))_{ij}\in M(n\times n,K)$
\footnote{
Es bezeichnet $M(n\times m,k)$ die Menge der $n$ mal
$m$ dimensionalen Matrizen mit Einträgen in $k$.
}
definieren wir das
\emph{System von gewöhnlichen Differentialgleichungen (kurz System von ODEs)}
als
\begin{equation}
\label{eq:KlassischesODE}
%TODO: partial oder d
\frac{d}{dx}u(x)=Au(x) \,,
\end{equation}
wobei $u(x)=\,^t(u_1(x),\dots,u_n(x))$ ein
Spaltenvektor\footnote{Für $v=(v_1,\dots,v_n)$ ein Vektor, bezeichnet
$ \,^tv:= \begin{pmatrix} v_{1}\\ \vdots\\ v_{n} \end{pmatrix} $
den transponierten Vektor.} von unbekannten Funktionen.
Wir sagen $v(x)=\,^t(v_1(x),\dots,v_n(x))$ ist eine \emph{Lösung} von
(\ref{eq:KlassischesODE}), falls $ v_i\in V$ für alle
$i\in\{1,\dots,n\}$ und $v$ die Gleichung (\ref{eq:KlassischesODE}), auf einer
Umgebung um die $0$, erfüllt.
\end{defn}

Durch setzen von $w(x)=Tu(x)$ für eine invertierbare Matrix $T\in GL(n,K)$
erhält man aus \ref{eq:KlassischesODE} das System 
\[
\frac{d}{dx}w(x)=(T^{-1}AT - T^{-1}\frac{d}{dx}T)w(x) \,,
\]
und deshalb erhalten wir die folgende Definition.
\begin{defn}[Differenziell Äquivalent]
Man nennt $A$ und $B\in M(n\times n,K)$ \emph{(differenziell) äquivalent}
($A\sim B$) genau dann, wenn es ein $T\in GL(n,K)$ gibt, mit
$B=T^{-1}AT-T^{-1}\frac{d}{dx}T$.
Dementsprechend sind zwei Systeme von ODEs äquivalent, wenn ihre zugeordneten
Matrizen differenziell äquivalent sind.
\end{defn}

\begin{comment}
$1=TT^{-1}$ $\rightsquigarrow$ $T'T^{-1}+T(T^{-1})'=0$\\
$1=T^{-1}T$ $\rightsquigarrow$ $(T^{-1})'T+T^{-1}T'=0$
\end{comment}

Mit elementarer Theorie über Gewöhnliche Differentialgleichungen lässt sich
ein System von ODEs in eine einzige ODE der Form
\begin{equation} \label{eq:UmgeformteODE}
(\underset{=:P}{\underbrace{ a_n\partial_x^n + a_{n-1}\partial_x^{n-1} +
  \cdots + a_1\partial_x + a_0}})\cdot u(t)=0 \,,
\end{equation}
mit $a_i\in K$ für alle $i\in\{0,\dots,n\}$, umschreiben. Dieses so erhaltene
$P\in\cD_K$ ist ein sogenannter linearer Differentialoperator und es gilt
\begin{align*}
v(x) \text{ ist Lösung von } P\cdot u(x)=0
&&\Rightarrow && \forall Q\in \cD_K \text{ ist } v(x) \text{ Lösung von }
QP\cdot u(x)=0 \,.
\end{align*}
Also ist eine Lösung von $P\cdot u(x)=0$ auch eine Lösung von $Q\cdot u(x)=0$
für alle $Q\in\cD_K\cdot P\vartriangleleft \cD_K$.

\begin{comment}
Sei $P$ ein linearer Differentialoperator mit Koeffizienten in $a_i(x)\in\Ckx$
geschrieben als $P=\sum^{d}_{i=0}{a_{i}(x)\partial_x^i}$.
Man sagt eine Funktion $u\in\cF$ ist Lösung von $P$, falls $u$ die Gleichung
$Pu=0$ erfüllt.
Man sagt $0$ ist ein singulärer Punkt falls $a_d(0)=0$.
Falls $0$ kein singulärer Punkt ist, hat $P$ genau $d$ über $\C$ Unabhängige
Lösungen in $\Ckx$. %TODO: oder \tilde\cO
\end{comment}

\section{Meromorphe Zusammenhänge}
Nun wollen wir dieses klassische Gebilde nun in die moderne Sprache der
meromorphen Zusammenhänge übersetzen.
%Quelle ist \cite{sabbah_cimpa90}
\begin{defn}[Meromorpher Zusammenhang] \label{def:merom-zush}
Ein \emph{meromorpher Zusammenhang} (bei $x=0$) ist ein Tupel
$(\cM_K,\partial)$ und besteht aus folgenden Daten:
\begin{itemize}
\item $\cM_K$, ein endlich dimensionaler $K$-Vektor Raum
\item einer $\C$-linearen Abbildung $\partial:\cM_K\rightarrow \cM_K$,
genannt \emph{Derivation} oder \emph{Zusammenhang}, welche für alle $f\in K$
und $u\in \cM_K$ die \emph{Leibnitzregel}
\begin{equation}\label{eq:Leibnitzregel}
\partial(fu)=f'u+f\partial u %TODO: hier klammern um das letzte u ?
\end{equation}
erfüllen soll.
\end{itemize}
\end{defn}
\begin{bem} %[Formaler meromorpher Zusammenhang]
Analog definiert man einen \emph{formalen meromorphen Zusammenhang}
$(\cM_{\hat K},\partial)$ bestehend, analog wie in Definition
\ref{def:merom-zush}, aus folgenden Daten:
\begin{itemize}
\item $\cM_{\hat K}$, ein endlich dimensionaler $\hat K$-Vektor Raum
\item einer $\C$-linearen Derivation $\partial:\cM_{\hat K}\rightarrow
\cM_{\hat K}$, welche die \emph{Leibnitzregel} (\ref{eq:Leibnitzregel})
erfüllen soll.
\end{itemize}
\end{bem}

\begin{bem}
Später wird man auf die Angabe von $\partial$ verzichten und einfach $\cM_K$
als den meromorphen Zusammenhang bezeichnen, auch wird manchmal auf die Angabe
von $K$ im Subscript verzichtet, sofern klar ist, welches $K$ gemeint ist.
\end{bem}

\begin{defn}
Seien $(\cM,\partial_\cM)$ und $(\cN,\partial_\cN)$ zwei meromorphe
Zusammenhänge über $k$. Eine $k$-lineare Abbildung $\phi:\cM\rightarrow\cN$
ist ein Morphismus von meromorphen Zusammenhängen, falls sie
$\phi\circ\partial_\cM=\phi\circ\partial_\cN$ erfüllt. In diesem Fall
schreiben wir auch $\phi:(\cM,\partial_\cM)\rightarrow(\cN,\partial_\cN)$.
Ein solcher Morphismus ist ein Isomorphismus, falls die Abbildung
$\phi:\cM\rightarrow\cN$ ein Isomorphismus von $k$-Vektorräumen ist.
\end{defn}

\begin{defn}
Wir erhalten damit die Kategorie der meromorphen Zusammenhänge über $k$ mit
\begin{align*}
\text{Objekte: } & (M,\partial)\text{ meromorpher Zusammenhang über }k
\\\text{Morphismen: } & (M,\partial)\overset{f}{\rightarrow}(M',\partial')
  \text{ Morphismus von meromorphen Zusammenhängen.}
\end{align*}
\end{defn}

\begin{lem} Sei $\cM_K$ ein endlich dimensionaler $K$-Vektor Raum mit
$\partial_1$ und $\partial_2$ zwei darauf definierte Derivationen, so gilt, die
Differenz zweier Derivationen ist $K$-linear.
\end{lem}
\begin{proof}
Seien $\partial_1$ und $\partial_2$ zwei Derivationen auf $\cM_K$.
Da $\partial_1$ und $\partial_2$ $\C$-linear, ist $\partial_1-\partial_2$
$\C$-linear, also muss nur noch gezeigt werden, dass
$(\partial_1-\partial_2)(fu)=f\cdot(\partial_1-\partial_2)(u)$ $\forall f\in
K$ und $u\in\cM_K$ gilt.\\
%TODO: wieso gilt das?
\begin{align*}
(\partial_1-\partial_2)(fu) &= \partial_1(fu)-\partial_2(fu)\\
&= f'u+f\partial_1u-f'u-f\partial_2u\\
&= \underset{=0}{\underbrace{f'u-f'u}}+f\cdot(\partial_1u-\partial_2u)\\
&= f\cdot(\partial_1-\partial_2)(u)\\
\end{align*}
\end{proof}
\begin{comment}
\begin{cor}
Für $(K^r,\partial)$ ein meromorpher Zusammenhang existiert ein $A\in M(r\times
r,K)$, so dass $\partial=\frac{d}{dx}-A$.
\end{cor}
\begin{proof}
Es sei $(K^r,\partial)$ ein meromorpher Zusammenhang.  So ist
$\frac{d}{dx}-\partial:K^r\rightarrow K^r$ $K$-linear, also lässt sich durch
eine Matrix $A\in M(r\times r,K)$ darstellen
%$\frac{d}{dx}-\partial=A$
, also ist, wie
behauptet, $\partial=\frac{d}{dx}-A$.
\end{proof}
\end{comment}

\begin{comment}
\cite[Seite 129]{hotta2007d}
\end{comment}
\begin{defn}[Zusammenhangsmatrix]
Sei $(\cM_K,\partial)$ ein meromorpher Zusammenhang so wähle eine $K$-Basis
$\{e_i\}_{i\in\{1,\dots,n\}}$ von $\cM$. Dann ist die
\emph{Zusammenhangsmatrix bzgl. der Basis $\{e_i\}_{i\in\{1,\dots,n\}}$} die
Matrix $A=(a_{ij}(x))_{i,j\in\{1,\dots,n\}}\in M(n\times n,K)$ definiert
durch
%\[ \partial e_j=-\sum_{i=1}^na_{ij}(x)e_i \,. \]
\[ a_{ij}(x) = -^te_i \partial e_j \,. \]
\end{defn}

Damit ist, bezüglich der Basis $\{e_i\}_{i\in\{1,\dots,n\}}$, die Wirkung von
$\partial$ auf $u=:\,^t(u_1,\dots,u_n)$ beschrieben durch
\iffalse
  \[
  \partial(u) = \partial \Big( \sum_{i=1}^nu_i(x)e_i \Big)
  =\sum_{i=1}^n \Big( u_i'(x)-\sum_{j=1}^na_{ij}u_j(x) \Big)e_i \,.
  \]
\fi
\begin{align*}
\partial(u) &= \partial \Big( \sum_{i=1}^nu_i(x)e_i \Big)
\\&=\sum_{i=1}^n \Big( u_i'(x)-\sum_{j=1}^na_{ij}u_j(x) \Big)e_i
\\&=\underbracket{\sum_{i=1}^n u_i'(x)e_i} 
  - \underbracket{\sum_{i=1}^n\sum_{j=1}^na_{ij}u_j(x)e_i}
\\&=\overbracket{u'(x)} 
  - \overbracket{Au(x)}
\end{align*}
Damit ist die Bedingung $\partial u(x)=0$, für $u(x)= \sum_{i=1}^n u_ie_i$,
äquivalent zu der Gleichung
\begin{equation*}
u'(x)=Au(x)
\end{equation*}
für $u(x)=\,^t(u_1(x),\dots,u_n(x))$. Damit haben wir gesehen,
dass jeder meromorphe Zusammanhang $(\cM,\partial)$ ausgestattet mit einer
$K$-Basis $\{e_i\}_{i\in\{1,\dots,n\}}$ von $\cM$ zu einem System von
gewöhnlichen Differentialgleichungen zugeordnet werden kann.

\begin{comment}
\begin{prop}[Transformationsformel] \cite[Chap 5.1.1]{hotta2007d}
In der Situation
\begin{center}
\begin{tikzpicture} [scale=3.3, descr/.style={fill=white,inner sep=2.5pt} ]
\matrix (m) [
  matrix of math nodes
  , row sep=3em
  , column sep=3em
  %, text height=3em
  %, text depth=0.25em
]
{
  K^r &   &   & K^r \\
      & M & M & \\
  K^r &   &   & K^r \\
};
\path[->,font=\scriptsize,>=angle 90]
(m-1-1) edge node[descr]{$\cong$} node[above]{$\phi$} (m-2-2)
(m-3-1) edge node[descr]{$\cong$} node[above]{$\psi$} (m-2-2)
(m-1-4) edge node[descr]{$\cong$} node[above]{$\phi$} (m-2-3)
(m-3-4) edge node[descr]{$\cong$} node[above]{$\psi$} (m-2-3)

(m-2-2) edge node[above]{$\partial$} (m-2-3)

(m-1-1) edge node[above]{$\frac{d}{dz}+A$} (m-1-4)
(m-3-1) edge node[above]{$\frac{d}{dz}+B$} (m-3-4)

(m-3-1) edge node[descr]{$\cong$} node[right]{$T$} (m-1-1)
(m-3-4) edge node[descr]{$\cong$} node[left]{$T$} (m-1-4)
;

\path[>=stealth,|->]
;
\end{tikzpicture}
\end{center}
mit $\phi,\psi$ und $T$ $K$-Linear und $\partial,(\frac{d}{dx}+A)$ und
$(\frac{d}{dx}+B)$ $\C$-Linear, gilt:\\
Der meromorphe Zusammenhang. $\frac{d}{dx}+A$ auf $K^r$ wird durch
Basiswechsel
$T\in GL(r,K)$ zu
\[
\frac{d}{dx}+(T^{-1}\cdot T'+T^{-1}AT) = \frac{d}{dx}+B
\]
\end{prop}
\end{comment}

\begin{defn}
Wenn wir umgekehrt mit einer Matrix $A=(a_{ij}(x))\in M(n\times n,K)$,
welche eine System von ODEs beschreibt, beginnen, können wir durch
\begin{align*}
\cM_A & :=\bigoplus_{i=1}^n Ke_i &&\text{und}
& \partial_A \sum_{i=1}^n u_ie_i &:= \sum_{i=1}^n\left(
  \frac{\partial u_i}{\partial x}-\sum_{j=1}^na_{ij}(x)u_j \right)e_i
\end{align*}
den \emph{assoziierten meromorphen Zusammenhang} $(\cM_A,\partial_A)$ definieren.
\end{defn}
\begin{lem}
Sind $A_1,$ $A_2\in M(n\times n,K)$ zwei Matrizen, die zwei Systeme von ODEs
beschreiben, dann gilt
\begin{align*}
\frac{d}{dx}u(x)=A_1u(x) && \text{ ist äquivalent zu }
  && \frac{d}{dx}u(x)=A_2u(x)
\end{align*}
genau dann, wenn
\begin{align*}
(\cM_{A_1},\partial_{A_1}) && \text{ ist äquivalent zu }
  && (\cM_{A_2},\partial_{A_2}) \,.
\end{align*}
\end{lem}
Damit haben wir eine Eins zu Eins Beziehung zwischen meromorphen
Zusammenhängen und Systemen von ODEs.
Genauer betrachtet wird dies beispielsweise in \cite[Sec 5.1]{hotta2007d}.

\section{Äquivalenz zu holonomen lokalisierten $\cD$-Moduln}
%Hier nun einige Eigenschaften meromorpher Zusammenhänge.
\begin{thm}
Ein meromorpher Zusammenhang bestimmt ein holonomes lokalisiertes
$\cD_K$-Modul und umgekehrt.
\end{thm}
\begin{proof}
Dies wird beispielsweise in \cite[Thm 4.3.2]{sabbah_cimpa90} bewiesen.
\end{proof}

\begin{lemdef}
\cite[Satz 4.12]{ZulaBarbara}
\cite[Thm 4.3.2]{sabbah_cimpa90}
Ist $\cM_K$ ein meromorpher Zusammenhang, dann existiert ein $P\in\cD_K$ so
dass $\cM_K\cong\cD_K/\cD_K\cdot P$. So ein wird $P$ dann als
\emph{Minimalpolynom} von $\cM_K$ bezeichnet.
\end{lemdef}
\begin{proof}
\cite[Satz 4.12]{ZulaBarbara}
\end{proof}
\begin{comment}
\begin{rem}
\cite[Proof of Theorem 5.4.7]{sabbah_cimpa90}
\[
\dim_{\hat K}\cM_{\hat K} =\deg P \mbox{ wenn } \cM_{\hat K}=\cD/\cD\cdot P
\]
\end{rem}
\end{comment}
\begin{comment}
\cite[4.2]{sabbah_cimpa90}
Let $\cM$ be a left $\cD$-module. First we consider it only as a
$\C\{x\}$-module and let $\cM[x^{-1}]$ be the localized module.
\end{comment}

%TODO: Reihenfolge bei sabbah anderst!!!
\begin{lem}[Lemma vom zyklischen Vektor] \label{lem:Zyklischer-Vektor}
Sei $\cM_K$ ein meromorpher Zusammenhang. Es existiert ein Element
$m\in\cM_K$ und eine ganze Zahl $d$ so dass
$m,\partial_xm,\dots,\partial_x^{d-1}m$ eine $K$-Basis von $\cM_K$ ist.
\end{lem}
\begin{proof}
Ein Beweis ist beispielsweise in \cite[Thm 4.3.3]{sabbah_cimpa90} oder
ausführlicher in \cite[Satz 4.8]{ZulaBarbara}.
\end{proof}
\begin{cor}
In der Situation von Lemma \ref{lem:Zyklischer-Vektor} gibt es ein $P\in
\cD_K$ mit $\partial$-Grad von $P$ ist gleich $d$ und $P \cdot m=0$, in
diesem Fall ist $P$ ein Minimalpolynom zu $\cM_K$, also gilt
$\cM_K=\cD_K/\cD_K\cdot P$. Explizit ergibt sich aus der Basisdarstellung
\begin{align*}
\partial_x^d m&= \alpha_{d-1}\partial_x^{d-1}m + \alpha_{d-2}\partial_x^{d-2}m
  + \cdots + \alpha_1\partial_xm + \alpha_0m & \alpha_i\in K
\end{align*}
von $\partial_x^d m$, dass
\[
\cM_K=\cD_K/\cD_K\cdot( \underset{=:P}{\underbrace{
  \partial^d - \alpha_{d-1}\partial_x^{d-1} - \alpha_{d-2}\partial_x^{d-2} -
  \cdots - \alpha_1\partial_x - \alpha_0 }} )
\]
gilt.
\end{cor}

\begin{thm}\label{thm:basic-splitting}
\cite[Seite 64]{ZulaBarbara} %TODO: hat die sich hier verschrieben???
Ist $P=P_1\cdot P_2$ mit $P_1,P_2\in \cD_K$ so gilt
\[
\cD_K/\cD_K\cdot P \cong \cD_K/\cD_K\cdot P_1 \oplus \cD_K/\cD_K\cdot P_2 \,.
\]
\end{thm}
\begin{proof}
\cite[Seite 57-64]{ZulaBarbara}
\end{proof}
\begin{cor}
Sei $P=P_1\cdot P_2$ mit $P_1,P_2\in \cD_K$ wie in Satz
\ref{thm:basic-splitting} so gilt
\[
\cD_K/\cD_K\cdot (P_1\cdot P_2) \cong \cD_K/\cD_K\cdot (P_2 \cdot P_1)
\]
\end{cor}
\begin{proof} Denn:
\begin{align*}
\cD_K/\cD_K\cdot P &= \cD_K/\cD_K\cdot (P_1\cdot P_2)\\
  &\cong \cD_K/\cD_K\cdot P_1 \oplus \cD_K/\cD_K\cdot P_2\\
  &= \cD_K/\cD_K\cdot P_2 \oplus \cD_K/\cD_K\cdot P_1\\
  &\cong \cD_K/\cD_K\cdot (P_2 \cdot P_1)
\end{align*}
\end{proof}

\section{Newton Polygon} \label{sec:NewtonPolygon}
% ist dies eine Invariante??
% gestohlen aus der ZulaBarbara Seite 46
\begin{comment}
Quelle: sabbah?\\
sabbah mach alles formal, Barbara mach alles konvergent
\end{comment}
Jedes $P\in \cD_{\hat K}$, also insbesondere auch jedes $P\in\cD_K$, lässt sich
eindeutig als
\[
P=\sum^{n}_{k=0}a_k(x)\partial_x^k}
=\sum^{n}_{k=0}\big(\sum^{\infty}_{l=-N}{\alpha_{kl}x^l\big)\partial_x^k} \]
mit $\alpha_{ml}\in\C$ schreiben. Betrachte das zu $P$ dazugehörige
\begin{align*}
H(P):&=\underset{m,l\mbox{ mit }\alpha_{ml}\neq0}{\bigcup}\Big( (m,l-m) +
    \R_{\leq 0}\times \R_{\geq 0} \Big) \subset \R^2\\
  &=\underset{m\mbox{ mit }a_{m}\neq0}{\bigcup}\Big( (m,deg(a_m)-m) +
    \R_{\leq 0}\times \R_{\geq 0} \Big) \subset \R^2 \,.
%TODO: welche Definition ist besser?
\end{align*}

\begin{defn}
Das Randpolygon der konvexen Hülle $\conv(H(P))$ von $H(P)$ heißt das
\emph{Newton Polygon} von $P$ und wird als $N(P)$ geschrieben.
\end{defn}

\begin{bem}
Claude Sabbah definiert das Newton-Polygon in \cite[5.1]{sabbah_cimpa90}
auf eine andere Weise. Er schreibt
\[
P=\sum_ka_k(x)(x\partial_x)^k
\]
mit $a_k(x)\in \Ckx$ und definiert das Newton-Polygon als das
Randpolygon der konvexen Hülle von
\[
H'(P):=\underset{m\mbox{ mit }a_{m}\neq0}{\bigcup}\Big( (m,deg(a_m)) +
\R_{\leq 0}\times \R_{\geq 0} \Big) \subset \R^2 \,.
\]
%TODO: zeige, das beide Definitionen gleich sind
\end{bem}

\begin{defn} % aus der zula
Die Menge $\slopes(P)$ sind die nicht-vertikalen Steigungen von $N(P)$, die
sich echt rechts von $\{0\}\times\R$ befinden.\\ % der $y$-Achse
%TODO: bessere Formulierung
\begin{itemize}
\item Schreibe $\cP(\cM)$ für die Menge der zu $\cM$ gehörigen slopes.
\item P heißt \emph{regulär} oder \emph{regulär singulär} $:\Leftrightarrow$
$\slopes(P)=\{0\}$ oder $\deg P=0$, sonst \emph{irregulär singulär}.
\item Ein meromorpher Zusammenhang $\cM_{\hat K}$ (bzw. $\cM_K$) heißt regulär
singulär, falls es ein regulär singuläres $P\in \cD_{\hat K}$ (bzw. $P\in
\cD_K$) gibt, mit $\cM_{\hat K}\cong\cD_{\hat K}/\cD_{\hat K}\cdot P$ (bzw.
$\cM_K\cong\cD_K/\cD_K\cdot P$).
\end{itemize}
\end{defn}

\begin{exmp} \label{exmp:Newton-Polygon}
\begin{enumerate}
\item Ein einfaches Beispiel ist
$P_1=x^{\textcolor{red}1}\partial_x^{\textcolor{blue}2}$.  Es ist abzulesen,
dass
\begin{align*}
\textcolor{blue}m &= \textcolor{blue}2 &
\text{und}&&
\textcolor{red}l  &= \textcolor{red}1
\end{align*}
so dass
\[
H(P_1)=\Big( (\textcolor{blue}{2},\textcolor{red}{1}-\textcolor{blue}{2}) +
\R_{\leq 0}\times \R_{\geq 0} \Big) =\{(u,v)\in\R^2|u\leq 2, v\geq -1\} \,.
\]
In Abbildung \ref{fig:Newton-Polygon1} ist der Quadrant, der zum Monom
$x\partial_x^2$ gehört (blau) sowie das Newton Polygon eingezeichnet.
Offensichtlich ist $\slopes(P_1)=\{0\}$ und damit ist $P_1$ regulär singulär.
\item 
\begin{comment}
\cite[Bsp 5.3. 2.]{ZulaBarbara}
\end{comment}
Sei $P_2=x^4(x+1)\partial_x^4+x\partial_x^2+\frac{1}{x}\partial_x+1$, so kann
man das entsprechende Newton Polygon konstruieren.
Das Newton Polygon wurde in Abbildung \ref{fig:Newton-Polygon2} visualisiert.
Man erkennt, dass $\cP(P_2)=\{0,\frac{2}{3}\}$ ist.
\end{enumerate}
\end{exmp}
\begin{figure}[htbp]
  \begin{minipage}[hbt]{0,49\textwidth}
  \begin{center}
    \begin{tikzpicture}[scale=1,descr/.style={fill=white,inner sep=2.5pt}]
    \def\myPoints{2/-1}
    \def\myPath{-- (0,-1) -- (2,-1)}
    \myPlotFunction[coordsize]{\myPoints}{\myPath}{2}{-1}{-1}{$N(P_1)$}
    \end{tikzpicture}
  \end{center}
  \caption{Newton-Polygon zu $P_1=x\partial_x^2$}
  \label{fig:Newton-Polygon1}
  \end{minipage}
  \begin{minipage}[hbt]{0,49\textwidth}
  \begin{center}
    \begin{tikzpicture}[scale=1,descr/.style={fill=white,inner sep=2.5pt}]
    \def\myPoints{0/0,1/-2,2/-1,4/0,4/1}
    \def\myPath{-- (1,-2) -- node[descr]{$\frac{2}{3}$} (4,0)}
    \myPlotFunction[coordsize]{\myPoints}{\myPath}{4}{-2}{-1}{$N(P_2)$}
    \end{tikzpicture}
  \end{center}
  \caption{Newton-Polygon zu $P_2$}
  \label{fig:Newton-Polygon2}
  \end{minipage}
\end{figure}

\begin{comment}
ZUM LÖSCHEN
\begin{bem}
Für alle $f\in \cD_{\hat K}^\times$
\footnote{ Für einen Ring $R$, bezeichnet $R^\times$ die Einheitengruppe von
$R$. }
gilt allgemein, dass das zu $P\in
\cD_{\hat K}$ gehörige Newton Polygon, bis auf vertikale Verschiebung mit dem
von $f\cdot P$ übereinstimmt.
\end{bem}
\begin{proof}
Siehe \cite[Seite 25]{sabbah_cimpa90} bzw.
\cite[Bem 5.4]{ZulaBarbara}.
%TODO
\end{proof}
\end{comment}
\begin{bem}
Nach \cite[Seite 25]{sabbah_cimpa90} gilt, dass das Newton-Polygon, bis auf
vertikales verschieben, nur von dem assoziierten meromorphen Zusammenhang
abhängt.
Dies wird auch in \cite[Bem 5.4]{ZulaBarbara} diskutiert.
\end{bem}

\begin{comment}
Damit Lässt sich das Newton Polygon, durch ein $f$, immer so Verschieben, dass
$(0,0)\in N(f\cdot P)$, und es gilt, dass
\[
\cD_K\cdot P=\cD_K\cdot(f\cdot P) \vartriangleleft \cD_K
\]
ist.
\end{comment}

\begin{defn}
In einem Polynom
$P=\epsilon x^{p}\partial_x^{q}
+\sum^{n}_{k=0}\big(\sum^{\infty}_{l=-N}{\alpha_{kl}x^l\big)\partial_x^k}$,
mit $\epsilon, \alpha_{kl}\in \C, p,q \in \Z$ sind die restlichen Monome
\emph{Therme im Quadranten} von $\epsilon x^{p}\partial_x^{q}$, falls für alle
$k\in\N$ und $l\in\Z_{\geq -N}$ mit $\alpha_{kl}\neq 0$ gilt:
$k \leq q$ und $l-k \geq p-q$.
\end{defn}
\begin{bem}
\begin{itemize}
\item Anschaulich bedeutet das, dass
\[
H(\epsilon x^{p}\partial_x^{q})
=\Big( (q,p-q) + \R_{\leq 0} \times \R_{\geq 0} \Big) \supset
\Big( (k,l-k) + \R_{\leq 0} \times \R_{\geq 0} \Big)
=H(\alpha_{kl}x^l\partial_x^k)
\,,
\]
für alle relevanten $k$ und $l$.
\item Sei $P$ ein Polynom, bei dem alle Koeffizienten im Quadranten von
$\epsilon x^{p}\partial_x^{q}$ sind, dann gilt:
\begin{align*}
H(P)&=H(\epsilon x^{p}\partial_x^{q}
  +\sum^{n}_{k=0}\big(\sum^{\infty}_{l=-N}{\alpha_{kl}x^l\big)\partial_x^k})
\\&=H(\epsilon x^{p}\partial_x^{q} + \textbf{T.i.Q. von }x^{p}\partial_x^{q})
\\&=H(\epsilon x^{p}\partial_x^{q})
\\\Rightarrow N(P)&=N(\epsilon x^{p}\partial_x^{q}) \,.
\end{align*}
Also können Therme, die sich bereits im Quadranten eines anderen Therms
befinden und nicht der Therm selbst sind, vernachlässigt werden, wenn das
Newton-Polygon gesucht ist. Das \textbf{T.i.Q.} ist eine hier Abkürzung für
Therme im Quadranten.
\end{itemize}
\end{bem}
\begin{comment}
\begin{exmp}
\[
(x^a\partial_x^b)^c
=x^{ac}\partial_x^{bc}+\textbf{T.i.Q. von }x^{ac}\partial_x^{bc}
\]
und somit gilt
\begin{align*}
N((x^a\partial_x^b)^c)
  &=N(x^{ac}\partial_x^{bc}+\textbf{T.i.Q. von }x^{ac}\partial_x^{bc})
\\&=N(x^{ac}\partial_x^{bc})
\end{align*}
\end{exmp}
\end{comment}

\begin{lem}
%TOTO: sabbah redet hier schon immer von \hat K, ist das nötig?
\cite[5.1]{sabbah_cimpa90} %Seite 25f
\begin{enumerate}
\item $\cP(\cM_K)$ ist nicht Leer, wenn $\cM_K\neq\{0\}$
\item Wenn man eine exakte Sequenz
$0\rightarrow{\cM'}_K\rightarrow{\cM}_K\rightarrow{\cM''}_K\rightarrow0$
hat, so gilt $\cP(\cM_K)=\cP({\cM'}_K)\cup\cP({\cM''}_K)$.
\begin{comment}
Siehe auch \cite[Thm 5.3.4]{sabbah_cimpa90}

Dort Steht:\\
Wir erhalten die Exakte Sequenz
\[
0 \rightarrow \cD_{\hat K}/\cD_{\hat K} \cdot P_1
  \rightarrow \cD_{\hat K}/\cD_{\hat K} \cdot P
  \rightarrow \cD_{\hat K}/\cD_{\hat K} \cdot P_2
  \rightarrow 0
\]
\begin{cor}
\cite[Thm 5.3.4]{sabbah_cimpa90}
$\cP(P)=\cP(P_1)\cup\cP(P_2)$ und $\cP(P_1)\cap\cP(P_2)=\emptyset$
\end{cor}
\end{comment}
% es gibt noch 2 weitere punkte
\end{enumerate}
\end{lem}

\begin{thm} \label{thm:Split-after-slopes}
\cite[Thm 5.3.1]{sabbah_cimpa90} \cite[5.15]{ZulaBarbara}
Sei $\cM_{\hat{K}}$ ein formaler meromorpher Zusammenhang und sei
$\cP(\cM_{\hat K})=\{\Lambda_1,\dots,\Lambda_r\}$ die Menge seiner
slopes. Es existiert eine (bis auf Permutation) eindeutige Zerlegung
\[
\cM_{\hat K}=\bigoplus_{i=1}^r\cM_{\hat K}^{(i)}
\]
in formale meromorphe Zusammenhänge
mit $\cP(\cM_{\hat K}^{(i)})=\{\Lambda_i\}$.
\end{thm}
\begin{proof}
\cite[Thm 5.3.1]{sabbah_cimpa90} oder \cite[5.15]{ZulaBarbara}
\end{proof}
\begin{bem}
In Satz \ref{thm:Split-after-slopes} ist es wirklich notwendig, formale
meromorphe Zusammenhänge zu betrachten, denn das Resultat gilt nicht für
konvergente meromorphe Zusammenhänge.
\end{bem}
\begin{comment}
\begin{exmp}
\cite[Ex 5.3.6]{sabbah_cimpa90}
Sei $P=x(x\partial_x)^2+x\partial_x+\frac{1}{2}$. So sieht das Newton-Polygon
wie folgt aus
\begin{figure}[H] % htbp
\begin{center}
  \begin{tikzpicture}[scale=1.5,descr/.style={fill=white,inner sep=2.5pt}]
  \def\myPoints{0/0,1/0,2/1}
  \def\myPath{ -- (1,0) -- node[descr]{$1$} (2,1)}
  \myPlotFunction{\myPoints}{\myPath}{2}{0}{1}{$N(P)$}
  \end{tikzpicture}
\end{center}
\caption{Newton Polygon zu $P=x(x\partial_x)^2+x\partial_x+\frac{1}{2}$}
\end{figure}
mit den slopes $\cP(P)=\{0,1\}=:\{\Lambda_1,\Lambda_2\}$. Nach dem Satz
\ref{thm:Split-after-slopes} existiert eine Zerlegung $P=P_1\cdot P_2$ mit
$\cP(P_1)=\{\Lambda_1\}$ und $\cP(P_2)=\{\Lambda_2\}$. Durch scharfes hinsehen
erkennt man, dass
\begin{align*}
P &= x(x\partial_x)^2+x\partial_x+\frac{1}{2}\\
  &\dots\\
  &= (x(x\partial_x)+\dots)\cdot(x\partial_x+\dots)\\
  &\dots\\
  &= P_1\cdot P_2
\end{align*}
\paragraph{anders geschrieben}
\begin{align*}
P &= x(x\partial_x)^2+x\partial_x+\frac{1}{2}\\
  &= xx\partial_xx\partial_x+x\partial_x+\frac{1}{2}\\
  &= x^2(x\partial_x+1)\partial_x+x\partial_x+\frac{1}{2}\\
  &= x^3\partial_x^2+x^2\partial_x+x\partial_x+\frac{1}{2}\\
  &= x^3\partial_x^2+(x^2+x)\partial_x+\frac{1}{2}\\
\end{align*}
So sieht das Newton-Polygon
wie folgt aus
\begin{figure}[H]
\begin{center}
  \begin{tikzpicture}[scale=1.5,descr/.style={fill=white,inner sep=2.5pt}]
  \def\myPoints{0/0,1/0,1/1,2/1}
  \def\myPath{ -- (1,0) -- node[descr]{$1$} (2,1)}
  \myPlotFunction{\myPoints}{\myPath}{2}{0}{1}{$N(P)$}
  \end{tikzpicture}
\end{center}
\caption{Newton Polygon zu $P$}
\end{figure}
\end{exmp}
\end{comment}

\begin{comment}
\begin{cor}
\cite[Cor 5.2.6]{sabbah_cimpa90}
Falls $\cM_{\hat K}$ ein regulärer formaler meromorpher Zusammenhang ist, dann
ist $\cM_{\hat K}$ isomorph zu einer direkten Summe von elementaren formalen
Zusammenhängen. Wobei die elementaren formalen Zusammenhänge die sind, die zu
passendem $\cD_{\hat K}/\cD_{\hat K}\cdot(x\partial_x-\alpha)^p$ isomorph
sind.
\end{cor}
\end{comment}

\subsection{Die Filtrierung $\,^\ell V\cD_{\hat K}$ und das $\ell$-Symbol}
\begin{comment}
TODO: mache alle Linearformen $L$ zu $\ell$
\end{comment}
Sei $\Lambda=\frac{\lambda_0}{\lambda_1}\in \Q_{\geq 0}$ vollständig gekürtzt,
also mit $\lambda_0$ und $\lambda_1$ in $\N$ relativ prim. Definiere die
Linearform $\ell(s_0,s_1)=\lambda_0s_0+\lambda_1s_1$ in zwei Variablen, sei
$P\in\cD_{\hat K}$.  Falls $P=x^a\partial_x^b$ mit $a\in \Z$ und $b\in \N$,
setzen wir
\[
\ord_\ell(P)=\ell(b,b-a)
\]
und falls $P=\sum_{i=0}^d b_i(x)\partial_x^i$ mit $b_i\in\hat K$, setzen wir
\[
\ord_\ell(P)=\max_{\{i\mid a_i\neq 0\}} \ell(i,i-v(b_i))\,.
%Hier ist ein fehler im Sabbah script a_i <-> b_i
\]
\begin{defn}[Die Filtrierung $\,^\ell V\cD_{\hat K}$]
\cite[Seite 25]{sabbah_cimpa90}
Nun können wir die aufsteigende Filtration $\,^\ell V\cD_{\hat K}$, welche mit
$\Z$ indiziert ist, durch
\[
\,^\ell V_\lambda\cD_{\hat K}:=\{P\in\cD_{\hat K}\mid \ord_\ell(P)\leq \lambda\}
\]
definieren.
\end{defn}
\begin{bem}
Man hat $\ord_\ell(PQ)=\ord_\ell(P)+\ord_\ell(Q)$ und falls $\lambda_0\neq 0$,
hat man auch, das $\ord_\ell([P,Q])\leq \ord_\ell(P)+\ord_\ell(Q)-1$.
\end{bem}
\begin{defn}[$\ell$-Symbol]
\cite[Seite 25]{sabbah_cimpa90}
Falls $\lambda_0\neq 0$, ist der graduierte Ring $gr^{\,^\ell V}\cD_{\hat
K}\bydef \bigoplus_{\lambda \in \Z}gr_\lambda^{\,^\ell V}\cD_{\hat K}$ ein
kommutativer Ring. Bezeichne die Klasse von $\partial_x$ in dem Ring durch
$\xi$, dann ist der Ring isomorph zu $\hat K[\xi]$.
%
Sei $P\in \cD_{\hat K}$, so ist $\sigma_\ell(P)$ definiert als die Klasse von
$P$ in $gr_{\ord_\ell(P)}^{\,^\ell V}\cD_{\hat K}$. $\sigma_\ell$ wird hierbei
als das $\ell$-Symbol Bezeichnet.
\end{defn}
Zum Beispiel ist $\sigma_\ell(x^a\partial_x^b)=x^a\xi^b$.
\begin{bem}
Bei \cite{sabbah_cimpa90} wird der Buchstabe $L$ anstatt $\ell$ für
Linearformen verwendet, dieser ist hier aber bereits für $\Ckt$ reserviert.
Dementsprechend ist die Filtrierung dort als $\,^L V\cD_{\hat K}$ und das
$\ell$-Symbol als $L$-Symbol zu finden.
\end{bem}
\begin{bem}
Ist $P\in \cD_{\hat K}$ geschrieben als
$P=\sum_i\sum_j\alpha_{ij}x^j\partial_x^i$.
So erhält man $\sigma_\ell(P)$ durch die Setzung
\[
\sigma_\ell(P)=\sum_{\{(i,j)\mid\ell(i,i-j)=\ord_\ell(P)\}}\alpha_{ij}x^j\xi^i \,.
\]
\end{bem}
\begin{proof}
TODO
\end{proof}
\begin{comment}
Ich will die Linearform vermeiden und direkt die skalare Steigung verwenden
\end{comment}
\begin{defn}[Stützfunktion]
Die Funktion
\[
\omega_P:[0,\infty)\rightarrow\R, \omega_P(t):=\inf\{v-tu \mid (u.v) \in N(P)\}
\]
heißt Stützfunktion und wird in \cite{ZulaBarbara} als Alternative zu dieser
Ordnung verwendet.
\end{defn}
\begin{bem}
Wenn $\ell(x_0,s_1)$ wie oben aus $\Lambda$ entstanden ist, so gilt
\[
\omega_P(\Lambda)=ord_\ell(P) \,.
\]
\end{bem}
\begin{comment}
TODO: ist $\ell$ Slope (gehört zu Slope) dann hat $\sigma_\ell(P)$ zumindest 2
Monome
\end{comment}

%TODO: titel: Operations on vector spaces with connection
%TODO: verzweigung statt pull-back?
\section{Operationen auf meromorphen Zusammenhängen}
\subsection{Tensorprodukt}
\begin{comment}
\begin{defn}[Tensorprodukt]
\cite[3(Algebra).11.21]{stacks-project}
% von Vorlesung Algebra 2
\begin{center}
\begin{tikzpicture} [scale=3.3, descr/.style={fill=white,inner sep=2.5pt} ]
  \matrix (m) [
    matrix of math nodes
    , row sep=2em
    , column sep=3em
    %, text height=3em
    %, text depth=0.25em
  ]{
    M\times N & M\otimes_RN \\
              & T \\
  };
  %TODO: Pfeile
  \path[->,font=\scriptsize,>=angle 90]
  (m-1-1) edge node[above]{$  $} (m-1-2)
  (m-1-1) edge node[below]{$f$} (m-2-2)
  ;
  \path[->,font=\scriptsize,>=angle 90,dashed]
  (m-1-2) edge node[right]{$\exists!\gamma$} (m-2-2)
  ;
\end{tikzpicture}
\end{center}
Für eine Abbildung $f:M\rightarrow M'$ definiere das Tensorprodukt davon über
$R$ mit $N$ als
\[
\id_N \otimes f:
\begin{array}[t]{ccc}
N\otimes_{R}M & \rightarrow & N\otimes_{R}M'\\
n\otimes m & \mapsto & n\otimes f(m)
\end{array}
\]
\end{defn}
\end{comment}
\begin{bem} \label{bem:Rechenregeln-Tensorprodukt}
Hier einige Rechenregeln für das Tensorprodukt,
\begin{align}
(M\otimes_R N)\otimes_S L &\cong M\otimes_R (N \otimes_S L)
  \label{bem:Rechenregeln-Tensorprodukt1}\\
M\otimes_R R &\cong M \label{bem:Rechenregeln-Tensorprodukt2}
\end{align}
Sei $f:M'\rightarrow M$ eine Abbildung, so gilt
\begin{align}
N\otimes_R(M/\im(f)) &\cong N\otimes_R M / \im(\id_{R}\otimes f)
\label{bem:Rechenregeln-Tensorprodukt3}
\end{align}
\end{bem}

\begin{prop} \cite[Prop 4.1.1]{SchneidersDmod} \label{prop:tensorZusammenhang}
Seien $(\cM,\partial_\cM)$ und $(\cN,\partial_\cN)$ meromorphe Zusammenhänge.
Sei $n\otimes n\in \cM\otimes_K\cN$.
Durch Setzten von
\begin{equation} \label{eq:TensorAbleiten}
\partial_\otimes(m\otimes n)=\partial_\cM(m)\otimes n +m\otimes \partial_\cN(n)
\end{equation}
als die Wirkung von $\partial$ auf das $K$-Modul $\cM\otimes_K\cN$, wird
$(\cM\otimes_K\cN,\partial)$ zu einem meromorphen Zusammenhang.
\end{prop}
\begin{comment}
\begin{proof}
Klar
\end{proof}
\end{comment}
\begin{lem} \cite[Ex 5.3.7]{sabbah_cimpa90}
%TODO: move after slope definition!!!
Falls $\cN$ regulär und nicht Null, dann ist die Menge der Slopes von
$\cM\otimes\cN$ genau die Menge der Slopes von $\cM$.
\end{lem}
\begin{proof}
TODO
\end{proof}

\subsection{pull-back und push-forward}
\begin{comment}
Nach \cite[1.a]{sabbah_Fourier-local} und \cite[1.3]{hotta2007d}.
\end{comment}
Es sei
\begin{align*}
\rho&:\C\rightarrow \C , t\mapsto x:=\rho(t) &\in t\Cft
\end{align*}
eine polynomielle Abbildung mit  Bewertung $p\geq1$.
Hier werden wir meistens $\rho(t)=t^p$ für ein $p\in \N$ betrachten. Diese
Funktion induziert eine Abbildung
\begin{align*}
\rho^*&:\Ckx\hookrightarrow \Ckt, f \mapsto f\circ \rho & \mbox{bzw.} &&
\rho^*&:\Cfx\hookrightarrow \Cft, f \mapsto f\circ \rho \,.
\end{align*}
Analog erhalten wir
\begin{align*}
\rho^*&:K\hookrightarrow L:=\Cktl, f \mapsto f\circ \rho & \mbox{bzw.} &&
\rho^*&:\hat K\hookrightarrow \hat L:=\Cftl, f \mapsto f\circ \rho \,,
\end{align*}
wobei $L$ (bzw. $\hat L$) eine endliche Körpererweiterung von $K$ (bzw. $\hat
K$) ist.
\begin{comment}
TODO: damit wird $\hat L$ zu einem $\hat K$ Vektorraum.
\end{comment}
Sei $\cM_{\hat K}$ ein endlich dimensionaler $\Cftl$ Vektorraum ausgestattet mit
einem Zusammenhang $\nabla$.
%
\begin{defn}[pull-back] \label{defn:pull-back}
\cite[1.a]{sabbah_Fourier-local} und
\cite[Page 34]{sabbah_cimpa90}
Der \emph{pull-back} oder das \emph{inverse Bild} $\rho^{+}\cM_{\hat K}$ von
$(\cM_{\hat K},\nabla)$ ist der Vektorraum
\[
\rho^{*}\cM_{\hat K}:=\hat L\otimes_{\hat K}\cM_{\hat K}
\bydef\Cftl\otimes_{\Cfxl}\cM_{\Cfxl}
\]
 mit dem \emph{pull-back Zusammenhang} $\rho^*\nabla$ definiert durch
%TODO: ist das der Zusammenhang oder die Wirkung oder was?
\begin{equation} \label{eq:pull-back-zusammenhang}
\partial_t(1\otimes m):=\rho'(t)\otimes\partial_xm \,.
\end{equation}
\end{defn}
Für ein allgemeines $\phi\otimes m\in \rho^{*}\cM_{\hat K}$ gilt somit
\begin{equation} \label{eq:pull-back-zusammenhang-2}
\partial_t(\phi\otimes m):=\rho'(t)(\phi\otimes\partial_xm) +
  \frac{\partial\phi}{\partial t}\otimes m \,.
\end{equation}
\begin{comment}
Nun wollen wir uns noch genauer mit dem pull-back beschäftigen, und stellen uns
die Frage:
\paragraph{Wie sieht die Wirkung der Derivation auf dem pull-back Zusammenhang
aus?} Für $\rho(t)=t^p$ betrachten wir beispielsweise ein Element der Form
$f(x)m=f(\rho(t))m\in\rho^*\cM_{\hat K}$, dann gilt
\begin{align*}
\partial_x(f(x)m) &= \partial_{\rho(t)}(f(\rho(t))m) \\
  &= f'(\rho(t))\cdot \underset{=1}
  {\underbrace{\frac{\partial(f(t))}{\partial(f(t))}}}m +
  f(\rho(t))\underset{=\partial_x} {\underbrace{\partial_{\rho(t)}}}m
\\&= f'(\rho(t))m + f(\rho(t))\partial_x m
  = (\star)
\\ \rho'(t)^{-1}\partial_t(f(x)m) &= \frac{1}{pt^{p-1}}\partial_t(f(t^p)m)
\\ &= f'(t^p)m+f(t^p)\frac{1}{pt^{p-1}}\partial_t m = (\star) \\
\end{align*}
Also gilt $\partial_t(f(t)m) = \rho'(u)^{-1}\partial_u(f(t)m)$ und somit
lässt sich vermuten, dass die Wirkung von $\partial_x$ gleich der Wirkung von
$\rho'(t)^{-1}\partial_t$ ist. In der Tat stimmt diese Vermutung, wie das
folgende Lemma zeigt.
\end{comment}
%
\begin{thm} \label{thm:pull-back-berechnung}
%Wie erhält man den pull-back Zusammenhang bzw. wie ist er berechenbar?
In der Situation von Lemma \ref{defn:pull-back}, mit
$\cM_{\hat K}=\cD_{\hat K}/\cD_{\hat K}\cdot P(x,\partial_x)$ für ein
$P(x,\partial_x)\in\cD_{\hat K}$, gilt
\[\rho^*\cM_{\hat K}\cong \cD_{\hat L}\Big/\cD_{\hat L}\cdot
  P(\rho(t),\rho'(t)^{-1}\partial_t) \,. \]
Für $P(\rho(t),\rho'(t)^{-1}\partial_t)$ werden wir auch
$\rho^*P(t,\partial_t)$ schreiben.
\end{thm}
\begin{comment}
\cite[Seite 130]{coutinho1995primer} Holonomic modules are preserved under
this construction.
\end{comment}
%
\begin{comment}
\cite[Page 34]{sabbah_cimpa90}
Sei $\cM_{\hat K}$ ein formaler meromorpher Zusammenhang. Man definiert
$\pi^*\cM_{\hat K}$ als den Vektor Raum über $\hat L:\pi^*\cM_{\hat K}=\hat
L\otimes_{\hat K}\cM_{\hat K}$. Dann definiert man die Wirkung von $\partial_t$
durch: $t\partial_t\cdot(1\otimes m)=q(1\otimes(x\partial_x\otimes m))$ und
damit
\[
t\partial_t\cdot(\phi\otimes m)=q(\phi\otimes(x\partial_x\cdot
m))+((t\frac{\partial\phi}{\partial t})\otimes m) \,.
\]
Man erhält damit die Wirkung von $\partial_t=t^{-1}(t\partial_t)$.
\end{comment}

Für den Beweis von Satz \ref{thm:pull-back-berechnung} werden zunächst ein paar Lemmata bewiesen.

\begin{lem} \label{lem:pull-back-hilfslemma1pre}
Es gilt $\rho^*\cD_{\hat K }\bydef\hat L\otimes_{\hat K}\cD_{\hat K } \cong
\cD_{\hat L }$ als $\cD_{\hat L}$-Vektorräume, mittels
\begin{comment} TODO: VR oder Moduln??  \end{comment}
\begin{center}
\begin{tikzpicture} [scale=3.3, descr/.style={fill=white,inner sep=2.5pt} ]
\matrix (m) [
  matrix of math nodes
  %, row sep=1.5em
  , column sep=3em
  %, text height=3em
  %, text depth=0.25em
]{
  \Phi:\hat L\otimes_{\hat K}\cD_{\hat K }  & \cD_{\hat L } \\
  %1\otimes m(t,\partial_t) & m(\rho(u),\rho'(u)^{-1}\partial_u) \\
  %f(u)\otimes m(t,\partial_t) & f(u)m(\rho(u),\rho'(u)^{-1}\partial_u) \\
  f(t)\otimes Q(x,\partial_x) & f(t)Q(\rho(t),\rho'(t)^{-1}\partial_t) \\
};
%TODO: Pfeile
\path[->,font=\scriptsize,>=angle 90]
(m-1-1) edge node[above]{$\cong$} (m-1-2)
;
\path[|->,font=\scriptsize,>=angle 90]
(m-2-1) edge (m-2-2)
%(m-3-1) edge (m-3-2)
;
\end{tikzpicture}
\end{center}
\end{lem}
\begin{comment}
\begin{proof}
Wir wollen zeigen, dass $\cD_{\hat L}$ die universelle Eigenschaft für das
Tensorprodukt $\hat L\otimes_{\hat K}\cD_{\hat K }$ erfüllt, in diesem Fall
folgt die Behauptung. Zunächst sei die bilineare Abbildung
\[
\kappa: \hat
L\times\cD_{\hat K }\rightarrow\cD_{\hat L},\, (f(t),Q(x,\partial_x))\mapsto
f(t)Q(\rho(t),\rho'(t)^{-1}\partial_t)
\]
gegeben, und nach der universellen Eigenschaft des Tensorproduktes gibt es
genau eine lineare Abbildung, so dass das folgende Diagramm kommutiert.
\begin{center}
\begin{tikzpicture} [scale=3.3, descr/.style={fill=white,inner sep=2.5pt} ]
\matrix (m) [
  matrix of math nodes
  , row sep=3em
  , column sep=3em
  %, text height=3em
  %, text depth=0.25em
]{
  \hat L\times\cD_{\hat K } & \hat L\otimes_{\hat K}\cD_{\hat K }\\
  & \cD_{\hat L } \\
};
\path[->,font=\scriptsize,>=angle 90]
(m-1-1) edge node[above]{$\otimes$} (m-1-2)
        edge node[above]{$\kappa$} (m-2-2)
;
\path[->,dashed]
(m-1-2) edge node[right]{$\exists!$} (m-2-2)
;
\end{tikzpicture}
\end{center}
Dieser so erhaltene eindeutige Morphismus ist genau unser $\Phi$.
\begin{center}
\begin{tikzpicture} [scale=3.3, descr/.style={fill=white,inner sep=2.5pt} ]
\matrix (m) [
  matrix of math nodes
  , row sep=3em
  , column sep=3em
  %, text height=3em
  %, text depth=0.25em
]{
  \hat L\times\cD_{\hat K } & \hat L\otimes_{\hat K}\cD_{\hat K }\\
  & \cD_{\hat L } \\
  & V \\
};
\path[->,font=\scriptsize,>=angle 90]
(m-1-2) edge node[right]{$\Phi$} (m-2-2)
(m-1-1) edge node[above]{$\otimes$} (m-1-2)
        edge node[above]{$\kappa$} (m-2-2)
        edge node[above]{$\gamma$} (m-3-2)
;
\path[->,dashed,bend left=60]
(m-1-2) edge node[right]{$\exists!$} (m-3-2)
;
\path[->,dashed]
(m-2-2) edge node[right]{$\exists?$} (m-3-2)
;
\end{tikzpicture}
\end{center}
\end{proof}
\end{comment}
\begin{proof}
Prüfe zunächst die Injektivität. Sei $f(t)\otimes Q(x,\partial_x)\in
\ker(\Phi)$ so, dass
\begin{align*}
0 &= \Phi(f(t)\otimes Q(x,\partial_x))
\\&= f(t)Q(\rho(t),\rho'(t)^{-1}\partial_t)
\end{align*}
und, da hier alles nullteilerfrei ist, ist die Bedingung äquivalent zur
folgenden
\begin{align*}
\Leftrightarrow && 0&=f(t) &\text{oder}&& 0&=Q(\rho(t),\rho'(t)^{-1}\partial_t)
\\\Leftrightarrow && 0&=f(t) &\text{oder}&& 0&=Q(x,\partial_x)
\\\Leftrightarrow&& 0&=f(t)\otimes Q(x,\partial_x) \,.
\end{align*}
\begin{comment} TODO: korrekt? \end{comment}
Nun zur Surjektivität.
Sei $g(t,\partial_t)=\sum_ka_k(t)\partial_t^k\in\cD_{\hat L}$ so gilt
\begin{align*}
g(t,\partial_t)&=\sum_ka_k(t)\partial_t^k
\\&=\sum_ka_k(t)(\underset{=1}{\underbrace{\rho'(t)\rho'(t)^{-1}}})^k
  \partial_t^k
\\&=\sum_ka_k(t)\rho'(t)^k(\rho'(t)^{-1} \partial_t)^k
\end{align*}
und zerlege $a_k(t)\rho'(t)^k=\sum_{i=0}^{p-1}t^ia_{k,i}(t^p)$. Damit gilt dann
\begin{align*}
g(t,\partial_t)&= \sum_k\sum_{i=0}^{p-1}t^ia_{k,i}(t^p)
  (\rho'(t)^{-1} \partial_t)^k
\\&= \sum_{i=0}^{p-1}t^i\Big(\sum_ka_{k,i}(t^p)
  (\rho'(t)^{-1} \partial_t)^k\Big)
\\&= \Phi\Big(\sum_{i=0}^{p-1}t^i\otimes(\sum_ka_{k,i}(x)
  (\partial_x)^k)\Big) \,.
\end{align*}
\end{proof}

\begin{lem} \label{lem:pull-back-hilfslemma1}
Das in Lemma \ref{lem:pull-back-hilfslemma1pre} definierte $\Phi$ ist sogar ein
Morphismus von meromorphen Zusammenhängen, also gilt sogar
$\rho^*\cD_{\hat K }\bydef\hat L\otimes_{\hat K}\cD_{\hat K } \cong \cD_{\hat L
}$
als meromorphe Zusammenhänge.
\end{lem}
\begin{proof}
Sei $\partial_t$ wie gewohnt und $\partial_\otimes$ der Zusammenhang auf
$\hat L\otimes_{\hat K}\cD_{\hat K }$, welcher wie in Proposition
\ref{prop:tensorZusammenhang} definiert sei.
Wir wollen noch zeigen, dass $\partial_t\circ\Phi = \Phi \circ
\partial_\otimes$ gilt, also dass $\Phi$ ein Morphismus von meromorphen
Zusammenhängen ist. Betrachte dazu das Diagramm
\begin{center}
\begin{tikzpicture} [scale=3.3, descr/.style={fill=white,inner sep=2.5pt} ]
\matrix (m) [
  matrix of math nodes
  , row sep=2.5em
  , column sep=5em
  %, text height=3em
  %, text depth=0.25em
]{
\hat L\otimes_{\hat K}\cD_{\hat K } & \hat L\otimes_{\hat K}\cD_{\hat K } \\
\cD_{\hat L } & \cD_{\hat L }\\
};
\path[->,font=\scriptsize,>=angle 90]
(m-1-1) edge node[above]{$\partial_\otimes$} (m-1-2)
(m-1-1) edge node[descr]{$\cong$} node[right]{$\Phi$} (m-2-1)
(m-1-2) edge node[descr]{$\cong$} node[right]{$\Phi$} (m-2-2)
(m-2-1) edge node[above]{$\partial_t$} (m-2-2)
;
\end{tikzpicture}
\end{center}
und für einen Elementartensor $f(t)\otimes Q(x,\partial_x)\in\hat
L\otimes_{\hat K}\cD_{\hat K }$
\begin{comment}
Q wie in großen Beweis später, Namenskollision
\end{comment}
folgt dann
\begin{center}
\begin{tikzpicture} [scale=3.3, descr/.style={fill=white,inner sep=2.5pt} ]
\matrix (m) [
  matrix of math nodes
  , row sep=2.5em
  , column sep=2em
  %, text height=3em
  %, text depth=0.25em
]{
f(t)\otimes Q(x,\partial_x) &
  \partial_tf(t)\otimes Q(x,\partial_x)
  + \rho'(t)\otimes \partial_xQ(x,\partial_x)\\
& \partial_tf(t)Q(x,\partial_x)
  + \underset{=1}{\underbrace{\rho'(t)\cdot \rho'(t)^{-1}}}
  \partial_tQ(\rho(t),\rho'(t)^{-q}\partial_t)\\
f(t)Q(\rho(t),\rho'(t)^{-1}\partial_t) &  \partial_tf(t)Q(x,\partial_x)
  + \partial_tQ(\rho(t),\rho'(t)^{-q}\partial_t)\\
};
\path[|->,font=\scriptsize,>=angle 90]
(m-1-1) edge node[right]{$\Phi$} (m-3-1)
(m-1-1) edge node[above]{$\partial_\otimes$} (m-1-2)
(m-3-1) edge node[above]{$\partial_t$} (m-3-2)
(m-1-2) edge node[right]{$\Phi$} (m-2-2)
;
\path[bend right=90]
(m-3-2) edge node[descr]{$=$} (m-2-2)
;
\end{tikzpicture} , %TODO: hier wirklich ein Komma??
\end{center}
also kommutiert das Diagramm.
\end{proof}
\begin{comment}
\begin{bem}
BENÜTZT BEREITS DAS NÄCHSTE LEMMA...

Das soeben, in Lemma \ref{lem:pull-back-hilfslemma1pre}, definierte $\Phi$ erfüllt
für Elementartensoren $1\otimes m\in \hat L\otimes_{\hat K}\cD_{\hat K}$
\begin{align*}
\partial_u(1\otimes m) &\overset{\mbox{def}}{=} \rho'(t)\otimes\partial_x m \\
&\overset{\Phi}{\mapsto} \underset{=1}{\underbrace{\rho'(t)\rho'(t)^{-1}}}
  \partial_t m(\rho(t),\rho'(t)^{-1}\partial_t) \\
&= \partial_t m(\rho(t),\rho'(t)^{-1}\partial_t)\\
&=\dots
\end{align*}
und somit (\ref{eq:pull-back-zusammenhang}) wie gewollt.
\end{bem}
\end{comment}
%
\begin{lem} \label{lem:pull-back-hilfslemma2}
Sei $P(x,\partial_x)\in \cD_K$. In der Situation
\begin{center}
\begin{tikzpicture} [scale=3.3, descr/.style={fill=white,inner sep=2.5pt} ]
\matrix (m) [
  matrix of math nodes
  , row sep=2.5em
  , column sep=5em
  %, text height=3em
  %, text depth=0.25em
]{
\hat L\otimes_{\hat K}\cD_{\hat K } & \hat L\otimes_{\hat K}\cD_{\hat K } \\
\cD_{\hat L } & \cD_{\hat L } \\
};
%TODO: Pfeile
%\path (m-1-1) edge[white] node{$\%$} (m-2-2);
\path[->,font=\scriptsize,>=angle 90]
(m-1-1) edge node[above]{$\id\otimes\_\!\cdot\! P(x,\partial_x)$} (m-1-2)
(m-1-1) edge node[descr]{$\cong$} node[right]{$\Phi$} (m-2-1)
(m-1-2) edge node[descr]{$\cong$} node[right]{$\Phi$} (m-2-2)
(m-2-1) edge node[above]{$\alpha$} (m-2-2)
;
\end{tikzpicture}
\end{center}
mit $\Phi$ wie in Lemma \ref{lem:pull-back-hilfslemma1pre}
macht $\alpha:=\_\!\cdot\! P(\rho(t),\rho'(t)^{-1}\partial_t)$ das Diagramm
kommutativ.
\end{lem}
\begin{proof}
Betrachte ein $f(t)\otimes Q(x,\partial_x)\in\hat L\otimes_{\hat K}\cD_{\hat
K}$. So gilt
\begin{center}
\begin{tikzpicture} [scale=3.3, descr/.style={fill=white,inner sep=2.5pt} ]
\matrix (m) [
  matrix of math nodes
  , row sep=2.5em
  , column sep=5em
  %, text height=3em
  %, text depth=0.25em
]{
f(t)\otimes Q(x,\partial_x) & f(t)\otimes Q(x,\partial_x)\cdot P(x,\partial_x)\\
& f(t) Q(\rho(t),\rho'(t)^{-1}\partial_t)
  \cdot P(\rho(t),\rho'(t)^{-1}\partial_t) \\
};
%TODO: Pfeile
%\path (m-1-1) edge[white] node{$\%$} (m-2-2);
\path[|->,font=\scriptsize,>=angle 90]
(m-1-1) edge node[above]{$\id\otimes\_\!\cdot\! P(x,\partial_x)$} (m-1-2)
(m-1-2) edge node[right]{$\Phi$} (m-2-2)
;
\end{tikzpicture}
\end{center}
und
\begin{center}
\begin{tikzpicture} [scale=3.3, descr/.style={fill=white,inner sep=2.5pt} ]
\matrix (m) [
  matrix of math nodes
  , row sep=2.5em
  , column sep=10em
  %, text height=3em
  %, text depth=0.25em
]{
f(t)\otimes Q(x,\partial_x) \\
f(t) Q(\rho(t),\rho'(t)^{-1}\partial_t) &
f(t) Q(\rho(t),\rho'(t)^{-1}\partial_t)
  \cdot P(\rho(t),\rho'(t)^{-1}\partial_t) \\
};
%TODO: Pfeile
%\path (m-1-1) edge[white] node{$\%$} (m-2-2);
\path[|->,font=\scriptsize,>=angle 90]
(m-1-1) edge node[right]{$\Phi$} (m-2-1)
(m-2-1) edge node[above]{$\_\!\cdot\! P(\rho(t),\rho'(t)^{-1}\partial_t)$}
  (m-2-2)
;
\end{tikzpicture}
\end{center}
also kommutiert das Diagramm mit $\alpha=\_\!\cdot\!
P(\rho(t),\rho'(t)^{-1}\partial_t)$.
\end{proof}

\begin{proof}[Beweis zu Satz \ref{thm:pull-back-berechnung}]
%TODO: warum hier alles Lokalisiert?
Sei $P\in\cD_{\hat K}$ und $\cM_{\hat K }:=\cD_{\hat K }/\cD_{\hat K }\cdot P$.
Wir wollen zeigen, dass
\begin{align*}
\rho^*\cM_{\hat K } &\overset{!}{\cong}\cD_{\hat L }/\cD_{\hat L }\cdot Q
\end{align*}
für $Q=P(\rho(t),\rho'(t)^{-1}\partial_t)$ gilt.
Betrachte dazu die kurze Sequenz
\begin{center}
\begin{tikzpicture} [scale=3.3, descr/.style={fill=white,inner sep=2.5pt} ]
  \matrix (m) [
    matrix of math nodes
    %, row sep=2em
    , column sep=2.7em
    %, text height=3em
    %, text depth=0.25em
  ]{
  0 & \cD_{\hat K } & \cD_{\hat K } & \cM_{\hat K }            & 0\\
    & u             & u\cdot P \\
    &               & u             & u\mod\cD_{\hat K}\cdot P \\
  };
  \path[->,font=\scriptsize,>=angle 90]
  (m-1-1) edge (m-1-2)
  (m-1-2) edge node[above]{$\_\!\cdot\! P$} (m-1-3)
  (m-1-3) edge node[above]{$\pi_{\hat K}$} (m-1-4)
  (m-1-4) edge (m-1-5)
  ;
  \path[|->,font=\scriptsize,>=angle 90]
  (m-2-2) edge (m-2-3)
  (m-3-3) edge (m-3-4)
  ;
\end{tikzpicture}
\end{center}
%TODO: muss \Cful oder \Cftl flach sein???
ist exakt, weil $\cM_{\hat K } \cong\cD_{\hat K }\Big/\cD_{\hat K
}\cdot P=\coker(\_\cdot P)$.  Weil $\hat K$ flach ist, da  Körper, ist
auch, nach Anwenden des
%exakten
Funktors $\hat L\otimes_{\hat K}\_$, die Sequenz
%TODO: Funktorialität von Pull-Back? NEIN: Funktorialität von tensor
\begin{center}
\begin{tikzpicture} [scale=3.3, descr/.style={fill=white,inner sep=2.5pt} ]
  \matrix (m) [
    matrix of math nodes
    , row sep=-.5em
    , column sep=2.7em
    %, text height=3em
    %, text depth=0.25em
  ]{
  0 & \hat L\otimes_{\hat K}\cD_{\hat K}
    & \hat L\otimes_{\hat K}\cD_{\hat K}
    & \hat L\otimes_{\hat K}\cM_{\hat K}
    & 0\\
    & & & \shortparallel\\
    & & & \rho^*\cM_{\hat K} \\
  };
  %TODO: Pfeile
  \path[->,font=\scriptsize,>=angle 90]
  (m-1-1) edge (m-1-2)
  (m-1-2) edge node[above]{$\id\otimes\_\!\cdot\! P$} (m-1-3)
  (m-1-3) edge node[above]{$\id\otimes\pi_{\hat K}$} (m-1-4)
  (m-1-4) edge (m-1-5)
  ;
\end{tikzpicture}
\end{center}
exakt.
\begin{comment}
Deshalb ist
\begin{align*}
\rho^*\cM_{\hat K} &\cong \coker(\id\otimes\_\cdot P)
  & \mbox{(weil exakt)}\\
&\cong \hat L\otimes_{\hat K}\cD_{\hat K } \Big/
  \Big(( \hat L\otimes_{\hat K}\cD_{\hat K } )
  \cdot (\id\otimes\_\!\cdot\!  P) \Big)
  & \mbox{(nach def. von $\coker$)}
\end{align*}
\end{comment}
Also mit $\Phi$ wie in Lemma \ref{lem:pull-back-hilfslemma1pre} und
$Q(t,\partial_t):=P(\rho(t),\rho'(t)^{-1}\partial_t)$
nach Lemma \ref{lem:pull-back-hilfslemma2}
ergibt sich
\begin{center}
\begin{tikzpicture} [scale=3.3, descr/.style={fill=white,inner sep=2.5pt} ]
  \matrix (m) [
    matrix of math nodes
    , row sep=2em
    , column sep=2.7em
    %, text height=3em
    %, text depth=0.25em
  ]{
  0 & \hat L\otimes_{\hat K}\cD_{\hat K}
    & \hat L\otimes_{\hat K}\cD_{\hat K}
    & \hat L\otimes_{\hat K}\cM_{\hat K}
    & 0\\
 & \cD_{\hat L }
    & \cD_{\hat L }
    &
    & \\
  };
  %TODO: Pfeile
  \path[->,font=\scriptsize,>=angle 90]
  (m-1-1) edge (m-1-2)
  (m-1-2) edge node[above]{$\id\otimes\_\!\cdot\! P$} (m-1-3)
  (m-1-3) edge (m-1-4)
  (m-1-4) edge (m-1-5)

  (m-1-2) edge node[descr]{$\cong$} node[right]{$\Phi$} (m-2-2)
  (m-1-3) edge node[descr]{$\cong$} node[right]{$\Phi$} (m-2-3)

  (m-2-2) edge node[above]{$\_\!\cdot\! Q$} (m-2-3)
  ;
\end{tikzpicture}
\end{center}
als kommutatives Diagramm. Nun, weil $\_\!\cdot\! Q$ injektiv ist, lässt sich
die untere Zeile zu einer exakten Sequenz fortsetzen
\begin{center}
\begin{tikzpicture} [scale=3.3, descr/.style={fill=white,inner sep=2.5pt} ]
  \matrix (m) [
    matrix of math nodes
    , row sep=2em
    , column sep=2.7em
    %, text height=3em
    %, text depth=0.25em
  ]{
  0 & \hat L\otimes_{\hat K}\cD_{\hat K}
    & \hat L\otimes_{\hat K}\cD_{\hat K}
    & \hat L\otimes_{\hat K}\cM_{\hat K}
    & 0\\
  0 & \cD_{\hat L }
    & \cD_{\hat L }
    & \cD_{\hat L }\Big/\cD_{\hat L }\cdot Q
    & 0 \\
  };
  %TODO: Pfeile
  \path[->,font=\scriptsize,>=angle 90]
  (m-1-1) edge (m-1-2)
  (m-1-2) edge node[above]{$\id\otimes\_\!\cdot\! P$} (m-1-3)
  (m-1-3) edge node[above]{$\id\otimes\pi_{\hat K}$} (m-1-4)
  (m-1-4) edge (m-1-5)

  (m-1-2) edge node[descr]{$\cong$} node[right]{$\Phi$} (m-2-2)
  (m-1-3) edge node[descr]{$\cong$} node[right]{$\Phi$} (m-2-3)
  %(m-1-4) edge[dashed,lightgray] node[right]{$\phi$} (m-2-4)

  (m-2-1) edge (m-2-2)
  (m-2-2) edge node[above]{$\_\!\cdot\! Q$} (m-2-3)
  (m-2-3) edge node[above]{$\pi_{\hat L}$} (m-2-4)
  (m-2-4) edge (m-2-5)
  ;
\end{tikzpicture}
\end{center}
und damit folgt, wegen Isomorphie der Kokerne, die Behauptung.
\end{proof}

%
\begin{lem}\label{lem:slope-pb-multiplikation}
Sei $\cP(\cM_{\hat K})=\{\Lambda_1,\dots,\Lambda_r\}$ die Menge der Slopes von
$\cM_{\hat K}$ und $\rho:t\mapsto x:=t^p$, dann gilt für $\cP(\rho^*\cM_{\hat
K})=\{\Lambda_1',\dots,\Lambda_r'\}$, dass $\Lambda_n'=p\cdot\Lambda_n$.
\end{lem}
\begin{proof}
Siehe \cite[5.4.3]{sabbah_cimpa90}.
\end{proof}
\begin{comment}
\begin{proof}
Sei $\cM_{\hat K}=\cD_{\hat K}\slash \cD_{\hat K}\cdot P$ mit $P=\sum
a_i(x)\partial_x^i$, dann ist
$\rho^*\cM_{\hat K}\cong\cD_{\hat L}\slash \cD_{\hat L}\cdot P'$ mit
\begin{align*}
H(P'(t,\partial_t)) &=H(P(\rho(t),\rho'(t)^{-1}\partial_t))
\\&=H(\sum_i a_i(\rho(t))(\rho'(t)^{-1}\partial_t)^ii)
\\&=H(\sum_i a_i(\rho(t))(\rho'(t)^{-1}\partial_t)^ii)
\\&=H(\sum_i a_i(t^p)((p\cdot t^{p-1})^{-1}\partial_t)^i)
\\&=H(\sum_i a_i(t^p)(p\cdot t^{p-1})^{-i}\partial_t^i)
\\&=H(\sum_i a_i(t^p)t^{-i(p-1)}\partial_t^i)
\\&=\dots %TODO: Hier weiter...
\end{align*}
\end{proof}
\end{comment}
%
\begin{exmp}[pull-back]\label{exmp:Pull-Back}
%from Beispiel/formal_b.tex
Hier nun ein explizit berechneter pull-back.
Wir wollen $\cM_{\hat K}:=\cD_{\hat K}/\cD_{\hat K}\cdot P$ bzgl. $P:=
x^3\partial_x^2-4x^2\partial_x-1$ betrachten.
Unser Ziel ist es hier ganzzahlige Slopes zu erhalten.
%TODO: erkläre wieso --> Elementare Zusammenhänge
Es gilt $\slopes(P)=\{\frac{1}{2}\}$ (siehe Abbildung \ref{fig:Pull-Back1}).
Wende den pull-back mit $\rho:t\rightarrow x:=t^2$ an.
Zunächst ein paar Nebenrechnungen, damit wir Satz
\ref{thm:pull-back-berechnung} einfacher anwenden können:
\begin{align*}
\partial_x   &\rightsquigarrow
  \frac{1}{\rho'(t)}\partial_t=\frac{1}{2t}\partial_t \,,
\\\partial_x^2 &\rightsquigarrow (\frac{1}{2t}\partial_t)^2
  = \frac{1}{2t}\partial_t (\frac{1}{2t}\partial_t)
  = \frac{1}{2t}(-\frac{1}{2t^2}\partial_t + \frac{1}{2t}\partial_t^2)
  = \frac{1}{4t^2}\partial_t^2-\frac{1}{4t^3}\partial_t \,.
\end{align*}
Also ergibt Einsetzen
\begin{align*}
\rho^*P &= (t^2)^3(\frac{1}{4t^2}\partial_t^2-\frac{1}{4t^3}\partial_t)-
    4(t^2)^2\frac{1}{2t}\partial_t-1
\\&= \frac{1}{4}t^4\partial_t^2 \underbracket{-t^3\frac{1}{4}\partial_t-
    4t^{3}\frac{1}{2}\partial_t}-1
\\&= \frac{1}{4}t^4\partial_t^2 \overbracket{-2\frac{1}{4}t^3\partial_t}-1 \,.
\end{align*}
%
Also ist $\rho^*P= \frac{1}{4}t^4\partial_t^2 -\frac{1}{2}t^3\partial_t-1$ mit
$ \slopes(\rho^*P)=\{1\} $ (siehe Abbildung \ref{fig:Pull-Back2}) und somit
$\rho^*\cM_{\hat K}=\cD_{\hat L}/\cD_{\hat L}
\cdot(\frac{1}{4}t^4\partial_t^2-\frac{1}{2}t^3\partial_t-1)$.
\begin{figure}[htbp]
  \begin{minipage}[hbt]{0,49\textwidth}
  \begin{center}
    \begin{tikzpicture}[scale=1.5,descr/.style={fill=white,inner sep=2.5pt}]
    \def\myPoints{0/0, 1/1, 2/1}
    \def\myPath{-- node[descr]{$\frac{1}{2}$} (2,1)}
    \myPlotFunction{\myPoints}{\myPath}{2}{0}{2}{$N(P))$}
    \end{tikzpicture}
  \end{center}
  \caption[Newton Polygon zu $P=x^3\partial_x^2-4x^2\partial_x-1$]
    {Newton Polygon zu \newline $P=x^3\partial_x^2-4x^2\partial_x-1$}
  \label{fig:Pull-Back1}
  \end{minipage}
  \begin{minipage}[hbt]{0,49\textwidth}
  \begin{center}
    \begin{tikzpicture}[scale=1.5,descr/.style={fill=white,inner sep=2.5pt}]
    \def\myPoints{0/0, 1/2, 2/2}
    \def\myPath{-- node[descr]{$1$} (2,2)}
    \myPlotFunction{\myPoints}{\myPath}{2}{0}{2}{$N(\rho^*P))$}
    \end{tikzpicture}
  \end{center}
  \caption[Newton Polygon zu $\rho^*P=%
    \frac{1}{4}t^4\partial_t^2-\frac{1}{2}t^3\partial_t-1$]
    {Newton Polygon zu \newline $\rho^*P=%
    \frac{1}{4}t^4\partial_t^2-\frac{1}{2}t^3\partial_t-1$}
  \label{fig:Pull-Back2}
  \end{minipage}
\end{figure}
\end{exmp}

Sei $\cN_{\hat L}$ ein endlich dimensionaler $\hat L$-VR mit Verknüpfung, so
definiere den push-forward wie folgt.
\begin{defn}[push-forward]
\cite[1.a]{sabbah_Fourier-local}
Der \emph{push-forward} oder das \emph{direktes Bild} $\rho_+\cN_{\hat L}$ von
$\cN_{\hat L}$ ist
\begin{itemize}
\item der $\hat K$-VR $\rho_*\cN$ ist definiert als der $\C$-Vektor Raum
$\cN_{\hat L}$ mit der $\hat K$-Vektor Raum Struktur durch
%$f(x)\cdot m:=f(\rho(t))m$
die skalare Multiplikation
$\begin{array}[t]{cccc}
\cdot: & \hat{K}\times\cN_{\hat{L}} & \rightarrow & \cN_{\hat{L}}\\
 & (f(x),m) & \mapsto & f(x)\cdot m:=f(\rho(t))m
\end{array}$ und
% für alle m aus ???
\item mit der Wirkung $\partial_x$ beschrieben durch
$\rho'(t)^{-1}\partial_t$.
\end{itemize}
\end{defn}

\begin{comment}
  \begin{minipage}[hbt]{0,49\textwidth}
  \begin{center}
    \begin{tikzpicture}[scale=1.5,descr/.style={fill=white,inner sep=2.5pt}]
    \def\myPoints{0/-3, 1/-1}
    \def\myPath{-- (1,-1)}
    \myPlotFunction{\myPoints}{\myPath}{1}{-3}{-1}{$N(P))$}
    \end{tikzpicture}
  \end{center}
  \caption{Newton-Polygon zu $P$}
  \label{fig:Push-Forward1}
  \end{minipage}
  \begin{minipage}[hbt]{0,49\textwidth}
  \begin{center}
    \begin{tikzpicture}[scale=1.5,descr/.style={fill=white,inner sep=2.5pt}]
    \def\myPoints{0/-2, 1/-1}
    \def\myPath{-- (1,-1)}
    \myPlotFunction{\myPoints}{\myPath}{1}{-2}{-1}{$N(\rho_*P))$}
    \end{tikzpicture}
  \end{center}
  \caption{Newton-Polygon zu $\rho_+P$}
  \label{fig:Push-Forward2}
  \end{minipage}
\begin{exmp}[push-forward]\label{exmp:Push-Forward}
%ACHTUNG: variablem müssen noch geändert werden!\\
%ACHTUNG: wenn das hier richtig wäre, müsste es zu einer dimensionsänderung
         %kommen!\\
Für $\rho:t\rightarrow u^2$, $\phi=\frac{1}{u^2}$ betrachte
\begin{align*}
\sE^\phi &\cong\hat\cD/\hat\cD\cdot(\partial_u+\partial_u\frac{1}{u^2})\\
&= \hat\cD/\hat\cD\cdot
(\underset{=:P}{\underbrace{\partial_u+\frac{2}{u^3}}})
\end{align*}
%also $P=\partial_u+\frac{2}{u^3}$
mit $ \slopes(P)=\{2\} $ (siehe Abbildung \ref{fig:Push-Forward1}).
Bilde nun das Direkte Bild über $\rho$, betrachte dazu
\begin{align*}
\partial_u+\frac{2}{u^3} &= 2u(\frac{1}{2u}\partial_u+\frac{1}{u^4}) \\
&= 2u(\rho'(u)^{-1}\partial_u+\frac{1}{u^4}) \\
&= 2u(\partial_t+\frac{1}{t^2})\\
\end{align*}
Also ist
$\rho_+\sE^\phi\cong \hat\cD/\hat\cD\cdot(\partial_t+\frac{1}{t^2})$
mit $\rho_+P=\partial_t+\frac{1}{t^2}$ und $ \slopes(\rho_+P)=\{1\} $ (siehe
Abbildung \ref{fig:Push-Forward2})
\end{exmp}
\end{comment}

\begin{thm} \label{thm:Projektionsformel}
\cite[1.a]{sabbah_Fourier-local}
Es gilt die Projektionsformel
\begin{equation} \label{eq:Projektionsformel}
\rho_+(\cN_{\hat L}\otimes_{\hat L}\rho^+\cM_{\hat K}) \cong
\rho_+\cN_{\hat L}\otimes_{\hat K}\cM_{\hat K}\,.
\end{equation}
\end{thm}
\begin{proof}
\begin{align*}
\rho_+(\cN_{\hat L}\otimes_{\hat L}\rho^+\cM_{\hat K}) &=
\rho_+(\underbracket{\cN_{\hat L}\otimes_{\hat L}(\hat L\otimes_{\hat
  K}\cM_{\hat L})})
  & \mbox{(def von $\rho^+\cM_{\hat K}$)}\\
&\cong\rho_+(\overbracket{\underbracket{(\cN_{\hat L}\otimes_{\hat L}\hat L)}
  \otimes_{\hat K}\cM_{\hat K}})
  & \mbox{(Rechenregeln Tensorprodukt)}\\ %TODO: hinzufügen
&\cong \rho_+(\overbracket{\cN_{\hat L}}\otimes_{\hat K}\cM_{\hat K})
  & \mbox{(Rechenregeln Tensorprodukt)}\\
&= \rho_+\cN_{\hat L}\otimes_{\hat K}\cM_{\hat K}
  & \mbox{(?)}
\end{align*}
\end{proof}

\begin{comment}
%von treffen ? auf seite 1
Sei $\rho(u)=u^p=t$ und $\phi(t)$ gegeben.
\begin{align*}
\rho^+\sE^{\phi(t)}&=\sE^{\phi(\rho(u))}=\sE^{\phi(u^p)}\\
\rho^+\rho_+\sE^{\phi(u)}
&=\underset{\zeta\in\mu_p}{\bigoplus}\sE^{\phi(\zeta\cdot u)}\\
\end{align*}
\end{comment}

\subsection{Fouriertransformation}
\begin{defn}[Fouriertransformation]
Sei $P=\sum_{i=0}^da_i(x)\partial_x^i$, dann ist die
\emph{fouriertransformierte} von
$P$ gegeben durch
\[
\cF_P:=\cF_P(z,\partial_z)=\sum_{i=0}^da_i(\partial_z)(-z)^i \,.
\]
\begin{comment}
\cite[Def 3.1]{Bloch_localfourier}
\cite{GarciaLopez04}
\cite[Def 6.1]{ZulaBarbara}
\end{comment}
\end{defn}
\begin{defn}[Fouriertransformation von lokalisierten holonomen D-Moduln]
Ist $\cM_{\hat K}=\hat K / \hat K \cdot P$ so ist die Fouriertransformierte
davon $\,^\cF\cM_{\hat K}=\hat K / \hat K \cdot \cF_P(x,\partial_x)$.
\end{defn}
\begin{exmp}
Sei $P=x^3\partial_x^4+x^2\partial_x^2+x$ dann ist die Fouriertransformierte
davon
\begin{align*}
\cF_P&=\partial_z^3(-z)^4+\partial_z^2(-z)^2+\partial_z
\\&=\underbracket{\partial_z^2z^2} + \underbracket{\partial_z^3z^4}+\partial_z
\\&=\overbracket{z^4\partial_z^3 + \underbracket{[\partial_z^3,z^4]}}
  + \overbracket{z^2\partial_z^2 + \underbracket{[\partial_z^2,z^2]}}
  + \partial_z
\\&\begin{aligned}
  =&z^4\partial_z^3 + \overbracket{\underbracket{
    \sum_{i=1}^3\frac{4\cdot3\dots(5-i)\cdot 3 \cdot 2  \dots
    (4-i)}{i!}z^{4-i}\partial_z^{3-i}
  }} + z^2\partial_z^2
\\&\qquad+ \overbracket{\underbracket{
    \sum_{i=1}^2\frac{2\cdot1\dots(3-i)\cdot 2 \cdot 1  \dots
    (3-i)}{i!}z^{2-i}\partial_z^{2-i}
  }}
  + \partial_z
\end{aligned}
\\&=z^4\partial_z^3 + \overbracket{
    12z^3\partial_z^2 + \frac{72}{2}z^2\partial_z + \frac{144}{6}z
  }
  + z^2\partial_z^2 + \overbracket{ 4z\partial_z + \frac{4}{2} } + \partial_z
\\&=z^4\partial_z^3
  + (12z^3 + z^2)\partial_z^2
  + (36z^2 + 4z + 1)\partial_z
  + 24z + 2
\end{align*}
mit den Newton Polygonen wie in Abbildung \ref{fig:fourierA} und
\ref{fig:fourierB}.
\begin{figure}[htbp]
  \begin{minipage}[hbt]{0,49\textwidth}
  \begin{center}
    \begin{tikzpicture}[scale=1,descr/.style={fill=white,inner sep=2.5pt}]
    \def\myPoints{0/1,2/0,4/-1}
    \def\myPath{-- (4,-1)}
    \myPlotFunction{\myPoints}{\myPath}{4}{-1}{1}{}
    \end{tikzpicture}
  \end{center}
  \caption{Newton-Polygon zu $P$}
    \label{fig:fourierA}
  \end{minipage}
  \begin{minipage}[hbt]{0,49\textwidth}
  \begin{center}
    \begin{tikzpicture}[scale=1,descr/.style={fill=white,inner sep=2.5pt}]
    \def\myPoints{0/0,0/1,%
                  1/-1,1/0,1/1,%
                  2/0,2/1,%
                  3/1 }
    \def\myPath{-- (1,-1) -- (3,1)}
    \myPlotFunction{\myPoints}{\myPath}{3}{-1}{1}{}
    \end{tikzpicture}
  \end{center}
  \caption{Newton-Polygon zu $\cF_P$}
    \label{fig:fourierB}
  \end{minipage}
\end{figure}
\end{exmp}

\begin{comment}
\subsection{Betrachten bei Unendlich}
\end{comment}
% vim: set ft=tex :


%!TeX root = main.tex
\chapter{Levelt-\!Turrittin-\!Theorem}
\section{Elementare meromorphe Zusammenhänge}
%\section{Elementare formale meromorphe Zusammenhänge}
\begin{defn}
Ein \emph{elementarer regulärer meromorpher Zusammenhang} ist ein Zusammenhang
$\cM$, welcher isomorph zu $\cD_{\hat K}/\cD_{\hat K}\cdot
(x\partial_x-\alpha)^p$, mit passendem $\alpha$ und $p$, ist.
\end{defn}
\begin{bem}
Es ist leicht zu sehen, dass jeder elementare reguläre meromorphe Zusammenhang
tatsächlich auch regulär ist.
\end{bem}

\begin{lem}
Es existiert eine Basis von $\cM_{\hat K}$ über $\hat K$ mit der Eigenschaft,
dass die Matrix, die $x\partial_x$ beschreibt, nur Einträge in $\Cfx$ hat.
\end{lem}
\begin{comment}
\cite[Lem 5.2.1.]{sabbah_cimpa90}
\end{comment}
\begin{proof}
Wähle einen zyklischen Vektor $m\in\cM_{\hat K}$ % TODO: richtiger Raum?
 und betrachte die Basis $m,\partial_x m,\dots,\partial_x^{d-1}m$ (siehe Lemma
\ref{lem:Zyklischer-Vektor}).
Schreibe $\partial_x^dm=\sum_{i=0}^{d-1}(-b_i(x))\partial_x^im$ in
Basisdarstellung mit Koeffizienten $b_i\in\hat K$.
Also erfüllt $m$ die Gleichung
$\partial_x^dm+\sum_{i=0}^{d-1}b_i(x)\partial_x^im=0$.\\
\begin{comment} TODO: bis hier schon klar \end{comment}
Tatsächlich kann man $b_i(x)=x^ib_i'(x)$ mit $b_i'\in \Cfx$ schreiben (wegen
Regularität).\\
Dies impliziert, dass $m,x\partial_xm,\dots,(x\partial_x)^{d-1}m$ ebenfalls
eine Basis von $\cM_{\hat K}$ ist.\\
Die Matrix von $x\partial_x$ zu dieser neuen Basis hat nur Einträge in $\Cfx$.
\end{proof}
\begin{lem}
Es existiert sogar eine Basis von $\cM_{\hat K}$ über $\hat K$ so dass die
Matrix zu $x\partial_x$ konstant ist.
\end{lem}
\begin{proof}
Siehe \cite[Thm 5.2.2]{sabbah_cimpa90}.
\end{proof}

\begin{thm} \label{thm:regulaerInDirSumme}
Ein regulärer formaler Zusammenhang $\cM_{\hat K}$ ist isomorph zu einer
direkten Summe von elementaren regulären meromorphen Zusammenhängen.
\end{thm}
\begin{proof}[Beweisskizze]
Man wählt eine Basis von $\cM_{\hat K}$, in der die Matrix zu $x\partial_x$
konstant ist. Diese Matrix kann in Jordan Normalform gebracht werden und damit
erhält man das Ergebnis.
Ausgeführt wurde das in \cite[Cor. 5.2.6]{sabbah_cimpa90}.
\end{proof}

\begin{comment}
einführen als Bausteine oder kleinste meromorphe Zusammenhänge
\end{comment}

Durch twisten der elementaren regulären meromorphen Zusammenhänge erhält man
wie folgt die elementaren meromorphen Zusammenhänge.
\begin{defn} \label{defn:elemMerZsh}
Ein \emph{elementarer meromorpher Zusammenhang} ist ein Zusammenhang $\cM$, für
den es $\psi \in \Cfxl$, $\alpha\in\C$ und $p\in \N$ gibt, so dass
\[
\cM\cong \sE^{\psi}\otimes R_{\alpha,p}\,,
\]
mit $R_{\alpha,p}:=\cD/\cD(x\partial_x-\alpha)^p$, also ein elementarer
regulärer meromorpher Zusammenhang, ist.
\end{defn}

\begin{lem} In der Situation von Definition \ref{defn:elemMerZsh} gilt
$\sE^{\psi}\otimes R_{\alpha,p}\cong
\cD/\cD\cdot(x\partial_x-(\alpha+x\frac{\partial \psi}{\partial x}))^p$.
\end{lem}
\begin{proof} Denn
\begin{align*}
\sE^{\psi}\otimes R_{\alpha,p}&=\sE^{\psi}\otimes\cD/\cD(x\partial_x-\alpha)^p
\\&\!\!\overset{\ref{lem:twistRechenregel}}{=}
  \cD/\cD(x(\partial_x-\frac{\partial \psi}{\partial x})-\alpha)^p
\\&=\cD/\cD(x\partial_x-(\alpha+x\frac{\partial \psi}{\partial x}))^p \,.
\end{align*}
\end{proof}

\begin{comment}
\section{Definition in \cite{sabbah_Fourier-local}}
%TODO: auch nicht formal
\begin{defn}[Elementarer formaler Zusammenhang]
\cite[Def 2.1]{sabbah_Fourier-local}
Zu einem gegebenen $\rho\in t\C\llbracket t\rrbracket$, $\phi\in \hat L \bydef
\C(\!(t)\!)$ und einem endlich dimensionalen $\hat L$-Vektorraum $R$ mit
regulärem Zusammenhang $\nabla$, definieren wir den assoziierten elementaren
endlich dimensionalen $\hat K$-Vektorraum mit Zusammenhang, durch:
\[
El(\rho,\phi,R)=\rho_+(\sE^\phi\otimes R)
\]
\end{defn}
\cite[nach Def 2.1]{sabbah_Fourier-local}
Bis auf Isomorphismus hängt $El(\rho,\phi,R)$ nur von $\phi\mod\Cft$ ab.
\begin{lem}
\cite[Lem 2.2]{sabbah_Fourier-local}
\end{lem}
\begin{lem} \cite[Lem 2.6.]{sabbah_Fourier-local}
Es gilt $El([t\mapsto t^p],\phi,R)\cong El([t\mapsto t^p],\psi,S)$ genau dann,
wenn
\begin{itemize}
\item es ein $\zeta$ gibt, mit $\zeta^p=1$ und
$\psi\circ\mu_\zeta\equiv\phi\mod\Cft$
\item und $S\cong R$ als $\hat L$-Vektorräume mit Zusammenhang.
\end{itemize}
\end{lem}
\begin{proof}
Siehe \cite[Lem 2.6.]{sabbah_Fourier-local}
\end{proof}
%
\begin{prop} \cite[Prop 3.1]{sabbah_Fourier-local}
Jeder irreduzible endlich dimensionale $\hat K$-Vektorraum $\cM$ mit
Zusammenhang ist isomorph zu $\rho_+(\sE^\phi\otimes L)$, wobei $\phi\in
t^{-1}\C[t^{-1}]$, $\rho:t\rightarrow t^p$ vom Grad $p\geq 1$ und ist minimal
unter $\phi$. (siehe \cite[Rem  2.8]{sabbah_Fourier-local}) und $L$ ist ein
Rang $1$ $\hat L$-Vektrorraum mit regulärem Zusammenhang.
\end{prop}
\begin{proof}
%TODO: verwendet hier schon das klassische Levelt-Turittin
Siehe \cite[Prop 3.1]{sabbah_Fourier-local}
\end{proof}
\end{comment}

\section{Die Filtrierung $\,^\ell V\cD_{\hat K}$ und das $\ell$-Symbol}
\begin{comment}
TODO: mache alle Linearformen $L$ zu $\ell$
\end{comment}
Sei $\Lambda=\frac{\lambda_0}{\lambda_1}\in \Q_{\geq 0}$ vollständig gekürtzt,
also mit $\lambda_0$ und $\lambda_1$ in $\N$ relativ prim. Definiere die
Linearform $\ell(s_0,s_1)=\lambda_0s_0+\lambda_1s_1$ in zwei Variablen, sei
$P\in\cD_{\hat K}$.  Falls $P=x^a\partial_x^b$ mit $a\in \Z$ und $b\in \N$,
setzen wir
\[
\ord_\ell(P)=\ell(b,b-a)
\]
und falls $P=\sum_{i=0}^d b_i(x)\partial_x^i$ mit $b_i\in\hat K$, setzen wir
\[
\ord_\ell(P)=\max_{\{i\mid a_i\neq 0\}} \ell(i,i-v(b_i))\,.
%Hier ist ein fehler im Sabbah script a_i <-> b_i
\]
\begin{defn}[Die Filtrierung $\,^\ell V\cD_{\hat K}$]
\cite[Seite 25]{sabbah_cimpa90}
Nun können wir die aufsteigende Filtration $\,^\ell V\cD_{\hat K}$, welche mit
$\Z$ indiziert ist, durch
\[
\,^\ell V_\lambda\cD_{\hat K}:=\{P\in\cD_{\hat K}\mid \ord_\ell(P)\leq \lambda\}
\]
definieren.
\end{defn}
\begin{bem}
Man hat $\ord_\ell(PQ)=\ord_\ell(P)+\ord_\ell(Q)$ und falls $\lambda_0\neq 0$,
hat man auch, das $\ord_\ell([P,Q])\leq \ord_\ell(P)+\ord_\ell(Q)-1$.
\end{bem}
\begin{defn}[$\ell$-Symbol]
\cite[Seite 25]{sabbah_cimpa90}
Falls $\lambda_0\neq 0$, ist der gradierte Ring $gr^{\,^\ell V}\cD_{\hat
K}\bydef \bigoplus_{\lambda \in \Z}gr_\lambda^{\,^\ell V}\cD_{\hat K}$ ein
kommutativer Ring. Bezeichne die Klasse von $\partial_x$ in dem Ring durch
$\xi$, dann ist der Ring isomorph zu $\hat K[\xi]$.
%
Sei $P\in \cD_{\hat K}$, so ist $\sigma_\ell(P)$ definiert als die Klasse von
$P$ in $gr_{\ord_\ell(P)}^{\,^\ell V}\cD_{\hat K}$. $\sigma_\ell$ wird hierbei
als das $\ell$-Symbol bezeichnet.
\end{defn}
Zum Beispiel ist $\sigma_\ell(x^a\partial_x^b)=x^a\xi^b$.
\begin{bem}
Bei \cite{sabbah_cimpa90} wird der Buchstabe $L$ anstatt $\ell$ für
Linearformen verwendet, dieser ist hier aber bereits für $\Ckt$ reserviert.
Dementsprechend ist die Filtrierung dort als $\,^L V\cD_{\hat K}$ und das
$\ell$-Symbol als $L$-Symbol zu finden.
\end{bem}
\begin{bem}
Ist $P\in \cD_{\hat K}$ geschrieben als
$P=\sum_i\sum_j\alpha_{ij}x^j\partial_x^i$.
So erhält man $\sigma_\ell(P)$ durch die Setzung
\[
\sigma_\ell(P)=\sum_{\{(i,j)\mid\ell(i,i-j)=\ord_\ell(P)\}}\alpha_{ij}x^j\xi^i \,.
\]
\end{bem}
\begin{proof}
TODO
\end{proof}
\begin{comment}
Ich will die Linearform vermeiden und direkt die skalare Steigung verwenden
\end{comment}
\begin{defn}[Stützfunktion]
Die Funktion
\[
\omega_P:[0,\infty)\rightarrow\R, \omega_P(t):=\inf\{v-tu \mid (u.v) \in N(P)\}
\]
heißt Stützfunktion und wird in \cite{ZulaBarbara} als Alternative zu dieser
Filtrierung verwendet.
\end{defn}
\begin{bem}
Wenn $\ell(x_0,s_1)$ wie oben aus $\Lambda$ entstanden ist, so gilt
\[
\omega_P(\Lambda)=ord_\ell(P) \,.
\]
\end{bem}
\begin{comment}
TODO: ist $\ell$ Slope (gehört zu Slope) dann hat $\sigma_\ell(P)$ zumindest 2
Monome
\end{comment}

\section{Levelt-\!Turrittin-\!Theorem}
\begin{comment}
Das Levelt-Turrittin-Theorem ist ein Satz, der hilft, meromorphe Zusammenhänge
in ihre irreduziblen Komponenten zu zerlegen.
\end{comment}

\begin{comment}
\subsection{Klassische Version}
\end{comment}
\begin{thm}
\cite[Thm 5.4.7]{sabbah_cimpa90}
Sei $\cM_{\hat K}$ ein formaler meromorpher Zusammenhang, so gibt es eine
ganze Zahl $p$, so dass der Zusammenhang $\cM_{\hat L}:=\rho^+\cM_{\hat K}$,
mit $\rho:t\mapsto x:=t^p$, isomorph zu einer direkten Summe von formalen
elementaren meromorphen Zusammenhänge
ist.
\end{thm}
Der folgende Beweis stammt hauptsächlich aus \cite[Seite 35]{sabbah_cimpa90}.
\begin{proof}
Zum Beweis wird Induktion auf die lexicographisch geordnetem Paare
$(\dim_{\hat K}\cM_{\hat K},\kappa)$ angewendet. Wobei
$\kappa\in\N\cup\{\infty\}$ dem größtem Slope von $\cM_{\hat K}$, falls dieser
ganzzahlig ist, entspricht. Sonsts wird $\kappa=\infty$ gesetzt.

\begin{comment}
TODO: Induktionsanfang und -schritt kennzeichnen
\end{comment}
Wir nehmen oBdA an, dass $\cM_{\hat K}$ genau einen Slope $\Lambda$ hat, sonst
Teile $\cM_{\hat K}$ mittels Satz \ref{thm:Split-after-slopes} in meromorphe
Zusammenhänge mit je einem Slope und wende jeweils die Induktion an.
Mit $\Lambda=:\frac{\lambda_0}{\lambda_1}$ (vollständig gekürtzt) definieren
wir die dem Slope entsprechende Linearform
$L(s_0,s_1):=\lambda_0s_0+\lambda_1s_1$.  Wir nennen $\sigma_L(P)\in \hat
K[\xi]$ die \emph{Determinanten Gleichung} von $P$. Da $L$ zu einem Slope von
$P$ gehört, besteht $\sigma_L(P)$ aus zumindest zwei Monomen.
\begin{comment}
and is homogeneous of degree $\ord_L(P)=0$ because $P$ is chosen with
coefficients in $\Cfx$, one of them, being a unit.
\end{comment}
Schreibe
\begin{align*}
\sigma_L(P)&=\sum_{L(i,i-j)=\ord_L(P)}\alpha_{ij}x^j\xi^i\\
  &=\sum_{L(i,i-j)=0}\alpha_{ij}x^j\xi^i \,.
\end{align*}
Sei $\theta:=x^{\lambda_0+\lambda_1}xi^{\lambda_1}$ so können wir
\[
\sigma_L(P) = \sum_{k\geq 0}\alpha_k\theta^k
\]
schreiben, wobei $\alpha_0\neq0$ ist.

\paragraph{Erster Fall: $\lambda_1=1$.} Das bedeutet, dass der Slope ganzzahlig
ist. Betrachte die Faktorisierung
\[
\sigma_L(P)=\epsilon\prod_{\beta}(\theta-\beta)^{\gamma_\beta}\,.
\]
Wobei $\epsilon\in\C$ eine Konstante ist.  Sei $\beta_0$  eine der Nullstellen,
so setze $R(z):=(\beta_0/(\lambda_0+1))z^{\lambda_0+1}$ und betrachte
$\cM_{\hat K}\otimes \cF_{\hat K}^R$.
\begin{comment}
AB HIER VLT NICHT RICHTIG, nur versuch
\end{comment}
Falls $P(x,\partial_x)\cdot e=0$ gilt
\[
P\Big(x,\partial_x-\frac{\partial R(x^{-1})}{\partial x}\Big)
  \cdot e\otimes e(R)=0
\]
und hier haben wir 
\begin{align*}
\frac{\partial R(x^{-1})}{\partial x} 
  &=\frac{\partial(\frac{\beta_0}{\lambda_0+1}x^{-(\lambda_0+1)})}{\partial
  x}\\
  &=-\beta_0z^{-(\lambda_0+2)} \,.
\end{align*}
Schreibe $P'=P(x,\partial_x+\beta_0x^{-(\lambda_0+2)})$.
\begin{lem}
Es gilt, dass $P'$ Koeffizienten in $\Cfx$ hat.
\end{lem}
\begin{proof}
TODO
\end{proof}
Des weiteren ist $\sigma_L(P')=\sum_{k\geq 0}\alpha_k(\theta+\beta_0)^k$. Wir
unterscheiden nun 2 Unterfälle:
\begin{enumerate}
\item \textbf{Die Determinanten Gleichung $\sigma_L(P)$ hat nur eine
Nullstelle.}
\begin{comment}TODO: Hier weiter \end{comment}
\item \textbf{Die Determinanten Gleichung $\sigma_L(P)$ hat mehrere
Nullstellen.}
\begin{comment}TODO: Hier weiter \end{comment}
\end{enumerate}

\paragraph{Zweiter Fall: $\lambda_1\neq1$.} In diesem Fall ist einzige Slope
$\Lambda$ nicht ganzzahlig. Mache deshalb einen pull-back mit $\lambda_1$. Sei
$\rho:t\mapsto x:=t^{\lambda_1}$ und erhalte $P'$ so dass $\rho^*\cM_{\hat
K}=\cD_{\hat L}/\cD_{\hat L}\cdot P'$.
Nach Lemma \ref{lem:slope-pb-multiplikation} hat $P'$ den einen Slope
$\Lambda\cdot\lambda_1=\lambda_0$.
Damit können wir nun die zugehörige Linearform $L':=\lambda_0s_0+s_1$
definieren. Es gilt dass
\[
\sigma_{L'}(P')=\dots
\]
ist, welches zumindest zwei unterschiedliche Nullstellen hat. Nun wendet man
den zweiten Unterfall des ersten Fall an.

\end{proof} % ende von beweis zu levelt aus sabbah_cimpa90

\begin{comment}
\subsection{Sabbah's Refined version}
\begin{prop}
\cite[Prop 3.1]{sabbah_Fourier-local}
Jeder irreduzible endlich dimensionale formale meromorphe Zusammenhang
$\cM_{\hat L}$ ist isomorph zu $\rho_+(\sE^\phi\otimes_{\hat K} S)$, wobei
$\phi\in x^{-1}\C[x^-1]$, $\rho:x\mapsto t=x^p$ mit grad $p\geq1$ minimal bzgl.
$\phi$ (siehe \cite[Rem 2.8]{sabbah_Fourier-local}), und $S$ ist ein Rang $1$
$\hat K$-Vektor Raum mit regulärem Zusammenhang.
\end{prop}
\begin{proof}
\cite[Prop 3.1]{sabbah_Fourier-local}
\end{proof}

\begin{thm}[Refined Turrittin-Levelt]
\cite[Cor 3.3]{sabbah_Fourier-local}
Jeder endlich dimensionale meromorphe Zusammenhang $\cM_{\hat K}$ kann in
eindutiger weiße geschrieben werden als direkte Summe $\bigoplus
El(\rho,\phi,R)\bydef\bigoplus\rho_+(\sE^\phi)\otimes R$, so dass
jedes $\rho_+\sE^\phi$ irreduzibel ist und keine zwei $\rho_+\sE^\phi$ isomorph
sind.
\end{thm}
\begin{proof}
\cite[Cor 3.3]{sabbah_Fourier-local}
\end{proof}
\end{comment}

% vim: set ft=tex :


%\part{Beispiele}
%% Bisher nur Notizen
% also nicht viel verwertbares

\chapter{Beispiele/Anwendung}
\section{Einfache Beispiele}
%\subsection{erstes}
\[
  P_a=t^2\partial_t^2+t\partial_t+(t^2-n^2)=\sum_{k=0}^2\sum_{l=0}^2
  \alpha_{kl}t^l\partial_t^k
\]

Mit: $\alpha_{2,2}=1$, $\alpha_{1,1}=1$, $\alpha_{0,2}=t^2$ und
$\alpha_{0,0}=n^2$

$
P_a=t^2\partial_t^2+t\partial_t+(t^2-n^2) \Rightarrow 
\begin{cases}
  k=2,l=2 & \Rightarrow u\leq l=2, v\geq l-k=0\\
  k=1,l=1 & \Rightarrow u\leq 1, v\geq 0\\
  k=0,l=0 & \Rightarrow u\leq 0, v\geq 0\\
  k=0,l=2 & \Rightarrow u\leq 0, v\geq 2\\
\end{cases}
$

\begin{center}
  \includegraphics[width=6cm]{beispiele/img/a.png}
\end{center}
also $\slopes(P_a)=\{0\}$ also ist $P_a$ regulär singulär

% vim: set ft=tex :
 % nur ein slope: 1
%\subsection{zweites}
\[
  P_b=t\partial_t^2+2\partial_t-1
\]

$
P_b=t\partial_t^2+2\partial_t-1 \Rightarrow 
\begin{cases}
  k=2,l=1 & \Rightarrow u\leq k=2, v\geq l-k=-1\\
  k=1,l=0 & \Rightarrow u\leq 1, v\geq -1\\
  k=0,l=0 & \Rightarrow u\leq 0, v\geq 0\\
\end{cases}
$

\begin{center}
  \includegraphics[width=6cm]{beispiele/img/b.png}
\end{center}
also $\slopes(P_b)=\{0\}$ also ist $P_b$ regulär singulär

% vim: set ft=tex :
 % regulär singulär
%\subsection{drittes}

\begin{comment}
  zula Barbara Seite 46
\end{comment}


\[
  P_c=t^2\partial_t+1
\]
$
P_c=t^2\partial_t+1
\Rightarrow
\begin{cases}
  k=1, l=2 & \Rightarrow u \leq 1, v \geq 1\\
  k=0, v=1 & \Rightarrow u \leq 0  v \geq 0\\
\end{cases}
$
\begin{center}
  \includegraphics[width=6cm]{beispiele/img/c.png}
\end{center}

also $\slopes(P_c)=\{1\}$ also ist $P_c$ \textbf{irregulär} singulär.\\
Hauptnenner aller Slopes ist $1$, also wieder trivial.

% vim:set ft=tex :
 % nur ein slope: 1
%\subsection{viertes}

\begin{comment}
  zula Barbara Seite 46
\end{comment}

\begin{comment}
  Original aus der Zula:
  \[
    P_d=-3t^{14}\partial_t^6+t^{11}(t+3)\partial_t^5 + 2t^8\partial_t^4
    -t^6(t^3+1)\partial_t^3 + t^4\partial_t
  \]

  $ P_d \Rightarrow
  \begin{cases}
    k=6,l=14 & \Rightarrow u\leq k=6, v\geq l-k=8\\
    k=5,l=12 & \Rightarrow u\leq 5, v\geq 7\\
    k=5,l=11 & \Rightarrow u\leq 5, v\geq 6\\
    k=4,l=8 & \Rightarrow u\leq 4, v\geq 4\\
    k=3,l=9 & \Rightarrow u\leq 3, v\geq 6\\
    k=3,l=6 & \Rightarrow u\leq 3, v\geq 3\\
    k=1,l=4 & \Rightarrow u\leq 1, v\geq 3\\
  \end{cases} $

  also ist Abbildung 5.8 auf seite 53 der zula falsch?
\end{comment}

\[
  P_d=-3t^{14}\partial_t^6+t^{11}(t+3)\partial_t^5 + 2t^8\partial_t^4
  -t^6(t^3+1)\partial_t^3 + t^3\partial_t
\]

$ P_d \Rightarrow
\begin{cases}
  k=6,l=14 & \Rightarrow u\leq k=6, v\geq l-k=8\\
  k=5,l=12 & \Rightarrow u\leq 5, v\geq 7\\
  k=5,l=11 & \Rightarrow u\leq 5, v\geq 6\\
  k=4,l=8 & \Rightarrow u\leq 4, v\geq 4\\
  k=3,l=9 & \Rightarrow u\leq 3, v\geq 6\\
  k=3,l=6 & \Rightarrow u\leq 3, v\geq 3\\
  k=1,l=3 & \Rightarrow u\leq 1, v\geq 2\\
\end{cases} $

\begin{center}
  \includegraphics[width=6cm]{beispiele/img/d.png}
\end{center}
also $\slopes(P_b)=\{0,\frac{1}{2},1,2\}$ also ist $P_d$ irregulär singulär.\\
Offenbar ist der Hauptnenner der Steigugnen gleich $2$.\\
Betrachte also $\rho:t\mapsto u^2$\\
und erhalte: ???

% vim: set ft=tex :
 % zu kompliziert
%\subsection{fünftes - bsp e}
Für $ P_e=t^4(t+1)\partial_t^4 + t\partial_t^2+\frac{1}{t}\partial_t+1 $ sieht
das Newton-Polygon wie folgt aus:

\begin{center}
  \includegraphics[width=6cm]{beispiele/img/e.png}
\end{center}

also sind die Slopes $\slopes(P_e)=\{0,\frac{2}{3}\}$.

Dies gilt Analog für das \emph{einfachere}:
\[ \bar P_e=t^4\partial_t^4 +\frac{1}{t}\partial_t \]

\begin{center}
  \includegraphics[width=6cm]{beispiele/img/bar_e.png}
\end{center}
Also offensichtlich gilt $\slopes(\bar P_e)=\{0,\frac{2}{3}\}$, also haben die
Slopes den Hauptnenner $3$, deshalb mache einen Pullback mit $\rho:t\mapsto u^3$

\begin{comment}
  Versuch:
  \[ \rho^+ \bar P_e=u^{12}\partial_u^4 +\frac{1}{u^3}\partial_u \]
  \begin{center}
    FALSCH: $\partial_t\not\mapsto\partial_u$ 
  \end{center}

  $ \rho^+ \bar P_e \Rightarrow
  \begin{cases}
    k=4, l=12 & \Rightarrow u \leq 4, v \geq 8\\
    k=1, l=-3 & \Rightarrow u \leq 1, v \geq -4\\
  \end{cases} $

  \begin{center}
    \includegraphics[width=6cm]{beispiele/img/rho_e.png}
  \end{center}

  \begin{center}
    zu steil
  \end{center}
\end{comment}

% vim: set ft=tex :
 % verbesserte version: e_new
\myinput{beispiele/e_new.tex}
\section{Meromorpher Zusammenhang der formal, aber nicht Konvergent, zerfällt}
%\begin{comment}
  %Quellen??
  %\[
    %\sum n!x^n
  %\]
%\end{comment}

\myinput{beispiele/formal_a.tex}
%% entsteht aus b durch t --> 1/\tau
%TODO: besserer Titel
\subsection{Beispiel ohne namen}
\begin{comment}
  Beginne mit
  \[ \tilde P=\tau\partial_\tau^2+2\partial_\tau-1 \]
  und gehe von $\tau$ über zu $t$ via $\tau\rightarrow\frac{1}{t}$:\\
  %TODO: rightarrow oder mapsto?
  \begin{itemize}
    \item was passiert mit der Ableitung $\partial_\tau$? Es gilt:
      \[
        \partial_\tau (f(\frac{1}{\tau}))=
        \partial_t(f)\cdot (-\frac{1}{\tau^2})=
        -\partial_t(f)\cdot t^2= %TODO: wegen klammerung?
        - t^2 \cdot \partial_t(f)
      \]
      also:
      \[
        \partial_\tau=-t^2\partial_t
        % stimmt das VZ?
      \]
    \item was ist $\partial_t(t^2\partial_t)$?
      \begin{align*}
        \partial_tt^2\partial_t &= (\partial_tt)t\partial_t\\
        &= (t\partial_t-1)t\partial_t\\
        &= t(\partial_tt)\partial_t-t\partial_t\\
        &= t(t\partial_t-1)\partial_t-t\partial_t\\
        &= t^2\partial_t^2-2t\partial_t\\
      \end{align*}
    \item was passiert mit $ \tilde P=\tau\partial_\tau^2+2\partial_\tau-1 $?
      \begin{align*}
        \tilde P &= \tau\partial_\tau^2+2\partial_\tau-1\\
        &\overset{\tau\rightarrow\frac{1}{t}}{\longrightarrow}
          \frac{1}{t}(-t^2\partial_t)^2+2(-t^2\partial_t)-1\\
        &= \frac{1}{t}t^2(\partial_t(t^2\partial_t))-2t^2\partial_t-1\\
        &= t(\partial_t(t^2\partial_t))-2t^2\partial_t-1\\
        &= t(t^2\partial_t^2-2t\partial_t)-2t^2\partial_t-1\\
        &= t^3\partial_t^2-4t^2\partial_t-1 =: P\\
      \end{align*}
  \end{itemize}
\end{comment}

Wir wollen nun den zum folgendem $P$ assoziierten Meromorphen Zusammenhang
betrachten:\\
\begin{minipage}[hbt]{0,39\textwidth}
  \[ P= t^3\partial_t^2-4t^2\partial_t-1 \]
  mit $ \slopes(P)=\{\frac{1}{2}\} $
\end{minipage}
\begin{minipage}[hbt]{0,59\textwidth}
  \begin{center}
    \includegraphics[width=6cm]{img/formal_b.png}
  \end{center}
\end{minipage}
%Es ist offensichtlich, dass $\slopes(P)=\{\frac{1}{2}\}$. 
Wir wollen ganzzahlige slopes haben, also wernde den pull-back
$\rho:t\rightarrow u^2$ an.

Zunächst ein paar nebenrechnungen: 
\begin{align*}
  \partial_t   &= \frac{1}{\rho'}\partial_u=\frac{1}{2u}\partial_u \\
  \partial_t^2 &= (\frac{1}{2u}\partial_u)^2\\
               &= \frac{1}{2u}(-\frac{1}{2u^2}\partial_u + 
                 \frac{1}{2u}\partial_u^2) \\
               &= -\frac{1}{4u^3}\partial_u+\frac{1}{4u^2}\partial_u^2 \\
\end{align*}
also
\begin{align*}
  \rho^+P &= u^6(-\frac{1}{4u^3}\partial_u+\frac{1}{4u^2}\partial_u^2)- 
            4u^{4}\frac{1}{2u}\partial_u-1\\
          &= -u^3\frac{1}{4u^3}\partial_u+\frac{1}{4}u^4\partial_u^2-
            4u^{3}\frac{1}{2}\partial_u-1\\
          &= \frac{1}{4}u^4\partial_u^2 -2\frac{1}{4}u^3\partial_u-1
\end{align*}

\begin{minipage}[hbt]{0,39\textwidth}
  \[ \rho^+P= \frac{1}{4}u^4\partial_u^2 -2\frac{1}{4}u^3\partial_u-1 \]
  mit $ \slopes(\rho^+P)=\{1\} $
\end{minipage}
\begin{minipage}[hbt]{0,59\textwidth}
  \begin{center}
    \includegraphics[width=6cm]{img/formal_b_pb.png}
  \end{center}
\end{minipage}

% vim: set ft=tex :


%\begin{comment}
  %Weiteres Beispiel:
  %\[
    %Sabbah\_Fourier-local.pdf \rightarrow 5.b.
  %\]
%\end{comment}

% vim: set ft=tex :


%%%%%%%%%%%%%%%%%%%%%%%%%%%%%%%%%%%%%%%%%%%%%%%%%%%%%%%%%%%%%%%%%%%%%
\appendix
\addcontentsline{toc}{chapter}{Anhang}

\begin{landscape}
  %\chapter{zur aufteilung in Lemma 2.4}
  \chapter{Aufteilung von ...}
  \label{chap:aufteilung}
  Sei $\phi\in u^{-1}\C[u^{-1}]$, so ist $\phi'=:\sum_{i=2}^N a_{-i}u^{-i}\in u^{-2}\C[u^{-1}]$ also 
  $u\phi'(u)=\sum_{i=1}^N a_{-i-1}u^{-i} \in u^{-1}\C[u^{-1}]$, welches wir zerlegen wollen.\\
  Zerlege also $u\phi'(u)=\sum_{j=0}^{p-1}u^j\psi_j(u^p)$
  mit $\psi_j\in\C[t^{-1}]$ für alle $j>0$ und $\psi_0\in
  t^{-1}\C[t^{-1}]$:\\
  \begin{center}
    \begin{tikzpicture} [descr/.style={fill=white,inner sep=2.5pt}]
    \matrix (m) [
      matrix of math nodes,
      row sep=1em,
      %column sep=-0.7em,
      text height=1.5ex,
      text depth=0.25ex]
    {
      & & & & & & & & & & & & & & \, \\
      u\phi'(u)= &
      a_{-2}u^{-1} &
      +...+ &
      a_{-p}u^{-(p-1)} &
      + &
      a_{-(p+1)}u^{-p} &
      + &
      a_{-(p+2)}u^{-(p+1)} &
      +...+ &
      a_{-2p}u^{-(2p-1)} &
      + &
      a_{-(2p+1)}u^{-2p} &
      + &
      a_{-(2p+3)}u^{-(2p+1)} &
      + ...  \\
      & & & & & & & & & & & & & & \, \\
    };

      \path[solid]
      (m-2-6) edge [bend left=20] node[descr]{$\psi_0(u^p)$} (m-2-12)
      (m-2-12)  edge [bend left=20] (m-1-15);

      \path[dotted]
      (m-2-4) edge [bend right=20] node[descr]{$u\psi_1(u^p)$} (m-2-10)
      (m-2-10)  edge [bend right=20] (m-3-15);

      \path[dashed]
      (m-2-2) edge [bend right=20] node[descr]{$u^{p-1}\psi_{p-1}(u^{p})$} (m-2-8)
      (m-2-8) edge [bend right=20] (m-2-14)
      (m-2-14) edge [bend right=20] (m-3-15);
    \end{tikzpicture}
  \end{center}
  also:\\
  \begin{align*}
    \psi_0(u^p) &= a_{-(p+1)}u^{-p}+a_{-(2p+1)}u^{-2p}+...\\
    \psi_1(u^p) &= a_{-p}u^{-p}+a_{-2p}u^{2p}+...\\
    & \vdots & \\
    \psi_{p-1}(u^p) &= a_{-2}u^p+a_{-(p+2)}u^{2p}+...\\
  \end{align*}
\end{landscape}

\lstdefinestyle{myLatex} { 
  language=[LaTeX]TeX
  , texcsstyle=*\bf\color{blue} 
  , basicstyle=\ttfamily
  , numbers=none
  , breaklines=true
  , commentstyle=\color{red}
  %, otherkeywords={$, \{, \}, \[, \]} 
  %, frame=lines 
  , xleftmargin=30pt          % linker Abstand vom Rand (framesep+framrule)
  , tabsize=2
  %, caption=LaTeX example
}

\chapter{Wie ich Newton Polygone zeichne}
Der code
\begin{lstlisting}[style=myLatex]
\newcommand{\myNewtonPlot}[6]{
  \draw[color=black,thick] #2;
  \foreach \pos in #1 { \fill[blue,opacity=.2] (-.5,#5) rectangle \pos; }
  \draw[->] (-.5,0) -- (#3+.7,0);
  \draw[->] (0,#4-.2) -- (0,#5+.2);
  \draw (1,0) -- (1,-.2);
  \draw (0,1) -- (-.2,1);
  \foreach \pos in #1 { \node[draw,circle,inner sep=1.5pt,fill=white] at \pos {}; }
  \node [below right] at (#3,#5/2) {#6};
}
\begin{tikzpicture}[scale=0.7]
\def\myPoints{{(0,0)}, {(1,-2)}, {(2,-1)}, {(4,0)}}
\def\myPath{(-.5,-2) -- (1,-2) -- (4,0) -- (4,2)}
\myNewtonPlot{\myPoints}{\myPath}{4}{-2}{2}{$N(P)$}
\end{tikzpicture}
\end{lstlisting}

ergibt

\begin{center}
\begin{tikzpicture}[scale=0.7]
\def\myPoints{{(0,0)}, {(1,-2)}, {(2,-1)}, {(4,0)}}
\def\myPath{(-.5,-2) -- (1,-2) -- (4,0) -- (4,2)}
\myNewtonPlot{\myPoints}{\myPath}{4}{-2}{2}{$N(P)$}
\end{tikzpicture}
\end{center}

% vim: set ft=tex :


\nocite{*}
%\bibliographystyle{dinat}
%\bibliographystyle{plain}
\bibliographystyle{amsalpha}
\bibliography{main}

\end{document}
