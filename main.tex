\documentclass[ngerman
  %,12pt
  %,titelpage % unused
  ,numbers=noenddot % obsolete
  ,headsepline
  ,parskip=half*
  ,openany
  ,DIV=15
  %,biblography=totoc  % unused 
]{scrbook}
%,landscape,twocolumn
\usepackage[ngerman]{babel}
\usepackage[T1]{fontenc}
%\usepackage[ansinew]{inputenc}
\usepackage[utf8]{inputenc}
\usepackage{lmodern} %Type1-Schriftart für nicht-englische Texte

\usepackage{lscape} % alt: pdflscape
\usepackage{amsmath,amsthm,amssymb}
\usepackage{latexsym}
\usepackage{enumerate}

\usepackage{verbatim}

\allowdisplaybreaks

%\usepackage{natbib}

\usepackage{thmbox}

\usepackage[automark]{scrpage2} % Headline styles
\usepackage[square,numbers]{natbib}

\pagestyle{scrheadings}
\chead{\today}

% Section Numbers moved left out
%\usepackage{sectsty}
%\makeatletter\def\@seccntformat#1{%
  %\protect\makebox[0pt][r]{\csname the#1\endcsname\hspace{12pt}}}\makeatother

% semibold instead of bold
%\renewcommand{\bfdefault}{sb}

\renewcommand*{\thefootnote}{[\arabic{footnote}]}

% Zeichen
\usepackage{nicefrac}
\usepackage{mathrsfs}
\usepackage{stmaryrd}

\usepackage{tikz}
\usetikzlibrary{matrix,arrows,decorations.pathmorphing}

\numberwithin{equation}{chapter}
\numberwithin{figure}{chapter}

%%%%  Theoreme Styles  %%%%%%%%%%%%%%%%%%%%%%%%%%%%%%%%%%%%%%%%%%%%%%
\makeatletter

\theoremstyle{plain}% default
\newtheorem{thm}{Satz}[chapter]
%\newtheorem{thm}[satz]{Satz}
\newtheorem{lemma}[thm]{Lemma}
\newtheorem{lem}[thm]{Lemma}
\newtheorem{kor}[thm]{Korollar}
\newtheorem{prop}[thm]{Proposition}
\newtheorem{cor}[thm]{Korollar}

\theoremstyle{definition}
\newtheorem{defn}[thm]{Definition}
\newtheorem{conj}[thm]{Conjecture}
\newtheorem{exmp}[thm]{Beispiel}

\theoremstyle{remark}
\newtheorem{rem}[thm]{Bemerkung}
\newtheorem{note}[thm]{Notiz}
\newtheorem{case}{Fall}

\usepackage[german]{varioref}
%\usepackage[nameinlink,german]{cleveref}
\usepackage[colorlinks=true,linkcolor=black]{hyperref}

%% Autoref Names %%%%%%%%%%%%%%%%%
%\crefname{lemma}{Lemma}{Lemmas}
%\crefname{equation}{Gleichung}{Gleichungen}
%\crefname{definition}{Definition}{Definitionen}
%\crefname{algorithmus}{Algorithmus}{Algorithmen}
%\crefname{kor}{Korollar}{Korollare}

%%%%  new Commands  %%%%%%%%%%%%%%%%%%%%%%%%%%%%%%%%%%%%%%%%%%%%%%%%%
%\renewcommand{\headrulewidth}{0.2pt}
\renewcommand{\P}{{\mathbb P}}
\newcommand{\R}{{\mathbb R}}
\newcommand{\K}{{\mathbb K}}
\newcommand{\N}{{\mathbb N}}
\newcommand{\Z}{{\mathbb Z}}
\renewcommand{\O}{{\mathbb O}}
\newcommand{\A}{{\mathbb A}}
\newcommand{\I}{{\mathbb I}}
\newcommand{\T}{{\mathbb T}}
\newcommand{\C}{{\mathbb C}}
\newcommand{\Q}{{\mathbb Q}}
\newcommand{\e}{\varepsilon}
\newcommand{\wto}{\rightharpoonup}
\newcommand{\wsto}{\stackrel{*}{\rightharpoonup}}
\newcommand{\css}{\subset\subset}
\newcommand{\myhr}{\rule{0.3\textwidth}{1pt}}

\renewcommand{\phi}{\varphi}
\renewcommand{\epsilon}{\varepsilon}

\DeclareMathOperator{\modulo}{mod}
\DeclareMathOperator{\conv}{conv}
\DeclareMathOperator{\Cof}{Cof}
\DeclareMathOperator{\grad}{grad}
\DeclareMathOperator{\Id}{Id}
\DeclareMathOperator{\dist}{dist}
\DeclareMathOperator{\diam}{diam}
\DeclareMathOperator{\co}{co}
\DeclareMathOperator{\supp}{supp}
\DeclareMathOperator{\graph}{graph}
\DeclareMathOperator{\slopes}{slopes}

% specific:
\newcommand{\D}{{\mathcal{D}}}
\newcommand{\Dm}{{\mathcal{D}\mbox{-Modul}}}
\newcommand{\M}{{\mathcal{M}}}
\newcommand{\E}{{\mathscr E}}

\newcommand{\Cfx}{{\mathbb C\rrbracket x\llbracket}}
\newcommand{\Cft}{{\mathbb C\rrbracket t\llbracket}}
\newcommand{\Cfu}{{\mathbb C\rrbracket u\llbracket}}

\makeatother
%}}}

\usepackage{xcolor}
\usepackage{color}
\usepackage{framed}

\newenvironment{fshaded}{%
\def\FrameCommand{\fcolorbox{framecolor}{shadecolor}}%
\MakeFramed {\FrameRestore}}%
{\endMakeFramed}

%\newenvironment{comment}{\definecolor{shadecolor}{rgb}{1,.6,.6}%
%\definecolor{framecolor}{rgb}{0,0,0}%
%\begin{fshaded}}{\end{fshaded}}

\renewcommand{\comment}{\definecolor{shadecolor}{rgb}{1,.6,.6}%
\definecolor{framecolor}{rgb}{0,0,0}%
\fshaded}
\renewcommand{\endcomment}{\endfshaded}

%Zeilenhöhe, für bessere lesbarkeit
\linespread{1.3}

%\usepackage{thmbox}


\begin{document}

%TODO: set titel variable??

%%%%%%%%%%%%%%%%%%%%%%%%%%%%%%%%%%%%%%%%%%%%%%%%%%%%%%%%%%%%%%%%%%%%%
\frontmatter

\begin{titlepage}
  \thispagestyle{empty}
  \newcommand{\Rule}{\rule{\textwidth}{1mm}}
  \begin{center}\sffamily\bfseries
    \LARGE Bachelorarbeit
    \vfill
    \Rule\vspace{5mm}
    \Huge
    mein thema
    \vspace{1mm}\Rule
    \vfill
    \normalfont\sffamily\large vorgelegt von\par
    \bfseries\LARGE Maximilian Huber
    \vfill
    \normalfont\sffamily\large am\\
    \bfseries\Large Institut für Mathematik\\
    \normalfont\sffamily\large der\\
    \bfseries\Large Universität Augsburg
    \vfill
    \normalfont\sffamily\large betreut durch \\
    \bfseries\Large Prof. Dr. Marco Hien \par
    \vfill
    \normalfont\sffamily\large abgegeben am \\
    \bfseries\Large noch nicht\\
  \end{center}
\begin{comment}
  \begin{center}
    stand: \today
  \end{center}
\end{comment}
\end{titlepage}
% vim: set ft=tex :


%\begin{center}
  %\today
%\end{center}
\tableofcontents{}
%\newpage

\begin{titlepage}
  \thispagestyle{empty}
  \newcommand{\Rule}{\rule{\textwidth}{1mm}}
  \begin{center}\sffamily\bfseries
    \LARGE Bachelorarbeit
    \vfill
    \Rule\vspace{5mm}
    \Huge
    mein thema
    \vspace{1mm}\Rule
    \vfill
    \normalfont\sffamily\large vorgelegt von\par
    \bfseries\LARGE Maximilian Huber
    \vfill
    \normalfont\sffamily\large am\\
    \bfseries\Large Institut für Mathematik\\
    \normalfont\sffamily\large der\\
    \bfseries\Large Universität Augsburg
    \vfill
    \normalfont\sffamily\large betreut durch \\
    \bfseries\Large Prof. Dr. Marco Hien \par
    \vfill
    \normalfont\sffamily\large abgegeben am \\
    \bfseries\Large noch nicht\\
  \end{center}
\begin{comment}
  \begin{center}
    stand: \today
  \end{center}
\end{comment}
\end{titlepage}
% vim: set ft=tex :


%%%%%%%%%%%%%%%%%%%%%%%%%%%%%%%%%%%%%%%%%%%%%%%%%%%%%%%%%%%%%%%%%%%%%
\mainmatter

\part{Grundlagen}

\chapter{Mathematische Grundlagen}

Hier werde ich mich auf \cite{sabbah_cimpa90} und \cite{coutinho1995primer} beziehen.

\section{Einige Ergebnise aus der Kommutativen Algebra}~

In dieser Arbeit spielen die folgenden Ringe eine Große Rolle:
\begin{itemize}
  \item $\C[x]:=\{ \sum^{N}_{i=1}a_ix^i | N\in \N \}$
  \item $\C\{x\}:=\{ \sum^{\infty}_{i=1}a_ix^i | \mbox{pos.
        Konvergenzradius} \}$
  \item $\C\llbracket x\rrbracket:=\{ \sum^{\infty}_{i=1}a_ix^i \}$
  \item $K:=\C(\{x\}):=\C\{x\}[x^{-1}]$
  \item $\hat{K}:=\C((x)):=\C\llbracket x\rrbracket[x^{-1}]$
\end{itemize}

wobei offensichtlich gilt $\C[x]\subset\C\{x\}\subset\C\llbracket x\rrbracket$.

\begin{comment}
  \begin{lem}[Seite 2]
    ein paar eigenschaften
    \begin{enumerate}
      \item $\C[x]$ ist ein graduierter Ring, durch die Grad der
        Polynome. Diese graduierung induziert eine aufsteigende Filtrierung.

        alle Ideale haben die form $(x-a)$ mit $,a\in \C$
      \item wenn $\mathfrak{m}$ das maximale Ideal von $\C[x]$ (erzeugt von
        $x$ ist), so ist
        \[
          \C[[x]]=
          \underset{k}{\underleftarrow{\lim}} \C[X]\backslash\mathfrak{m}^k
        \]
        The ring $\C[[x]]$ ist ein nöterscher lokaler Ring:
        jede Potenzreihe mit konstantem term $\neq 0$ ist invertierbar.

        Der ring ist ebenfalls ein diskreter ??? Ring (discrete valuation
        ring)

        Die Filtrierung nach grad des Maximalen Ideals, genannt
        $\mathfrak{m}$-adische Fitration, ist die Filtrierung
        $\mathfrak{m}^k=\{f\in \C[[x]]|v(f)\geq k\}$

        und es gilt $gr_\mathfrak{m}(\C[[x]])=\C[x]$
      \item $\C\{x\}\subset \C[[x]]$ ist ein Untering der Potenzreihen, wobei
        der Konvergenzradius echt positiv ist.

        ist ähnlich zu $\C[[x]]$
    \end{enumerate}
  \end{lem}
\end{comment}

\section{Direkte Summe}
\begin{defn}[Direkte Summe] \cite[4.5.1]{stacks-project}
  Seien $x,y\in \Ob(\cC)$, eine \emph{Direkte Summe} oder das \emph{coprodukt}
  von $x$ und $y$ ist ein Objekt $x\oplus y\in \Ob(\cC)$ zusammen mit
  Morphismen $i\in\Mor_\cC(x,x\otimes y)$ und $j\in\Mor_\cC(y,x\otimes y)$ so
  dass die folgende universelle Eigenschaft gilt: für jedes $w\in Ob(\cC)$ mit
  Morphismen $\alpha\in\Mor_\cC(x,w)$ und $\beta\in\Mor_\cC(y,w)$ existiert ein
  eindeutiges $\gamma\in\Mor_\cC(x\otimes y,w)$ so dass das Diagram
  \begin{center}
    \begin{tikzpicture} [scale=3.3, descr/.style={fill=white,inner sep=2.5pt} ]
      \matrix (m) [
        matrix of math nodes
        , row sep=2em
        , column sep=3em
        %, text height=3em
        %, text depth=0.25em
      ]
      {
        & y          & & &   \\
        x & x\otimes y & &   \\
          &            & & w \\
      };
      %TODO: Pfeile
      \path[->,font=\scriptsize,>=angle 90]
      (m-1-2) edge node[left]{$j$} (m-2-2)
      (m-2-1) edge node[above]{$i$} (m-2-2)
      (m-1-2) edge node[right]{$\beta$} (m-3-4)
      (m-2-1) edge node[below]{$\alpha$} (m-3-4)
      ;
      \path[->,font=\scriptsize,>=angle 90,dashed]
      (m-2-2) edge node[above]{$\gamma$} (m-3-4)
      ;
    \end{tikzpicture}
  \end{center}
  kommutiert.
\end{defn}


\section{Weyl-Algebra und der Ring $\cD$} %TODO: sabah-cimpa90.pdf  seite 3
Ich werde hier die Weyl Algebra, wie in
\cite[Chapter~1]{sabbah_cimpa90}, in einer Veränderlichen einführen.
Sei $\frac{\partial}{\partial x}=\partial_x$ der Ableitungsoperator nach $x$
und sei $f \in\C[x]$ (bzw. $\C\{x\}$ bzw. $\C\llbracket x\rrbracket$).
Man hat die folgende Kommutations-Relation zwischen dem
\emph{Ableitungsoperator}
und dem \emph{Multiplikations Operator} $f$:
% TODO; ist das der Kommutator??
\begin{equation}\label{eq:weyl_relation}
  [\frac{\partial}{\partial x},f]=\frac{\partial f}{\partial x}
\end{equation}
wobei die Rechte Seite die Multiplikation mit $\frac{\partial f}{\partial x}$
darstellt. Dies bedeutet, für alle $g\in\C[x]$ hat man:
\[
  [\frac{\partial}{\partial x},f]\cdot g
  =\frac{\partial fg}{\partial x} - f\frac{\partial g}{\partial x}
  =\frac{\partial f}{\partial x} \cdot g
\]
\begin{defn}[Weyl Algebra]
  Definiere nun die Weyl Algebra $A_1(\C)$ (bzw. die Algebra $\cD$ von
  linearen Operatoren mit Koeffizienten in $\C\{x\}$ bzw. die Algebra
  $\hat{\cD}$ (Koeffizienten in $\C\llbracket x\rrbracket$)) als die
  Quotientenalgebra der freien Algebra, welche von dem Koeffizientenring
  zusammen mit dem Element $\partial_x$, erzeugt wird, Modulo der Relation
  \eqref{eq:weyl_relation}.
\end{defn}
Wir werden die Notation $A_1(\C):=\C[x]<\partial_x>$ (bzw.
$\cD:=\C\{x\}<\partial_x>$ bzw. 
$\hat{\cD}:=\C\llbracket x\rrbracket<\partial_x>$) verwenden.

%TODO: vlt umsortieren
\begin{lem} %TODO: vervollständigen
  Sei $A$ einder der 3 soeben eingeführten Objekten, die Addition 
  \[
    +:A\times A \rightarrow A
  \]
  und die Multiplikation
  \[
    \cdot:A\times A \rightarrow A
  \]
  definieren auf $A$ eine Ringstruktur $(A,+,\cdot)$.
\end{lem}
\begin{proof}
  \cite[Kapittel 2 Section 1]{ZulaBarbara}
\end{proof}

\begin{rem}
  $A_1(\C),~\cD$ und $\hat\cD$ sind nicht kommutative Algebren.
\end{rem}

% ist das nichtkommutativ??
\begin{defn}[Kommutator]%zula seite 15
  Sei $R$ ein Ring. Für $a,b\in R$ wird
  \[[a,b]=a\cdot b-b\cdot a\]
  der \emph{Kommutator von a und b} genannt.
\end{defn}
% komutativ, dann immer kommutator gleich 0

\begin{prop} % geklaut aus Zula Barbara
  \begin{enumerate}
    \item Es gilt
      \[[ \partial_x,x] = \partial_xx-x\partial_x=1 \]
    \item Sei $f\in \C[x]$, so gilt:
      \[ [\partial_x,f] = \frac{\partial,f}{\partial x} \,. \]
      Denn für $g\in \C[x]$ ist
      \[
        [\partial_x,f]\cdot g=\partial_x(fg)-f\partial_xg=
        (\partial_xf)g+\underset{=0}{\underbrace{ 
            f(\partial_xg)-f(\partial_xg)}}=
        (\partial_xf)g
      \]
    \item Es gelten die Formeln\\
    \begin{align*}
      [\partial_x,x^k]   &= kx^{k-1}\\
      [\partial_x^j,x]   &= j\partial_x^{j-1}\\
      [\partial_x^j,x^k] &= \sum_{i\geq1}\frac{k(k-1)\cdots(k-i+1)
        \cdot j(j-1)\cdots(j-i+1)}{i!}x^{k-i}\partial_x^{j-i} \\
    \end{align*}
  \end{enumerate}
\end{prop}
\begin{proof}
  \cite{ZulaBarbara}
\end{proof}

\begin{prop} \label{prop:weyl_eindeutige_schreibung}
  Jedes Element in $A_1(\C)$ (bzw. $\cD$ oder $\hat{\cD}$) kann auf eindeutige
  weiße als $P=\sum_{i=0}^na_i(x)\partial_x^i$, mit $a_i(x)\in A_1(\C)$ (bzw.
  $\cD$ oder $\hat{\cD}$), geschrieben werden. 
\end{prop}
\begin{proof}
  \cite[Proposition 1.2.3]{sabbah_cimpa90}
  \begin{comment}
    ein teil des Beweises ist "left as an exersice"
  \end{comment}
\end{proof}

%TODO: Beispiele??

%TODO: definition Filtrierung

\begin{defn}
  Sei $P=\sum_{i=0}^na_i(x)\partial_x^i$ gegeben, so definiere 
  \[
    \deg P:=\max\{i|a_i\neq 0\}
  \]
  In natürlicher Weise erhält man $F_N\cD:=\{P\in\cD|\deg P\leq N\}$ sowie die
  entsprechende aufsteigende Filtrierung
  \[
    \cdots\subset F_{-1}\cD\subset F_{0}\cD\subset
    F_{1}\cD\subset\cdots\subset\cD
  \]
  und erhalte $gr_k^F\cD\underset{\mbox{def}}{=}F_N\cD\slash F_{N-1}\cD
  =\{P\in\cD|\deg P=N\}\cong\C\{x\}$.
\end{defn}

\begin{proof}
  Sei $P\in F_N\cD$ so betrachte den Isomorphismus:
  \[
    F_N\cD\slash F_{N-1}\cD\rightarrow \C\{x\}; [P]=P+F_{N-1}\cD\mapsto a_n(x)
  \]
\end{proof}

\begin{prop}
  Es gilt:
  \begin{center}
    \begin{tikzpicture} [descr/.style={fill=white,inner sep=2.5pt}]
    \matrix (m) [
      matrix of math nodes,
      row sep=1em,
      %column sep=-0.7em,
      text height=1.5ex,
      text depth=0.25ex]
    {
      gr^F\cD &
      := &
      \bigoplus_{\N\in\Z}gr_N^F\cD &
      = &
      \bigoplus_{\N\in\N_0}gr_N^F\cD &
      \cong &
      \bigoplus_{\N\in\N_0}\C\{x\} &
      \cong &
      \C\{x\}[\xi] &
      = &
      \bigoplus_{\N\in\N_0}\C\{x\}\cdot \xi^N \\
    };
      \path[solid]
      (m-1-1) edge [bend right=15] node[descr]{$\cong$}
        node[above]{$\mbox{isomorph als grad. Ringe}$} (m-1-11);
    \end{tikzpicture}
  \end{center}
\end{prop}
\begin{proof} TODO
  \begin{comment}
    Treffen?
  \end{comment}
\end{proof}

\begin{comment}
  \subsection{Weyl Algebra als Graduierter Ring}
  % Treffen 2
  Sei $A$ nun einer der drei Koeffizienten Ringe, welche zuvor behandelt
  wurden.  Der Ring $A<\partial_x>$ kommt zusammen mit einer aufsteigenden
  Filtrierung, welche wir mit $F(A<\partial_x)$ bezeichen werden.  Sei $P$ ein
  bzgl. \ref{prop:weyl_eindeutige_schreibung} minimal geschriebener Operator,
  so ist $P$ in $F_k$ falls der maximale Grad von $\partial_x$ in $P$ kleiner
  oder gleich $k$. So definiere den Grad $deg P$ von $P$ als die Eindeutige
  ganze Zahl $k$ mit $P\in F_kA<\partial_x>\slash F_{k-1}<\partial_x>$

  Unabhängigkeit von Schreibung wird in Sabbah Script behauptet
\end{comment}

\section{Struktur von Links-Idealen auf $\cD$}

\section{Lokalisierung eines $\C\{x\}$-Moduls}

\begin{defn}
  Sei $M$ ein $\C\{x\}$-Modul und $K=\C\{x\}[x^{-1}]$, dann ist die
  Lokalisierung
  \[ M[x^{-1}]:=M\otimes_{\C\{x\}}K \,. \]
\end{defn}

\section{Lokalisierung eines holonomen $\cD$-Moduls}

% vim: set ft=tex :

%\newpage

\chapter{Der Meromorpher Zusammenhang}
Quelle ist \cite{sabbah_cimpa90}
\section{Definition}~

\begin{defn}[Meromorpher Zusammenhang]
  Ein (\emph{Keim eines}) \emph{Meromorpher Zusammenhang} (an $x=0$)
  $(\cM_K,\partial)$ besteht aus folgenden Daten:
  \begin{itemize}
    \item $\cM_K$, ein endlich dimensionaler $K$-Vr
    \item einer $\C$-linearen Abbildung $\partial:\cM_K\rightarrow \cM_K$, genannt
      \emph{Derivation}, welche für alle $f\in K$ und $u\in \cM_K$ die
      \emph{Leibnitzregel}
      \begin{equation}\label{eq:Leibnitzregel}
        \partial(fu)=f'u+f\partial u
      \end{equation}
      erfüllen soll.
  \end{itemize}
\end{defn}

\begin{bem}
  Später wird man auf die angabe von $\partial$ verichten und einfach $\cM$ als
  den Meromorphen Zusammenhang bezeichnen.
\end{bem}

\section{Eigenschaften}
Hier nun einige Eigenschaften Meromorpher Zusammenhänge.

\begin{lem}
  %TODO: hier $\cM ??
  Sei $(\cM,\partial)$ ein gegebener Meromorpher Zusammenhang, und $\phi$ ein
  Basisisomorphismus von $K^r$ nach $\cM$, also in der Situation
  \begin{center}
    \begin{tikzpicture} [scale=3.3, descr/.style={fill=white,inner sep=2.5pt} ]
    \matrix (m) [
      matrix of math nodes
      , row sep=3em
      , column sep=3em
      %, text height=3em
      %, text depth=0.25em
      ]
    {
      \cM & \cM \\
      K^r & K^r \\
    };
      \path[->,font=\scriptsize,>=angle 90]
      (m-1-1) edge node[above]{$\partial$} (m-1-2)
      (m-2-1) edge node[above]{$\phi^{-1}\partial\phi$} (m-2-2)
      ;
      %TODO: make this harpoon arrows
      \path[->,font=\scriptsize,>=angle 90]
      (m-2-1) edge node[descr]{$\cong$} node[right]{$\phi$} (m-1-1)
      (m-2-2) edge node[descr]{$\cong$} node[left]{$\phi$} (m-1-2)
      ;

      \path[>=stealth,|->]
      ;
    \end{tikzpicture}
  \end{center}
  gilt: $(K^r,\phi^{-1}\partial\phi)$ ist ebenfalls ein Meromorpher Zusammenhang.
\end{lem}
\begin{proof}
  TODO, (3. Treffen)
\end{proof}

Sind $\partial_1$ und $\partial_2$ zwei Meromorphe Zusammenhänge auf $\cM_K\cong
K^r$. So betrachte $\partial_1-\partial_2:\cM\rightarrow\cM$ für alle $f\in K$ 
und $u\in \cM_K$ :
\begin{align*}
  (\partial_1-\partial_2)(fu) &= \partial_1(fu)-\partial_2(fu)\\
  &= f'u+f\partial_1u-f'u-f\partial_2u\\
  &= f\cdot(\partial_1-\partial_2)(u)\\
\end{align*}
\begin{lem}
  Da $\partial_1-\partial_2$ $\C$-linear und, wie eben gezeigt,
  $(\partial_1-\partial_2)(fu)=f\cdot(\partial_1-\partial_2)(u)$ allgemein
  gilt: Die differenz zweier Meromorpher Zusammenhäge ist $K$-linear.
  %Differenz zweier Meromorpher Zshg. ist K-linear
\end{lem}
Insbesondere ist $\frac{d}{dz}-\partial:K^r\rightarrow K^r$ $K$-linear, also es
existiert eine Matrix $A\in M(r\times r,K)$ mit $\frac{d}{dz}-\partial=A$, also
ist $\partial=\frac{d}{dz}-A$.

%TODO: beobachtung...

%TODO: differenz ist linear

\begin{defn}[Transformationsformel]
  In der Situation

  \begin{center}
    \begin{tikzpicture} [scale=3.3, descr/.style={fill=white,inner sep=2.5pt} ]
    \matrix (m) [
      matrix of math nodes
      , row sep=3em
      , column sep=3em
      %, text height=3em
      %, text depth=0.25em
      ]
    {
      K^r & & & K^r \\
       & M & M & \\
      K^r & & & K^r \\
    };
      \path[->,font=\scriptsize,>=angle 90]
      (m-1-1) edge node[descr]{$\cong$} node[above]{$\phi$} (m-2-2)
      (m-3-1) edge node[descr]{$\cong$} node[above]{$\psi$} (m-2-2)
      (m-1-4) edge node[descr]{$\cong$} node[above]{$\phi$} (m-2-3)
      (m-3-4) edge node[descr]{$\cong$} node[above]{$\psi$} (m-2-3)

      (m-2-2) edge node[above]{$\partial$} (m-2-3)

      (m-1-1) edge node[above]{$\frac{d}{dz}+A$} (m-1-4)
      (m-3-1) edge node[above]{$\frac{d}{dz}+B$} (m-3-4)

      (m-3-1) edge node[descr]{$\cong$} node[right]{$T$} (m-1-1)
      (m-3-4) edge node[descr]{$\cong$} node[left]{$T$} (m-1-4)
      ;

      \path[>=stealth,|->]
      ;
    \end{tikzpicture}
  \end{center}
  mit $\phi,\psi$ und $T$ $K$-Linear und $\partial,(\frac{d}{dz}+A)$ und
  $(\frac{d}{dz}+B)$ $\C$-Linear, gilt:\\
  Der Merom. Zush. $\frac{d}{dz}+A$ auf $K^r$ wird durch Basiswechsel $T\in
  GL(r,K)$ zu
  \[
    \frac{d}{dz}+(T^{-1}\cdot T'+T^{-1}AT) = \frac{d}{dz}+B
  \]
\end{defn}
\begin{defn}
  $A\sim B$ differenziell Äquivalent $:\Leftrightarrow$ $\exists T\in GL(r,K)$
  mit $B=T^{-1}\cdot T'+T^{-1}AT$
\end{defn}

\begin{comment}
  $1=TT^{-1}$ $\rightsquigarrow$ $T'T^{-1}+T(T^{-1})'=0$\\
  $1=T^{-1}T$ $\rightsquigarrow$ $(T^{-1})'T+T^{-1}T'=0$
\end{comment}

\section{Formale Meromorphe Zusammenhänge}

\begin{defn}[Formaler Meromorpher Zusammenhang]
  Ein \emph{Formaler Meromorpher zusammenhang} $(\cM_K,\partial)$ besteht aus 
  folgenden Daten:
  \begin{itemize}
    \item $\cM_K$, ein endlich dimensionaler $\hat K$-Vr
    \item eine \emph{Derivation} $\partial$, für die die \emph{Leibnitzregel} 
      \eqref{eq:Leibnitzregel}, für alle $f\in \hat K$ und $u\in \cM_{\hat K}$,
      erfüllt sein soll.
  \end{itemize}
\end{defn}

\begin{comment}
  Oder einfach ein Meromorpher Zshg. über anderes $K$ also $\hat K$
\end{comment}

\section{Elementare Meromorphe Zusammenhänge}

%TODO: auch nicht formal
\begin{defn}[Elementarer formaler Zusammenhang]
  Zu einem gegebenen $\rho\in u\C\llbracket u\rrbracket$, $\phi\in\C((u))$ und
  einem endlich dimensionalen $\C((u))$-Vektorraum $R$ mit regulärem 
  Zusammenhang $\nabla$, definieren wir den assoziierten Elementaren endlich
  dimensionalen $\C((t))$-Vektorraum mit Zusammenhang, durch:
  \[
    El(\rho,\phi,R)=\rho_+(\sE\otimes R)
  \]
\end{defn}

%TODO: weitere Eigenschaften

% vim: set ft=tex :

%\newpage

\chapter{Weiterführende Aussagen}
\begin{titlepage}
  \thispagestyle{empty}
  \newcommand{\Rule}{\rule{\textwidth}{1mm}}
  \begin{center}\sffamily\bfseries
    \LARGE Bachelorarbeit
    \vfill
    \Rule\vspace{5mm}
    \Huge
    mein thema
    \vspace{1mm}\Rule
    \vfill
    \normalfont\sffamily\large vorgelegt von\par
    \bfseries\LARGE Maximilian Huber
    \vfill
    \normalfont\sffamily\large am\\
    \bfseries\Large Institut für Mathematik\\
    \normalfont\sffamily\large der\\
    \bfseries\Large Universität Augsburg
    \vfill
    \normalfont\sffamily\large betreut durch \\
    \bfseries\Large Prof. Dr. Marco Hien \par
    \vfill
    \normalfont\sffamily\large abgegeben am \\
    \bfseries\Large noch nicht\\
  \end{center}
\begin{comment}
  \begin{center}
    stand: \today
  \end{center}
\end{comment}
\end{titlepage}
% vim: set ft=tex :

%\newpage

\part{Beispiele}
\section{Meromorpher zusammenhang der Formal zuerfällt aber nicht Konvergent}
\begin{comment}
  Quellen??
\end{comment}

%%%%%%%%%%%%%%%%%%%%%%%%%%%%%%%%%%%%%%%%%%%%%%%%%%%%%%%%%%%%%%%%%%%%%
\appendix
\addcontentsline{toc}{chapter}{Anhang}

\input{lemma2-4/anhang.tex}

\nocite{*}
%\bibliographystyle{dinat} 
\bibliographystyle{plain}
\bibliography{main}

\end{document}
