%!TeX root = main.tex
\chapter{Levelt-\!Turrittin-\!Theorem}
\begin{comment}
Quellen:\\
sabbah\_cimpa90 seite 28 / 30

\begin{thm}[Levelt-Turittin]
Es ex. endliche Körper $\hat{L}|\hat{K}$ mit $\hat{L}=\C((u))$ mit
$\hat{K}\hookrightarrow\hat{L},x\mapsto u^p$ so dass:
\[
\hat{M}\otimes_{\hat{K}}\hat{L}=\bigoplus_{i=1}^r\hat{M}_i
\]
mit $#\slopes(\hat{M}_i)=1 \forall i$ bzw. genauer
$\hat{M}_i=\xi^{\phi_i}\otimes R$
\end{thm}

\begin{thm}[Levelt-Turrittin-Malgrange]
$\exist \hat L|\hat K$ mit $\hat M_i\otimes _{\hat K}\hat L =
\bigoplus_{j=1}^s \hat N_j$
mit 
\[
\hat N_i=\xi^{\phi_j}\otimes R
\]
und 
\begin{itemize}
\item $\dim_L\xi^{\phi_j}=1$, $\phi_j\in \C[u^{-1}]\cdot u^{-1}$
\item R regulär singulär, also mit $\slopes=\{0\}$ % oder ={k} ??
\end{itemize}
\end{thm}
\end{comment}

Ab hier werden wir nur noch formale Meromorphe Zusammenhänge betrachten. 
%Alle bisher getroffenen Aussagen gelten für diese aber analog.

\begin{comment}
% sabbah\_cimpba90 Seite 30
Sei $M_{\hat{K}}=\cD_{\hat{K}}/\cD_{\hat{K}}\cdot P$ und nehme an, dass $N(P)$
zumindes 2 nichttriviale Steigungen hat. Spalte $N(P)=N_1\dot\cup N_2$ in 2
Teile. Dann gilt:

\begin{lem}
Es existiert eine Aufteilung $P=P_1P_2$ mit:
\begin{itemize}
\item $N(P_1)\subset N_1$ und $N(P_2)\subset N_2$
\item A ist eine kante von ...
\end{itemize}
\end{lem}
\end{comment}

\section{Klassische Definition}
\begin{comment}
\cite[Page 34]{sabbah_cimpa90}
Sei $\cM_{\hat K}$ ein formaler Meromorpher Zusammenhang. Man definiert
$\pi^*\cM_{\hat K}$ als den Vektor Raum über $\hat L:\pi^*\cM_{\hat K}=\hat
L\otimes_{\hat K}\cM_{\hat K}$. Dann definiert man die Wirkung von
$\partial_t$ durch: $t\partial_t\cdot(1\otimes
m)=q(1\otimes(x\partial_x\otimes m))$ und damit
\[
t\partial_t\cdot(\phi\otimes m)=q(\phi\otimes(x\partial_x\cdot
m))+((t\frac{\partial\phi}{\partial t})\otimes m) \,.
\]
\end{comment}
\begin{thm}
\cite[Thm 5.4.7]{sabbah_cimpa90}
Sie $\cM_{\hat K}$ ein formaler Meromorpher Zusammenhang. So gibt es eine
ganze Zahl $q$ so dass der Zusammenhang $\pi^*\cM_{\hat K}=\cM_{\hat L}$
isomorph zu einer direkten Summe von elementaren Meromorphen Zusammenhänge
ist.
\end{thm}

\begin{exmp}
Sei hier $P=\frac{1}{4}u^4\partial_u^2-\frac{1}{2}u^3\partial_u-1$, wie in
Beispiel \ref{exmp:pull-back}.
Wir wollen $\cD/\cD\cdot P$ mittels des Levelt-Turrittin-Theorems Zerlegen.
%TODO: all
\end{exmp}

\section{Sabbah's Refined version}

\begin{prop}
\cite[Prop 3.1]{sabbah_Fourier-local}
Jeder irreduzible endlich dimensionale formale Meromorphe Zusammenhang
$\cM_{\hat K}$ ist isomorph zu $\rho_+(\sE^\phi\otimes L)$, wobei $\phi\in
u^{-1}\C[u^-1]$, $\rho:u\mapsto t=u^p$ mit grad $p\geq1$ minimal bzgl. $\phi$
(siehe \cite[Rem 2.8]{sabbah_Fourier-local}), und $L$ ist ein Rang $1$
$\Cful$-Vektor Raum mit regulärem Zusammenhang.
\end{prop}
\begin{proof}
\cite[Prop 3.1]{sabbah_Fourier-local}
\end{proof}

\begin{thm}[Refined Turrittin-Levelt]
\cite[Cor 3.3]{sabbah_Fourier-local}
Jeder endlich dimensionale Meromorphe Zusammenhang $\cM_{\hat K}$ kann in
eindutiger weiße geschrieben werden als direkte Summe $\bigoplus
El(\rho,\phi,R)\underset{\mbox{def}}{=}\rho_+(\sE^\phi)\otimes R$, so dass
jedes $\rho_+\sE^\phi$ irreduzibel ist und keine zwei $\rho_+\sE^\phi$ isomorph
sind.
\end{thm}
\begin{proof}
\cite[Cor 3.3]{sabbah_Fourier-local}
\end{proof}

% vim: set ft=tex :
