%!TeX root = main.tex
\chapter{Levelt-\!Turrittin-\!Theorem}
\section{Elementare meromorphe Zusammenhänge}
%\section{Elementare formale meromorphe Zusammenhänge}
\begin{defn}
Ein \emph{elementarer regulärer meromorpher Zusammenhang} ist ein Zusammenhang
$\cM$, welcher isomorph zu $\cD_{\hat K}/\cD_{\hat K}\cdot
(x\partial_x-\alpha)^p$, mit passendem $\alpha$ und $p$, ist.
\end{defn}
\begin{bem}
Es ist leicht zu sehen, dass jeder elementare reguläre meromorphe Zusammenhang
tatsächlich auch regulär ist.
\end{bem}

\begin{lem}
Es existiert eine Basis von $\cM_{\hat K}$ über $\hat K$ mit der Eigenschaft,
dass die Matrix, die $x\partial_x$ beschreibt, nur Einträge in $\Cfx$ hat.
\end{lem}
\begin{comment}
\cite[Lem 5.2.1.]{sabbah_cimpa90}
\end{comment}
\begin{proof}
Wähle einen zyklischen Vektor $m\in\cM_{\hat K}$ % TODO: richtiger Raum?
 und betrachte die Basis $m,\partial_x m,\dots,\partial_x^{d-1}m$ (siehe Lemma
\ref{lem:Zyklischer-Vektor}).
Schreibe $\partial_x^dm=\sum_{i=0}^{d-1}(-b_i(x))\partial_x^im$ in
Basisdarstellung mit Koeffizienten $b_i\in\hat K$.
Also erfüllt $m$ die Gleichung
$\partial_x^dm+\sum_{i=0}^{d-1}b_i(x)\partial_x^im=0$.\\
\begin{comment} TODO: bis hier schon klar \end{comment}
Tatsächlich kann man $b_i(x)=x^ib_i'(x)$ mit $b_i'\in \Cfx$ schreiben (wegen
Regularität).\\
Dies impliziert, dass $m,x\partial_xm,\dots,(x\partial_x)^{d-1}m$ ebenfalls
eine Basis von $\cM_{\hat K}$ ist.\\
Die Matrix von $x\partial_x$ zu dieser neuen Basis hat nur Einträge in $\Cfx$.
\end{proof}
\begin{lem}
Es existiert sogar eine Basis von $\cM_{\hat K}$ über $\hat K$ so dass die
Matrix zu $x\partial_x$ konstant ist.
\end{lem}
\begin{proof}
Siehe \cite[Thm 5.2.2]{sabbah_cimpa90}.
\end{proof}

\begin{thm} \label{thm:regulaerInDirSumme}
Ein regulärer formaler Zusammenhang $\cM_{\hat K}$ ist isomorph zu einer
direkten Summe von elementaren regulären meromorphen Zusammenhängen.
\end{thm}
\begin{proof}[Beweisskizze]
Man wählt eine Basis von $\cM_{\hat K}$, in der die Matrix zu $x\partial_x$
konstant ist. Diese Matrix kann in Jordan Normalform gebracht werden und damit
erhält man das Ergebnis.
Ausgeführt wurde das in \cite[Cor. 5.2.6]{sabbah_cimpa90}.
\end{proof}

\begin{comment}
einführen als Bausteine oder kleinste meromorphe Zusammenhänge
\end{comment}

Durch twisten der elementaren regulären meromorphen Zusammenhänge erhält man
wie folgt die elementaren meromorphen Zusammenhänge.
\begin{defn} \label{defn:elemMerZsh}
Ein \emph{elementarer meromorpher Zusammenhang} ist ein Zusammenhang $\cM$, für
den es $\psi \in \Cfxl$, $\alpha\in\C$ und $p\in \N$ gibt, so dass
\[
\cM\cong \sE^{\psi}\otimes R_{\alpha,p}\,,
\]
mit $R_{\alpha,p}:=\cD/\cD(x\partial_x-\alpha)^p$, also ein elementarer
regulärer meromorpher Zusammenhang, ist.
\end{defn}

\begin{lem} In der Situation von Definition \ref{defn:elemMerZsh} gilt
$\sE^{\psi}\otimes R_{\alpha,p}\cong
\cD/\cD\cdot(x\partial_x-(\alpha+x\frac{\partial \psi}{\partial x}))^p$.
\end{lem}
\begin{proof} Denn
\begin{align*}
\sE^{\psi}\otimes R_{\alpha,p}&=\sE^{\psi}\otimes\cD/\cD(x\partial_x-\alpha)^p
\\&\!\!\overset{\ref{lem:twistRechenregel}}{=}
  \cD/\cD(x(\partial_x-\frac{\partial \psi}{\partial x})-\alpha)^p
\\&=\cD/\cD(x\partial_x-(\alpha+x\frac{\partial \psi}{\partial x}))^p \,.
\end{align*}
\end{proof}

\begin{comment}
\section{Definition in \cite{sabbah_Fourier-local}}
%TODO: auch nicht formal
\begin{defn}[Elementarer formaler Zusammenhang]
\cite[Def 2.1]{sabbah_Fourier-local}
Zu einem gegebenen $\rho\in t\C\llbracket t\rrbracket$, $\phi\in \hat L \bydef
\C(\!(t)\!)$ und einem endlich dimensionalen $\hat L$-Vektorraum $R$ mit
regulärem Zusammenhang $\nabla$, definieren wir den assoziierten elementaren
endlich dimensionalen $\hat K$-Vektorraum mit Zusammenhang, durch:
\[
El(\rho,\phi,R)=\rho_+(\sE^\phi\otimes R)
\]
\end{defn}
\cite[nach Def 2.1]{sabbah_Fourier-local}
Bis auf Isomorphismus hängt $El(\rho,\phi,R)$ nur von $\phi\mod\Cft$ ab.
\begin{lem}
\cite[Lem 2.2]{sabbah_Fourier-local}
\end{lem}
\begin{lem} \cite[Lem 2.6.]{sabbah_Fourier-local}
Es gilt $El([t\mapsto t^p],\phi,R)\cong El([t\mapsto t^p],\psi,S)$ genau dann,
wenn
\begin{itemize}
\item es ein $\zeta$ gibt, mit $\zeta^p=1$ und
$\psi\circ\mu_\zeta\equiv\phi\mod\Cft$
\item und $S\cong R$ als $\hat L$-Vektorräume mit Zusammenhang.
\end{itemize}
\end{lem}
\begin{proof}
Siehe \cite[Lem 2.6.]{sabbah_Fourier-local}
\end{proof}
%
\begin{prop} \cite[Prop 3.1]{sabbah_Fourier-local}
Jeder irreduzible endlich dimensionale $\hat K$-Vektorraum $\cM$ mit
Zusammenhang ist isomorph zu $\rho_+(\sE^\phi\otimes L)$, wobei $\phi\in
t^{-1}\C[t^{-1}]$, $\rho:t\rightarrow t^p$ vom Grad $p\geq 1$ und ist minimal
unter $\phi$. (siehe \cite[Rem  2.8]{sabbah_Fourier-local}) und $L$ ist ein
Rang $1$ $\hat L$-Vektrorraum mit regulärem Zusammenhang.
\end{prop}
\begin{proof}
%TODO: verwendet hier schon das klassische Levelt-Turittin
Siehe \cite[Prop 3.1]{sabbah_Fourier-local}
\end{proof}
\end{comment}

\section{Die Filtrierung $\,^\ell V\cD_{\hat K}$ und das $\ell$-Symbol}
\begin{comment}
TODO: Problem mit vertikaler verschiebung? kompensieren??
\end{comment}
Dieser Abschnitt bezieht sicht auf \cite[Seite 25]{sabbah_cimpa90} und
beschreibt das $\ell$-Symbol, welches im Beweis des Levelt-Turittin Theorems
verwendung findet.\\
Sei $\Lambda=\frac{\lambda_0}{\lambda_1}\in \Q_{\geq 0}$ vollständig gekürtzt,
also mit $\lambda_0$ und $\lambda_1$ in $\N$ relativ prim. Definiere die
Linearform $\ell(s_0,s_1)=\lambda_0s_0+\lambda_1s_1$ in zwei Variablen, sei
$P\in\cD_{\hat K}$.  Falls $P=x^a\partial_x^b$ mit $a\in \Z$ und $b\in \N$,
setzen wir
\[
\ord_\ell(P):=\ell(b,b-a)
\]
und falls $P=\sum_{i=0}^d b_i(x)\partial_x^i$ mit $b_i\in\hat K$, setzen wir
\[
\ord_\ell(P):=\max_{\{i\mid a_i\neq 0\}} \ell(i,i-v(b_i))\,.
%Hier ist ein fehler im Sabbah script a_i <-> b_i
\]

\begin{bem}
Seien $P$ ein linearer Differenzialoperator und
$\ell(s_0,s_1)=\lambda_0s_0+\lambda_1s_1$ gegeben.
Betrachte die Geradenschar $g_a(x)=\frac{\lambda_0}{\lambda_1}+a$, dann gibt es
genau ein $a$, welches minimal unter der Eigenschaft, das Newton-Polygon zu $P$
zu berühren bzw. zu schneiden, ist.
Dieses so definierte $a$ entspricht $\ord_\ell(P)$.
\end{bem}

\begin{defn}[Die Filtrierung $\,^\ell V\cD_{\hat K}$]
Nun können wir die aufsteigende Filtration $\,^\ell V\cD_{\hat K}$, welche mit
$\Z$ indiziert ist, durch
\[
\,^\ell V_\lambda\cD_{\hat K}:=\{P\in\cD_{\hat K}\mid \ord_\ell(P)\leq \lambda\}
\]
definieren.
\end{defn}
\begin{bem}
Man hat $\ord_\ell(PQ)=\ord_\ell(P)+\ord_\ell(Q)$ und falls $\lambda_0\neq 0$,
hat man auch, das $\ord_\ell([P,Q])\leq \ord_\ell(P)+\ord_\ell(Q)-1$.
\end{bem}
\begin{defn}[$\ell$-Symbol]
Falls $\lambda_0\neq 0$, ist der gradierte Ring $gr^{\,^\ell V}\cD_{\hat
K}\bydef \bigoplus_{\lambda \in \Z}gr_\lambda^{\,^\ell V}\cD_{\hat K}$ ein
kommutativer Ring. Bezeichne die Klasse von $\partial_x$ in dem Ring durch
$\xi$, dann ist der Ring isomorph zu $\hat K[\xi]$.
%
Sei $P\in \cD_{\hat K}$, so ist $\sigma_\ell(P)$ definiert als die Klasse von
$P$ in $gr_{\ord_\ell(P)}^{\,^\ell V}\cD_{\hat K}$. $\sigma_\ell$ wird hierbei
als das $\ell$-Symbol bezeichnet.
\end{defn}
Zum Beispiel ist $\sigma_\ell(x^a\partial_x^b)=x^a\xi^b$.
\begin{bem}
Bei \cite{sabbah_cimpa90} wird der Buchstabe $L$ anstatt $\ell$ für
Linearformen verwendet, dieser ist hier aber bereits für $\Ckt$ reserviert.
Dementsprechend ist die Filtrierung dort als $\,^L V\cD_{\hat K}$ und das
$\ell$-Symbol als $L$-Symbol zu finden.
\end{bem}
\begin{bem}
Ist $P\in \cD_{\hat K}$ geschrieben als
$P=\sum_i\sum_j\alpha_{ij}x^j\partial_x^i$.
So erhält man $\sigma_\ell(P)$ durch die Setzung
\[
\sigma_\ell(P)=\sum_{\{(i,j)\mid\ell(i,i-j)=\ord_\ell(P)\}}\alpha_{ij}x^j\xi^i \,.
\]
\end{bem}
\begin{proof}
TODO
\end{proof}
\begin{comment}
Ich will die Linearform vermeiden und direkt die skalare Steigung verwenden
\end{comment}
\begin{defn}[Stützfunktion]
Die Funktion
\[
\omega_P:[0,\infty)\rightarrow\R, \omega_P(t):=\inf\{v-tu \mid (u.v) \in N(P)\}
\]
heißt Stützfunktion und wird in \cite{ZulaBarbara} als Alternative zu dieser
Filtrierung verwendet.
\end{defn}
\begin{bem}
Wenn $\ell(x_0,s_1)$ wie oben aus $\Lambda$ entstanden ist, so gilt
\[
\omega_P(\Lambda)=ord_\ell(P) \,.
\]
\end{bem}
\begin{comment}
TODO: ist $\ell$ Slope (gehört zu Slope) dann hat $\sigma_\ell(P)$ zumindest 2
Monome
\end{comment}

\section{Levelt-\!Turrittin-\!Theorem}
\begin{comment}
Das Levelt-Turrittin-Theorem ist ein Satz, der hilft, meromorphe Zusammenhänge
in ihre irreduziblen Komponenten zu zerlegen.
\end{comment}

\begin{comment}
\subsection{Klassische Version}
\end{comment}
\begin{thm}
\cite[Thm 5.4.7]{sabbah_cimpa90}
Sei $\cM_{\hat K}$ ein formaler meromorpher Zusammenhang, so gibt es eine
ganze Zahl $p$, so dass der Zusammenhang $\cM_{\hat L}:=\rho^+\cM_{\hat K}$,
mit $\rho:t\mapsto x:=t^p$, isomorph zu einer direkten Summe von formalen
elementaren meromorphen Zusammenhänge
ist.
\end{thm}
\begin{comment}
Der folgende Beweis stammt hauptsächlich aus \cite[Seite 35]{sabbah_cimpa90}.
\end{comment}
\begin{proof}[Beweisskizze]
Zum Beweis wird Induktion auf die lexicographisch geordnetem Paare
$(\dim_{\hat K}\cM_{\hat K},\kappa)$ angewendet. Wobei
$\kappa\in\N\cup\{\infty\}$ dem größtem Slope von $\cM_{\hat K}$, falls dieser
ganzzahlig ist, entspricht. Sonsts wird $\kappa=\infty$ gesetzt. In jedem
Induktionsschritt wird entweder die Dimension oder das $\kappa$ verringert.

\begin{comment}
TODO: Induktionsanfang und -schritt kennzeichnen
\end{comment}
Wir nehmen oBdA an, dass $\cM_{\hat K}$ genau einen Slope $\Lambda$ hat, sonst
Teile $\cM_{\hat K}$ mittels Satz \ref{thm:Split-after-slopes} in meromorphe
Zusammenhänge mit je einem Slope und wende jeweils die Induktion an.
Mit $\Lambda=:\frac{\lambda_0}{\lambda_1}$ (vollständig gekürtzt) definieren
wir die dem Slope entsprechende Linearform
$\ell(s_0,s_1):=\lambda_0s_0+\lambda_1s_1$.  
Wir nehmen oBdA auch an, dass $\ord_\ell(P)=0$. Dies geht nach Bemerkung
\ref{bem:NPverschieben}.
Da $\ell$ zu einem Slope von
$P$ gehört, besteht $\sigma_\ell(P)$ aus zumindest zwei Monomen.
Schreibe
\begin{align*}
\sigma_L(P)&=\sum_{\ell(i,i-j)=\ord_\ell(P)}\alpha_{ij}x^j\xi^i\\
  &=\sum_{\ell(i,i-j)=0}\alpha_{ij}x^j\xi^i
\end{align*}
und setze $\theta:=x^{\lambda_0+\lambda_1}xi^{\lambda_1}$ so erhalten wir
\[
\sigma_\ell(P) = \sum_{k\geq 0}\alpha_k\theta^k \,,
\]
wobei $\alpha_0\neq0$ ist.

\paragraph{Erster Fall: $\lambda_1=1$.} Das bedeutet, dass der Slope ganzzahlig
ist. Betrachte die Faktorisierung
\[
\sigma_\ell(P)=
  \epsilon\prod_{\beta\text{ Nullstelle}}(\theta-\beta)^{\gamma_\beta}\,.
\]
Wobei $\epsilon\in\C^\times$ eine Konstante ist.  Sei $\beta$  eine der
Nullstellen, so setze $\psi(x):=(\beta_0/\lambda_0)x^{-\lambda_0}$ und
betrachte $\cM_{\hat K}\otimes_{\hat K} \sE_{\hat K}^{\psi}$.
Sei $P$ ein Minimalpolynom von $\cM_{\hat K}$, dann ist nach Lemma
\ref{lem:twistRechenregel} ein Minimalpolynom von 
$\cM_{\hat K}\otimes_{\hat K} \sE_{\hat K}^{\psi}$ gegeben durch
\begin{align*}
P'(x,\partial_x)&=P(x,\partial_x-\frac{\partial \psi}{\partial x})
\\&=P(x,\partial_x+\frac{\beta}{x^{\lambda_0+1}})
\end{align*}
mit Koeffizienten in
$\Cfx$.
Des weiteren ist $\sigma_\ell(P')=\sum_{k\geq 0}\alpha_k(\theta+\beta_0)^k$.
Wir unterscheiden nun 2 Unterfälle:
\begin{enumerate}
\item \textbf{Die Determinanten Gleichung $\sigma_\ell(P)$ hat nur eine
Nullstelle.}
In diesem fall wurde die maximale Steigung echt verringert.
\begin{comment}TODO: Hier weiter \end{comment}
\item \textbf{Die Determinanten Gleichung $\sigma_\ell(P)$ hat mehrere
Nullstellen.}
In diesem fall hat $\cM_{\hat K}\otimes_{\hat K} \sE_{\hat K}^{\psi}$ mehr als
einen Slope und kann deshalb mit Satz \ref{thm:Split-after-slopes} in eine
direkte Summe von Meromorphen Zusammenhängen, mit echt niedrigerer Dimension,
zerlegt werden.
\begin{comment}TODO: Hier weiter \end{comment}
\end{enumerate}
In beiden Unterfällen muss danach das Twisten, nach Anwenden der Induktion,
durch ein tensorieren mit $\sE_{\hat K}^{-\psi}$ rückgängig gemacht werden.

\paragraph{Zweiter Fall: $\lambda_1\neq1$ (bzw. $\kappa = +\infty$).} 
In diesem Fall ist einzige Slope $\Lambda$ nicht ganzzahlig. Mache deshalb
einen pull-back mit $\lambda_1$. Sei $\rho:t\mapsto x:=t^{\lambda_1}$ und
erhalte $P'$ so dass $\rho^*\cM_{\hat K}=\cD_{\hat L}/\cD_{\hat L}\cdot P'$.
Nach Lemma \ref{lem:slope-pb-multiplikation} hat $P'$ den einen Slope
$\Lambda'=\Lambda\cdot\lambda_1=\lambda_0\in\N$.
\begin{comment}
Damit können wir nun die zugehörige Linearform $\ell':=\lambda_0s_0+s_1$
definieren. Es gilt dass
\[
\sigma_{\ell'}(P')=\dots
\]
ist, welches zumindest zwei unterschiedliche Nullstellen hat. Nun wendet man
den zweiten Unterfall des ersten Fall an.
\end{comment}
\end{proof} % ende von beweis zu levelt aus sabbah_cimpa90
\begin{bem}
Eine sehr detailierte Version dieses Beweises, ist beispielsweise in \cite[Thm
5.16]{DiplHedwig} zu finden.
\end{bem}

\begin{comment}
\subsection{Sabbah's Refined version}
\begin{prop}
\cite[Prop 3.1]{sabbah_Fourier-local}
Jeder irreduzible endlich dimensionale formale meromorphe Zusammenhang
$\cM_{\hat L}$ ist isomorph zu $\rho_+(\sE^\phi\otimes_{\hat K} S)$, wobei
$\phi\in x^{-1}\C[x^-1]$, $\rho:x\mapsto t=x^p$ mit grad $p\geq1$ minimal bzgl.
$\phi$ (siehe \cite[Rem 2.8]{sabbah_Fourier-local}), und $S$ ist ein Rang $1$
$\hat K$-Vektor Raum mit regulärem Zusammenhang.
\end{prop}
\begin{proof}
\cite[Prop 3.1]{sabbah_Fourier-local}
\end{proof}

\begin{thm}[Refined Turrittin-Levelt]
\cite[Cor 3.3]{sabbah_Fourier-local}
Jeder endlich dimensionale meromorphe Zusammenhang $\cM_{\hat K}$ kann in
eindutiger weiße geschrieben werden als direkte Summe $\bigoplus
El(\rho,\phi,R)\bydef\bigoplus\rho_+(\sE^\phi)\otimes R$, so dass
jedes $\rho_+\sE^\phi$ irreduzibel ist und keine zwei $\rho_+\sE^\phi$ isomorph
sind.
\end{thm}
\begin{proof}
\cite[Cor 3.3]{sabbah_Fourier-local}
\end{proof}
\end{comment}

% vim: set ft=tex :
