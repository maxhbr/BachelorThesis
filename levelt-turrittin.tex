%!TeX root = main.tex
\chapter{Levelt-\!Turrittin-\!Theorem}
Das Levelt-Turrittin-Theorem ist ein Satz, der hilft, Meromorphe Zusammenhänge
in ihre irreduziblen Komponenten zu zerlegen.

\section{Klassische Version}
\begin{thm}
\cite[Thm 5.4.7]{sabbah_cimpa90}
Sie $\cM_{\hat K}$ ein formaler Meromorpher Zusammenhang. So gibt es eine
ganze Zahl $p$ so dass der Zusammenhang $\cM_{\hat L}:=\rho^+\cM_{\hat K}$,
mit $\rho:t\mapsto x:=t^p$, isomorph zu einer direkten Summe von elementaren
Meromorphen Zusammenhänge
ist.
\end{thm}
Der folgende Beweis stammt aus \cite[Seite 35]{sabbah_cimpa90}.
\begin{proof}
Der Beweis geht, mittels induktion auf den Lexicographisch geordnetem Paar
$(\dim_{\hat K}\cM_{\hat K},\kappa)$ wobei $\kappa\in\N\cup\{\infty\}$ dem
größtem Slope von $\cM_{\hat K}$. Es wird $\kappa=\infty$ gesetzt, falls der
größte Slope nicht Ganzzahlig ist.

Wir nehmen oBdA an, dass $\cM_{\hat K}$ genau einen Slope $\Lambda$ hat, sonst
Teile $\cM_{\hat K}$ mittels Satz \ref{thm:Split-after-slopes} in Meromorphe
Zusammenhänge mit je einem Slope und wende jeweils die Induktion an.
Mit $\Lambda=:\frac{\lambda_0}{\lambda_1}$ (vollständig gekürtzt) Definieren
wir die dem Slope entsprechende Linearform
$L(s_0,s_1):=\lambda_0s_0+\lambda_1s_1$.  Wir nennen $\sigma_L(P)\in \hat
K[\xi]$ die \emph{Determinanten Gleichung} von $P$. Da $L$ zu einem Slope von
$P$ gehört, besteht $\sigma_L(P)$ aus zumindest zwei Monomen.
\begin{comment}
and is homogeneous of degree $\ord_L(P)=0$ because $P$ is chosen with
coefficients in $\Cfx$, one of them, being a unit.
\end{comment}
Schreibe
\begin{align*}
\sigma_L(P)&=\sum_{L(i,i-j)=\ord_L(P)}\alpha_{ij}x^j\xi^i\\
  &=\sum_{L(i,i-j)=0}\alpha_{ij}x^j\xi^i \,.
\end{align*}
Sei $\theta:=x^{\lambda_0+\lambda_1}}xi^{\lambda_1}$ so können wir
\[
\sigma_L(P) &= \sum_{k\geq 0}\alpha_k\theta^k
\]
schreiben, wobei $\alpha_0\neq0$ ist.

\paragraph{Erster Fall: $\lambda_1=1$.} Das bedeutet, dass der Slope ganzzahlig
ist. Betrachte die Faktorisierung
\[
\sigma_L(P)=\epsilon\prod_{\beta}(\theta-\beta)^{\gamma_\beta}\,.
\]
Wobei $\epsilon\in\C$ eine Konstante ist.  Sei $\beta_0$  eine der Nullstellen.
So setze $R(z):=(\beta_0/(\lambda_0+1))z^{\lambda_0+1}$ und betrachte
$\cM_{\hat K}\otimes \cF_{\hat K}^R$.
\begin{lem}
Falls $e$ ein zyklischer Vektor für $\cM_{\hat K}$ ist, so ist $e\otimes e(R)$
ein zyklischer Vektor für $\cM_{\hat K}\otimes \cF_{\hat K}^R$.
\end{lem}
\begin{proof}
TODO
\end{proof}
\begin{comment}
AB HIER VLT NICHT RICHTIG, nur versuch
\end{comment}
Falls $P(x,\partial_x)\cdot e=0$ gilt
\[
P\Big(x,\partial_x-\frac{\partial R(x^{-1})}{\partial x}\Big)
  \cdot e\otimes e(R)=0
\]
und hier haben wir 
\begin{align*}
\frac{\partial R(x^{-1})}{\partial x} 
  &=\frac{\partial(\frac{\beta_0}{\lambda_0+1}x^{-(\lambda_0+1)})}{\partial
  x}\\
  &=-\beta_0z^{-(\lambda_0+2)} \,.
\end{align*}
Schreibe $P'=P(x,\partial_x+\beta_0x^{-(\lambda_0+2)})$.
\begin{lem}
Es gilt, dass $P'$ Koeffizienten in $\Cfx$ hat.
\end{lem}
\begin{proof}
TODO
\end{proof}
Des weiteren ist $\sigma_L(P')=\sum_{k\geq 0}\alpha_k(\theta+\beta_0)^k$. Wir
unterscheiden nun 2 Fälle:
\begin{enumerate}
\item \textbf{Die Determinanten Gleichung $\sigma_L(P)$ hat nur eine
Nullstelle.}
\begin{comment}TODO: Hier weiter \end{comment}
\item \textbf{Die Determinanten Gleichung $\sigma_L(P)$ hat mehrere
Nullstellen.}
\begin{comment}TODO: Hier weiter \end{comment}
\end{enumerate}

\paragraph{Zweiter Fall: $\lambda_1\neq1$.} In diesem Fall ist einzige Slope
$\Lambda$ nicht ganzzahlig. Mache deshalb einen pull-back mit $\lambda_1$. Sei
$\rho:t\mapsto x:=t^{\lambda_1}$ und erhalte $P'$ so dass $\rho^*\cM_{\hat
K}=\cD_{\hat L}/\cD_{\hat L}\cdot P'$. Nach Lemma
\ref{lem:slope-pb-multiplikation} hat $P'$ den einen Slope
$\Lambda\cdot\lambda_1=\lambda_0$.
\begin{comment}TODO: Hier weiter \end{comment}

\end{proof} % ende von beweis zu levelt aus sabbah_cimpa90

\section{Sabbah's Refined version}
\begin{prop}
\cite[Prop 3.1]{sabbah_Fourier-local}
Jeder irreduzible endlich dimensionale formale Meromorphe Zusammenhang
$\cM_{\hat L}$ ist isomorph zu $\rho_+(\sE^\phi\otimes_{\hat K} S)$, wobei
$\phi\in x^{-1}\C[x^-1]$, $\rho:x\mapsto t=x^p$ mit grad $p\geq1$ minimal bzgl.
$\phi$ (siehe \cite[Rem 2.8]{sabbah_Fourier-local}), und $S$ ist ein Rang $1$
$\hat K$-Vektor Raum mit regulärem Zusammenhang.
\end{prop}
\begin{proof}
\cite[Prop 3.1]{sabbah_Fourier-local}
\end{proof}

\begin{thm}[Refined Turrittin-Levelt]
\cite[Cor 3.3]{sabbah_Fourier-local}
Jeder endlich dimensionale Meromorphe Zusammenhang $\cM_{\hat K}$ kann in
eindutiger weiße geschrieben werden als direkte Summe $\bigoplus
El(\rho,\phi,R)\bydef\bigoplus\rho_+(\sE^\phi)\otimes R$, so dass
jedes $\rho_+\sE^\phi$ irreduzibel ist und keine zwei $\rho_+\sE^\phi$ isomorph
sind.
\end{thm}
\begin{comment}
In welchem Raum ist $\cM$ ?? in $L$ oder in $K$
\end{comment}
\begin{proof}
\cite[Cor 3.3]{sabbah_Fourier-local}
\end{proof}

% vim: set ft=tex :
