\chapter{Numerische berechnung der Koeffizienten} \label{anh:Programm}
Hier nun ein Haskell Programm, dass in der Funktion \textbf{main} die
Koeffizienten von $v$ und $u$ für Abschnitt \ref{sec:konvergenzDerPotReihen} 
numerisch berechnet. Für die beispielhaften Berechnungen hier, wählen wir
$a=\frac{1}{8}$, dadurch gilt $u_{-2}=i$.
\lstinputlisting[language=HaskellUlisses]{../haskell/koeff.hs}

Ist der Code in einer Datei \textbf{/Pfad/zu/koeff.hs} gespeicher, so lässt er
sich in Unix-Artigen Systemen beispielsweise mit den folgenden Befehlen
compilieren und ausführen.
%\begin{lstlisting}[style=Bash]
%$ ghc --make -threaded /Pfad/zu/koeff.hs 
%$ /Pfad/zu/koeff +RTS -N3
%\end{lstlisting}
%\lstinputlisting[style=Bash]{../haskell/run.sh}
Durch das Ausführen berechnet das Programm die Koeffizienten von $v$ und $u$
bis zum Index $15$ sowie einzelne Werte an $20$, $30$, $40$, $50$, $100$ und
$150$ und produziert einen Ausgang, der wie folgt aussieht
\begin{lstlisting}[style=Bash]
n       | v_n   u_n
--------+-----------------------------------------------
-2      | 0.0 :+ 0.0    0.0 :+ 1.0
-1      | 0.5 :+ 0.0    (-1.5) :+ (-0.0)
0       | (-0.0) :+ 0.75    0.0 :+ (-0.75)
1       | 1.5 :+ 0.0    (-1.5) :+ (-0.0)
2       | 0.0 :+ (-3.9375)    (-0.0) :+ 3.9375
3       | (-13.5) :+ (-0.0)    13.5 :+ 0.0
4       | 0.0 :+ 59.34375    (-0.0) :+ (-59.34375)
5       | 324.0 :+ 0.0    (-324.0) :+ (-0.0)
6       | 0.0 :+ (-2122.98046875)    (-0.0) :+ 2122.98046875
7       | (-16213.5) :+ (-0.0)    16213.5 :+ 0.0
8       | 0.0 :+ 141115.447265625    (-0.0) :+ (-141115.447265625)
9       | 1376311.5 :+ 0.0    (-1376311.5) :+ (-0.0)
10      | 0.0 :+ (-1.4850124677246094e7)    (-0.0) :+ 1.4850124677246094e7
11      | (-1.75490226e8) :+ (-0.0)    1.75490226e8 :+ 0.0
12      | 0.0 :+ 2.2530628205925293e9    (-0.0) :+ (-2.2530628205925293e9)
13      | 3.1217145174e10 :+ 0.0    (-3.1217145174e10) :+ (-0.0)
14      | 0.0 :+ (-4.641652455250599e11)    (-0.0) :+ 4.641652455250599e11
15      | (-7.3709524476135e12) :+ (-0.0)    7.3709524476135e12 :+ 0.0
20      | 0.0 :+ 1.753906248830001e19    (-0.0) :+ (-1.753906248830001e19)
30      | 0.0 :+ (-2.7520294973343126e33)    (-0.0) :+ 2.7520294973343126e33
40      | 0.0 :+ 1.1055855646065139e49    (-0.0) :+ (-1.1055855646065139e49)
50      | 0.0 :+ (-5.0878905001062135e65)    (-0.0) :+ 5.0878905001062135e65
100     | 0.0 :+ 3.045728894141079e159    (-0.0) :+ (-3.045728894141079e159)
150     | 0.0 :+ (-2.7737283214890534e264)    (-0.0) :+ 2.7737283214890534e264
\end{lstlisting}
In Haskell ist das \textbf{:+} ein Infix-Konstruktor der Klasse
\textbf{Data.Complex}. So erzeugt ein Aufruf der Form \textbf{a :+ b} eine
Imaginärzahl, die $a+ib$ entspricht.

Übersetzt in unsere Zahlenschreibweise sieht das Ergebnis also wie folgt aus:

\begin{table}[htbp]
\begin{center} \scriptsize
\begin{tabular}{|r||r|r|}
\hline
n        & $v_n$                             & $u_n$
\\\hline\hline
-2       & $0$                               & $i$
\\-1     & $0,5$                             & $-1,5 $
\\0      & $0,75i$                          & $-0,75i$
\\1      & $1,5$                            & $-1,5$
\\2      & $-3,9375i$                         & $3,9375i$
\\3      & $-13,5$                            & $13,5$
\\4      & $59,34375i$                      & $-59,34375i$
\\5      & $324,0$                          & $-324,0$
\\6      & $-2122,98046875i$                  & $2122,98046875i$
\\7      & $-16213,5$                         & $16213,5$
\\8      & $141115,447265625i$              & $-141115,447265625i$
\\9      & $1376311,5$                      & $-1376311,5$
\\10     & $-1,4850124677246094\cdot10^7i$    & $1,4850124677246094\cdot10^7i$
\\11     & $-1,75490226\cdot10^8$             & $1,75490226\cdot10^8$
\\12     & $2,2530628205925293\cdot10^9i$   & $-2,2530628205925293\cdot10^9i$
\\13     & $3,1217145174\cdot10^{10}$       & $-3,1217145174\cdot10^{10}$
\\14     & $-4,641652455250599\cdot10^{11}i$ &
$4,641652455250599\cdot10^{11}i$
\\15     & $-7,3709524476135\cdot10^{12}$     & $7,3709524476135\cdot10^{12}$
\\\vdots & \vdots                            & \vdots
\\20     & $1.753906248830001\cdot10^{19}i$ & $-1.753906248830001\cdot10^{19}i$
\\\vdots & \vdots                            & \vdots
\\30     & $-2.7520294973343126\cdot10^{33}i$ &
$2.7520294973343126\cdot10^{33}i$
\\\vdots & \vdots                            & \vdots
\\40     & $1.1055855646065139\cdot10^{49}i$&
$-1.1055855646065139\cdot10^{49}i$
\\\vdots & \vdots                            & \vdots
\\50     & $-5.0878905001062135\cdot10^{65}i$ &
$5.0878905001062135\cdot10^{65}i$
\\\vdots & \vdots                            & \vdots
\\100    & $3.045728894141079\cdot10^{159}i$&
$-3.045728894141079\cdot10^{159}i$
\\\vdots & \vdots                            & \vdots
\\150    & $-2.7737283214890534\cdot10^{264}i$&
$2.7737283214890534\cdot10^{264}i$
\\\vdots & \vdots                            & \vdots
\\ \hline
\end{tabular} 
\caption{Numerisch berechnete Koeffizienten von $u(t)$ und $v(t)$ für $a=\frac{1}{8}$}
\label{tab:koeff_a=0.125}
\end{center} 
\end{table}

% vim: set ft=tex :
