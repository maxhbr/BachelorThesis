\chapter{Numerische berechnung der Koeffizienten}
Hier wird nun ein Haskell Programm, dass in der Funktion \textbf{main} die
Koeffizienten von $v$ und $u$ numerisch berechnet. Wir wählen $a=\frac{1}{8}$,
dadurch gilt $u_{-2}=i$.
\lstinputlisting[language=HaskellUlisses]{../haskell/programm.hs}

Ist der Code in einer Datei \textbf{koeff.hs} gespeicher, so lässt er sich
in Unix-Artigen Systemen beispielsweise mit den folgenden Befehlen compilieren
und ausführen.
\begin{lstlisting}[style=Bash]
$ ghc koeff.hs
$ ./koeff
\end{lstlisting}
Durch das Ausführen berechnet das Programm die Koeffizienten von $v$ und $u$
und produziert einen Ausgang, der wie folgt aussieht
\begin{lstlisting}[style=Bash]
n       | v_n   u_n
-------------------
-2      | 0.0 :+ 0.0    0.0 :+ 1.0
-1      | 0.5 :+ 0.0    (-1.5) :+ (-0.0)
0       | 0.0 :+ (-0.75)    (-0.0) :+ 0.75
1       | (-1.5) :+ (-0.0)    1.5 :+ 0.0
2       | 0.0 :+ 3.9375    (-0.0) :+ (-3.9375)
3       | 13.5 :+ 0.0    (-13.5) :+ (-0.0)
4       | 0.0 :+ (-59.34375)    (-0.0) :+ 59.34375
5       | (-324.0) :+ (-0.0)    324.0 :+ 0.0
6       | 0.0 :+ 2122.98046875    (-0.0) :+ (-2122.98046875)
7       | 16213.5 :+ 0.0    (-16213.5) :+ (-0.0)
8       | 0.0 :+ (-141115.447265625)    (-0.0) :+ 141115.447265625
9       | (-1376311.5) :+ (-0.0)    1376311.5 :+ 0.0
10      | 0.0 :+ 1.4850124677246094e7    (-0.0) :+ (-1.4850124677246094e7)
11      | 1.75490226e8 :+ 0.0    (-1.75490226e8) :+ (-0.0)
12      | 0.0 :+ (-2.2530628205925293e9)    (-0.0) :+ 2.2530628205925293e9
13      | (-3.1217145174e10) :+ (-0.0)    3.1217145174e10 :+ 0.0
14      | 0.0 :+ 4.641652455250599e11    (-0.0) :+ (-4.641652455250599e11)
15      | 7.3709524476135e12 :+ 0.0    (-7.3709524476135e12) :+ (-0.0)
\end{lstlisting}
übersetzt in unsere Zahlenschreibweise sieht das Ergebnis, wie folgt aus:
\begin{center}
\begin{tabular}{c || c | c}
n    & $v_n$                            & $u_n$
\\\hline\hline
  -2 & $0$                              & $i$
\\-1 & $0,5$                            & $-1,5 $
\\0  & $-0,75i$                         & $-0,75i$
\\1  & $-1,5$                           & $1,5$
\\2  & $3,9375i$                        & $-3,9375i$
\\3  & $13,5$                           & $-13,5$
\\4  & $-59,34375i$                     & $59,34375i$
\\5  & $-324,0$                         & $324,0$
\\6  & $2122,98046875i$                 & $-2122,98046875i$
\\7  & $16213,5$                        & $-16213,5$
\\8  & $-141115,447265625i$             & $141115,447265625i$
\\9  & $-1376311,5$                     & $1376311,5$
\\10 & $1,4850124677246094\cdot10^7i$   & $-1,4850124677246094\cdot10^7i$
\\11 & $1,75490226\cdot10^8$            & $-1,75490226\cdot10^8$
\\12 & $-2,2530628205925293\cdot10^9i$  & $2,2530628205925293\cdot10^9i$
\\13 & $-3,1217145174\cdot10^{10}$      & $3,1217145174\cdot10^{10}$
\\14 & $4,641652455250599\cdot10^{11}i$ & $-4,641652455250599\cdot10^{11}i$
\\15 & $7,3709524476135\cdot10^{12}$    & $-7,3709524476135\cdot10^{12}$
\end{tabular}
\end{center}

% vim: set ft=tex :
