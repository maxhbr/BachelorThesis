\section{Formale Meromorphe Zusammenhänge}

\begin{defn}[Formaler Meromorpher Zusammenhang]
  Ein \emph{Formaler Meromorpher zusammenhang} $(\cM_{\hat{K}},\partial)$ besteht aus
  folgenden Daten:
  \begin{itemize}
    \item $\cM_{\hat{K}}$, ein endlich dimensionaler $\hat K$-Vr
    \item eine \emph{Derivation} $\partial$, für die die \emph{Leibnitzregel}
      \eqref{eq:Leibnitzregel}, für alle $f\in \hat K$ und $u\in \cM_{\hat K}$,
      erfüllt sein soll.
  \end{itemize}
\end{defn}

\begin{comment}
  Oder einfach ein Meromorpher Zshg. über anderes $K$ also $\hat K$
\end{comment}

\begin{bem}
  alle bisher gegebenen Definitionen und Lemmata gelten für formale Meromorphe
  Zusammenhänge analog wie für konvergente Meromorphe Zusammenhänge.
\end{bem}

\begin{comment}
  \section{Elementare Meromorphe Zusammenhänge}

  %TODO: auch nicht formal
  \begin{defn}[Elementarer formaler Zusammenhang]
    \cite[Def 2.1]{sabbah_Fourier-local}
    Zu einem gegebenen $\rho\in u\C\llbracket u\rrbracket$,
    $\phi\in\C(\!(u)\!)$ und einem endlich dimensionalen
    $\C(\!(u)\!)$-Vektorraum $R$ mit regulärem Zusammenhang $\nabla$,
    definieren wir den assoziierten Elementaren endlich dimensionalen
    $\C(\!(t)\!)$-Vektorraum mit Zusammenhang, durch:
    \[
      El(\rho,\phi,R)=\rho_+(\sE^\phi\otimes R)
    \]
  \end{defn}
\end{comment}

% vim: set ft=tex :
