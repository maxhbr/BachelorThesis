%!TeX root = main.tex
\chapter{DIE Klasse der Fourier-Transformationen}
In diesem Kapittel werden Beispiele einer speziellen Klasse von $\cD$-Moduln
diskutiert. Dazu wird im folgendem zu 2 Beispielen explizit der Beweis aus
\cite{sabbah_cimpa90} zur Levelt-Turrittin-Zerlegung nachvollzogen. 
\begin{comment}
Eine weitere
explizite Berechnung findet sich bei \cite[5.b oder 5.c]{sabbah_cimpa90}.
\end{comment}

\section{Rezept für allgemeine $\phi$} \label{sec:allgemeinProblem}

\begin{comment}
siehe: \cite[5.b]{sabbah_Fourier-local}

bzeug zu $\sE^\phi$??
\end{comment}

\begin{comment}
sei $\phi\in\{\frac{1}{t^k},\frac{1}{t^2}+\frac{1}{t^3},\dots\}$
\begin{enumerate}
\item Starte mit: $P(t,\partial_t):=(\partial_t-\frac{d}{dt}\phi(t)) \cdot
\mbox{Hauptnenner }\in\C[t]<\partial_t>$
\item Furiertrafo: $F_P(z,\partial_z)=P(\partial_z,-z)\in\C[z]<\partial_z>$
\item $x=z^{-1}$ und $\partial_x=-z^2\partial_z$ \\
\[
Q(x,\partial_x):=F_P(x^{-1},-x^2\partial_x)\cdot \mbox{Hauptnenner
}\in\C[x]<\partial_x>
\]

\textbf{Hauptnenner unnötig?!?}
\item Berechne für $Q$ das NP usw...
\end{enumerate}
\end{comment}

Hier wollen wir nun eine Spezielle Klasse von Meromorphen Zusammenhängen, die
die durch das folgende Rezept entstehen.
\begin{enumerate}
\item Wähle zunächst ein $\phi$
$\in=
  \{\phi=\sum_{k\in I}\frac{a_k}{t^{k}}|I\subset\N\mbox{ endlich},a_k\in \C\}$ 
aus 
\item und definiere dann
$ \tilde Q(t,\partial_t):=\partial_t-\frac{d}{dt}\phi(t)$
$\in\C[t][t^{-1}]<\partial_t>$.
\item Wir wollen aber ein Element in $\C[t]<\partial_t>$,
deshalb multipliziere mit Hauptnenner und erhalte
\[
Q(t,\partial_t):=(\partial_t-\frac{d}{dt}\phi(t)) \cdot
\underset{\in\C[t]}{\underbrace{\mbox{Hauptnenner }}}\in\C[t]<\partial_t>
\]
Dies ändert den Assozierten Meromorphen Zusammenhang nicht.
\begin{comment}
Lemma?
\end{comment}
\item Fouriertransformiere $Q$ und erhalte
$\cF_Q(z,\partial_z)=Q(\partial_z,-z)$ in $\C[z]<\partial_z>$
\item Wende den Übergang $x\rightsquigarrow z^{-1}$ an.\\
Was passiert mit der Ableitung $\partial_x$? Es gilt
\[
\partial_x (f(\frac{1}{x}))=
\partial_z(f)\cdot (-\frac{1}{x^2})=
-\partial_z(f)\cdot z^2= %TODO: wegen klammerung?
- z^2 \cdot \partial_z(f)
\]
also $ \partial_x\rightsquigarrow-z^2\partial_z $.
\[
P_\phi(x,\partial_x):=F_Q(x^{-1},-x^2\partial_x) \in \C[t]<\partial_t>
\]
\item Erhalte den zu $P_\phi$ assoziierten Meromorphen Zusammenhang $\cM_\phi$.
\end{enumerate}

\begin{comment}
warum sind diese wichtig??
\end{comment}

\paragraph{Wende das Rezept allgemein für
$\phi=\sum_{k\in I}\frac{a_k}{t^{k}}$ an.}
So ist
\begin{align*}
\tilde Q(t,\partial_t) &=\partial_t-\frac{d}{dt}\phi(t) \\
                       &=\partial_t+\sum_{k\in I} k\frac{a_k}{t^{k+1}}
                       & &\in \C[t][t^{-1}]<\partial_t>\\
Q(t,\partial_t)        &=\partial_tt^{\max(I)+1}
                         +\sum_{k\in I} k\frac{a_k}{t^{k-\max(I)}} \\
                       &=\partial_tt^{\max(I)+1}
                         +\sum_{k\in I}k a_k t^{\max(I)-k}
                       & &\in \C[t]<\partial_t>\\
\cF_Q(z,\partial_z)    &=Q(\partial_z,-z)\\
                       &=-z\partial_z^{\max(I)+1}
                         +\sum_{k\in I}k a_k\partial_z^{\max(I)-k} \\
P_{\phi}(x,\partial_x) &=F_Q(x^{-1},-x^2\partial_x) \\
                       &=x\partial_x(-x^2\partial_x)^{\max(I)}
                         +\sum_{k\in I}k a_k(-x^2\partial_x)^{\max(I)-k}
                       & &\in \C[x]<\partial_x>
\end{align*}

Nun müssen wir noch $(x^2\partial_x)^{k+1}$ besser verstehen.
\begin{align*}
(x^2\partial_x)^{k+1} &=x^2\underbracket{\partial_xx^2}\partial_x
                        (x^2\partial_x)^{k-1}\\
                      &=x^2\overbracket{(2x+x^{2}\partial_x)}\partial_x
                        (x^2\partial_x)^{k-1}\\
                      &=(2x^3\partial_x+x^{4}\partial_x^2)
                        (x^2\partial_x)^{k-1}\\
                      &=(2x^3\partial_x+x^{4}\partial_x^2)(x^2\partial_x)
                        (x^2\partial_x)^{k-2}\\
                      &=(2x^3\underbracket{\partial_xx^2}\partial_x
                        +x^{4}\underbracket{\partial_x^2x^2}\partial_x)
                        (x^2\partial_x)^{k-2}\\
                      &=(2x^3\overbracket{(2x+x^{2}\partial_x)}\partial_x
                        +x^{4}\overbracket{(2x\partial_x+1+x^2\partial_x^2)}
                        \partial_x) (x^2\partial_x)^{k-2}\\
                      &=(4x^4\partial_x+2x^{5}\partial_x^2
                        +2x^{5}\partial_x^2
                        +x^4\partial_x
                        +x^6\partial_x^3)
                        (x^2\partial_x)^{k-2}\\
                      &=(5x^4\partial_x+4x^{5}\partial_x^2
                        +x^6\partial_x^3)
                        (x^2\partial_x)^{k-2}\\
                      &=\dots \mbox{geschlossene Formel??}
%TODO: induktion
\end{align*}
also gilt für spezielle $k$
\begin{equation} \label{eq:rezeptNeben1}
(x^2\partial_x)^{k+1}=
\begin{cases}
2x^3\partial_x+x^{4}\partial_x^2 & \mbox{ falls } k=1\\
5x^4\partial_x+4x^{5}\partial_x^2 +x^6\partial_x^3 & \mbox{ falls } k=2\\
\dots
\end{cases}
\end{equation}

\section{Angewendet für $\phi_1:=\frac{a}{x}$}
Das wohl einfachste Beispiel ist $\cM_{\phi_1}$ bezüglich
$\phi_1:=\frac{a}{x}$.
\begin{align*}
P_{\phi_1} &=a-x\underbracket{\partial_xx^2}\partial_x\\
           &\overbox{=}{(\ref{eq:kommutator1})}
             a-x\overbracket{(2x + x^2\partial_x)}\partial_x\\
           &=a-2x^2\partial_x - x^3\partial_x^2\\
\end{align*}
Finde nun das Newton-Polygon mit den Slopes $\cP(\cM_{\phi_1})$
\begin{figure}[H]
\caption{Newton Polygon zu $P_{\phi_1}$}
\begin{center}
\fbox{
  \begin{tikzpicture}[scale=1.5,descr/.style={fill=white,inner sep=2.5pt}]
  \def\myPoints{0/0,1/1,2/1}
  \myinput{myNewtonPoly.sty}
  \end{tikzpicture}
}
\end{center}
\end{figure}

\subsection{Levelt-Turrittin-Zerlegung für $\phi_1$}
\begin{comment}
Zeige zunächst: ist noch nicht zerlegt!\\
% ist das nötig?
Es ist $\cM_{\phi_1}\bydef \cD_{\hat K}/\cD_{\hat K}\cdot P_{}\phi_1}$. Wäre
dieser bereits Zerlegt, also gleich einem Elementarem Meromorphen Zusammenhang,
so müsste es $\rho\in t\Cft$, $\phi\in\hat L$ und $R$ geben, so dass
$\cM_{\phi_1}\cong El(\rho,\phi,R)=\rho_+(\sE^\phi\otimes R)$.
\end{comment}
\begin{comment}
vergleiche mit \cite[Proof of theorem 5.4.7]{sabbah_cimpa90}\\
was ist $\sigma_L(P)\in \hat K[\xi]$?
\end{comment}
$\cM_{\phi_1}$ hat genau den Slope $\frac{1}{2}$ mit Nenner $2$.\\
Sei $\rho:t\mapsto x:=t^2$ und betrachte 
\begin{align*}
\rho^+\cM_{\phi_1} &= \rho^+\Big( \cD_{\hat K}/\cD_{\hat
  K}\cdot(a-2x^2\partial_x -x^3\partial_x^2) \Big) \\
&= \cD_{\hat L}/\cD_{\hat L}\cdot(a-2\rho(t)^2(\rho'(t)^{-1}\partial_t)
  -\rho(t)^3(\rho'(t)^{-1}\partial_t)^2)\\
&= \cD_{\hat L}/\cD_{\hat L}\cdot(a-2t^4(\frac{1}{2}t^{-1}\partial_t)
  -t^6(\frac{1}{2}t^{-1}\partial_t)^2)\\
&= \cD_{\hat L}/\cD_{\hat L}\cdot(a-t^3\partial_t
  -\frac{1}{4}t^6(t^{-1}\underbracket{\partial_tt^{-1}}\partial_t))\\
&= \cD_{\hat L}/\cD_{\hat L}\cdot(a-t^3\partial_t
  -\frac{1}{4}t^6(t^{-1}\overbracket{(t^{-1}\partial_t -t^{-2})}\partial_t))\\
&= \cD_{\hat L}/\cD_{\hat L}\cdot(a-t^3\partial_t
  -\frac{1}{4}t^6(t^{-2}\partial_t^2 -t^{-1}t^{-2}\partial_t))\\
&= \cD_{\hat L}/\cD_{\hat L}\cdot(a-t^3\partial_t
  -\frac{1}{4}t^4\partial_t^2+\frac{1}{4}t^{3}\partial_t)\\
&= \cD_{\hat L}/\cD_{\hat L}\cdot(a-\frac{3}{4}t^3\partial_t
  -\frac{1}{4}t^4\partial_t^2)\\
\end{align*}
\begin{figure}[H]
\caption{Newton Polygon zu $\rho^*P_{\phi_1}$}
\begin{center}
\fbox{
  \begin{tikzpicture}[scale=1.5,descr/.style={fill=white,inner sep=2.5pt}]
  \def\myPoints{0/0,1/2,2/2}
  \myinput{myNewtonPoly.sty}
  \end{tikzpicture}
}
\end{center}
\end{figure}
\begin{comment}
Ab hier nochmal neu / verbessern!
\end{comment}
Nun hat $\rho^*P$ nur noch den Slope
$1=\frac{1}{1}=:\frac{\lambda_0}{\lambda_1}$ und definiere damit die Linearform
$L(s_0,s_1)=\lambda_0s_0+\lambda_1s_1$.  Berechne nun die \emph{Determinanten
Gleichung} $\sigma_L(\rho^*P_{\phi_1})\in \hat
K[\xi]$ von $\rho^*P$. 
\[
\sigma_L(\rho^*P)=
\]
Setze $\theta=x^{\lambda_0+\lambda_1}\xi^{\lambda_1}$ so können wir 
\[
\sigma_L(\rho^*P)=\sum_{k\geq 0}\alpha_k\theta^k
\]
schreiben, mit $\alpha_0\neq 0$ ...

\section{Angewendet für $\phi_2:=\frac{a}{x^2}$}
\textbf{also für $\phi_2:=\frac{a}{x^2}$} ist
\begin{align*}
P_{\phi_2} &=2+\frac{1}{x}\underbracket{(x^2\partial_x)^{3}}\\
           &\overbox{=}{(\ref{eq:rezeptNeben1})}2+\frac{1}{x}
             \overbracket{(5x^4\partial_x+4x^5\partial_x^2+x^6\partial_x^3)}\\
           &=2+5x^3\partial_x+4x^{4}\partial_x^2+x^5\partial_x^3\\
%TODO: BESSER:???
P_{\phi_2} &=2a+x\partial_x\underbracket{(-x^2\partial_x)^{2}}\\
           &=2a +x\partial_x \overbracket{(2x^3\partial_x+x^4\partial_x^2)} \\
           &=2a 
             +2x\underbracket{\partial_xx^3}\partial_x
             +x\underbracket{\partial_xx^4}\partial_x^2 \\
           &=2a 
             +2x\overbracket{(3x^2+x^3\partial_x)}\partial_x
             +x\overbracket{(4x^3+x^4\partial_x)}\partial_x^2 \\
           &=2a+5x^3\partial_x+4x^{4}\partial_x^2+x^5\partial_x^3
\end{align*}
\begin{figure}[H]
\caption{Newton Polygon zu $P_{\phi_2}$}
\begin{center}
\fbox{
  \begin{tikzpicture}[scale=1.5,descr/.style={fill=white,inner sep=2.5pt}]
  \def\myPoints{0/0,1/2,2/2,3/2}
  \myinput{myNewtonPoly.sty}
  \end{tikzpicture}
}
\end{center}
\end{figure}

\subsection{Levelt-Turrittin-Zerlegung für $\phi_2$}
$\cM_{\phi_2}$ hat genau den Slope $\frac{2}{3}$ mit Nenner $3$.\\
Sei $\rho:t\mapsto x:=t^3$ und betrachte 
\begin{align*}
\rho^+\cM_{\phi_1} &= \rho^+\Big( \cD_{\hat K}/\cD_{\hat
  K}\cdot(2a+5x^3\partial_x+4x^{4}\partial_x^2+x^5\partial_x^3) \Big) \\
&= \cD_{\hat L}/\cD_{\hat L}\cdot(2a+
  5\rho(t)^3(\rho'(t)^{-1}\partial_t)
  +4\rho(t)^{4}(\rho'(t)^{-1}\partial_t)^2
  +\rho(t)^5(\rho'(t)^{-1}\partial_t)^3) \\
&= \cD_{\hat L}/\cD_{\hat L}\cdot(2a+
  5t^9(\frac{1}{3}t^{-2}\partial_t)
  +4t^{12}(\frac{1}{3}t^{-2}\partial_t)^2
  +t^{15}(\frac{1}{3}t^{-2}\partial_t)^3) \\
&= \cD_{\hat L}/\cD_{\hat L}\cdot(2a+
  \frac{5}{3}t^7\partial_t
  +\frac{4}{9}t^{12}(t^{-2}
    \underbracket{\partial_tt^{-2}}
    \partial_t)
  +\frac{1}{27}t^{15}(t^{-2}
    \underbracket{\partial_tt^{-2}}
    \underbracket{\partial_tt^{-2}}
    \partial_t)) \\
&\begin{aligned}
&= \cD_{\hat L}/\cD_{\hat L}\cdot(2a+
  \frac{5}{3}t^7\partial_t
  +\frac{4}{9}t^{10}
    \overbracket{(t^{-2}\partial_t-2t^{-3})}
    \partial_t\\
  &\qquad+\frac{1}{27}t^{13}
    \underbracket{
      \overbracket{(t^{-2}\partial_t-2t^{-3})}
      \overbracket{(t^{-2}\partial_t-2t^{-3})}
    }
    \partial_t) \\
\end{aligned}\\
&\begin{aligned}
&= \cD_{\hat L}/\cD_{\hat L}\cdot(2a+
  \frac{5}{3}t^7\partial_t
  +\frac{4}{9}t^{8}\partial_t^2
  -\frac{8}{9}t^{7}\partial_t\\
  &\qquad+\frac{1}{27}t^{13}
    \overbracket{(
      t^{-2}\underbracket{\partial_tt^{-2}}\partial_t
      -2t^{-2}\underbracket{\partial_tt^{-3}}
      -2t^{-5}\partial_t
      +4t^{-6}
    )}
    \partial_t) \\
\end{aligned}\\
&\begin{aligned}
  &= \cD_{\hat L}/\cD_{\hat L}\cdot(2a+
    (\frac{5}{3}-\frac{7}{9}+\frac{4}{27})t^7\partial_t
    +(\frac{4}{9}-\frac{2}{27})t^{8}\partial_t^2
    +\frac{1}{27}t^{11}\overbracket{(t^{-2}\partial_t-2t^{-3})}\partial_t^2\\
    &\qquad-\frac{2}{27}t^{11}\overbracket{(t^{-3}\partial_t-3t^{-4})}
      \partial_t
  )
\end{aligned}\\
&\begin{aligned}
  &= \cD_{\hat L}/\cD_{\hat L}\cdot(2a+
    (\frac{5}{3}-\frac{7}{9}+\frac{4}{27})t^7\partial_t
    +(\frac{4}{9}-\frac{2}{27})t^{8}\partial_t^2
    +\frac{1}{27}t^{9}\partial_t^3-\frac{2}{27}t^{8}\partial_t^2\\
    &\qquad -\frac{2}{27}t^{8}\partial_t^2 +\frac{6}{27}t^{7}\partial_t
  )
\end{aligned}\\
&= \cD_{\hat L}/\cD_{\hat L}\cdot(2a+ \frac{34}{27}t^7\partial_t
  +\frac{6}{27}t^{8}\partial_t^2 +\frac{1}{27}t^{9}\partial_t^3)
\end{align*}
\begin{figure}[H]
\caption{Newton Polygon zu $\rho^*P_{\phi_2}$}
\begin{center}
\fbox{
  \begin{tikzpicture}[scale=1.5,descr/.style={fill=white,inner sep=2.5pt}]
  \def\myPoints{0/0,1/6,2/6,3/6}
  \myinput{myNewtonPoly.sty}
  \end{tikzpicture}
}
\end{center}
\end{figure}

\section{Angewendet für $\phi_3:=\frac{1}{x}+\frac{1}{x^2}$}
\textbf{also für $\phi_3:=\frac{1}{x}+\frac{1}{x^2}$} ist
\begin{align*}
P_{\phi_3} &=x\partial_x(-x^2\partial_x)^{\max_j(k_j)}
             +\sum_{i\in I} k_i(-x^2\partial_x)^{\max_j(k_j)-k_i}\\
           &=x\partial_x\underbracket{(x^2\partial_x)^{2}}
             +1(-x^2\partial_x)^{1}+2(-x^2\partial_x)^{0}\\
           &\overbox{=}{(\ref{eq:rezeptNeben1})}
             x\partial_x \overbracket{(2x^3\partial_x+x^4\partial_x^2)}
             -x^2\partial_x+2\\
           &=2x\underbracket{\partial_xx^3}\partial_x
             +x\underbracket{\partial_xx^4}\partial_x^2
             -x^2\partial_x+2\\
           &\overbox{=}{(\ref{eq:kommutator1})}
             \underbracket{2x\overbracket{(3x^2+x^3\partial_x)}\partial_x}
             +\underbracket{x\overbracket{(4x^3+x^4\partial_x)}\partial_x^2}
             -x^2\partial_x+2\\
           &=\overbracket{6x^3\partial_x+2x^4\partial_x^2}
             +\overbracket{4x^4\partial_x^2+x^5\partial_x^3}
             -x^2\partial_x+2\\
           &= x^5\partial_x^3+6x^4\partial_x^2+(6x^3-x^2)\partial_x+2
\end{align*}
\begin{figure}[H]
\caption{Newton Polygon zu $P_{\phi_3}$}
\begin{center}
\fbox{
  \begin{tikzpicture}[scale=1.5,descr/.style={fill=white,inner sep=2.5pt}]
  \def\myPoints{0/0,1/1,1/2,2/2,3/2}
  \myinput{myNewtonPoly.sty}
  \end{tikzpicture}
}
\end{center}
\end{figure}

\section{Angewendet für $\phi_4:=\frac{1}{x^2}+\frac{1}{x^3}$}
\textbf{also für $\phi_4:=\frac{1}{x^2}+\frac{1}{x^3}$} ist
\begin{align*}
P_{\phi_4} &=x\partial_x(-x^2\partial_x)^{\max_j(k_j)}
             +\sum_{i\in I} k_i(-x^2\partial_x)^{\max_j(k_j)-k_i}\\
           &=-x\partial_x\underbracket{(x^2\partial_x)^{3}}
             -2x^2\partial_x +3\\
           &\overbox{=}{(\ref{eq:rezeptNeben1})}
             -x\partial_x\overbracket{(5x^4\partial_x+4x^{5}\partial_x^2
             +x^6\partial_x^3)}-2x^2\partial_x +3\\
           &=-5x\underbracket{\partial_xx^4}\partial_x
             -4x\underbracket{\partial_xx^{5}}\partial_x^2
             -x\underbracket{\partial_xx^6}\partial_x^3
             -2x^2\partial_x +3\\
           &\overbox{=}{(\ref{eq:kommutator1})}
             \underbracket{-5x\overbracket{(4x^3+x^4\partial_x)}\partial_x}
             \underbracket{-4x\overbracket{(5x^4+x^{5}\partial_x)}\partial_x^2}
             \underbracket{-x\overbracket{(6x^5+x^6\partial_x)}\partial_x^3}
             -2x^2\partial_x+3\\
           &=\overbracket{-20x^4\partial_x-5x^5\partial_x^2}
             \overbracket{-20x^5\partial_x^2-4x^{6}\partial_x^3}
             \overbracket{-6x^6\partial_x^3-x^7\partial_x^4}
             -2x^2\partial_x +3\\
           &=-x^7\partial_x^4-10x^6\partial_x^3-25x^5\partial_x^2
             -(20x^4+2x^2)\partial_x+3\\
\end{align*}
\begin{figure}[H]
\caption{Newton Polygon zu $P_{\phi_4}$}
\begin{center}
\fbox{
  \begin{tikzpicture}[scale=1.5,descr/.style={fill=white,inner sep=2.5pt}]
  \def\myPoints{0/0,1/1,1/3,2/3,3/3,4/3}
  \myinput{myNewtonPoly.sty}
  \end{tikzpicture}
}
\end{center}
\end{figure}


% vim: set ft=tex :
