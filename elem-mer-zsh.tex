%!TeX root = main.tex
\chapter{Elementare Meromorphe Zusammenhänge}

%TODO: auch nicht formal
\begin{defn}[Elementarer formaler Zusammenhang]
\cite[Def 2.1]{sabbah_Fourier-local}
\begin{comment}
Alternative. ausfürlichere / komplexe definition \cite[Def 5.4.5.]{sabbah_cimpa90}
\end{comment}
Zu einem gegebenen $\rho\in u\C\llbracket u\rrbracket$,
$\phi\in\C(\!(u)\!)$ und einem endlich dimensionalen
$\C(\!(u)\!)$-Vektorraum $R$ mit regulärem Zusammenhang $\nabla$,
definieren wir den assoziierten Elementaren endlich dimensionalen
$\C(\!(t)\!)$-Vektorraum mit Zusammenhang, durch:
\[
El(\rho,\phi,R)=\rho_+(\sE^\phi\otimes R)
\]
\end{defn}
\cite[nach Def 2.1]{sabbah_Fourier-local}
Bis auf isomorphismus hängt $El(\rho,\phi,R)$ nur von $\phi\mod\Cfu$ ab.
\begin{lem}
\cite[Lem 2.2]{sabbah_Fourier-local}
\end{lem}

\myinput{lemma2-4.tex}

\begin{lem} \cite[Lem 2.6.]{sabbah_Fourier-local}
Es gilt $El([u\mapsto u^p],\phi,R)\cong El([u\mapsto u^p],\psi,S)$ genau dann,
wenn
\begin{itemize}
\item es ein $\zeta$ gibt, mit $\zeta^p=1$ und
$\psi\circ\mu_\zeta\equiv\phi\mod\Cfu$
\item und $S\cong R$ als $\Cful$-Vektorräume mit Zusammenhang.
\end{itemize}
\end{lem}
\begin{proof}
\cite[Lem 2.6.]{sabbah_Fourier-local}
\end{proof}

\begin{prop} \cite[Prop 3.1]{sabbah_Fourier-local}
Jeder irreduzible endlich dimensionale $\Cfxl$-Vektorraum $\cM$ mit
Zusammenhang ist isomorph zu $\rho_+(\sE^\phi\otimes L)$, wobei $\phi\in
u^{-1}\C[u^{-1}]$, $\rho:u\rightarrow u^p$ vom grad $p\geq 1$ und ist minimal
unter $\phi$. (siehe \cite[Rem  2.8]{sabbah_Fourier-local}) und $L$ ist ein
Rang $1$ $\Cfxl$-Vektrorraum mit regulärem Zusammenhang.
\end{prop}
\begin{proof}
%TODO: verwendet hier schon das klassische Levelt-Turittin
\cite[Prop 3.1]{sabbah_Fourier-local}
\end{proof}

% vim: set ft=tex :
