%!TeX root = main.tex
\chapter{Elementare Meromorphe Zusammenhänge}
\begin{comment}
einführen als Bausteine oder kleinste Meromorphe Zusammenhänge
\end{comment}
\begin{defn}
\cite[1.a]{sabbah_Fourier-local}
Sei $\phi\in\hat K$.
Wir schreiben $\sE_{\hat K}^\phi$ für den (formalen) Rang 1 Vektorraum $\Cfxl
\bydef \hat K$ ausgestattet mit dem Zusammenhang
$\nabla=\partial_x+\partial_x\phi$, im speziellen also
$\nabla_{\partial_x}1=\partial_x1=\phi'$.\\
\begin{comment}
Also
\begin{align*}
\sE^\phi=\Cfxl & \overset{\partial_x}{\rightarrow} \Cfxl\\
1              & \mapsto \phi'(x)\\
f(x)           & \mapsto f'(x)+f(x)\phi'(x)\\
\end{align*}
\end{comment}
\end{defn}
\begin{bem}
\begin{enumerate}
\item Auf die Angabe von des Rang 1 Vektorraums im Subscript wird im folgendem
meist verzichtet.
\item Offensichtlich ist $\sE^\phi\cong\cD_{\hat K}/\cD_{\hat
K}\cdot(\partial_x-\phi'(x))$, weil für den zyklischen Vektor $1$ gilt, dass
$\partial_x \cdot 1 = \phi'(x) \cdot 1$.
\end{enumerate}
\end{bem}

\begin{bem} \label{bem:FormRang1VR}
\cite[1.a]{sabbah_Fourier-local}
Es gilt $\sE^\phi\cong\sE^\psi$ genau dann wenn $\phi\equiv\psi \mod \Cfx$.
%TODO: hier formal oder konvergent?
\end{bem}

\begin{comment}
ausformulierte version eines beweises im paper sabbah_Fourier-local.pdf zu
lemma 2.4
\end{comment}

Sei $\rho:t\mapsto x:=t^p$ und $\mu_\xi:t\mapsto\xi t$.
\begin{lem}
\cite[Lem 2.4]{sabbah_Fourier-local}
Für alle $\phi \in \hat L$ gilt
\[
\rho^+\rho_+\sE^\phi=\bigoplus_{\xi^p=1}\sE^{\phi\circ\mu_\xi} \,.
\]
\end{lem}
%
\begin{proof}
Wir wollen zeigen, dass das folgende Diagram, für einen passenden
Isomorphismus, kommutiert:
\begin{center}
  \begin{tikzpicture} [scale=3.3, descr/.style={fill=white,inner sep=2.5pt} ]
  \matrix (m) [
    matrix of math nodes
    ,row sep=2em
    ,column sep=6em
    %,text height=3em
    %,text depth=0.25em
    ]
  {
    \rho^+\rho_+\sE^{\phi(u)} &
    \bigoplus_{\xi^p=1}\sE^{\phi\circ\mu_\xi} \\
    \rho^+\rho_+\sE^{\phi(u)} &
    \bigoplus_{\xi^p=1}\sE^{\phi\circ\mu_\xi} \\
  };
    \path[->,font=\scriptsize,>=angle 90]
    (m-1-1) edge node[right]{$\partial_t$} (m-2-1)
    (m-1-2) edge node[right]{$\partial_t$} (m-2-2)
    (m-1-1) edge node[below]{$\cong$} (m-1-2)
    (m-2-1) edge node[below]{$\cong$} (m-2-2)
    ;
  \end{tikzpicture}
\end{center}
Es sei oBdA $\phi\in t^{-1}\C[t^{-1}]$, dies ist nach Bemerkung 
\ref{bem:FormRang1VR} berechtigt.
Wir wählen eine $\hat L$ Basis $e$ des Rang 1 $\hat L$-Vektorraum $\sE^\phi$
und damit erhält man die Familie $e,te,...,t^{p-1}e$ als $\hat K$-Basis von
$\rho_+\sE^\phi$.\\
Durch die Setzung $e_k:=t^{-k}\otimes_{\hat K}t^ke$ wird die Familie
$\mathbf{e}:=(e_0,...,e_{p-1})$ eine $\hat L$-Basis von
$\rho^+\rho_+\sE^\phi$.\\
Zerlege nun $t\phi'(t)=\sum_{j=0}^{p-1}t^j\psi_j(t^p)\in t^{-2}\C[t^{-1}]$
mit $\psi_j\in\C[x^{-1}]$ für alle $j>0$ und $\psi_0\in x^{-1}\C[x^{-1}]$
(siehe: Anhang \ref{chap:aufteilung}). 
Es gilt:
\[
t\partial_te_k= \sum_{i=0}^{p-1-k}t^i\psi_i(t^p)e_{k+1} +
  \sum_{i=p-k}^{p-1}t^i\psi_i(t^p)e_{k+i-p} 
\]
denn:
\begin{align*}
t\partial_te_k &= t\partial_t(t^{-k}\otimes_{\hat K}t^ke)\\
  &= t(-kt^{-k-1}\otimes_{\hat K}t^ke +
    pt^{p-1}\cdot t^{-k}\otimes_{\hat K}\partial_x
    (\underset{\in\rho_+\sE^\phi}{\underbrace{t^ke}}))\\
  &= -kt^{-k}\otimes_{\hat K}t^ke +
    pt^{p-1}t^{-k+1}\otimes_{\hat K}(pt^{p-1})^{-1} (kt^{k-1}e
    + t^k\phi'(t)e)\\
  &= -kt^{-k}\otimes_{\hat K}t^ke +
    t^{-k+1}\otimes_{\hat K}(kt^{k-1}e + t^k\phi'(t)e)\\
  &= \underset{=0}{\underbrace{-kt^{-k}\otimes_{\hat K}t^ke +
    t^{-k+1}\otimes_{\hat K}kt^{k-1}e}} +
    t^{-k+1}\otimes_{\hat K}t^k\phi'(t)e\\
  &= t^{-k}\otimes_{\hat K}t^{k+1}\phi'(t)e\\
  &= \sum_{i=0}^{p-1}t^{-k}\otimes_{\hat K}
    t^{k}t^i\underset{\in\hat K}{\underbrace{\psi_i(t^p)}}e\\
  &= \sum_{i=0}^{p-1}t^i\psi_i(t^p)(t^{-k}\otimes_{\hat K} t^{k}e)\\
  &= \sum_{i=0}^{p-1-k}t^i\psi_i(t^p)e_{k+1} +
  \sum_{i=p-k}^{p-1}t^i\psi_i(t^p)e_{k+i-p}
\end{align*}
Sei
\[
V:=\begin{pmatrix}
0 &        &          & 1\\
1 & 0\\
  & \ddots & \ddots\\
  &        & 1        & 0
\end{pmatrix}
\]
so dass $\mathbf{e}\cdot V=(e_1,...,e_{p-1},e_0)$ gilt, 
so dass gilt:
\[
t\partial_t\mathbf{e}=\mathbf{e}[\sum_{j=0}^{p-1}t^j\psi_jV^j]
\]
denn:
\begin{align*}
  t\partial_t\mathbf{e} &= (t\partial_te_0,...,t\partial_te_{p-1})\\
  &= \Bigg(\sum_{i=0}^{p-1-k}t^i\psi_i(t^p)e_{k+1} +
    \sum_{i=p-k}^{p-1}t^i\psi_i(t^p)e_{k+i-p}\Bigg)_{k\in\{0,..,p-1\}}\\
  &= \mathbf{e}
{ %\small
  \footnotesize
  \begin{pmatrix}u^{p-1}\psi_{p-1}(t^p) &  & \cdots & t^{3}\psi_{3}(t^p) & t^{2}
  \psi_{2}(t^p) & t^{1}\psi_{1}(t^p)\\
  t^{1}\psi_{1}(t^p) & t^{p-1}\psi_{p-1}(t^p) &  &
  & \ddots & t^{2}\psi_{2}(t^p)\\
  t^{2}\psi_{2}(t^p) & t^{1}\psi_{1}(t^p) & \ddots &  &  & t^{3}\psi_{3}(t^p)\\
  t^{3}\psi_{3}(t^p) & \ddots & \ddots & \ddots &  & \vdots\\
  \vdots &  & \ddots & t^{1}\psi_{1}(t^p) & t^{p-1}\psi_{p-1}(t^p)\\
  t^{p-2}\psi_{p-2}(t^p) & \cdots & t^{3}\psi_{3}(t^p) & t^{2}\psi_{2}(t^p) &
  t^{1}\psi_{1}(t^p) & t^{p-1}\psi_{p-1}(t^p)
  \end{pmatrix}
} \\
  &= \mathbf{e}[\sum_{j=0}^{p-1}t^j\psi_j(t^p)V^j]
\end{align*}
Die Wirkung von $\partial_t$ auf die Basis von $\rho^+\rho_+\sE^{\phi(t)}$ ist
also Beschrieben durch
\[
\partial_t\mathbf{e}=\mathbf{e}[\sum_{j=0}^{p-1}t^{j-1}\psi_jV^j] \,.
\]
Da $V$ das Minimalpolynom $\chi_V(x)=X^p-1$ hat, können wir diese Matrix durch
Passendes $T$ auf die Form
\[
D:=TVT^{-1}=\begin{pmatrix}\xi^{0}\\
 & \xi^{1}\\
 &  & \ddots\\
 &  &  & \xi^{p-1}
\end{pmatrix} \,,
\]
mit $\xi^p=1$, bringen.
So dass gilt:
\begin{align*}
  T[\sum_{j=0}^{p-1}t^{j-1}\psi_j(t^p)V^j]T^{-1} 
  &= [\sum_{j=0}^{p-1}t^{j-1}\psi_j(t^p) (TVT^{-1})^j]\\
  &= [\sum_{j=0}^{p-1}t^{j-1}\psi_j(t^p)D^j]\\
  &=
{
  \footnotesize
  \begin{pmatrix}\sum_{j=0}^{p-1}t^{j-1}\psi_{j}\\
    & \sum_{j=0}^{p-1}t^{j-1}\psi_{j}\left(\xi^{1}\right)^{j}\\
    & & \ddots\\
    &  &  & \sum_{j=0}^{p-1}t^{j-1}\psi_{j}\left(\xi^{p-1}\right)^{j}
  \end{pmatrix}
}\\
  &=
{
  \footnotesize
  \begin{pmatrix}\sum_{j=0}^{p-1}t^{j-1}\psi_{j}\\
    & \sum_{j=0}^{p-1}(t\xi^1)^{j-1}\psi_{j}\xi^{1}\\
    & & \ddots\\
    &  &  & \sum_{j=0}^{p-1}(t\xi^{p-1})^{j-1}\psi_{j}\xi^{p-1}
  \end{pmatrix}
  %\mbox{\footnote{$(\xi^i)^p=1\forall i$}} % TODO: correct footnote
}\\
  &= \begin{pmatrix}\phi'(t)\\
    & \phi'(\xi t)\xi^{1}\\
    & & \ddots\\
    &  &  & \phi'(\xi^{p-1} t)\xi^{p-1}
  \end{pmatrix}\\
\end{align*}
Damit wissen wir bereits, das im Diagram
\begin{center}
  \begin{tikzpicture} [scale=3.3, descr/.style={fill=white,inner sep=2.5pt} ]
  \matrix (m) [
    matrix of math nodes
    ,row sep=2em
    ,column sep=6em
    %,text height=3em
    %,text depth=0.25em
    ]
  {
    \rho^+\rho_+\sE^{\phi(u)} &
    \hat L^p &
    \hat L^p &
    \bigoplus_{\xi^p=1}\sE^{\phi\circ\mu_\xi} \\
    & \\
    & \\
    \rho^+\rho_+\sE^{\phi(u)} &
    \hat L^p &
    \hat L^p &
    \bigoplus_{\xi^p=1}\sE^{\phi\circ\mu_\xi} \\
  };
    \path[->,font=\scriptsize,>=angle 90]
    (m-1-2) edge node[below]{$\cong$} (m-1-1)
    (m-1-1) edge node[right]{$\partial_t$} (m-4-1)
    (m-4-2) edge node[below]{$\cong$} (m-4-1)
    (m-1-3) edge node[above]{$T$} node[below]{$\cong$} (m-1-2)
    (m-4-3) edge node[above]{$T$} node[below]{$\cong$} (m-4-2)
    (m-1-2) edge node[descr]{$\sum_{j=0}^{p-1}t^{j-1}\psi_jV^j$} (m-4-2)
    (m-1-3) edge node[descr]{$\sum_{j=0}^{p-1}t^{j-1}\psi_jD^j$} (m-4-3)
    (m-1-3) edge node[above]{$\Phi$} node[below]{$\cong$} (m-1-4)
    (m-4-3) edge node[above]{$\Phi$} node[below]{$\cong$} (m-4-4)
    (m-1-4) edge node[right]{$\partial_t$} (m-4-4)
    ;

    \draw [decoration={brace,amplitude=0.5em},decorate]
      (m.south -| m-4-3.east) -- node[below]{$(\star)$} 
        (m.south -| m-4-1.west);
  \end{tikzpicture}
\end{center}
der mit $(\star)$ bezeichnete Teil kommutiert. Um zu zeigen, dass alles
kommutiert, zeigen wir noch, dass
\begin{align*}
\partial_t(\Phi(x))=\Phi\big(\sum_{j=0}^{p-1}t^{j-1}\psi_j(x)D^j\big)
& & \forall x\in \hat L^p
\end{align*}
gilt.
\begin{comment}
TODO: zeige das noch
\end{comment}
Sei $x=\,^t(x_1,\dots,x_p)\in\hat L^p$. So ist 
\begin{align*}
\partial_t(\Phi(x)) &= \partial_t(\,^t(\dots))
\end{align*}
und
\begin{align*}
\Phi\Big(\,^tx\big(\sum_{j=0}^{p-1}t^{j-1}\psi_j(t^p)D^j\big)\Big) 
&= \Phi\Big( (x_1,\dots,x_p)
  \begin{pmatrix}\phi'(t)\\
    & \phi'(\xi t)\xi^{1}\\
    & & \ddots\\
    &  &  & \phi'(\xi^{p-1} t)\xi^{p-1}
  \end{pmatrix}
  \Big)\\
&= \Phi\Big(
(x_1\phi'(t),x_2\phi'(\xi t)\xi,\dots,x_p\phi'(\xi^{p-1} t)\xi^{p-1}) \Big)
\end{align*}
\end{proof}

%%%%%%%%%%%%%%%%%%%%%%%%%%%%%%%%%%%%%%%%%%%%%%%%%%%%%%%%%%%%%%%%%%%%%%%%%%%%%%%
\section{Defnintion von Notizen und \cite[Cor 5.2.6]{sabbah_cimpa90}}
\begin{defn}
Ein \emph{Elementarer Meromorpher Zusammenhang} ist ein Zusammenhang $\cM$, für
den es $\psi \in \Cfx$, $\alpha\in\C$ und $p\in \N$ gibt, so dass
\[
\cM\cong \sE^{\psi}\otimes R_{\alpha,p}\,,
\]
mit $R_{\alpha,p}:=\cD/\cD(x\partial_x-\alpha)^p$, ist.
\end{defn}

\begin{lem}
\sE^{\psi}\otimes R_{\alpha,p}\cong
\cD/\cD\cdot(x\partial_x-(\alpha+x\frac{\partial \psi}{\partial x}))^p
\end{lem}
\begin{proof}
\cite[Lem 5.12]{DiplHedwig}
\end{proof}

%%%%%%%%%%%%%%%%%%%%%%%%%%%%%%%%%%%%%%%%%%%%%%%%%%%%%%%%%%%%%%%%%%%%%%%%%%%%%%%
\section{Defnintion in \cite{sabbah_cimpa90}}
\begin{comment}
in \cite{sabbah_cimpa90} Teil 5.4.4 Seite 34
\end{comment}
\begin{defn}
Sei $R(z)=\sum_{i=0}^k\alpha_iz^i\in z\C[z]$.
So ist der Meromorphe Zusammenhang $\cF_{\hat K}^R$ als Vektorraum isomorph zu
$\hat K$ und hat der Basis $e(R)$.
Die Wirkung von $x\partial_x$ ist definiert durch
\[
x\partial_x(\phi\cdot e(R))=\Big[ (x\frac{\partial \phi}{\partial x}) 
  +\phi x \frac{\partial R(x^{-1})}{\partial_x} \Big]\cdot e(R)
\]
\begin{comment}
hat das was mit fourier zu tun
\end{comment}
\begin{comment}
This means that $e(R)$ plays the role of $\exp R(x^{-1})$.
\end{comment}
\end{defn}
\begin{defn}
Ein \emph{Elementarer Meromorpher Zusammenhang} (über $\hat K$) ist ein Zusammenhang
welcher zu 
$\cF_{\hat K}^R\otimes_{\hat K} \cG_{\hat K}$
isomorph ist. Wobei hier $\cG_{\hat K}$ ein Elementarer regulärer Meromorpher
Zusammenhang.
\end{defn}

%%%%%%%%%%%%%%%%%%%%%%%%%%%%%%%%%%%%%%%%%%%%%%%%%%%%%%%%%%%%%%%%%%%%%%%%%%%%%%%
\section{Defnintion in \cite{sabbah_Fourier-local}}
%TODO: auch nicht formal
\begin{defn}[Elementarer formaler Zusammenhang]
\cite[Def 2.1]{sabbah_Fourier-local}
\begin{comment}
Alternative. ausfürlichere / komplexe definition \cite[Def 5.4.5.]{sabbah_cimpa90}
\end{comment}
Zu einem gegebenen $\rho\in t\C\llbracket t\rrbracket$, $\phi\in \hat L \bydef
\C(\!(t)\!)$ und einem endlich dimensionalen $\hat L$-Vektorraum $R$ mit
regulärem Zusammenhang $\nabla$, definieren wir den assoziierten Elementaren
endlich dimensionalen $\hat K$-Vektorraum mit Zusammenhang, durch:
\[
El(\rho,\phi,R)=\rho_+(\sE^\phi\otimes R)
\]
\end{defn}
\cite[nach Def 2.1]{sabbah_Fourier-local}
Bis auf Isomorphismus hängt $El(\rho,\phi,R)$ nur von $\phi\mod\Cft$ ab.
\begin{lem}
\cite[Lem 2.2]{sabbah_Fourier-local}
\end{lem}
\begin{lem} \cite[Lem 2.6.]{sabbah_Fourier-local}
Es gilt $El([t\mapsto t^p],\phi,R)\cong El([t\mapsto t^p],\psi,S)$ genau dann,
wenn
\begin{itemize}
\item es ein $\zeta$ gibt, mit $\zeta^p=1$ und
$\psi\circ\mu_\zeta\equiv\phi\mod\Cft$
\item und $S\cong R$ als $\hat L$-Vektorräume mit Zusammenhang.
\end{itemize}
\end{lem}
\begin{proof}
\cite[Lem 2.6.]{sabbah_Fourier-local}
\end{proof}
%
\begin{prop} \cite[Prop 3.1]{sabbah_Fourier-local}
Jeder irreduzible endlich dimensionale $\hat K$-Vektorraum $\cM$ mit
Zusammenhang ist isomorph zu $\rho_+(\sE^\phi\otimes L)$, wobei $\phi\in
t^{-1}\C[t^{-1}]$, $\rho:t\rightarrow t^p$ vom Grad $p\geq 1$ und ist minimal
unter $\phi$. (siehe \cite[Rem  2.8]{sabbah_Fourier-local}) und $L$ ist ein
Rang $1$ $\hat L$-Vektrorraum mit regulärem Zusammenhang.
\end{prop}
\begin{proof}
%TODO: verwendet hier schon das klassische Levelt-Turittin
\cite[Prop 3.1]{sabbah_Fourier-local}
\end{proof}

% vim: set ft=tex :
