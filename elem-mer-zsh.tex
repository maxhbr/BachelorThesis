%!TeX root = main.tex
\chapter{Elementare Meromorphe Zusammenhänge}
\begin{comment}
einführen als Bausteine oder kleinste Meromorphe Zusammenhänge
\end{comment}
\begin{defn}
\cite[1.a]{sabbah_Fourier-local}
Sei $\phi\in\hat K$.
Wir schreiben $\sE^\phi$ für den (formalen) Rang 1 Vektorraum $\Cfxl \bydef
\hat K$ ausgestattet mit dem Zusammenhang $\nabla=\partial_x+\partial_x\phi$,
im speziellen also $\nabla_{\partial_x}1=\partial_x1=\phi'$.\\
\begin{comment}
Also
\begin{align*}
\sE^\phi=\Cfxl & \overset{\partial_x}{\rightarrow} \Cfxl\\
1              & \mapsto \phi'(x)\\
f(x)           & \mapsto f'(x)+f(x)\phi'(x)\\
\end{align*}
\end{comment}
\end{defn}

\begin{bem} \label{bem:FormRang1VR}
\cite[1.a]{sabbah_Fourier-local}
Es gilt $\sE^\phi\cong\sE^\psi$ genau dann wenn $\phi\equiv\psi \mod \Cfx$.
%TODO: hier formal oder konvergent?
\end{bem}
%
\myinput{lemma2-4.tex}

%%%%%%%%%%%%%%%%%%%%%%%%%%%%%%%%%%%%%%%%%%%%%%%%%%%%%%%%%%%%%%%%%%%%%%%%%%%%%%%
\section{Defnintion von Notizen und \cite[Cor 5.2.6]{sabbah_cimpa90}}
\begin{defn}
Ein \emph{Elementarer Meromorpher Zusammenhang} ist ein Zusammenhang $\cM$, für
den es $\psi \in \Cfx$, $\alpha\in\C$ und $p\in \N$ gibt, so dass
\[
\cM\cong \sE^{\psi}\otimes R_{\alpha,p}\,,
\]
mit $R_{\alpha,p}:=\cD/\cD(x\partial_x-\alpha)^p$, ist.
\begin{comment}
Sind die Regulär?
\end{comment}
\end{defn}

%%%%%%%%%%%%%%%%%%%%%%%%%%%%%%%%%%%%%%%%%%%%%%%%%%%%%%%%%%%%%%%%%%%%%%%%%%%%%%%
\section{Defnintion in \cite{sabbah_cimpa90}}
\begin{comment}
in \cite{sabbah_cimpa90} Teil 5.4.4 Seite 34
\end{comment}
\begin{defn}
Sei $R(z)=\sum_{i=0}^k\alpha_iz^i\in z\C[z]$.
So ist der Meromorphe Zusammenhang $\cF_{\hat K}^R$ als Vektorraum isomorph zu
$\hat K$ und hat der Basis $e(R)$.
Die Wirkung von $x\partial_x$ ist definiert durch
\[
x\partial_x(\phi\cdot e(R))=\Big[ (x\frac{\partial \phi}{\partial x}) 
  +\phi x \frac{\partial R(x^{-1})}{\partial_x} \Big]\cdot e(R)
\]
\begin{comment}
This means that $e(R)$ plays the role of $\exp R(x^{-1})$.
\end{comment}
\end{defn}
\begin{defn}
Ein \emph{Elementarer Meromorpher Zusammenhang} (über $\hat K$) ist ein Zusammenhang
welcher zu 
$\cF_{\hat K}^R\otimes_{\hat K} \cG_{\hat K}$
isomorph ist. Wobei hier $\cG_{\hat K}$ ein Elementarer regulärer Meromorpher
Zusammenhang.
\end{defn}

%%%%%%%%%%%%%%%%%%%%%%%%%%%%%%%%%%%%%%%%%%%%%%%%%%%%%%%%%%%%%%%%%%%%%%%%%%%%%%%
\section{Defnintion in \cite{sabbah_Fourier-local}}
%TODO: auch nicht formal
\begin{defn}[Elementarer formaler Zusammenhang]
\cite[Def 2.1]{sabbah_Fourier-local}
\begin{comment}
Alternative. ausfürlichere / komplexe definition \cite[Def 5.4.5.]{sabbah_cimpa90}
\end{comment}
Zu einem gegebenen $\rho\in t\C\llbracket t\rrbracket$, $\phi\in \hat L \bydef
\C(\!(t)\!)$ und einem endlich dimensionalen $\hat L$-Vektorraum $R$ mit
regulärem Zusammenhang $\nabla$, definieren wir den assoziierten Elementaren
endlich dimensionalen $\hat K$-Vektorraum mit Zusammenhang, durch:
\[
El(\rho,\phi,R)=\rho_+(\sE^\phi\otimes R)
\]
\end{defn}
\cite[nach Def 2.1]{sabbah_Fourier-local}
Bis auf Isomorphismus hängt $El(\rho,\phi,R)$ nur von $\phi\mod\Cft$ ab.
\begin{lem}
\cite[Lem 2.2]{sabbah_Fourier-local}
\end{lem}
\begin{lem} \cite[Lem 2.6.]{sabbah_Fourier-local}
Es gilt $El([t\mapsto t^p],\phi,R)\cong El([t\mapsto t^p],\psi,S)$ genau dann,
wenn
\begin{itemize}
\item es ein $\zeta$ gibt, mit $\zeta^p=1$ und
$\psi\circ\mu_\zeta\equiv\phi\mod\Cft$
\item und $S\cong R$ als $\hat L$-Vektorräume mit Zusammenhang.
\end{itemize}
\end{lem}
\begin{proof}
\cite[Lem 2.6.]{sabbah_Fourier-local}
\end{proof}
%
\begin{prop} \cite[Prop 3.1]{sabbah_Fourier-local}
Jeder irreduzible endlich dimensionale $\hat K$-Vektorraum $\cM$ mit
Zusammenhang ist isomorph zu $\rho_+(\sE^\phi\otimes L)$, wobei $\phi\in
t^{-1}\C[t^{-1}]$, $\rho:t\rightarrow t^p$ vom grad $p\geq 1$ und ist minimal
unter $\phi$. (siehe \cite[Rem  2.8]{sabbah_Fourier-local}) und $L$ ist ein
Rang $1$ $\hat L$-Vektrorraum mit regulärem Zusammenhang.
\end{prop}
\begin{proof}
%TODO: verwendet hier schon das klassische Levelt-Turittin
\cite[Prop 3.1]{sabbah_Fourier-local}
\end{proof}

% vim: set ft=tex :
