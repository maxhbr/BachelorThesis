%\chapter*{Schluss} 
\addchap{Fazit / Ausblick}
\begin{comment}
%TODO: bekannt? woher?
Dies bestätigt die ohnehin bekannte Aussage, dass in unserem Beispiel keine
konvergente Zerlegung zu finden ist.
\end{comment}

Die erhaltene Zerlegung (\ref{eq:finaleZerlegung}) lässt sich, wie in
\cite[Cor 3.3]{sabbah_Fourier-local} beschrieben, nach einem Pushforward
sortieren und zusammenfassen. Das Ergebnis ist eine direkte Summe aus, in
\cite{sabbah_Fourier-local} eingeführten $El(\rho,\phi,R)$, welche die Slopes
von $\cM$ \glqq{}respektieren\grqq{}.

Möchte man meromorphe Zusammenhänge noch weiter Klassifizieren bzw. weitere
Aussagen treffen, so liefert die asymptotische Analysis mit der sogenannten
Stokes-Struktur dafür einen Ansatz.
Durch die Stokes Struktur wird die Änderung der asymtotischen Lösung auf
Sektoren der Pole kodiert.
Auf diesem Gebiet gibt es noch viel Forschungsbedarf, so hat sich bisher nur im
Eindimmensionalen eine einheitliche Theorie entwickelt. Für höherdimmensionale
Fragestellungen sind immer noch die meisten Fragen offen.
Das Ziel ist es, ein Ergebnis zu erhalten, welches zur
Riemann-Hilbert-Korrespondenz äquivalente Aussagen über irregulär singuläre
meromorphe Zusammenhänge trifft.
Vergleiche dazu beispielsweise \cite{sabbah2013introduction} oder
\cite{citeulike:8523004}.

Das Berechnen solch einer Stokes-Struktur ist im Allgemeinem schwer.
Ein Versuch dies zu vereinfachen ist, den fraglichen meromorphen Zusammenhang
zunächst, wie hier, in elementare meromorphe Zusammenhänge zu zerlegen.
Auf dieser Basis liese sich gut eine weiterführende Arbeit aufbauen, welche
das Ziel hat, explizit eine Stokes-Struktur zu berechnen.

\begin{comment}
Stokes Struktur ausrechen? Dazu die Lösung asymptotisch approximieren
$\rightsquigarrow$ offensichtlich schwer\\
deshalb suche andere Lösung
\end{comment}
\begin{comment}
Als nächstes könnte man die Stokes Struktur des zerlegten meromorphen
Zusammenhangs zu berechnen.
Dazu möchte man die Lösungen der einzelnen Summanden asymptotisch
approximieren. 
\end{comment}

% vim:set ft=tex foldmethod=marker foldmarker={{{,}}}:
