%\chapter*{Schluss} 
\addchap{Ausblick}
\begin{comment}
%TODO: bekannt? woher?
Dies bestätigt die ohnehin bekannte Aussage, dass in unserem Beispiel keine
konvergente Zerlegung zu finden ist.
\end{comment}

Ein sinnvoller nächster Schritt wäre es, zu untersuchen in wie fern sich die in
\ref{sec:LT-speziell} berechnete Zerlegung auf ein $\phi$ aus
$\{\frac{a}{x^q}\mid a\in\C, q\in\N\}$ verallgemeinern lässt.
Die kanonisch darauffolgende Verallgemeinerung wäre, ein $\phi$
$\in\{\phi=\sum_{k\in I}\frac{a_k}{t^{k}}|I\subset\N\mbox{ endlich} ,a_k\in
\C\}$ zu betrachten und für dieses die Zerlegung von $\cM_\phi$ explizit zu
berechnen.

Auch könnte man versuchen, die von Takuro Mochizuki genutzte
Monodromiebetrachtung nachzuvollziehen und mit dem hier berechneten Ergebnissen
in bezug bringen.

Möchte man irreguläre meromorphe Zusammenhänge bzw.  irregulär singuläre
$\cD$-Moduln genauer klassifizieren, können Ideen und Konzepte aus der
asymptotische Analysis verwendet werden.
Einen vielversprechenden Ansatz liefert die Berschreibung der Asymptotik der
Lösungen anhand der sogenannten \glqq{}Stokes-Struktur\grqq{}.
Durch die Stokes Struktur wird die Änderung der asymtotischen Lösung auf
Sektoren der Pole kodiert.
Auf diesem Gebiet gibt es noch viel Forschungsbedarf, so hat sich bisher nur im
Eindimmensionalen eine einheitliche Theorie entwickelt. Für höherdimmensionale
Fragestellungen sind immer noch die meisten Fragen offen.
Das Ziel ist es,
In Anlehnung an die Riemann-Hilbert-Korrespondenz, irregulär singuläre
$\cD$-Moduln anhand der Lösung und ihrer Stokes-Struktur vollständig
beschreiben zu können.
Vergleiche dazu beispielsweise \cite{sabbah2013introduction} oder
\cite{citeulike:8523004}.

Das Berechnen solch einer Stokes-Struktur ist im Allgemeinem sehr
anspruchsvoll.
Es gibt mehrere Ansätze dies zu vereinfachen. Ein solche Ansatz ist, den
formalisierten meromorphen Zusammenhang zunächst in elementare formale
meromorphe Zusammenhänge zu zerlegen.
Dann kann man die Asymptotik der Lösungen jedes einzelnen Summanden der
Levelt-Turrittin-Zerlegung gesondert untersuchen.
\begin{comment}
Ein Versuch, dies zu vereinfachen ist, den fraglichen meromorphen Zusammenhang
zunächst, wie hier, in elementare meromorphe Zusammenhänge zu zerlegen und
schließlich die Asymptotik der zugehörigen Lösungen zu betrachten.
\end{comment}
Auf diesen Ergebnissen aufbauend ließe sich in sinnvoller Weise eine
weiterführende Arbeit anschließen, in der die zugehörige Stokes-Struktur
explizit bestimmt wird.

\begin{comment}
Stokes Struktur ausrechen? Dazu die Lösung asymptotisch approximieren
$\rightsquigarrow$ offensichtlich schwer\\
deshalb suche andere Lösung
\end{comment}
\begin{comment}
Als nächstes könnte man die Stokes Struktur des zerlegten meromorphen
Zusammenhangs zu berechnen.
Dazu möchte man die Lösungen der einzelnen Summanden asymptotisch
approximieren. 
\end{comment}

% vim:set ft=tex foldmethod=marker foldmarker={{{,}}}:
