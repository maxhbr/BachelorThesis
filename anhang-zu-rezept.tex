\chapter{Genaueres zu $(x^2\partial_x)^{k}$} \label{chap:zu-rezept}
Nun wollen wir noch $(x^2\partial_x)^{k+1}$ besser verstehen.
\begin{align*}
(x^2\partial_x)^{k+1} &=x^2\myubracket{\partial_xx^2}\partial_x
                        (x^2\partial_x)^{k-1}\\
                      &=x^2\myobracket{(2x+x^{2}\partial_x)}\partial_x
                        (x^2\partial_x)^{k-1}\\
                      &=(2x^3\partial_x+x^{4}\partial_x^2)
                        (x^2\partial_x)^{k-1}\\
                      &=(2x^3\partial_x+x^{4}\partial_x^2)(x^2\partial_x)
                        (x^2\partial_x)^{k-2}\\
                      &=(2x^3\myubracket{\partial_xx^2}\partial_x
                        +x^{4}\myubracket{\partial_x^2x^2}\partial_x)
                        (x^2\partial_x)^{k-2}\\
                      &=(2x^3\myobracket{(2x+x^{2}\partial_x)}\partial_x
                        +x^{4}\myobracket{(2x\partial_x+1+x^2\partial_x^2)}
                        \partial_x) (x^2\partial_x)^{k-2}\\
                      &=(4x^4\partial_x+2x^{5}\partial_x^2
                        +2x^{5}\partial_x^2
                        +x^4\partial_x
                        +x^6\partial_x^3)
                        (x^2\partial_x)^{k-2}\\
                      &=(5x^4\partial_x+4x^{5}\partial_x^2
                        +x^6\partial_x^3)
                        (x^2\partial_x)^{k-2}\\
                      &=\sum_{n=1}^{k+1}\binom{k}{n-1}\frac{(k+1)!}{n!}x^{n+k}
                        \partial_x^{n}
\end{align*}
\begin{comment}
Stirlingzahlen
\end{comment}
also gilt für spezielle $k$
\begin{equation} \label{eq:rezeptNeben1}
(x^2\partial_x)^{k+1}=
\begin{cases}
  2x^3\partial_x+x^{4}\partial_x^2 & \mbox{ falls } k=1\\
  5x^4\partial_x+4x^{5}\partial_x^2 +x^6\partial_x^3 & \mbox{ falls } k=2\\
  \sum_{n=1}^{k+1}\binom{k}{n-1}\frac{(k+1)!}{n!}x^{n+k} \partial_x^{n}
\end{cases}
\end{equation}
