\chapter{Quelltexte}

\section{ComplRat.hs}
Das Modul \text{ComplRat} implementiert die Zahlen $\Q(i)$.
\lstinputlisting[language=HaskellUlisses
                ,firstnumber=1
                %,captionpos=b
                ,caption=ComplRat.hs]{../haskell/ComplRat.hs}
Hier ist \texttt{:+:} ein Infix-Konstruktor der Klasse \texttt{ComplRat} und
erzeugt mit einem Aufruf der Form \texttt{a :+: b} eine Imaginärzahl, die
$a+ib$ entspricht.

\section{Koeffs.hs}
Dieses Modul \texttt{Koeffs} stelle die Funktionen \texttt{vKoeffs} und
\texttt{uKoeffs} bereit, welche zu einem gegebenem Wert von $u_{-2}$ eine
unendliche Liste der Koeffizienten generieren. Die Einträge in der Liste sind
vom Typ \texttt{ComplRat}.
\texttt{ComplRat} ermöglicht es, dass die Berechung ohne numerische Fehler
erfolgt, da nie gerundet wird.
\lstinputlisting[language=HaskellUlisses
                ,firstnumber=1
                %,captionpos=b
                ,caption=Koeffs.hs]{../haskell/Koeffs.hs}
Beispielhaft kann man mit dem folgendem Programm die Koeffizienten von $v(t)$,
zu $u_{-2}=i$, erzeugen lassen.
\lstinputlisting[language=HaskellUlisses
                ,firstnumber=1
                %,captionpos=b
                ,caption=testKoeffs.hs]{../haskell/testKoeffs.hs}
Ist der Code in einer Datei \textbf{/Pfad/zu/testKoeffs.hs} gespeicher, so
lässt er sich in Unix-Artigen Systemen beispielsweise mit den folgenden
Befehlen compilieren und ausführen.
\begin{lstlisting}[style=Bash]
$ ghc --make /Pfad/zu/testKoeffs.hs
$ /Pfad/zu/testKoeffs 15
\end{lstlisting}
Durch das Ausführen berechnet das Programm die Koeffizienten von $v$ bis zum
Index $15$ und gibt in der Konsole das folgende aus
\begin{lstlisting}[style=Bash]
n       | v_n
--------+----------------------------------------------------------------------
-1      | 1 % 2
0       | (0 % 1+i3 % 4)
1       | 3 % 2
2       | (0 % 1+i(-63) % 16)
3       | (-27) % 2
4       | (0 % 1+i1899 % 32)
5       | 324 % 1
6       | (0 % 1+i(-543483) % 256)
7       | (-32427) % 2
8       | (0 % 1+i72251109 % 512)
9       | 2752623 % 2
10      | (0 % 1+i(-30413055339) % 2048)
11      | (-175490226) % 1
12      | (0 % 1+i9228545313147 % 4096)
13      | 31217145174 % 1
14      | (0 % 1+i(-30419533530730323) % 65536)
15      | (-14741904895227) % 2
\end{lstlisting}
Übersetzt in unsere Zahlenschreibweise ergibt sich daraus die folgende Tabelle:
%\ref{tab:koeff_a=0.125}
\begin{table}[H] %htbp]
\begin{center}
\begin{tabular}{|r||r|}
\hline
n        & $v_n$
\\\hline\hline
  -1 & $\frac{1}{2}$
\\0  & $\frac{3}{4}i$
\\1  & $\frac{3}{2}$
\\2  & $-\frac{63}{16}i$
\\3  & $-\frac{27}{2}$
\\4  & $\frac{1899}{32}i$
\\5  & $\frac{342}{1}$
\\6  & $-\frac{543483}{256}i$
\\7  & $-\frac{32427}{2}$
\\8  & $\frac{72251109}{512}i$
\\9  & $\frac{2752623}{2}$
\\10 & $-\frac{30413055339}{2048}i$
\\11 & $-\frac{175490226}{1}$
\\12 & $\frac{9228545313147}{4096}i$
\\13 & $\frac{31217145174}{1}$
\\14 & $-\frac{30419533530730323}{65536}i$
\\15 & $-\frac{14741904895227}{2}$
\\\hline
\end{tabular}
\caption{Numerisch berechnete Koeffizienten von $v(t)$ für $u_{-2}=i$ bzw.
  $a=\frac{1}{8}$}
\label{tab:koeff_a=0.125}
\end{center}
\end{table}

\section{GenDataToFile.hs}
\lstinputlisting[language=HaskellUlisses
                ,firstnumber=1
                %,captionpos=b
                ,caption=Main.hs]{../haskell/Main.hs}

% vim: set ft=tex :
