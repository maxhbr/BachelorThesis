\chapter{Quelltexte}

\section{ComplRat.hs}
Das Modul \text{ComplRat} implementiert die Zahlen $\Q(i)$.
\lstinputlisting[language=HaskellUlisses
                ,firstnumber=1
                %,captionpos=b
                ,caption=ComplRat.hs]{../haskell/ComplRat.hs}
Hier ist \texttt{:+:} ein Infix-Konstruktor der Klasse \texttt{ComplRat} und
erzeugt mit einem Aufruf der Form \texttt{a :+: b} eine Imaginärzahl, die
$a+ib$ entspricht.

\section{Koeffs.hs} \label{sec:Koeffs.hs}
Dieses Modul \texttt{Koeffs} stelle die Funktionen \texttt{vKoeffs} und
\texttt{uKoeffs} bereit, welche zu einem gegebenem Wert von $u_{-2}$ eine
unendliche Liste der Koeffizienten generieren. Die Einträge in der Liste sind
vom Typ \texttt{ComplRat}.
Dies ermöglicht es, dass die Berechung ohne numerische Fehler erfolgt, da nie
gerundet wird.
\lstinputlisting[language=HaskellUlisses
                ,firstnumber=1
                %,captionpos=b
                ,caption=Koeffs.hs]{../haskell/Koeffs.hs}
Beispielhaft kann man mit dem folgendem Programm die Koeffizienten von $v(t)$,
zu $a=\frac{1}{8}$ also $u_{-2}=i=2i\sqrt{2a}$, erzeugen lassen.
\lstinputlisting[language=HaskellUlisses
                ,firstnumber=1
                %,captionpos=b
                ,caption=testKoeffs.hs]{../haskell/testKoeffs.hs}
Ist der Code in einer Datei \textbf{/Pfad/zu/testKoeffs.hs} gespeicher, so
lässt er sich in Unix-Artigen Systemen beispielsweise mit den folgenden
Befehlen compilieren und ausführen.
\begin{lstlisting}[style=Bash]
$ ghc --make /Pfad/zu/testKoeffs.hs
$ /Pfad/zu/testKoeffs 15
\end{lstlisting}
Durch das Ausführen berechnet das Programm die Koeffizienten von $v$ bis zum
Index $15$ und gibt in der Konsole das folgende aus
\begin{lstlisting}[style=Bash]
n       | v_n
--------+----------------------------------------------------------------------
-1      | 1 % 2
0       | (0 % 1+i3 % 4)
1       | 3 % 2
2       | (0 % 1+i(-63) % 16)
3       | (-27) % 2
4       | (0 % 1+i1899 % 32)
5       | 324 % 1
6       | (0 % 1+i(-543483) % 256)
7       | (-32427) % 2
8       | (0 % 1+i72251109 % 512)
9       | 2752623 % 2
10      | (0 % 1+i(-30413055339) % 2048)
11      | (-175490226) % 1
12      | (0 % 1+i9228545313147 % 4096)
13      | 31217145174 % 1
14      | (0 % 1+i(-30419533530730323) % 65536)
15      | (-14741904895227) % 2
\end{lstlisting}
Übersetzt in unsere Zahlenschreibweise ergibt sich daraus die folgende Tabelle:
%\ref{tab:koeff_a=0.125}
\begin{table}[H] %htbp]
\begin{center}
\begin{tabular}{|r||r|}
\hline
n        & $v_n$
\\\hline\hline
  -1 & $\frac{1}{2}$
\\0  & $\frac{3}{4}i$
\\1  & $\frac{3}{2}$
\\2  & $-\frac{63}{16}i$
\\3  & $-\frac{27}{2}$
\\4  & $\frac{1899}{32}i$
\\5  & $\frac{342}{1}$
\\6  & $-\frac{543483}{256}i$
\\7  & $-\frac{32427}{2}$
\\8  & $\frac{72251109}{512}i$
\\9  & $\frac{2752623}{2}$
\\10 & $-\frac{30413055339}{2048}i$
\\11 & $-\frac{175490226}{1}$
\\12 & $\frac{9228545313147}{4096}i$
\\13 & $\frac{31217145174}{1}$
\\14 & $-\frac{30419533530730323}{65536}i$
\\15 & $-\frac{14741904895227}{2}$
\\\hline
\end{tabular}
\caption{Numerisch berechnete Koeffizienten von $v(t)$ für $u_{-2}=i$ bzw.
  $a=\frac{1}{8}$}
\label{tab:koeff_a=0.125}
\end{center}
\end{table}

\begin{comment}
\begin{align*}
v(t)&=
  \frac{1}{2}t^{-1}+
  \frac{3}{4}it^{0}+
  \frac{3}{2}t^{1}+
  \frac{-63}{16}it^{2}+
  \frac{-27}{2}t^{3}+
  \frac{1899}{32}it^{4}+
  \frac{324}{1}t^{5}+
  \frac{-543483}{256}it^{6}+
\\&\qquad\frac{-32427}{2}t^{7}+
  \frac{72251109}{512}it^{8}+
  \frac{2752623}{2}t^{9}+
  \frac{-30413055339}{2048}it^{10}+
  \frac{-175490226}{1}t^{11}+
\\&\qquad\frac{9228545313147}{4096}it^{12}+
  \frac{31217145174}{1}t^{13}+
  \frac{-30419533530730323}{65536}it^{14}+
\\&\qquad\frac{-14741904895227}{2}t^{15}+
  \frac{16317191917079376129}{131072}it^{16}+
  \frac{4456057685561073}{2}t^{17}+
\\&\qquad\frac{-22082325223708363779009}{524288}it^{18}+
  \frac{-1677161966915352627}{2}t^{19}+
\\&\qquad\frac{18391039987731669876160557}{1048576}it^{20}+
  \frac{384452768592440499024}{1}t^{21}+
\\&\qquad\frac{-73930258776609869550094166319}{8388608}it^{22}+
  \frac{-210878717949731493002826}{1}t^{23}+
\\&\qquad\frac{88204980719873920964105544038937}{16777216}it^{24}+
  \frac{136346686011011135869054074}{1}t^{25}+
\\&\qquad\frac{-246474684300724210330466557670749827}{67108864}it^{26}+
\\&\qquad\frac{-102614997677451303311734530276}{1}t^{27}+
\\&\qquad\frac{398608966820777951112056743321778108571}{134217728}it^{28}+
\\&\qquad\frac{88929857099067937229443324337874}{1}t^{29}+
\\&\qquad\frac{-11819876688678190917510659802435441505814403}{4294967296}it^{30}+
  \dots
\end{align*}
\end{comment}

\section{SaveToFile.hs}
\lstinputlisting[language=HaskellUlisses
                ,firstnumber=1
                %,captionpos=b
                ,caption=SaveToFile.hs]{../haskell/SaveToFile.hs}

In diesem Modul werden zusätzlich die Module \texttt{Data.Number.CReal} und
\texttt{Control.Monad.Parallel} eingebunden.
Die Datentyp \texttt{CReal} des ersten Moduls implementiert die Reellen Zahlen
und wird verwendet, da die berechneten Zahlen den Zahlenbereich des Floating
Datentypes übersteigen.
Das zweite Modul stellt eine veränderte Version von \texttt{sequence\_} bereit,
welches automatisch die ausführungen parallelisiert.
\begin{comment}
Beide Pakete lassen sich mittels dem Cabal Befehl
\begin{lstlisting}[style=Bash]
cabal install numbers monad-parallel
\end{lstlisting}
installieren
\end{comment}

\iffalse
Das folgende Script nutzt 7 Prozessoren, um mit \texttt{SaveToFile.hs} die
ersten 10000 werte zu berechnen. Anschließend werden mittels \texttt{gnuplot}
die Plots als PDF erzeugt.
\begin{lstlisting}[style=Bash,caption=GeneratePlots.sh]
#!/bin/sh
max=10000
ghc --make -threaded ./SaveToFile.hs
mkdir -p ./data
./SaveToFile $max +RTS -N7
mkdir -p ./plot
art[2]="betrag"; art[3]="cauchy"; art[4]="quot";
for i in 2 3 4; do
  name="${art[i]}"
  echo $name
  gnuplot << EOF
set samples 1001
set key below
set term push #TODO: unnötig?
set term post enh color lw 1 12 "Times-Roman"
set output "${name}.eps"
set log xy
plot for [fn in system("ls data/*")] fn every ::0::${max} using 1:${i}\
  with lines title system("basename ".fn)
EOF
  epstopdf "${name}.eps" --outfile "./plot/${name}.pdf"
  rm "${name}.eps"
done
\end{lstlisting}
\fi

% vim: set ft=tex :
