% ausformulierte version eines beweises im paper sabbah_Fourier-local.pdf zu
% lemma 2.4
% vollständig, bis auf schluss welcher vlt nicht ganz formal ist

\begin{comment}
  sabbah\_Fourier-local.pdf lemma 2.4
\end{comment}

Sei $\rho:u\mapsto u^p$ und $\mu_\xi:u\mapsto\xi u$.
\begin{lem}
  \cite[Lem 2.4]{sabbah_Fourier-local}
  Für alle $\phi \in \C(\!(u)\!)$ gilt
  \[ \rho^+\rho_+\sE^\phi=\bigoplus_{\xi^p=1}\sE^{\phi\circ\mu_\xi} \,. \]
\end{lem}

\begin{proof}

Wir wählen eine $\C(\!(u)\!)$ Basis $\{e\}$ von $\sE^\phi$ und zur
vereinfachung nehmen wir an, dass $\phi\in u^{-1}\C[u^{-1}]$
\footnote{$\sE^\phi=\sE^\psi$ $\Leftrightarrow$ $\phi\equiv\psi\mod\C[[u]]$}.\\
Dann ist die Familie $e,ue,...,u^{p-1}e$ eine $\C(\!(t)\!)$-Basis von
$\rho_+\sE^\phi$.\\
Setze $e_k=u^{-k}\otimes_{\C(\!(t)\!)}u^ke$.
Dann ist die Familie $\mathbf{e}=(e_0,...,e_{p-1})$ eine $\C((u))$-Basis von
$\rho^+\rho_+\sE^\phi$.\\
Zerlege nun $u\phi'(u)=\sum_{j=0}^{p-1}u^j\psi_j(u^p)$ $\in u^{-2}\C[u^{-1}]$
mit $\psi_j\in\C[t^{-1}]$ für alle $j>0$ und $\psi_0\in t^{-1}\C[u^{-1}]$
(siehe: Anhang \ref{chap:aufteilung}).

Sei $P$ die Permutationsmatrix, definiert durch
$\mathbf{e}\cdot P=(e_1,...,e_{p-1},e_0)$
\footnote{$P=\begin{pmatrix}0 &  &  & 1\\
1 & 0\\
 & \ddots & \ddots\\
 &  & 1 & 0
\end{pmatrix}$}.\\
Es gilt:\\
\[ u\partial_ue_k= \sum_{i=0}^{p-1-k}u^i\psi_i(u^p)e_{k+1} +
  \sum_{i=p-k}^{p-1}u^i\psi_i(u^p)e_{k+i-p} \]
denn:\\
\begin{align*}
  u\partial_ue_k &= u\partial_u(u^{-k}\otimes_{\C(\!(t)\!)}u^ke)\\
  &= u(-ku^{-k-1}\otimes_{\C(\!(t)\!)}u^ke +
    pu^{p-1}u^{-k}\otimes_{\C(\!(t)\!)}\partial_t
    (\underset{\in\rho_+\sE^\phi}{\underbrace{u^ke}}))\\
  &= -ku^{-k}\otimes_{\C(\!(t)\!)}u^ke +
    pu^{p-1}u^{-k+1}\otimes_{\C(\!(t)\!)}(pu^{p-1})^{-1} (ku^{k-1}e
    + u^k\phi'(u)e)\\
  &= -ku^{-k}\otimes_{\C(\!(t)\!)}u^ke +
    u^{-k+1}\otimes_{\C(\!(t)\!)}(ku^{k-1}e + u^k\phi'(u)e)\\
  &= \underset{=0}{\underbrace{-ku^{-k}\otimes_{\C((t))}u^ke +
    u^{-k+1}\otimes_{\C((t))}ku^{k-1}e}} +
    u^{-k+1}\otimes_{\C((t))}u^k\phi'(u)e\\
  &= u^{-k}\otimes_{\C((t))}u^{k+1}\phi'(u)e\\
  &= \sum_{i=0}^{p-1}u^{-k}\otimes_{\C((t))}
    u^{k}u^i\underset{\in\C((t))}{\underbrace{\psi_i(u^p)}}e\\
  &= \sum_{i=0}^{p-1}u^i\psi_i(u^p)(u^{-k}\otimes_{\C((t))} u^{k}e)\\
  &= \sum_{i=0}^{p-1-k}u^i\psi_i(u^p)e_{k+1} +
  \sum_{i=p-k}^{p-1}u^i\psi_i(u^p)e_{k+i-p}
\end{align*}

so dass gilt:
\[ u\partial_u\mathbf{e}=\mathbf{e}[\sum_{j=0}^{p-1}u^j\psi_jP^j] \]
denn:\\
\begin{align*}
  u\partial_u\mathbf{e} &= (u\partial_ue_0,...,u\partial_ue_{p-1})\\
  &= \Bigg(\sum_{i=0}^{p-1-k}u^i\psi_i(u^p)e_{k+1} +
    \sum_{i=p-k}^{p-1}u^i\psi_i(u^p)e_{k+i-p}\Bigg)_{k\in\{0,..,p-1\}}\\
  &= \mathbf{e}
  \begin{pmatrix}u^{p-1}\psi_{p-1}(u^p) &  & \cdots & u^{3}\psi_{3}(u^p) & u^{2}
    \psi_{2}(u^p) & u^{1}\psi_{1}(u^p)\\
    u^{1}\psi_{1}(u^p) & u^{p-1}\psi_{p-1}(u^p) &  &
    & \ddots & u^{2}\psi_{2}(u^p)\\
    u^{2}\psi_{2}(u^p) & u^{1}\psi_{1}(u^p) & \ddots &  &  & u^{3}\psi_{3}(u^p)\\
    u^{3}\psi_{3}(u^p) & \ddots & \ddots & \ddots &  & \vdots\\
    \vdots &  & \ddots & u^{1}\psi_{1}(u^p) & u^{p-1}\psi_{p-1}(u^p)\\
    u^{p-2}\psi_{p-2}(u^p) & \cdots & u^{3}\psi_{3}(u^p) & u^{2}\psi_{2}(u^p) &
    u^{1}\psi_{1}(u^p) & u^{p-1}\psi_{p-1}(u^p)
  \end{pmatrix}\\
  &= \mathbf{e}[\sum_{j=0}^{p-1}u^j\psi_j(u^p)P^j]
\end{align*}


Die Wirkung von $\partial_u$ auf die Basis von $\rho^+\rho_+\sE^{\phi(u)}$ ist
also Beschrieben durch:
\[ \partial_u\mathbf{e}=\mathbf{e}[\sum_{j=0}^{p-1}u^{j-1}\psi_jP^j] \]

Diagonalisiere nun $TPT^{-1}=D=\begin{pmatrix}\xi^{0}\\
 & \xi^{1}\\
 &  & \ddots\\
 &  &  & \xi^{p-1}
\end{pmatrix}$\footnote{Klar, da mipo $X^p-1$}, mit $\xi^p=1$ und
$T\in Gl_p(\C)$.\\
So dass gilt:\\
\begin{align*}
  T[\sum_{j=0}^{p-1}u^{j-1}\psi_j(u^p)P^j]T^{-1} &=
    [\sum_{j=0}^{p-1}u^{j-1}\psi_j(u^p) (TPT^{-1})^j]\\
  &= [\sum_{j=0}^{p-1}u^{j-1}\psi_j(u^p)D^j]\\
  &= \begin{pmatrix}\sum_{j=0}^{p-1}u^{j-1}\psi_{j}\\
    & \sum_{j=0}^{p-1}u^{j-1}\psi_{j}\left(\xi^{1}\right)^{j}\\
    & & \ddots\\
    &  &  & \sum_{j=0}^{p-1}u^{j-1}\psi_{j}\left(\xi^{p-1}\right)^{j}
  \end{pmatrix}\\
  &= \begin{pmatrix}\sum_{j=0}^{p-1}u^{j-1}\psi_{j}\\
    & \sum_{j=0}^{p-1}(u\xi^1)^{j-1}\psi_{j}\xi^{1}\\
    & & \ddots\\
    &  &  & \sum_{j=0}^{p-1}(u\xi^{p-1})^{j-1}\psi_{j}\xi^{p-1}
  \end{pmatrix} \mbox{\footnote{$(\xi^i)^p=1\forall i$}}\\ % TODO: correct footnote
  &= \begin{pmatrix}\phi'(u)\\
    & \phi'(\xi u)\xi^{1}\\
    & & \ddots\\
    &  &  & \phi'(\xi^{p-1} u)\xi^{p-1}
  \end{pmatrix}\\
\end{align*}

Wie sieht denn die Wirkung auf die Basis von
$\bigoplus_{\xi^p=1}\sE^{\phi\circ\mu_\xi}\overset{\Phi}{\cong}\C((u))^p$ aus?

$\partial_u \begin{pmatrix} 1\\
  0\\
  \vdots\\
  0\\
\end{pmatrix}
= \begin{pmatrix} \partial_u1\\
  \partial_u0\\
  \vdots\\
  \partial_u0\\
\end{pmatrix}
= \begin{pmatrix} \phi'(u)\\
  0\\
  \vdots\\
  0\\
\end{pmatrix}
$\\
$\partial_u \begin{pmatrix} 0\\
  1\\
  0\\
  \vdots\\
  0\\
\end{pmatrix}
= \begin{pmatrix} \partial_u0\\
  \partial_u1\\
  \partial_u0\\
  \vdots\\
  \partial_u0\\
\end{pmatrix}
= \begin{pmatrix} 0\\
  \phi'(u)\\
  0\\
  \vdots\\
  0\\
\end{pmatrix}
\overset{\Phi}{\mapsto} \begin{pmatrix} 0\\
  \phi'(u)\xi\\
  0\\
  \vdots\\
  0\\
\end{pmatrix}
$
%TODO: this

Also kommutiert das Diagram:

\begin{center}
  \begin{tikzpicture} [scale=3.3, descr/.style={fill=white,inner sep=2.5pt} ]
  \matrix (m) [
    matrix of math nodes
    ,row sep=2em
    ,column sep=3em
    %,text height=3em
    %,text depth=0.25em
    ]
  {
    \rho^+\rho_+\sE^{\phi(u)} &
    \C((u))^p &
    \C((u))^p &
    \bigoplus_{\xi^p=1}\sE^{\phi\circ\mu_\xi} \\
    & \\
    & \\
    \rho^+\rho_+\sE^{\phi(u)} &
    \C((u))^p &
    \C((u))^p &
    \bigoplus_{\xi^p=1}\sE^{\phi\circ\mu_\xi} \\
  };
    \path[->,font=\scriptsize,>=angle 90]
    (m-1-2) edge node[below]{$\cong$} (m-1-1)
    (m-1-1) edge node[right]{$\partial_u$} (m-4-1)
    (m-4-2) edge node[below]{$\cong$} (m-4-1)
    (m-1-3) edge node[above]{$T$} node[below]{$\cong$} (m-1-2)
    (m-4-3) edge node[above]{$T$} node[below]{$\cong$} (m-4-2)
    (m-1-2) edge node[descr]{$\sum_{j=0}^{p-1}u^{j-1}\psi_jP^j$} (m-4-2)
    (m-1-3) edge node[descr]{$\sum_{j=0}^{p-1}u^{j-1}\psi_jD^j$} (m-4-3)
    (m-1-3) edge node[above]{$\Phi$} node[below]{$\cong$} (m-1-4)
    (m-4-3) edge node[below]{$\cong$} (m-4-4)
    (m-1-4) edge node[right]{$\partial_u$} (m-4-4)
    ;
  \end{tikzpicture}
\end{center}

Und deshalb ist klar ersichtlich das auf $\rho^+\rho_+\sE^{\phi(u)}$ und
$\sum_{j=0}^{p-1}u^{j-1}\psi_jD^j$ ein Äquivalenter Meromorpher Zusammenhang
definiert ist.

\end{proof}
