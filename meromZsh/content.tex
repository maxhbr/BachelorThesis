% TODO; wie darf ich einen Meromorphen verändern, (das P verändern) ohne das
% sich effektiv was ändert?

\chapter{Der Meromorpher Zusammenhang}
Quelle ist \cite{sabbah_cimpa90}
\section{Definition}~

\begin{defn}[Meromorpher Zusammenhang]
  Ein (\emph{Keim eines}) \emph{Meromorpher Zusammenhang} (an $x=0$)
  $(\cM_K,\partial)$ besteht aus folgenden Daten:
  \begin{itemize}
    \item $\cM_K$, ein endlich dimensionaler $K$-Vr
    \item einer $\C$-linearen Abbildung $\partial:\cM_K\rightarrow \cM_K$, genannt
      \emph{Derivation}, welche für alle $f\in K$ und $u\in \cM_K$ die
      \emph{Leibnitzregel}
      \begin{equation}\label{eq:Leibnitzregel}
        \partial(fu)=f'u+f\partial u
      \end{equation}
      erfüllen soll.
  \end{itemize}
\end{defn}

\begin{bem}
  Später wird man auf die angabe von $\partial$ verichten und einfach $\cM$ als
  den Meromorphen Zusammenhang bezeichnen.
\end{bem}

\section{Eigenschaften}
Hier nun einige Eigenschaften Meromorpher Zusammenhänge.

\begin{lem}
  %TODO: hier $\cM ??
  Sei $(\cM,\partial)$ ein gegebener Meromorpher Zusammenhang, und $\phi$ ein
  Basisisomorphismus von $K^r$ nach $\cM$, also in der Situation
  \begin{center}
    \begin{tikzpicture} [scale=3.3, descr/.style={fill=white,inner sep=2.5pt} ]
    \matrix (m) [
      matrix of math nodes
      , row sep=3em
      , column sep=3em
      %, text height=3em
      %, text depth=0.25em
      ]
    {
      \cM & \cM \\
      K^r & K^r \\
    };
      \path[->,font=\scriptsize,>=angle 90]
      (m-1-1) edge node[above]{$\partial$} (m-1-2)
      (m-2-1) edge node[above]{$\phi^{-1}\partial\phi$} (m-2-2)
      ;
      %TODO: make this harpoon arrows
      \path[->,font=\scriptsize,>=angle 90]
      (m-2-1) edge node[descr]{$\cong$} node[right]{$\phi$} (m-1-1)
      (m-2-2) edge node[descr]{$\cong$} node[left]{$\phi$} (m-1-2)
      ;

      \path[>=stealth,|->]
      ;
    \end{tikzpicture}
  \end{center}
  gilt: $(K^r,\phi^{-1}\partial\phi)$ ist ebenfalls ein Meromorpher Zusammenhang.
\end{lem}
\begin{proof}
  TODO, (3. Treffen)
\end{proof}

Sind $\partial_1$ und $\partial_2$ zwei Meromorphe Zusammenhänge auf $\cM_K\cong
K^r$. So betrachte $\partial_1-\partial_2:\cM\rightarrow\cM$ für alle $f\in K$ 
und $u\in \cM_K$ :
\begin{align*}
  (\partial_1-\partial_2)(fu) &= \partial_1(fu)-\partial_2(fu)\\
  &= f'u+f\partial_1u-f'u-f\partial_2u\\
  &= f\cdot(\partial_1-\partial_2)(u)\\
\end{align*}
\begin{lem}
  Da $\partial_1-\partial_2$ $\C$-linear und, wie eben gezeigt,
  $(\partial_1-\partial_2)(fu)=f\cdot(\partial_1-\partial_2)(u)$ allgemein
  gilt: Die differenz zweier Meromorpher Zusammenhäge ist $K$-linear.
  %Differenz zweier Meromorpher Zshg. ist K-linear
\end{lem}
Insbesondere ist $\frac{d}{dz}-\partial:K^r\rightarrow K^r$ $K$-linear, also es
existiert eine Matrix $A\in M(r\times r,K)$ mit $\frac{d}{dz}-\partial=A$, also
ist $\partial=\frac{d}{dz}-A$.

%TODO: beobachtung...

%TODO: differenz ist linear

\begin{defn}[Transformationsformel]
  In der Situation

  \begin{center}
    \begin{tikzpicture} [scale=3.3, descr/.style={fill=white,inner sep=2.5pt} ]
    \matrix (m) [
      matrix of math nodes
      , row sep=3em
      , column sep=3em
      %, text height=3em
      %, text depth=0.25em
      ]
    {
      K^r & & & K^r \\
       & M & M & \\
      K^r & & & K^r \\
    };
      \path[->,font=\scriptsize,>=angle 90]
      (m-1-1) edge node[descr]{$\cong$} node[above]{$\phi$} (m-2-2)
      (m-3-1) edge node[descr]{$\cong$} node[above]{$\psi$} (m-2-2)
      (m-1-4) edge node[descr]{$\cong$} node[above]{$\phi$} (m-2-3)
      (m-3-4) edge node[descr]{$\cong$} node[above]{$\psi$} (m-2-3)

      (m-2-2) edge node[above]{$\partial$} (m-2-3)

      (m-1-1) edge node[above]{$\frac{d}{dz}+A$} (m-1-4)
      (m-3-1) edge node[above]{$\frac{d}{dz}+B$} (m-3-4)

      (m-3-1) edge node[descr]{$\cong$} node[right]{$T$} (m-1-1)
      (m-3-4) edge node[descr]{$\cong$} node[left]{$T$} (m-1-4)
      ;

      \path[>=stealth,|->]
      ;
    \end{tikzpicture}
  \end{center}
  mit $\phi,\psi$ und $T$ $K$-Linear und $\partial,(\frac{d}{dz}+A)$ und
  $(\frac{d}{dz}+B)$ $\C$-Linear, gilt:\\
  Der Merom. Zush. $\frac{d}{dz}+A$ auf $K^r$ wird durch Basiswechsel $T\in
  GL(r,K)$ zu
  \[
    \frac{d}{dz}+(T^{-1}\cdot T'+T^{-1}AT) = \frac{d}{dz}+B
  \]
\end{defn}
\begin{defn}
  $A\sim B$ differenziell Äquivalent $:\Leftrightarrow$ $\exists T\in GL(r,K)$
  mit $B=T^{-1}\cdot T'+T^{-1}AT$
\end{defn}

\begin{comment}
  $1=TT^{-1}$ $\rightsquigarrow$ $T'T^{-1}+T(T^{-1})'=0$\\
  $1=T^{-1}T$ $\rightsquigarrow$ $(T^{-1})'T+T^{-1}T'=0$
\end{comment}

\section{Formale Meromorphe Zusammenhänge}

\begin{defn}[Formaler Meromorpher Zusammenhang]
  Ein \emph{Formaler Meromorpher zusammenhang} $(\cM_K,\partial)$ besteht aus 
  folgenden Daten:
  \begin{itemize}
    \item $\cM_K$, ein endlich dimensionaler $\hat K$-Vr
    \item eine \emph{Derivation} $\partial$, für die die \emph{Leibnitzregel} 
      \eqref{eq:Leibnitzregel}, für alle $f\in \hat K$ und $u\in \cM_{\hat K}$,
      erfüllt sein soll.
  \end{itemize}
\end{defn}

\begin{comment}
  Oder einfach ein Meromorpher Zshg. über anderes $K$ also $\hat K$
\end{comment}

\section{Elementare Meromorphe Zusammenhänge}

%TODO: auch nicht formal
\begin{defn}[Elementarer formaler Zusammenhang]
  Zu einem gegebenen $\rho\in u\C\llbracket u\rrbracket$, $\phi\in\C((u))$ und
  einem endlich dimensionalen $\C((u))$-Vektorraum $R$ mit regulärem 
  Zusammenhang $\nabla$, definieren wir den assoziierten Elementaren endlich
  dimensionalen $\C((t))$-Vektorraum mit Zusammenhang, durch:
  \[
    El(\rho,\phi,R)=\rho_+(\sE\otimes R)
  \]
\end{defn}

%TODO: weitere Eigenschaften

\section{Newton Polygon} % ist dies eine Invariante??
% gestohlen aus der ZulaBarbara Seite 46

Jedes $P\in \cD$ lässt sich eindeutig schreiben als
\[ P=\sum^{n}_{k=0}{\sum^{\infty}_{l=-N}{\alpha_{kl}t^l\partial_t^k}} \]
mit $\alpha_{kl}\in\C$ schreiben und betrachte das dazugehörige
\[ H:=\underset{k,l\mbox{ mit }\alpha_{kl}\neq0}{\bigcup}\{ (k,l-k) +
  \R_{\leq 0}\times \R_{\geq 0} \}\subset \R^2 \,. \]
\begin{comment}
  bei sabbah: $H\subset \N\times\Z$ und dann konvexe hülle davon in $\R^2$
\end{comment}

\begin{defn} % aus der zula
  Das Randpolygon von $\conv(H)$ heißt das \emph{Newton Polygon} von $P$ und
  wird geschrieben als $N(P)$.
\end{defn}

\begin{defn} % aus der zula
  Die \emph{Steigungen (engl. slopes)} sind die nicht-vertikalen Steigungen von
  $N(P)$, die sich echt rechts von $\{0\}\times\R$ befinden.\\ % der $y$-Achse
  %TODO: bessere formulierung
  %TODO: y-Achse korrekt?
  P heißt \emph{regulär singulär} $:\Leftrightarrow$ $\slopes{P}=\{0\}$, sonst
  \emph{irregulär singulär}.
\end{defn}

% vim: set ft=tex :
