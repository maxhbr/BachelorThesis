\chapter{Der Meromorpher Zusammenhang}
Quelle ist \cite{sabbah_cimpa90}
\section{Definition}~

\begin{defn}[Meromorpher Zusammenhang]
  Ein \emph{Meromorpher zusammenhang} (Bzw. besser \emph{Keim eines Meromorphen
  Zusammenhangs}) $(\M_K,\partial)$ besteht aus folgenden Daten:
  \begin{itemize}
    \item $\M_K$, ein endlich dimensionaler $K$-Vr
    \item einer $\C$-linearen Abbildung $\partial:\M_K\rightarrow \M_K$, genannt
      \emph{Derivation}. Wobei für alle $f\in K$ und $u\in \M_K$ die
      \emph{Leibnitzregel}
      \begin{equation}\label{eq:Leibnitzregel}
        \partial(fu)=f'u+f\partial u
      \end{equation}
      erfüllt sein soll.
  \end{itemize}
\end{defn}

\section{Eigenschaften}
Hier nun einige Eigenschaften von Meromorphen Zusammenhängen.

\begin{lem}
  Sei $(M,\partial)$ ein gegebener Meromorpher Zusammenhang, und $\phi$ ein
  Basisisomorphismus von $K^r$ nach $\M$, also in der Situation
  \begin{center}
    \begin{tikzpicture} [scale=3.3, descr/.style={fill=white,inner sep=2.5pt} ]
    \matrix (m) [
      matrix of math nodes
      , row sep=3em
      , column sep=3em
      %, text height=3em
      %, text depth=0.25em
      ]
    {
      \M & \M \\
      K^r & K^r \\
    };
      \path[->,font=\scriptsize,>=angle 90]
      (m-1-1) edge node[above]{$\partial$} (m-1-2)
      (m-2-1) edge node[above]{$\phi^{-1}\partial\phi$} (m-2-2)
      ;
      %TODO: make this harpoon arrows
      \path[->,font=\scriptsize,>=angle 90]
      (m-2-1) edge node[descr]{$\cong$} node[right]{$\phi$} (m-1-1)
      (m-2-2) edge node[descr]{$\cong$} node[left]{$\phi$} (m-1-2)
      ;

      \path[>=stealth,|->]
      ;
    \end{tikzpicture}
  \end{center}
  gilt: $(K^r,\phi^{-1}\partial\phi)$ ist ebenfalls ein Meromorpher Zusammenhang.
\end{lem}
\begin{proof}
  TODO, (3. Treffen)
\end{proof}

Sind $\partial_1$ und $\partial_2$ zwei Meromorphe Zusammenhänge auf $\M_K\cong
K^r$. So betrachte $\partial_1-\partial_2:\M\rightarrow\M$für alle $f\in K$ und
$u\in \M_K$ :
\begin{eqnarray*}
  (\partial_1-\partial_2)(fu) & = & \partial_1(fu)-\partial_2(fu)\\
  & = & f'u+f\partial_1u-f'u-f\partial_2u\\
  & = & f\cdot(\partial_1-\partial_2)(u)\\
\end{eqnarray*}
\begin{lem}
  Da $\partial_1-\partial_2$ $\C$-linear und, wie eben gezeigt,
  $(\partial_1-\partial_2)(fu)=f\cdot(\partial_1-\partial_2)(u)$ allgemein
  gilt: Die differenz zweier Meromorpher Zusammenhäge ist $K$-linear.
  %Differenz zweier Meromorpher Zshg. ist K-linear
\end{lem}
Insbesondere ist $\frac{d}{dz}-\partial:K^r\rightarrow K^r$ $K$-linear, also es
existiert eine Matrix $A\in M(r\times r,K)$ mit $\frac{d}{dz}-\partial=A$, also
ist $\partial=\frac{d}{dz}-A$.

%TODO: beobachtung...

%TODO: differenz ist linear

\begin{defn}[Transformationsformel]
  In der Situation

  \begin{center}
    \begin{tikzpicture} [scale=3.3, descr/.style={fill=white,inner sep=2.5pt} ]
    \matrix (m) [
      matrix of math nodes
      , row sep=3em
      , column sep=3em
      %, text height=3em
      %, text depth=0.25em
      ]
    {
      K^r & & & K^r \\
       & M & M & \\
      K^r & & & K^r \\
    };
      \path[->,font=\scriptsize,>=angle 90]
      (m-1-1) edge node[descr]{$\cong$} node[above]{$\phi$} (m-2-2)
      (m-3-1) edge node[descr]{$\cong$} node[above]{$\psi$} (m-2-2)
      (m-1-4) edge node[descr]{$\cong$} node[above]{$\phi$} (m-2-3)
      (m-3-4) edge node[descr]{$\cong$} node[above]{$\psi$} (m-2-3)

      (m-2-2) edge node[above]{$\partial$} (m-2-3)

      (m-1-1) edge node[above]{$\frac{d}{dz}+A$} (m-1-4)
      (m-3-1) edge node[above]{$\frac{d}{dz}+B$} (m-3-4)

      (m-3-1) edge node[descr]{$\cong$} node[right]{$T$} (m-1-1)
      (m-3-4) edge node[descr]{$\cong$} node[left]{$T$} (m-1-4)
      ;

      \path[>=stealth,|->]
      ;
    \end{tikzpicture}
  \end{center}
  mit $\phi,\psi$ und $T$ $K$-Linear und $\partial,(\frac{d}{dz}+A)$ und
  $(\frac{d}{dz}+B)$ $\C$-Linear, gilt:\\
  Der Merom. Zush. $\frac{d}{dz}+A$ auf $K^r$ wird durch Basiswechsel $T\in
  GL(r,K)$ zu
  \[
    \frac{d}{dz}+(T^{-1}\cdot T'+T^{-1}AT) = \frac{d}{dz}+B
  \]
\end{defn}
\begin{defn}
  $A\sim B$ differenziell Äquivalent $:\Leftrightarrow$ $\exists T\in GL(r,K)$
  mit $B=T^{-1}\cdot T'+T^{-1}AT$
\end{defn}

\begin{comment}
  $1=TT^{-1}$ $\rightsquigarrow$ $T'T^{-1}+T(T^{-1})'=0$\\
  $1=T^{-1}T$ $\rightsquigarrow$ $(T^{-1})'T+T^{-1}T'=0$
\end{comment}

\section{Elementare Meromorphe Zusammenhänge}
\begin{comment}
  Sabbah redet in \cite{sabbah_Fourier-local} von formal meromorphic
  connenctions
\end{comment}

\begin{defn}[Elementarer formaler Zusammenhang]
  Zu einem gegebenen $\rho\in u\C\llbracket u\rrbracket$, $\phi\in\C((u))$ und
  einem endlich dimensionalen $\C((u))$-Vektorraum $R$ mit regulärem 
  Zusammenhang $\nabla$, definieren wir den assoziierten Elementaren endlich
  dimensionalen $\C((t))$-Vektorraum mit Zusammenhang, durch:
  \[
    El(\rho,\phi,R)=\rho_+(\E\otimes R)
  \]
\end{defn}

%TODO: weitere Eigenschaften

% vim: set ft=tex :
