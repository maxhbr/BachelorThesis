\chapter{Der Meromorpher Zusammenhang}
% quelle ist sabbah
\section{Definition}~

\begin{defn}[Meromorpher Zusammenhang]
  Ein \emph{Meromorpher zusammenhang} (Bzw. besser \emph{Keim eines Meromorphen
    Zusammenhangs}) besteht aus folgenden Daten:
  \begin{itemize}
    \item $M$, ein endlich dimensionaler $K$-Vr wobei $K=\C\{z-x_0\}[(z-x_0)^{-1}]$
    \item einer $\C$-linearen Abbildung $\partial:M\rightarrow M$, welche die
      \emph{Leibnitzregel}
      $$\partial(fu)=f'u+f\partial u$$
      erfüllt.
  \end{itemize}
\end{defn}

\section{Eigenschaften}
Hier nun einige Eigenschaften von Meromorphen Zusammenhängen.

Sei $(M,\partial)$ ein gegebener Meromorpher Zusammenhang, so wähle eine Basis
$(m_i)_{i\in\{1,..,r\}}$ von M. Dann gilt:
\begin{center}
  \begin{tikzpicture} [scale=3.3, descr/.style={fill=white,inner sep=2.5pt} ]
  \matrix (m) [
    matrix of math nodes
    , row sep=3em
    , column sep=3em
    %, text height=3em
    %, text depth=0.25em
    ]
  {
    M & M \\
    K^r & K^r \\
  };
    \path[->,font=\scriptsize,>=angle 90]
    (m-1-1) edge node[above]{$\partial$} (m-1-2)
    (m-2-1) edge node[above]{$\phi^{-1}\partial\phi$} (m-2-2)
    ;
    %TODO: make this harpoon arrows
    \path[->,font=\scriptsize,>=angle 90]
    (m-2-1) edge node[descr]{$\cong$} node[right]{$\phi$} (m-1-1)
    (m-2-2) edge node[descr]{$\cong$} node[left]{$\phi$} (m-1-2)
    ;

    \path[>=stealth,|->]
    ;
  \end{tikzpicture}
\end{center}
%TODO: beobachtung...

%TODO: differenz ist linear

\begin{defn}[Transformationsformel]
  In der Situation

  \begin{center}
    \begin{tikzpicture} [scale=3.3, descr/.style={fill=white,inner sep=2.5pt} ]
    \matrix (m) [
      matrix of math nodes
      , row sep=3em
      , column sep=3em
      %, text height=3em
      %, text depth=0.25em
      ]
    {
      K^r & & & K^r \\
       & M & M & \\
      K^r & & & K^r \\
    };
      \path[->,font=\scriptsize,>=angle 90]
      (m-1-1) edge node[descr]{$\cong$} node[above]{$\phi$} (m-2-2)
      (m-3-1) edge node[descr]{$\cong$} node[above]{$\psi$} (m-2-2)
      (m-1-4) edge node[descr]{$\cong$} node[above]{$\phi$} (m-2-3)
      (m-3-4) edge node[descr]{$\cong$} node[above]{$\psi$} (m-2-3)

      (m-2-2) edge node[above]{$\partial$} (m-2-3)

      (m-1-1) edge node[above]{$\frac{d}{dz}+A$} (m-1-4)
      (m-3-1) edge node[above]{$\frac{d}{dz}+B$} (m-3-4)

      (m-3-1) edge node[descr]{$\cong$} node[right]{$T$} (m-1-1)
      (m-3-4) edge node[descr]{$\cong$} node[left]{$T$} (m-1-4)
      ;

      \path[>=stealth,|->]
      ;
    \end{tikzpicture}
  \end{center}
  mit $\phi,\psi$ und $T$ $K$-Linear und $\partial,(\frac{d}{dz}+A)$ und
  $(\frac{d}{dz}+B)$ $\C$-Linear, gilt:\\
  Der Merom. Zush. $\frac{d}{dz}+A$ auf $K^r$ wird durch Basiswechsel $T\in
  GL(r,K)$ zu
  $$
  \frac{d}{dz}+(T^{-1}\cdot T'+T^{-1}AT) = \frac{d}{dz}+B
  $$
\end{defn}

% vim: set ft=tex :
