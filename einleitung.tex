\chapter{Einleitung}

Die Theorie der $\cD$-Moduln ist aus einem Versuch, Systeme von partiellen
Differentialgleichungen algebraisch zu betrachten, entstanden. Seit den 1960er
Jahren wurde diese Theorie von beispielsweise B. Malgrange, J. Bernstein, J.E.
Björk, M.Sato, T. Kawai und M. Kashiwara entwickelt und vorangetrieben.
Auch aktuell wird noch viel auf diesem Thema geforscht, so ist hier vor allem
C. Sabbah zu erwähnen, der auch zwei der Hauptquellen dieser Arbeit, namentlich
\cite{sabbah_cimpa90} und \cite{sabbah_Fourier-local}, verfasst hat.

Eines der wichtigsten Resultate für $\cD$-Moduln ist die
Riemann-Hilbert-Korrespondenz. Sie beschreibt die Kategorienäquivalenzen
\begin{center}
\begin{tikzcd}[column sep=huge]
  \left\{ \overset{\text{\normalsize Regulär singuläre}}{
   \underset{\text{\normalsize holonome $\cD$-Moduln}}{\phantom{0}}}
   \right\}\arrow[yshift=0.7ex]{r}{\textbf{DR}}&
  \left\{ \text{Perverse Garben} \right\} \,,
   \right\}\arrow[yshift=-0.7ex]{l}
\end{tikzcd}
\end{center}
wobei \textbf{DR} der de Rahm Funktor ist.
Für unseren Fall, da wir uns auf meromorphe Zusammenhänge bzw. lokalisierte
holonome $\cD$-Moduln beschränken wollen, ergibt sich
\begin{center}
\begin{tikzcd}[column sep=large]
  \left\{ \overset{\text{\normalsize Regulär singuläre}}{
   \underset{\text{\normalsize holonome $\cD$-Moduln}}{\text{lokalisierte}}}
   \right\} \arrow[leftrightarrow]{r}[description]{$1:1$}&
  \left\{ \overset{\text{\normalsize Regulär singuläre}}{
   \underset{\text{\normalsize Zusammenhänge}}{\text{meromorphe}}}
   \right\}\arrow[yshift=0.7ex]{r}&
  \left\{ \text{Lokale Systeme} \right\} \,.
   \right\}\arrow[yshift=-0.7ex]{l}
\end{tikzcd}
\end{center}
Vergleiche hierzu \cite[Sec 6]{REFKashiwara1984} oder \cite[Thm
7.2.1]{hotta2007d}.
Dieses bekannte Resultat über regulär singuläre holonome $\cD$-Moduln,
lässt sich leider nicht kanonisch auf irregulär singuläre fortsetzen.
Einen Ansatz, um ein solches Ergebnis für irregulär singuläre meromorphe
Zusammenhänge trotzdem zu erhalten, liefert die Theorie der Stokes-Strukturen.
Durch die Stokes Struktur wird die Änderung der asymtotischen Lösung auf
Sektoren der Pole Kodiert.
Vergleiche dazu beispielsweise \cite{sabbah2013introduction} oder
\cite{citeulike:8523004}.

Das berechnen solch einer Stokes-Struktur ist im Allgemeinem schwer, ein
Versuch dies zu vereinfachen ist, den fraglichen meromorphen Zusammenhang
zunächst in gewisse elementare meromorphe Zusammenhänge zu zerlegen.
Dies macht man in der Hoffnung, dass man für diese die Strukturen einfacher
berechnen kann.
Ein Satz der solch eine Zerlegung definiert bzw. beschreibt ist das
Levelt-Turrittin-Theorem.
\begin{comment}
Die Levelt-Turrittin-Zerlegung ist eine sogenannte Slope-Filtration. Diese
Klasse, der Slope-Filtrationen wird in \cite{0812.3921} betrachtet.
\end{comment}
Das Ziel dieser Arbeit ist es, in die Theorie der $\cD$-Moduln (Kapitel
\ref{chap:dModuln}) bzw. der meromorphen Zusammenhänge (Kapitel
\ref{chap:meromZsh}) kurz einzuführen und danach eine
Levelt-Turrittin-Zerlegung explizit an einem Beispiel auszuführen und zu
berechnen. Die Riemann-Hilbert-Korrespondenz und die Theorie der
Stokes-Strukturen werden in dieser Arbeit nicht weiter betrachtet.

\begin{comment}
Es wird in dieser Arbeit kein Vorwissen über $\cD$-Moduln bzw. meromorphe
Zusammenhänge vorausgesetzt, diese beiden Begriffe werden in den ersten Zwei
Kapiteln eingeführt.
\end{comment}

Im ersten Kapitel wird sehr knapp die Theorie der $\cD$-Moduln eingeführt, da
uns diese im später sehr nützlich sein wird. Hier wurde, vor allem bei der
Definition von holonomen $\cD$-Moduln, viel Hintergrund weggelassen, da dieser
nicht in den Rahmen einer solchen Arbeit passt.\\
\begin{comment} \end{comment}
Im zweiten Kapitel werden die meromorphen Zusammenhänge definiert, diese sind
das wichtigste mathematische Objekt in dieser Arbeit.
Ein wichtiges Resultat in diesem Kapitel, ist die Eins-zu-Eins Korrespondenz
von meromorphen Zusammenhängen zu holonomen lokalisierten $\cD$-Moduln, da man
mit letzteren gut rechnen kann.\\
\begin{comment} \end{comment}
Das dritte Kapitel beschäftigt sich ausschließlich mit Operationen auf
meromorphen Zusammenhängen.
Vorgestellt werden Tensorprodukt, Pullback und Pushforward,
Fouriertransformation, die sogenannte Betrachtung bei Unendlich und das Twisten
eines meromorphen Zusammenhangs.
Diese Operationen werden in den folgenden zwei Kapiteln Verwendung finden.\\
\begin{comment} \end{comment}
Im vierten Kapitel soll das bereits erwähnte Levelt-Turrittin-Theorem
vorgestellt werden, welches es erlaubt, einen meromorphen Zusammenhang in
elementare meromorphe Zusammenhänge zu zerlegen.\\
\begin{comment} \end{comment}
Das letzte Kapitel hat das Ziel die Levelt-Turrittin-Zerlegung auf eine Klasse
von meromorphen Zusammenhängen anzuwenden.
Dazu wird zu Beginn ein Rezept gegeben, welches die gewünschten Zusammenhänge
beschreibt.
Zu diesen Zusammenhängen werden zunächst allgemeine Aussagen getroffen bevor in
Abschnitt \ref{sec:LT-speziell} ganz konkret eine Levelt-Turrittin-Zerlegung zu
einem meromorphen Zusammenhang berechnet wird.

Die im letzten Kapitel explizit berechnete Levelt-Turrittin-Zerlegung könnte
weiter verwendet werden, um die entsprechende Stokes-Struktur explizit zu
berechnen. Dies würde sich beispielsweise gut als Thema für eine weiterführende
Arbeit eignen.

\begin{comment}
Ich möchte diese Stelle nutzen, um Herrn Prof. Dr. Hien dafür zu danken, dass
er mir ermöglicht hat, mich mit diesem Thema zu beschäftigen.
Auch bedanke ich mich für die hervorragende Betreuung, welche diese Arbeit erst
ermöglicht hat.
\end{comment}

% vim:set ft=tex foldmethod=marker foldmarker={{{,}}}:
