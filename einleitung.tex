\chapter{Einleitung}

Das in dieser Arbeit am meisten betrachtete Objekt, sind die meromorphen
Zusammenhänge (siehe Kapitel \ref{chap:meromZsh}). Diese sind eine algebraische
Sichtweise der allseits bekannten Systemen von gewöhnlichen
Differentialgleichungen. 

Des weiteren werden holonome lokalisierte $\cD$-Moduln eingeführt, diese sind
in $1:1$ Korrespondenz zu den meromorphen Zusammenhängen und werden in dieser
Arbeit immer verwendet, da sich mit diesen leichter Rechnen lässt.

\begin{comment}
Riemann-Hilbert-Korrespondenz:
\begin{center}
\begin{tikzcd}
  \left\{ \text{Regulär singuläre holonome $\cD$-Moduln}} \right\}\arrow{r} &
  \left\{ \text{Perverse Garben} \right\} \\
  \left\{ \text{Regulär singuläre meromorphe Zusammenhänge} \right\}
  \arrow[hookrightarrow]{u} \arrow{r} & 
  \left\{ \text{Lokale Systeme} \right\}
\end{tikzcd}
\end{center}
Vergleiche hierzu \cite[Sec 6]{REFKashiwara1984}.
Dies ist ein bekanntes Resultat über regulär singuläre meromorphe
Zusammenhänge, jedoch lässt sich diese Aussage nicht auf irregulär singuläre
meromorphe Zusammenhänge fortsetzen.
\end{comment}

\begin{comment}
Ziel der Arbeit
\end{comment}
\begin{comment}
In dieser Arbeit soll zunächst die Theorie der meromorphen Zusammenhänge
eingeführt werden.
\end{comment}

\begin{comment}
Vorraussetzungen
\end{comment}
Es wird in dieser Arbeit kein Vorwissen über $\cD$-Moduln bzw. meromorphe
Zusammenhänge vorausgesetzt, diese beiden Begriffe werden in den ersten Zwei
Kapiteln eingeführt.

\begin{comment}
Aufbau / inhalt der Kapitel
\end{comment}
Im ersten Kapitel wird die Theorie der $\cD$-Moduln eingeführt.

Im zweiten Kapitel werden die meromorphen Zusammenhänge definiert, diese sind
das wichtigste Objekt, das in dieser Arbeit diskutiert werden soll.
Ein wichtiges Resultat, in diesen Kapiteln, ist die Äquivalenz von meromorphen
Zusammenhängen zu holonomen lokalisierten $\cD$-Moduln, da man mit letzteren
gut rechnen kann.

Das dritte Kapitel beschäftigt sich ausschließlich mit Operationen auf
meromorphen Zusammenhängen.
Vorgestellt werden Tensorprodukt, Pullback und Pushforward,
Fouriertransformation, die sogenannte Betrachtung bei Unendlich und das Twisten
eines meromorphen Zusammenhangs.

Im vierten Kapitel soll das Levelt-Turrittin-Theorem vorgestellt werden, 
welches Nützlich ist, um die Stokes Struktur eines 

\begin{comment}
Die Levelt-Turrittin-Theorem ist eine sogenannte Slope-Filtration. Diese
Klasse, der Slope-Filtrationen wird in \cite{0812.3921} betrachtet.
\end{comment}

Das letzte Kapitel wird die Levelt-Turrittin-Zerlegung auf eine Klasse von
meromorphen Zusammenhängen angewendet. Dazu wird zu Beginn ein Rezept gegeben,
welches die gewünschten Zusammenhänge beschreibt. Zu diesen Zusammenhängen
werden zunächst allgemeine Aussagen getroffen bevor in Abschnitt
\ref{sec:LT-speziell} ganz konkret eine Levelt-Turrittin-Zerlegung zu einem
meromorphen Zusammenhang berechnet wird.

\begin{comment}
Ich möchte diesen Zeitpunkt nutzen, um Herrn Prof. Dr. Hien dafür zu danken,
das er mir ermöglicht hat, mich mit diesem Thema zu beschäftigen.
Auch bedanke ich mich für die hervorragende Betreuung, welche diese Arbeit erst
ermöglicht hat.
\end{comment}

% vim:set ft=tex foldmethod=marker foldmarker={{{,}}}:
