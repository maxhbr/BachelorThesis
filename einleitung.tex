\chapter{Einleitung}

Die Theorie der $\cD$-Moduln ist aus einem Versuch, Systeme von partiellen
Differentialgleichungen algebraisch zu betrachten, entstanden. Seit den 1960er
Jahren wurde diese Theorie von beispielsweise B. Malgrange, J. Bernstein, P.
Deligne, M.Sato, und M. Kashiwara entwickelt und vorangetrieben.
Auch aktuell wird viel auf diesem Thema geforscht, so ist hier vor allem
C. Sabbah zu erwähnen, der auch zwei der Hauptquellen dieser Arbeit, namentlich
\cite{sabbah_cimpa90} und \cite{sabbah_Fourier-local}, verfasst hat.

Eines der wichtigsten Resultate für $\cD$-Moduln ist die
Riemann-Hilbert-Korrespondenz. Sie beschreibt mittels des de-Rahm-Funktors
\textbf{DR} die Kategorienäquivalenz
\begin{center}
\begin{tikzcd}[column sep=large]
  & \left\{ \overset{\text{\normalsize Regulär singuläre}}{
   \underset{\text{\normalsize holonome $\cD$-Moduln}}{\phantom{0}}}
   \right\}\arrow[yshift=0.7ex]{r}{\textbf{DR}}&
  \left\{ \text{Perverse Garben} \right\}
   \arrow[yshift=-0.7ex]{l}\\
  \left\{ \overset{\text{\normalsize Regulär singuläre}}{
   \underset{\text{\normalsize Zusammenhänge}}{\text{meromorphe}}}
   \right\} \arrow[leftrightarrow]{r}[description]{$1:1$}&
  \left\{ \overset{\text{\normalsize Regulär singuläre}}{
   \underset{\text{\normalsize holonome $\cD$-Moduln}}{\text{lokalisierte}}}
   \right\}\arrow[yshift=0.7ex]{r} \arrow[hook]{u}&
  \left\{ \text{Lokale Systeme} \right\} \,.
   \arrow[yshift=-0.7ex]{l} \arrow[hook]{u}
\end{tikzcd}
\end{center}
\iffalse
  \begin{center}
  \begin{tikzcd}[column sep=huge]
    \left\{ \overset{\text{\normalsize Regulär singuläre}}{
     \underset{\text{\normalsize holonome $\cD$-Moduln}}{\phantom{0}}}
     \right\}\arrow[yshift=0.7ex]{r}{\textbf{DR}}&
    \left\{ \text{Perverse Garben} \right\} \,,
     \right\}\arrow[yshift=-0.7ex]{l}
  \end{tikzcd}
  \end{center}
  wobei \textbf{DR} der de-Rahm-Funktor ist.
  Für unseren Fall, da wir uns auf meromorphe Zusammenhänge bzw. lokalisierte
  holonome $\cD$-Moduln beschränken wollen, ergibt sich
  \begin{center}
  \begin{tikzcd}[column sep=large]
    \left\{ \overset{\text{\normalsize Regulär singuläre}}{
     \underset{\text{\normalsize holonome $\cD$-Moduln}}{\text{lokalisierte}}}
     \right\} \arrow[leftrightarrow]{r}[description]{$1:1$}&
    \left\{ \overset{\text{\normalsize Regulär singuläre}}{
     \underset{\text{\normalsize Zusammenhänge}}{\text{meromorphe}}}
     \right\}\arrow[yshift=0.7ex]{r}&
    \left\{ \text{Lokale Systeme} \right\} \,.
     \right\}\arrow[yshift=-0.7ex]{l}
  \end{tikzcd}
  \end{center}
\fi
Vergleiche hierzu \cite[Sec 6]{REFKashiwara1984} oder \cite[Thm
7.2.1]{hotta2007d}.

Allerdings lässt sich dieses Resultat über regulär singuläre holonome
$\cD$-Moduln leider nicht kanonisch auf irregulär singuläre fortsetzen.
Jedoch gibt es zumindest für formale irregulär singuläre $\cD$-Moduln bzw. 
formale irregulär singuläre meromorphe Zusammenhänge eine allgemeine
Strukturaussage: das Levelt-Turrittin-Theorem, welches im Rahmen dieser Arbeit
genauer betrachtet werden soll.

Das Levelt-Turrittin-Theorem beschreibt, wie sich ein formaler meromorpher
Zusammenhang $\cM$ nach einem möglicherweise notwendigem Pullback in die
direkte Summe
\begin{center}
$ \cM \cong \bigoplus_{i=0}^n \sE^{\psi_i}\otimes R_i $
\end{center}
zerlegen lässt.
Die $\sE^{\psi_i}\otimes R_i$ stellen dabei elementare meromorphe Zusammenhänge
dar und $R_i$ ist jeweils einer der bereits erwähnten regulär singulären
meromorphen Zusammenhänge.

Das Ziel dieser Arbeit ist es, in die Theorie der $\cD$-Moduln (Kapitel
\ref{chap:dModuln}) bzw. der meromorphen Zusammenhänge (Kapitel
\ref{chap:meromZsh} und \ref{chap:operationen}) kurz einzuführen und danach
eine Levelt-Turrittin-Zerlegung explizit an einem Beispiel auszuführen und zu
berechnen.
\begin{comment}
Die Riemann-Hilbert-Korrespondenz und die Theorie der Stokes-Strukturen werden
in dieser Arbeit nicht weiter betrachtet.
\end{comment}

\begin{comment}
Es wird in dieser Arbeit kein Vorwissen über $\cD$-Moduln bzw. meromorphe
Zusammenhänge vorausgesetzt, diese beiden Begriffe werden in den ersten Zwei
Kapiteln eingeführt.
\end{comment}

Im ersten Kapitel erfolgt ein kurzer Überblick über die Theorie der
$\cD$-Moduln, dabei wird der Schwerpunkt auf diejenigen Themen gelegt, die für
das weitere Verständnis der Arbeit nötig sind. Eine ausführliche Einführung,
z.B. zur genauen Definition von Holonomie, kann in \cite{sabbah_cimpa90} oder
in \cite{hotta2007d} nachgeschlagen werden.\\
\iffalse
  Im ersten Kapitel wird sehr knapp die Theorie der $\cD$-Moduln eingeführt, da
  uns diese im später sehr nützlich sein wird. Hier wurde, vor allem bei der
  Definition von holonomen $\cD$-Moduln, viel Hintergrund weggelassen, da
  dieser zum Verständnis dieser Arbeit nicht nötig ist.\\
\fi
\begin{comment} \end{comment}
Im zweiten Kapitel werden die meromorphen Zusammenhänge definiert.
Diese sind das wichtigste mathematische Objekt in dieser Arbeit.
Ein wichtiges Resultat in diesem Kapitel ist die Eins-zu-Eins Korrespondenz
von meromorphen Zusammenhängen zu holonomen lokalisierten $\cD$-Moduln.\\
\begin{comment} \end{comment}
Das dritte Kapitel beschäftigt sich ausschließlich mit Operationen auf
meromorphen Zusammenhängen.
Vorgestellt werden Tensorprodukt, Pullback und Pushforward,
Fouriertransformation, die sogenannte \glqq{}Betrachtung bei Unendlich\grqq{} und das Twisten
eines meromorphen Zusammenhangs.
Diese Operationen werden in den folgenden zwei Kapiteln Verwendung finden.\\
\begin{comment} \end{comment}
Im vierten Kapitel soll das Levelt-Turrittin-Theorem vorgestellt werden.
Es erlaubt, einen meromorphen Zusammenhang in elementare meromorphe
Zusammenhänge zu zerlegen.\\
\begin{comment} \end{comment}
Das letzte Kapitel hat das Ziel, die Levelt-Turrittin-Zerlegung auf eine Klasse
von meromorphen Zusammenhängen anzuwenden.
Dazu wird zu Beginn ein Rezept gegeben, welches die gewünschten Zusammenhänge
beschreibt.
Zu diesen Zusammenhängen werden zunächst allgemeine Aussagen getroffen, bevor
in Abschnitt \ref{sec:LT-speziell} ganz konkret eine Levelt-Turrittin-Zerlegung
zu einem meromorphen Zusammenhang berechnet wird.
Zuletzt wird in Abschnitt \ref{sec:konvergenzDerPotReihen} das
Konvergenzverhalten der berechneten Zusammenhänge analysiert.

\begin{comment}
Ich möchte diese Stelle nutzen, um Herrn Prof. Dr. Hien dafür zu danken, dass
er mir ermöglicht hat, mich mit diesem Thema zu beschäftigen.
Auch bedanke ich mich für die hervorragende Betreuung, welche diese Arbeit erst
ermöglicht hat.
\end{comment}

% vim:set ft=tex foldmethod=marker foldmarker={{{,}}}:
