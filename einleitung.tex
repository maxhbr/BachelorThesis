\chapter{Einleitung}

\begin{comment}
Ziel der Arbeit
\end{comment}
In dieser Arbeit soll zunächst die Theorie der meromorphen Zusammenhänge
eingeführt werden.


\begin{comment}
Vorraussetzungen
\end{comment}

\begin{comment}
Aufbau / inhalt der Kapitel
\end{comment}
Im ersten Kapitel wird die Theorie der $\cD$-Moduln eingeführt.

Im zweiten Kapitel werden die meromorphen Zusammenhänge definiert, diese sind
das wichtigste Objekt, das in dieser Arbeit diskutiert werden soll.
Ein wichtiges Resultat, in diesen Kapiteln, ist die Äquivalenz von meromorphen
Zusammenhängen zu holonomen lokalisierten $\cD$-Moduln, da man mit letzteren
gut rechnen kann.

Das dritte Kapitel beschäftigt sich ausschließlich mit Operationen auf
meromorphen Zusammenhängen.

Im vierten Kapitel soll das Levelt-Turrittin-Theorem vorgestellt werden, 
welches

Das letzte Kapitel wendet die Levelt-Turrittin-Zerlegung auf einen meromorphen
Zusammenhang an.

\begin{comment}
    
\end{comment}

% vim:set ft=tex foldmethod=marker foldmarker={{{,}}}:
