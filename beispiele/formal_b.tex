% entsteht aus b durch t --> 1/\tau
%TODO: besserer Titel
\subsection{Beispiel ohne namen}
\begin{comment}
  Beginne mit
  \[ \tilde P=\tau\partial_\tau^2+2\partial_\tau-1 \]
  und gehe von $\tau$ über zu $t$ via $\tau\rightarrow\frac{1}{t}$:\\
  %TODO: rightarrow oder mapsto?
  \begin{itemize}
    \item was passiert mit der Ableitung $\partial_\tau$? Es gilt:
      \[
        \partial_\tau (f(\frac{1}{\tau}))=
        \partial_t(f)\cdot (-\frac{1}{\tau^2})=
        -\partial_t(f)\cdot t^2= %TODO: wegen klammerung?
        - t^2 \cdot \partial_t(f)
      \]
      also:
      \[
        \partial_\tau=-t^2\partial_t
        % stimmt das VZ?
      \]
    \item was ist $\partial_t(t^2\partial_t)$?
      \begin{align*}
        \partial_tt^2\partial_t &= (\partial_tt)t\partial_t\\
        &= (t\partial_t-1)t\partial_t\\
        &= t(\partial_tt)\partial_t-t\partial_t\\
        &= t(t\partial_t-1)\partial_t-t\partial_t\\
        &= t^2\partial_t^2-2t\partial_t\\
      \end{align*}
    \item was passiert mit $ \tilde P=\tau\partial_\tau^2+2\partial_\tau-1 $?
      \begin{align*}
        \tilde P &= \tau\partial_\tau^2+2\partial_\tau-1\\
        &\overset{\tau\rightarrow\frac{1}{t}}{\longrightarrow}
          \frac{1}{t}(-t^2\partial_t)^2+2(-t^2\partial_t)-1\\
        &= \frac{1}{t}t^2(\partial_t(t^2\partial_t))-2t^2\partial_t-1\\
        &= t(\partial_t(t^2\partial_t))-2t^2\partial_t-1\\
        &= t(t^2\partial_t^2-2t\partial_t)-2t^2\partial_t-1\\
        &= t^3\partial_t^2-4t^2\partial_t-1 =: P\\
      \end{align*}
  \end{itemize}
\end{comment}

Wir wollen nun den zum folgendem $P$ assoziierten Meromorphen Zusammenhang
betrachten:\\
\begin{minipage}[hbt]{0,39\textwidth}
  \[ P= t^3\partial_t^2-4t^2\partial_t-1 \]
  mit $ \slopes(P)=\{\frac{1}{2}\} $
\end{minipage}
\begin{minipage}[hbt]{0,59\textwidth}
  \begin{center}
    \includegraphics[width=6cm]{img/formal_b.png}
  \end{center}
\end{minipage}
%Es ist offensichtlich, dass $\slopes(P)=\{\frac{1}{2}\}$. 
Wir wollen ganzzahlige slopes haben, also wernde den pull-back
$\rho:t\rightarrow u^2$ an.

Zunächst ein paar nebenrechnungen: 
\begin{align*}
  \partial_t   &= \frac{1}{\rho'}\partial_u=\frac{1}{2u}\partial_u \\
  \partial_t^2 &= (\frac{1}{2u}\partial_u)^2\\
               &= \frac{1}{2u}(-\frac{1}{2u^2}\partial_u + 
                 \frac{1}{2u}\partial_u^2) \\
               &= -\frac{1}{4u^3}\partial_u+\frac{1}{4u^2}\partial_u^2 \\
\end{align*}
also
\begin{align*}
  \rho^+P &= u^6(-\frac{1}{4u^3}\partial_u+\frac{1}{4u^2}\partial_u^2)- 
            4u^{4}\frac{1}{2u}\partial_u-1\\
          &= -u^3\frac{1}{4u^3}\partial_u+\frac{1}{4}u^4\partial_u^2-
            4u^{3}\frac{1}{2}\partial_u-1\\
          &= \frac{1}{4}u^4\partial_u^2 -2\frac{1}{4}u^3\partial_u-1
\end{align*}

\begin{minipage}[hbt]{0,39\textwidth}
  \[ \rho^+P= \frac{1}{4}u^4\partial_u^2 -2\frac{1}{4}u^3\partial_u-1 \]
  mit $ \slopes(\rho^+P)=\{1\} $
\end{minipage}
\begin{minipage}[hbt]{0,59\textwidth}
  \begin{center}
    \includegraphics[width=6cm]{img/formal_b_pb.png}
  \end{center}
\end{minipage}

% vim: set ft=tex :
