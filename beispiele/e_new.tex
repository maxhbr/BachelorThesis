%Neuer versuch, ein geeignetes Beispiel zu finden

Hier soll ein einfaches Beispiel hergeleitet werden, an dem die Zerlegung nach
dem Levelt-Turittin-Theorem einmal explizit ausformuliert werden soll.

Beginne mit

\begin{minipage}[hbt]{0,49\textwidth}
  \[ t^4(t+1)\partial_t^4 + t\partial_t^2+\frac{1}{t}\partial_t+1 \]
\end{minipage}
\begin{minipage}[hbt]{0,49\textwidth}
  \begin{center}
    \includegraphics[width=3cm]{img/e.png}
  \end{center}
\end{minipage}

(von ZulaBarbara Seite 47)
und ignoriere zuerst die Terme, die zum Newton Polygon keinen Beitrag leisten

\begin{minipage}[hbt]{0,49\textwidth}
  \[ t^4\partial_t^4 +\frac{1}{t}\partial_t \]
\end{minipage}
\begin{minipage}[hbt]{0,49\textwidth}
  \begin{center}
    \includegraphics[width=3cm]{img/bar_e.png}
  \end{center}
\end{minipage}

multipliziere dieses mit $t$ und ändere aber dadurch den assoziierten
Meromorphen Zusammenhang nicht \cite[Chapter 5.1]{sabbah_cimpa90}

\begin{minipage}[hbt]{0,49\textwidth}
  \[ P:=t^5\partial_t^4 +\partial_t \]
\end{minipage}
\begin{minipage}[hbt]{0,49\textwidth}
  \begin{center} 
    \includegraphics[width=3cm]{img/bar_e_times_x.png} 
  \end{center}
\end{minipage}

und es gilt $\slopes(P)=\{0,\frac{2}{3}\}$. Eliminiere als nächstes nun die
Brüche in den Slopes mittels einem geeignetem Pullback. Da hier der Hauptnenner
$3$ ist bietet sich $\rho:t\mapsto u^3$ für den Pullback an.
\begin{comment}
  Dieser Pullback Multipliziert (indirekt) die Slopes mit $3$,
  \textbf{Quelle?}\\ aber wie wendet man ihn (explizit) an?
\end{comment}

\[ \rho^+P=??? \]
welches die Slopes $\slopes(\rho^+P)=\{0,2\}\subset\Z$ hat. Schreibe nun dieses
$\rho^+P=Q\cdot R$ mit $P,Q\in\Cfu$ wobei gilt $\slopes(Q)=\{0\}$ und
$\slopes(R)=\{2\}$.

Also gilt:
\[
  \hat\cD/(\hat\cD\cdot\rho^+P)\cong
  \hat\cD/(\hat\cD\cdot Q) \oplus  \hat\cD/(\hat\cD\cdot R)
\]

% vim: set ft=tex :
