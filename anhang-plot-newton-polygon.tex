\chapter{Wie ich Newton Polygone zeichne}
\lstdefinestyle{myLatex} {
  %language=[LaTeX]TeX
  language=LaTeX
  , texcsstyle=*\bf\color{blue}
  , basicstyle=\ttfamily
  , numbers=none
  , breaklines=true
  , commentstyle=\color{red}
  %, otherkeywords={$, \{, \}, \[, \]}
  %, frame=lines
  , xleftmargin=30pt          % linker Abstand vom Rand (framesep+framrule)
  , tabsize=2
  %, caption=LaTeX example
}

Ich benutze tikz
\begin{lstlisting}[style=myLatex]
\usepackage{tikz}
\usetikzlibrary{matrix,arrows,decorations.pathmorphing}
\end{lstlisting}
und ein eigenes Kommando
\begin{lstlisting}[style=myLatex]
\newcommand{\myNewtonPlot}[6]{
  \draw[color=black,thick] #2;
  \foreach \pos in #1 { \fill[blue,opacity=.2] (-.5,#5) rectangle \pos; }
  \draw[->] (-.5,0) -- (#3+.7,0);
  \draw[->] (0,#4-.2) -- (0,#5+.2);
  \draw (1,0) -- (1,-.1);
  \draw (0,1) -- (-.1,1);
  \foreach \pos in #1 { \node[draw,circle,inner sep=1.5pt,fill=white] at \pos {}; }
  \node [below right] at (#3,#5/2) {#6};
}
\end{lstlisting}
welche 6 Parameter verlangt:
\begin{enumerate}
\item ein array der Punkte
\item einen Pfad, der das Newton Polygon beschreibt
\item den maximalen x Wert
\item den minimalen y Wert
\item den maximalen y Wert
\end{enumerate}
Ein Aufruf
\begin{lstlisting}[style=myLatex]
\begin{tikzpicture}[scale=1.5]
\def\myPoints{{(0,0)}, {(1,-2)}, {(2,-1)}, {(4,0)}}
\def\myPath{(-.5,-2) -- (1,-2) -- (4,0) -- (4,2)}
\myNewtonPlot{\myPoints}{\myPath}{4}{-2}{2}{$N(P)$}
\end{tikzpicture}
\end{lstlisting}
ergibt dann
\begin{center}
\begin{tikzpicture}[scale=1.5]
\def\myPoints{{(0,0)}, {(1,-2)}, {(2,-1)}, {(4,0)}}
\def\myPath{(-.5,-2) -- (1,-2) -- (4,0) -- (4,2)}
\myNewtonPlot{\myPoints}{\myPath}{4}{-2}{2}{$N(P)$}
\end{tikzpicture}
\end{center}

\newpage
BEGIN

\begin{center}
\begin{tikzpicture}[scale=1.5,descr/.style={fill=white,inner sep=2.5pt}]

\def\myPoints{0/0,1/1,2/1,3/4}

\foreach \x/\y [count=\ii] in \myPoints{
  \ifnum\ii=1
    \xdef\myMinY{\y}
    \xdef\myMaxY{\y}
    \xdef\myMaxX{\x}
  \else
    \ifnum\myMinY>\y
      \xdef\myMinY{\y}
    \fi
    \ifnum\myMaxY<\y
      \xdef\myMaxY{\y}
    \fi
    \ifnum\myMaxX<\x
      \xdef\myMaxX{\x}
    \fi
  \fi
}
\ifnum\myMaxY<1
  \xdef\myMaxY{1}
\fi

%%axis
\draw[->] (-.5,0) -- (\myMaxX+.2,0);
\draw[->] (0,\myMinY-.2) -- (0,\myMaxY+.7);
\draw (1,0) -- (1,-.1);
\draw (0,1) -- (-.1,1);

%%rectangle
\foreach \x/\y in \myPoints
  { \fill[blue,opacity=.2] (-.5,\myMaxY+.5) rectangle (\x,\y); }

%%path
\xdef\myLastX{0}
\xdef\myLastY{\myMinY}
\draw (-.5,\myMinY) -- (\myLastX,\myLastY);
\foreach \x/\y [count=\ii] in \myPoints {
  \xdef\myAuslassen{0} % nicht auslassen
  \foreach \xx/\yy [count=\iii] in \myPoints {
    \ifnum\iii > \ii
      \pgfmathparse{add(\ii,1)}
      \ifnum\iii=\pgfmathresult
        \xdef\myNextX{\xx}
        \xdef\myNextY{\yy}
        \pgfmathparse{divide(subtract(\yy,\myLastY),subtract(\xx,\myLastX))}
        \xdef\mySteigung{\pgfmathresult}
        %\node[below] at (\xx,\yy) {\mySteigung (\myAuslassen)};
        %\node[below] at (\x,-\xx) {\mySteigung};
      \else
        \xdef\myNextX{\xx}
        \xdef\myNextY{\yy}
        %\pgfmathparse{subtract(\mySteigung,divide(subtract(\yy,\myLastY),subtract(\xx,\myLastX)))}
        \pgfmathparse{\yy-\myLastY}
        \node[below] at (\x-.2,-\xx+.7) {\pgfmathresult = \yy - \myLastY};
        \pgfmathparse{divide(\pgfmathresult,subtract(\xx,\myLastX))}
        \node[below] at (\x-.1,-\xx+.3) {\pgfmathresult};
        \pgfmathparse{subtract(\mySteigung,\pgfmathresult)}
        \node[below] at (\x,-\xx) {\pgfmathresult};

        %\ifnum\pgfmathresult<0
          %\xdef\myAuslassen{1}
        %\fi

        %\node[below] at (2*\x,-\xx)
          %{\mySteigung<\pgfmathresult(\myAuslassen)};
      \fi
    \fi
  }

  \ifnum\myAuslassen=0
    \draw (\myLastX,\myLastY) -- 
      %node[descr]{$\ii$} 
      (\myNextX,\myNextY);
    \xdef\myLastX{\myNextX}
    \xdef\myLastY{\myNextY}
  \fi
}
\draw (\myLastX,\myLastY) -- (\myMaxX,\myMaxY+.5);

%%nodes
\foreach \x/\y in \myPoints
  { \node[draw,circle,inner sep=1.5pt,fill=white] at (\x,\y) {}; }

\end{tikzpicture}
\end{center}

END

\myMaxY


% vim: set ft=tex :
