\chapter{Wie ich Newton Polygone zeichne}
\lstdefinestyle{myLatex} {
  %language=[LaTeX]TeX
  language=LaTeX
  , texcsstyle=*\bf\color{blue}
  , basicstyle=\ttfamily
  , numbers=none
  , breaklines=true
  , commentstyle=\color{red}
  %, otherkeywords={$, \{, \}, \[, \]}
  %, frame=lines
  , xleftmargin=30pt          % linker Abstand vom Rand (framesep+framrule)
  , tabsize=2
  %, caption=LaTeX example
}

Ich benutze tikz
\begin{lstlisting}[style=myLatex]
\usepackage{tikz}
\usetikzlibrary{matrix,arrows,decorations.pathmorphing}
\end{lstlisting}
und ein eigenes Kommando
\begin{lstlisting}[style=myLatex]
\newcommand{\myNewtonPlot}[6]{
  \draw[color=black,thick] #2;
  \foreach \pos in #1 { \fill[blue,opacity=.2] (-.5,#5) rectangle \pos; }
  \draw[->] (-.5,0) -- (#3+.7,0);
  \draw[->] (0,#4-.2) -- (0,#5+.2);
  \draw (1,0) -- (1,-.1);
  \draw (0,1) -- (-.1,1);
  \foreach \pos in #1 { \node[draw,circle,inner sep=1.5pt,fill=white] at \pos {}; }
  \node [below right] at (#3,#5/2) {#6};
}
\end{lstlisting}
welche 6 Parameter verlangt:
\begin{enumerate}
\item ein array der Punkte
\item einen Pfad, der das Newton Polygon beschreibt
\item den maximalen x Wert
\item den minimalen y Wert
\item den maximalen y Wert
\end{enumerate}
Ein Aufruf
\begin{lstlisting}[style=myLatex]
\begin{tikzpicture}[scale=1.5]
\def\myPoints{{(0,0)}, {(1,-2)}, {(2,-1)}, {(4,0)}}
\def\myPath{(-.5,-2) -- (1,-2) -- (4,0) -- (4,2)}
\myNewtonPlot{\myPoints}{\myPath}{4}{-2}{2}{$N(P)$}
\end{tikzpicture}
\end{lstlisting}
ergibt dann
\begin{center}
\begin{tikzpicture}[scale=1.5]
\def\myPoints{{(0,0)}, {(1,-2)}, {(2,-1)}, {(4,0)}}
\def\myPath{(-.5,-2) -- (1,-2) -- (4,0) -- (4,2)}
\myNewtonPlot{\myPoints}{\myPath}{4}{-2}{2}{$N(P)$}
\end{tikzpicture}
\end{center}

\begin{center}
\begin{tikzpicture}[scale=1.5,descr/.style={fill=white,inner sep=2.5pt}]

\def\myPoints{0/-1,1/0,2/0,3/3,3/4,4/3}

\foreach \x/\y [count=\ii] in \myPoints{
  \ifnum\ii=1
    \xdef\myMinY{\y}
    \xdef\myMaxY{\y}
    \xdef\myMaxX{\x}
  \else
    \ifnum\myMinY>\y
      \xdef\myMinY{\y}
    \fi
    \ifnum\myMaxY<\y
      \xdef\myMaxY{\y}
    \fi
    \ifnum\myMaxX<\x
      \xdef\myMaxX{\x}
    \fi
  \fi
}
\ifnum\myMaxY<1
  \xdef\myMaxY{1}
\fi

%%axis
\draw[->] (-.5,0) -- (\myMaxX+.2,0);
\draw[->] (0,\myMinY-.2) -- (0,\myMaxY+.7);
\draw (1,0) -- (1,-.1);
\draw (0,1) -- (-.1,1);

%%rectangle
\foreach \x/\y in \myPoints
  { \fill[blue,opacity=.2] (-.5,\myMaxY+.5) rectangle (\x,\y); }

%%path
\newcommand*{\steigung}[2]{%
\numexpr #1*#2\relax.%
}
\xdef\myLastX{0}
\xdef\myLastY{\myMinY}
\xdef\myAuslassen{1}
\draw (-.5,\myMinY) -- (\myLastX,\myLastY);
\foreach \x/\y [count=\ii] in \myPoints {
  \xdef\myAuslassen{0} % nicht auslassen
  \foreach \xx/\yy [count=\iii] in \myPoints {
    \ifnum\iii > \ii
      \pgfmathparse{\ii+1}
      \ifnum\iii=\pgfmathresult
        \xdef\myNextX{\xx}
        \xdef\myNextY{\yy}

        \FPneg\tmp\myLastY
        \FPadd\tmp\yy\tmp
        \FPneg\tmpp\myLastX
        \FPadd\tmpp\xx\tmpp
        \ifnum\tmpp=0
          %TODO
        \else
          \FPdiv\tmp\tmp\tmpp
        \fi
        \FPneg\tmp\tmp
        \xdef\mySteigung{\tmp}
      \else
        \FPneg\tmp\myLastY
        \FPadd\tmp\yy\tmp
        \FPneg\tmpp\myLastX
        \FPadd\tmpp\xx\tmpp
        \ifnum\tmpp=0
          %TODO
        \else
          \FPdiv\tmp\tmp\tmpp
        \fi
        \FPadd\tmp\tmp\mySteigung

        \FPiflt\tmp 0
          \xdef\myAuslassen{1}
        \fi
      \fi
    \fi
  }

  \FPifeq\myAuslassen 0
    \FPiflt\myLastX\myNextX
      \FPneg\tmp\myLastY
      \FPadd\tmp\myNextY\tmp
      \FPneg\tmpp\myLastX
      \FPadd\tmpp\myNextX\tmpp
      \FPtrunc\tmp\tmp 2
      \FPclip\tmp\tmp
      \FPtrunc\tmpp\tmpp 2
      \FPclip\tmpp\tmpp
      %\FPifeq\tmpp 1
        %\draw (\myLastX,\myLastY) -- node[descr]{$\tmp$} (\myNextX,\myNextY);
      %\else
        %\draw (\myLastX,\myLastY) -- node[descr]{$\frac{\tmp}{\tmpp}$} (\myNextX,\myNextY);
      %\fi

      \draw (\myLastX,\myLastY) -- node[descr]{$\frac{\tmp}{\tmpp}$} (\myNextX,\myNextY);

      \xdef\myLastX{\myNextX}
      \xdef\myLastY{\myNextY}
    \fi
  \fi
}
\draw (\myLastX,\myLastY) -- (\myMaxX,\myMaxY+.5);

%%nodes
\foreach \x/\y in \myPoints
  { \node[draw,circle,inner sep=1.5pt,fill=white] at (\x,\y) {}; }

\end{tikzpicture}
\end{center}

% vim: set ft=tex :
