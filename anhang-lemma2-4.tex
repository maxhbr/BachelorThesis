\begin{landscape}
%\chapter{zur aufteilung in Lemma 2.4}
\chapter{Aufteilung von ...}
\label{chap:aufteilung}
Sei $\phi\in t^{-1}\C[t^{-1}]$, so ist $\phi'=:\sum_{i=2}^N a_{-i}t^{-i}\in
t^{-2}\C[t^{-1}]$ also $u\phi'(t)=\sum_{i=1}^N a_{-i-1}t^{-i} \in
t^{-1}\C[t^{-1}]$, welches wir zerlegen wollen.\\
Zerlege also $t\phi'(t)=\sum_{j=0}^{p-1}t^j\psi_j(t^p)$ mit
$\psi_j\in\C[x^{-1}]$ für alle $j>0$ und $\psi_0\in x^{-1}\C[x^{-1}]$:\\
\begin{center}
\footnotesize
\begin{tikzpicture} [descr/.style={fill=white,inner sep=2.5pt}]
\matrix (m) [
  matrix of math nodes,
  row sep=1em,
  %column sep=-0.7em,
  text height=1.5ex,
  text depth=0.25ex]
{
  & & & & & & & & & & & & & & \, \\
  t\phi'(t)= &
  a_{-2}t^{-1} &
  +...+ &
  a_{-p}t^{-(p-1)} &
  + &
  a_{-(p+1)}t^{-p} &
  + &
  a_{-(p+2)}t^{-(p+1)} &
  +...+ &
  a_{-2p}t^{-(2p-1)} &
  + &
  a_{-(2p+1)}t^{-2p} &
  + &
  a_{-(2p+3)}t^{-(2p+1)} &
  + ...  \\
  & & & & & & & & & & & & & & \, \\
};

  \path[solid]
  (m-2-6) edge [bend left=20] node[descr]{$\psi_0(t^p)$} (m-2-12)
  (m-2-12)  edge [bend left=20] (m-1-15);

  \path[dotted]
  (m-2-4) edge [bend right=20] node[descr]{$t\psi_1(t^p)$} (m-2-10)
  (m-2-10)  edge [bend right=20] (m-3-15);

  \path[dashed]
  (m-2-2) edge [bend right=20] node[descr]{$t^{p-1}\psi_{p-1}(t^{p})$} (m-2-8)
  (m-2-8) edge [bend right=20] (m-2-14)
  (m-2-14) edge [bend right=20] (m-3-15);
\end{tikzpicture}
\end{center}
also:\\
\begin{align*}
\psi_0(t^p) &= a_{-(p+1)}t^{-p}+a_{-(2p+1)}t^{-2p}+...\\
\psi_1(t^p) &= a_{-p}t^{-p}+a_{-2p}t^{2p}+...\\
& \vdots & \\
\psi_{p-1}(t^p) &= a_{-2}t^p+a_{-(p+2)}t^{2p}+...\\
\end{align*}
\end{landscape}
