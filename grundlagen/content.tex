\chapter{Mathematische Grundlagen}

Hier werde ich mich auf \cite{sabbah_cimpa90} und \cite{coutinho1995primer} beziehen.

\section{Einige Ergebnise aus der Kommutativen Algebra}~

In dieser Arbeit spielen die folgenden Ringe eine Große Rolle:
\begin{itemize}
  \item $\C[x]:=\{ \sum^{N}_{i=1}a_ix^i | N\in \N \}$
  \item $\C\{x\}:=\{ \sum^{\infty}_{i=1}a_ix^i | \mbox{pos.
        Konvergenzradius} \}$
  \item $\C\llbracket x\rrbracket:=\{ \sum^{\infty}_{i=1}a_ix^i \}$
  \item $K:=\C(\{x\}):=\C\{x\}[x^{-1}]$
  \item $\hat{K}:=\C((x)):=\C\llbracket x\rrbracket[x^{-1}]$
\end{itemize}

wobei offensichtlich gilt $\C[x]\subset\C\{x\}\subset\C\llbracket x\rrbracket$.

\begin{comment}
  \begin{lem}[Seite 2]
    ein paar eigenschaften
    \begin{enumerate}
      \item $\C[x]$ ist ein graduierter Ring, durch die Grad der
        Polynome. Diese graduierung induziert eine aufsteigende Filtrierung.

        alle Ideale haben die form $(x-a)$ mit $,a\in \C$
      \item wenn $\mathfrak{m}$ das maximale Ideal von $\C[x]$ (erzeugt von
        $x$ ist), so ist
        \[
          \C[[x]]=
          \underset{k}{\underleftarrow{\lim}} \C[X]\backslash\mathfrak{m}^k
        \]
        The ring $\C[[x]]$ ist ein nöterscher lokaler Ring:
        jede Potenzreihe mit konstantem term $\neq 0$ ist invertierbar.

        Der ring ist ebenfalls ein diskreter ??? Ring (discrete valuation
        ring)

        Die Filtrierung nach grad des Maximalen Ideals, genannt
        $\mathfrak{m}$-adische Fitration, ist die Filtrierung
        $\mathfrak{m}^k=\{f\in \C[[x]]|v(f)\geq k\}$

        und es gilt $gr_\mathfrak{m}(\C[[x]])=\C[x]$
      \item $\C\{x\}\subset \C[[x]]$ ist ein Untering der Potenzreihen, wobei
        der Konvergenzradius echt positiv ist.

        ist ähnlich zu $\C[[x]]$
    \end{enumerate}
  \end{lem}
\end{comment}

\begin{defn}[Direkte Summe] \cite[4(Categories).5.1]{stacks-project}
  Seien $x,y\in \Ob(\cC)$, eine \emph{Direkte Summe} oder das \emph{coprodukt}
  von $x$ und $y$ ist ein Objekt $x\oplus y\in \Ob(\cC)$ zusammen mit
  Morphismen $i\in\Mor_\cC(x,x\oplus y)$ und $j\in\Mor_\cC(y,x\oplus y)$ so
  dass die folgende universelle Eigenschaft gilt: für jedes $w\in Ob(\cC)$ mit
  Morphismen $\alpha\in\Mor_\cC(x,w)$ und $\beta\in\Mor_\cC(y,w)$ existiert ein
  eindeutiges $\gamma\in\Mor_\cC(x\oplus y,w)$ so dass das Diagram
  \begin{center}
    \begin{tikzpicture} [scale=3.3, descr/.style={fill=white,inner sep=2.5pt} ]
      \matrix (m) [
        matrix of math nodes
        , row sep=2em
        , column sep=3em
        %, text height=3em
        %, text depth=0.25em
      ]
      {
          & y         &  & \\
        x & x\oplus y &  & \\
          &           &  & w \\
      };
      %TODO: Pfeile
      \path[->,font=\scriptsize,>=angle 90]
      (m-1-2) edge node[left]{$j$} (m-2-2)
      (m-2-1) edge node[above]{$i$} (m-2-2)
      (m-1-2) edge node[right]{$\beta$} (m-3-4)
      (m-2-1) edge node[below]{$\alpha$} (m-3-4)
      ;
      \path[->,font=\scriptsize,>=angle 90,dashed]
      (m-2-2) edge node[above]{$\gamma$} (m-3-4)
      ;
    \end{tikzpicture}
  \end{center}
  kommutiert.
\end{defn}

\begin{defn}[Tensorprodukt / Faserprodukt] 
  \cite[3(Algebra).11.21]{stacks-project}
  \cite[4(Categories).6.1]{stacks-project} 
  % von Vorlesung Algebra 2

  \begin{center}
    \begin{tikzpicture} [scale=3.3, descr/.style={fill=white,inner sep=2.5pt} ]
      \matrix (m) [
        matrix of math nodes
        , row sep=2em
        , column sep=3em
        %, text height=3em
        %, text depth=0.25em
      ]
      {
        M\times N & M\otimes_RN \\
                  & T \\
      };
      %TODO: Pfeile
      \path[->,font=\scriptsize,>=angle 90]
      (m-1-1) edge node[above]{$  $} (m-1-2)
      (m-1-1) edge node[below]{$f$} (m-2-2)
      ;
      \path[->,font=\scriptsize,>=angle 90,dashed]
      (m-1-2) edge node[right]{$\exists!\gamma$} (m-2-2)
      ;
    \end{tikzpicture}
  \end{center}
\end{defn}
%TODO: tensorprodukt / faserprodukt zum basiswechsel

\section{Weyl-Algebra und der Ring $\cD$} 
%TODO: sabah-cimpa90.pdf  seite 3
%script ginzburg.pdf seite 34 als garbe definiert
Ich werde hier die Weyl Algebra, wie in
\cite[Chapter~1]{sabbah_cimpa90}, in einer Veränderlichen einführen.
Sei $\frac{\partial}{\partial x}=\partial_x$ der Ableitungsoperator nach $x$
und sei $f \in\C[x]$ (bzw. $\C\{x\}$ bzw. $\C\llbracket x\rrbracket$).
Man hat die folgende Kommutations-Relation zwischen dem
\emph{Ableitungsoperator}
und dem \emph{Multiplikations Operator} $f$:
% TODO; ist das der Kommutator??
\begin{equation}\label{eq:weyl_relation}
  [\frac{\partial}{\partial x},f]=\frac{\partial f}{\partial x}
\end{equation}
wobei die Rechte Seite die Multiplikation mit $\frac{\partial f}{\partial x}$
darstellt. Dies bedeutet, für alle $g\in\C[x]$ hat man:
\[
  [\frac{\partial}{\partial x},f]\cdot g
  =\frac{\partial fg}{\partial x} - f\frac{\partial g}{\partial x}
  =\frac{\partial f}{\partial x} \cdot g
\]
\begin{defn}[Weyl Algebra]
  Definiere nun die Weyl Algebra $A_1(\C)$ (bzw. die Algebra $\cD$ von
  linearen Operatoren mit Koeffizienten in $\C\{x\}$ bzw. die Algebra
  $\hat{\cD}$ (Koeffizienten in $\C\llbracket x\rrbracket$)) als die
  Quotientenalgebra der freien Algebra, welche von dem Koeffizientenring
  zusammen mit dem Element $\partial_x$, erzeugt wird, Modulo der Relation
  \eqref{eq:weyl_relation}.
\end{defn}
Wir werden die Notation $A_1(\C):=\C[x]<\partial_x>$ (bzw.
$\cD:=\C\{x\}<\partial_x>$ bzw. 
$\hat{\cD}:=\C\llbracket x\rrbracket<\partial_x>$) verwenden.

%TODO: vlt umsortieren
\begin{lem} %TODO: vervollständigen
  Sei $A$ einder der 3 soeben eingeführten Objekten, die Addition 
  \[
    +:A\times A \rightarrow A
  \]
  und die Multiplikation
  \[
    \cdot:A\times A \rightarrow A
  \]
  definieren auf $A$ eine Ringstruktur $(A,+,\cdot)$.
\end{lem}
\begin{proof}
  \cite[Kapittel 2 Section 1]{ZulaBarbara}
\end{proof}

\begin{rem}
  $A_1(\C),~\cD$ und $\hat\cD$ sind nicht kommutative Algebren.
\end{rem}

% ist das nichtkommutativ??
\begin{defn}[Kommutator]%zula seite 15
  Sei $R$ ein Ring. Für $a,b\in R$ wird
  \[[a,b]=a\cdot b-b\cdot a\]
  der \emph{Kommutator von a und b} genannt.
\end{defn}
% komutativ, dann immer kommutator gleich 0

\begin{prop} % geklaut aus Zula Barbara
  \begin{enumerate}
    \item Es gilt
      \[[ \partial_x,x] = \partial_xx-x\partial_x=1 \]
    \item Sei $f\in \C[x]$, so gilt:
      \[ [\partial_x,f] = \frac{\partial,f}{\partial x} \,. \]
      Denn für $g\in \C[x]$ ist
      \[
        [\partial_x,f]\cdot g=\partial_x(fg)-f\partial_xg=
        (\partial_xf)g+\underset{=0}{\underbrace{ 
            f(\partial_xg)-f(\partial_xg)}}=
        (\partial_xf)g
      \]
    \item Es gelten die Formeln\\
    \begin{align*}
      [\partial_x,x^k]   &= kx^{k-1}\\
      [\partial_x^j,x]   &= j\partial_x^{j-1}\\
      [\partial_x^j,x^k] &= \sum_{i\geq1}\frac{k(k-1)\cdots(k-i+1)
        \cdot j(j-1)\cdots(j-i+1)}{i!}x^{k-i}\partial_x^{j-i} \\
    \end{align*}
  \end{enumerate}
\end{prop}
\begin{proof}
  \cite{ZulaBarbara}
\end{proof}

\begin{prop} \label{prop:weyl_eindeutige_schreibung}
  Jedes Element in $A_1(\C)$ (bzw. $\cD$ oder $\hat{\cD}$) kann auf eindeutige
  weiße als $P=\sum_{i=0}^na_i(x)\partial_x^i$, mit $a_i(x)\in A_1(\C)$ (bzw.
  $\cD$ oder $\hat{\cD}$), geschrieben werden. 
\end{prop}
\begin{proof}
  \cite[Proposition 1.2.3]{sabbah_cimpa90}
  \begin{comment}
    ein teil des Beweises ist "left as an exersice"
  \end{comment}
\end{proof}

%TODO: Beispiele??

%TODO: definition Filtrierung

\begin{defn}
  Sei $P=\sum_{i=0}^na_i(x)\partial_x^i$ gegeben, so definiere 
  \[
    \deg P:=\max\{i|a_i\neq 0\}
  \]
  In natürlicher Weise erhält man $F_N\cD:=\{P\in\cD|\deg P\leq N\}$ sowie die
  entsprechende aufsteigende Filtrierung
  \[
    \cdots\subset F_{-1}\cD\subset F_{0}\cD\subset
    F_{1}\cD\subset\cdots\subset\cD
  \]
  und erhalte $gr_k^F\cD\underset{\mbox{def}}{=}F_N\cD\slash F_{N-1}\cD
  =\{P\in\cD|\deg P=N\}\cong\C\{x\}$.
\end{defn}

\begin{proof}
  Sei $P\in F_N\cD$ so betrachte den Isomorphismus:
  \[
    F_N\cD\slash F_{N-1}\cD\rightarrow \C\{x\}; [P]=P+F_{N-1}\cD\mapsto a_n(x)
  \]
\end{proof}

\begin{prop}
  Es gilt:
  \begin{center}
    \begin{tikzpicture} [descr/.style={fill=white,inner sep=2.5pt}]
    \matrix (m) [
      matrix of math nodes,
      row sep=1em,
      %column sep=-0.7em,
      text height=1.5ex,
      text depth=0.25ex]
    {
      gr^F\cD &
      := &
      \bigoplus_{\N\in\Z}gr_N^F\cD &
      = &
      \bigoplus_{\N\in\N_0}gr_N^F\cD &
      \cong &
      \bigoplus_{\N\in\N_0}\C\{x\} &
      \cong &
      \C\{x\}[\xi] &
      = &
      \bigoplus_{\N\in\N_0}\C\{x\}\cdot \xi^N \\
    };
      \path[solid]
      (m-1-1) edge [bend right=15] node[descr]{$\cong$}
        node[above]{$\mbox{isomorph als grad. Ringe}$} (m-1-11);
    \end{tikzpicture}
  \end{center}
\end{prop}
\begin{proof} TODO
  \begin{comment}
    Treffen?
  \end{comment}
\end{proof}

\begin{comment}
  \subsection{Weyl Algebra als Graduierter Ring}
  % Treffen 2
  Sei $A$ nun einer der drei Koeffizienten Ringe, welche zuvor behandelt
  wurden.  Der Ring $A<\partial_x>$ kommt zusammen mit einer aufsteigenden
  Filtrierung, welche wir mit $F(A<\partial_x)$ bezeichen werden.  Sei $P$ ein
  bzgl. \ref{prop:weyl_eindeutige_schreibung} minimal geschriebener Operator,
  so ist $P$ in $F_k$ falls der maximale Grad von $\partial_x$ in $P$ kleiner
  oder gleich $k$. So definiere den Grad $deg P$ von $P$ als die Eindeutige
  ganze Zahl $k$ mit $P\in F_kA<\partial_x>\slash F_{k-1}<\partial_x>$

  Unabhängigkeit von Schreibung wird in Sabbah Script behauptet
\end{comment}

\subsection{Struktur von Links-Idealen auf $\cD$}

% vim: set ft=tex :
