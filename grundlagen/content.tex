\chapter{Mathematische Grundlagen}

Hier werde ich mich auf \cite{sabbah_cimpa90} und \cite{coutinho1995primer} beziehen.

\section{Einige Ergebnise aus der Kommutativen Algebra}~

In dieser Arbeit spielen die folgenden Ringe eine Große Rolle:
\begin{itemize}
  \item $\C[x]:=\{ \sum^{N}_{i=1}a_ix^i | N\in \N \}$
  \item $\C\llbracket x\rrbracket:=\{ \sum^{\infty}_{i=1}a_ix^i \}$
  \item $\C\{x\}:=\{ \sum^{\infty}_{i=1}a_ix^i | \mbox{pos.
        Konvergenzradius} \}$
  \item $K:=\C(\{x\}):=\C\{x\}[x^{-1}]$
  \item $\hat{K}:=\C((x)):=\C\llbracket x\rrbracket[x^{-1}]$
\end{itemize}

wobei offensichtlich gilt $\C[x]\subset\C\{x\}\subset\C\llbracket x\rrbracket$.

\begin{comment}
  \begin{lem}[Seite 2]
    ein paar eigenschaften
    \begin{enumerate}
      \item $\C[x]$ ist ein graduierter Ring, durch die Grad der
        Polynome. Diese graduierung induziert eine aufsteigende Filtrierung.

        alle Ideale haben die form $(x-a)$ mit $,a\in \C$
      \item wenn $\mathfrak{m}$ das maximale Ideal von $\C[x]$ (erzeugt von
        $x$ ist), so ist
        \[
          \C[[x]]=
          \underset{k}{\underleftarrow{\lim}} \C[X]\backslash\mathfrak{m}^k
        \]
        The ring $\C[[x]]$ ist ein nöterscher lokaler Ring:
        jede Potenzreihe mit konstantem term $\neq0$ ist invertierbar.

        Der ring ist ebenfalls ein diskreter ??? Ring (discrete valuation
        ring)

        Die Filtrierung nach grad des Maximalen Ideals, genannt
        $\mathfrak{m}$-adische Fitration, ist die Filtrierung
        $\mathfrak{m}^k=\{f\in \C[[x]]|v(f)\geq k\}$

        und es gilt $gr_\mathfrak{m}(\C[[x]])=\C[x]$
      \item $\C\{x\}\subset \C[[x]]$ ist ein Untering der Potenzreihen, wobei
        der Konvergenzradius echt positiv ist.

        ist ähnlich zu $\C[[x]]$
    \end{enumerate}
  \end{lem}
\end{comment}

\begin{defn}[Kommutator]%zula seite 15
  Sei $R$ ein Ring. Für $a,b\in R$ wird
  \[[a,b]=b\cdot a-a\cdot b\]
  der \emph{Kommutator von a und b} genannt.
\end{defn}

\section{Weiterführende Definitionen}~

\begin{defn}[pull-back]
  Der \emph{pull-back} $\rho^{+}M$ ist der Vektorraum
  $\rho^{*}M=\C((u))\otimes_{\C((u))}M$ mit
  der \emph{pull-back Verknüpfung(connection)} $\rho^*\nabla$ definiert durch
  $\partial_u(1\otimes m):=\rho'(u)\otimes\partial_tm$
\end{defn}

sei nun $N$ ein $\C((u))$-VR mit Verknüpfung
\begin{defn}[push-forward]
  Der \emph{push-forward} $\rho_+N$ ist definiert durch:
  \begin{itemize}
    \item der $\C((t))$-VR $\rho_*N$ ist der $\C$-VR N mit der $\C((t))$
      Struktur durch $f(t)\cdot 0:=f(\rho(t))m$
    \item die wirkung von $\partial_t$ ist die von $\rho'(u)^-1\partial_u$
  \end{itemize}
\end{defn}
\begin{thm} \label{thm:Projektionsformel}
  es gilt dir Projektionsformel
  \begin{equation} \label{eq:Projektionsformel}
    \rho_+(N\otimes_{\C((u))}\rho^+M)\cong\rho_+N\otimes_{\C((t))}M
  \end{equation}
\end{thm}

\begin{comment}
  TEST für ref\\
  \ref{thm:Projektionsformel}\\
  TEST für eqref\\
  \eqref{eq:Projektionsformel}\\
\end{comment}

\section{Weyl-Algebra und der Ring $\D$} %TODO: sabah-cimpa90.pdf  seite 3
Ich werde hier die Weyl Algebra, wie in
\cite[Chapter~1]{sabbah_cimpa90}, in einer Veränderlichen einführen.
Sei $\frac{\partial}{\partial x}=\partial_x$ der Ableitungsoperator nach $x$
und sei f $\in\C[x]$ (bzw. $\C\{x\}$ bzw. $\C\llbracket x\rrbracket$).
Man hat die folgende Kommutations-Relation zwischen dem
\emph{Ableitungsoperator}
und dem \emph{Multiplikations Operator} f:
\begin{equation}\label{eq:weyl_relation}
  [\frac{\partial}{\partial x},f]=\frac{\partial f}{\partial x}
\end{equation}
wobei die Rechte Seite die Multiplikation mit $\frac{\partial f}{\partial x}$
darstellt. Dies bedeutet für alle $g\in\C[x]$ hat man:
\[
  [\frac{\partial}{\partial x},f]\cdot g
  =\frac{\partial fg}{\partial x} - f\frac{\partial g}{\partial x}
  =\frac{\partial f}{\partial x} \cdot g
\]
\begin{defn}[Weyl Algebra]
  Definiere nun die Weyl Algebra $A_1(\C)$ (bzw. die Algebra $\D$ von
  linearen Operatoren mit Koeffizienten in $\C\{x\}$ bzw. die Algebra
  $\hat{\D}$ (Koeffizienten in $\C\llbracket x\rrbracket$)) als die
  Quotientenalgebra der freien Algebra, welche von dem Koeffizientenring
  zusammen mit dem Element $\partial_x$, erzeugt wird, Modulo der Relation
  \eqref{eq:weyl_relation}.
\end{defn}
Wir werden die Notation $A_1(\C)=\C[x]<\partial_x>$ (bzw.
$\D=\C\{x\}<\partial_x>$ bzw. $\hat{\D}=\C[[x]]<\partial_x>$) verwenden.

\begin{comment}
  \textbf{countinho's ansicht:}
  Sei $K\{z_1,...,z_{2n}\}$ eine freie Algebra in $2n$ variablen. Die
  Multiplikation von zwei Monomen ist als einfaches Nebeneinanderschreiben
  Definiert. Betrachte nun den folgenden Homomorphismus:
  \[ \phi:K\{z_1,...,z_{2n}\}\rightarrow A_n \]
  mit $\phi(z_i)=x_i$ und $\phi(z_{i+n})=\partial_i$ für $i\in\{1,..,n\}$.\\
  Sei $J$ das (two sided) Ideal von $K\{x_1,..,x_{2n}\}$ generiert durch:
  \begin{itemize}
    \item $[z_{i+n},z_i]-1$ für $i=1,...,n$
    \item $[z_i,z_j]$
      für $j\neq i+n$ und $1\leq i,j\leq 2n$
  \end{itemize}
  So dass $\phi$ einen homomorphismus
  \[ \phi:K\{z_1,...,z_{2n}\}/J\rightarrow A_n \]
  induziert.

\end{comment}

\begin{rem}
  $A_1(\C),~\D$ und $\hat\D$ sind nicht kommutative Algebren.
\end{rem}

\begin{prop} \label{prop:weyl_eundeutige_schreibung}
  Jedes Element in $A_1(\C)$ (bzw. $\D$ oder $\hat{\D}$) kann auf eindeutige
  weiße als $\sum_{i=0}^na_i(x)\partial_x^i$, mit $a_i(x)\in A_1(\C)$ (bzw.
  $\D$ oder $\hat{\D}$), geschrieben werden
\end{prop}
\begin{proof}
  \cite[Proposition 1.2.3]{sabbah_cimpa90}
  \begin{comment}
    ein teil des Beweises ist "left as an exersice"
  \end{comment}
\end{proof}

\begin{lem}
  Es gelten die Formeln\\
  %\begin{eqnarray*}
    %[\partial_x,x^k] & = & kx^{k-1}\\
    %[\partial_x^j,x] & = & j\partial_x^{j-1}\\
    %[\partial_x^j,x^k] & = & \sum_{i\geq1}\frac{}{}x^{k-i}\partial_x^{j-i}\\
  %\end{eqnarray*}
  $ [\partial_x,x^k] = kx^{k-1} $\\
  $ [\partial_x^j,x] = j\partial_x^{j-1} $\\
  $ [\partial_x^j,x^k] = \sum_{i\geq1}\frac{k(k-1)\cdots(k-i+1)
    \cdot j(j-1)\cdots(j-i+1)}{i!}x^{k-i}\partial_x^{j-i} $
\end{lem}
\begin{proof}
  Zula Barbara
\end{proof}

\subsection{Weyl Algebra als Graduierter Ring}
% Treffen 2
Sei $A$ nun einer der drei Koeffizienten Ringe, welche zuvor behandelt wurden.
Der Ring $A<\partial_x>$ kommt zusammen mit einer aufsteigenden Filtrierung,
welche wir mit $F(A<\partial_x)$ bezeichen werden.
Sei $P$ ein bzgl. \ref{prop:weyl_eundeutige_schreibung} minimal geschriebener
Operator, so ist $P$ in $F_k$ falls der maximale Grad von $\partial_x$ in $P$
kleiner oder gleich $k$. So definiere den Grad $deg P$ von $P$ als die 
Eindeutige ganze Zahl $k$ mit $P\in F_kA<\partial_x>\slash F_{k-1}<\partial_x>$
\begin{comment}
  Unabhängigkeit von Schreibung wird in Sabbah Script behauptet
\end{comment}

\section{Struktur von Links-Idealen auf $\D$}

\section{Lokalisierung eines $\C\{x\}$-Modules} %TODO: Modules or Moduls

\section{Lokalisierung eines holonomen $\D$-Modules}

% vim: set ft=tex :
