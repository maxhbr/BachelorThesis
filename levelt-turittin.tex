\begin{comment}
  Quellen:\\
  sabbah\_cimpa90 seite 28 / 30
\end{comment}

Ab hier werden wir nur noch formale Meromorphe Zusammenhänge betrachten. Alle
bisher getroffenen Aussagen gelten für diese aber analog.

\begin{comment}
  % sabbah\_cimpba90 Seite 30
  Sei $M_{\hat{K}}=\cD_{\hat{K}}/\cD_{\hat{K}}\cdot P$ und nehme an, dass $N(P)$
  zumindes 2 nichttriviale Steigungen hat. Spalte $N(P)=N_1\dot\cup N_2$ in 2
  Teile. Dann gilt:

  \begin{lem}
    Es existiert eine Aufteilung $P=P_1P_2$ mit:
    \begin{itemize}
      \item $N(P_1)\subset N_1$ und $N(P_2)\subset N_2$
      \item A ist eine kante von ...
    \end{itemize}
  \end{lem}
\end{comment}

\begin{thm}
  \cite[Thm 5.3.1]{sabbah_cimpa90}
\end{thm}

% vim: set ft=tex :
